\chapter{Geschichte und Legenden}
\label{legendenundgeschichten}

\begin{quotation}
\par \textit{\glqq Eine Zeit der Drachen\index{D!Drachen} und der Magie\index{M!Magie}.\grqq}\\
Arkan sprach etwas leiser. \textit{\glqq So musst Du Dir vorstellen, war dieses Land vor dem Umbruch.\grqq}
\par Arn wurde aufmerksamer. Wenn sein Meister anfing zu fl"ustern, dann stand eine Geschichte bevor. Eine jener Legenden\index{L!Legende}, die dieses Land gepr"agt und zu dem gemacht haben, was es jetzt ist. \textit{\glqq Was ist dann passiert? Was hat sich ver"andert.\grqq}
\par \textit{\glqq Um zu verstehen warum sich alles ver"anderte, muss man verstehen, wie es vorher war. Vor allem, bevor alles begann ... \grqq}
\end{quotation}

\section{Die Vier Zeitalter\index{Z!Zeitalter, die vier}}

\parpic[l]{\epsfig{file=pics/capitals/d.eps, scale=0.5}}ie Weltgeschichte l"asst sich grob in 3 Zeitabschnitte einteilen und wir stehen bereits am Anfang eines weiteren. Die Chronisten haben mit jedem einschneidenden Erlebniss eine neue Zeitrechnung angefangen, so dass man bis zum heutigen Tage von drei Zeitaltern oder drei Jahrtausenden spricht. Jedes Millenium hat das Land gezeichnet bzw. gepr"agt und die Entwicklung der Bewohner beeinflusst.
\par Die Jahreszahlen, die in den allgemeinen Tabellen verzeichnet sind, sollten f"ur den Leser jedoch weniger von Bedeutung, als die Ereignisse, die dahinter stehen. Da gerade in den Jahrhunderten der Finsterniss und des Krieges etwas weniger Genauigkeit auf das Datum als auf das "Uberleben und die "Uberlieferung gelegt wurde, m"ogen hier gro"sz"ugig einige Abweichungen in Kauf genommen werden.

	\subsection{Erstes Zeitalter\index{Z!Zeitalter, erstes}: Drachenkrieg\index{D!Drachenkrieg}}
\begin{quotation}
\par\textit{\glqq Um etwas Neues zu erschaffen muss Altes zerst"ort werden. In den Existenzebenen ist nur Platz f"ur Eines von Beidem. \grqq}
\end{quotation}
\parpic[l]{\epsfig{file=pics/capitals/n.eps, scale=0.5}}ur wenig ist
bekannt �ber die Welt vor dem ersten Zeitalter\index{Z!Zeitalter,
erstes}, liegt dieses doch mittlerweile knapp 3000 Jahre zur"uck und
die Augenzeugen jener Epoche\index{E!Epoche} sind rar geworden. In
Wahrheit gesprochen sind die einzigen Wesen, die ein solches Alter
erreichen k"onnen die Drachen\index{D!Drache}. Und von denen wei"s
ja nun jedes Kind, dass sie ausahmslos von der Bildfl"ache
verschwunden sind. Was die Chronisten\index{C!Chronist} aus dieser
Zeit zu berichten wissen beruht auf Schriften und einigen
Steinmei"seleien, die in ungef"ahr diese Zeit datiert wurden.
\par In einen sind sich jedoch alle einig: Das erste Zeitalter\index{Z!Zeitalter, erstes} begann mit dem Erscheinen der Drachen\index{D!Drache}.
\par Es muss der Ausbruch des Chaos\index{C!Chaos} gewesen sein.
Begleitet von Springfluten und Vulkanausbr"uchen erschienen sie in gigantischen Schw"armen.
Der Himmel verdunkelte sich und Donnerst"urme erschienen am Horizont.
Mit ihren Schwingen erschufen sie Wirbelst"urme, die das Land zerst"orten.
Und mit Ihnen kamen die Drachenkrieger\index{D!Drachenkrieger}.
Humanoide Lebensformen mit dem Aussehen eines Drachen;
Geschuppte Haut, einige ihrerseits mit Schwingen, andere mit der Gabe des Drachenodems\index{D!Drachenodem}.
\par Keiner kann so genau sagen woher sie gekommen sind oder warum. Doch die Frage nach dem \textit{Warum} schien eher ein verzweifeltes Flehen nach einer Erkl"arung als eine ernst gemeinte Frage zu sein. Die Kreaturen zogen durch das Land und s"ahten Zerst"orung. Wer nicht get"otet wurde, wurde versklavt.
\par Einige Theorien gaben den Schwarzmagiern\index{S!Schwarzmagier} Schuld an der Beschwrung der Drachen. Andere wiederum suchen die Erkl"arung in dem Zorn der G"otter.
\par Nach einem Krieg, der fast vier Jahrhunderte anhielt und dessen Sieger schon bei seinem Ausbruch feststand, schien Ruhe in das Land einzukehren. Dieses lag weniger an der Tatsache, dass der Kampfesmut versiegte sondern eher daran, dass einfach nichts mehr da war, was sie vernichten konnten.
\parpic[r]{\epsfig{file=pics/moonshadow.eps, scale=0.3}}
Die "Uberlebenden dieser Zeit wurden in Lagern als Sklaven gehalten und zum Bau neuer Geb"aude gezwungen. Die Krieger sprachen von den Vorbereitungen zur Ankunft der F"unf\index{F!Fnf, die}. Es wurden gigantische Tempel und unterirdische Anlagen gebaut. Die Oberfl"ache war unbewohnbar geworden.
\par Vor etwas mehr als 2000 Jahren, am Ende des ersten Zeitalters, erschienen sie. Wer bis dahin glaubte, dass die Drachen\index{D!Drache}, die die Kriege f"uhrten zu den gr"o"sten Kreaturen z"ahlte, wurde sp"atestens jetzt eines besseren belehrt. Die Herrscher der Drachen\index{D!Drache}, f"ur die Jahrhunderte eine neue Welt erschaffen wurde, hielten es f"ur an der Zeit sich zu offenbahren. Monstr"ose Kreaturen mit der Macht der Urgewalten\index{U!Urgewalt}. Jede von Ihnen als elementare Kraft. Sie sollten bekannt werden als die Inkarnation dessen, was sp"ater die Magie\index{M!Magie} wurde.
\\
\par \textit{Antares}\index{A!Antares}, die Kraft des Geistes,
\par \textit{Par}\index{P!Par}, die Kraft des Wassers,
\par \textit{Lucien}\index{L!Lucien}, die Kraft der Luft,
\par \textit{Epher}\index{E!Epher}, die Kraft des Feuers und
\par \textit{Orgoth}\index{O!Orgoth}, die Kraft der Erde.
\\
\par Die Welt wurde aufgeteilt und jede der Urgewalten\index{U!Urgewalt} herrschte "uber sein Reich unabh"angig von den Anderen. Zuerst "anderte sich nicht viel an dem Zustand. Die Menschen waren weiter Sklaven\index{I!Index} und jede Stunde ihres erb"armlichen Lebens damit besch"aftigt einen Stein nach dem anderen aus dem Fels zu schlagen, nur um dann einen Stein nach dem anderen "ubereinander zu Mauern zu stapeln.
\par Die meisten Zitadellen und Bergfesten stammen aus jener Zeit.

	\subsection{Zweites Zeitalter\index{Z!Zeitalter, zweites}: Herrschaft der Finsternis\index{H!Herrschafft der Finsternis}}
\par �ber das zweite Zeitalter existieren bisher nur wenige verl�ssliche Berichte, so
dass wir an dieser Stelle erst noch weitere Recherchen anstellen m�ssen.

	\subsection{Drittes Zeitalter\index{Z!Zeitalter, drittes}: Der Umbruch\index{U!Umbruch}}
\par Die Hauptausrichtung des dritten Zeitalters galt dem Wiederaufbau. Die ...

	\subsection{Viertes Zeitalter\index{Z!Zeitalter, viertes}: Drachend"ammerung\index{D!Drachendaemmerung}}
\parpic[l]{\epsfig{file=pics/capitals/d.eps, scale=0.5}}ie Zukunft\index{Z!Zukunft} diese Jahrtausends\index{J!Jahrtausend}
steht noch in den Sternen, hat es doch noch nicht einmal
angebrochen. Dennoch versprechen die Astrologen ein aufreibendes
Zeitalter\index{Z!Zeitalter}.
\par Nicht nur, dass mit diesem Jahrtausend\index{J!Jahrtausend}
der Bannspruch\index{B!Bannspruch} "uber die
Drachen\index{D!Drachen} seine Wirkung verliert, sondern auch die
Konzentration der Magie weckt Bef"urchtungen in den "Altesten. Zu
tief sitzt der Schock der ersten Zeitalter\index{Z!Zeitalter} und
die Unterdr"uckung. Jetzt, da der Bannspruch\index{B!Bannspruch}
bald seine Wirkung verliert, muss dringend ein Weg der Erneuerung
gefunden werden.
\par Doch die Kenntnisse vergangener Tage sind verlorend und die
Zeit zur Entwicklung neuer Zauber knapp. Die weisesten und
m"achtigsten Magier haben sich auf die Suche nach den Artefakten des
zweiten Zeitalters gemacht, um Hinweise auf das Auftauchen und die
Vernichtung der Drachen zu finden. Die Spuren jener Zeit sind stark
verwischt und die Ebenen des Krieges verw"ustet. Viel erschwerender
kommt jedoch hinzu, dass die Magier damals "uber den Kontinent
verteilt waren, als sie den Bannspruch wirkten, ebenso deren
Artefakte.

	\subsection{Zeitleiste}
\parpic[l]{\epsfig{file=pics/capitals/d.eps, scale=0.5}}ie Zeitrechnung wird mit Beginn eines jeden Zeitalters
neu gez"ahlt. Spricht man vom Jahr 546 ist das Jahr 546 in dem
aktuellen Zeitalter gemeint. Will man "uber ein Jahr in einem
anderen Zeitalter sprechen, dann setzt man die Nummer des Zeitalters
vorweg. Das Jahr 546 im zweiten Zeitalter ist also das Jahr 2.546.
\par Die Angaben in den Tabellen \ref{tabelle_zeitalter1},
\ref{tabelle_zeitalter2} und \ref{tabelle_zeitalter3} (ab Seite
\pageref{tabelle_zeitalter1}) sind historischen Schriften entnommen.
Da gerade in der dunklen Zeit keine M"oglichkeit zur genauen
Zeitbestimmung bestand, k"onnen hier Differenzen zu anderen
Schriften durchaus m"oglich sein.
\par Das geradezu faszinierende an der Vergangenheit dieses
Landes ist die Kontinuit"at mit der die Geschichte geschrieben wird.
In der Vergangenheit stand am Ende eines Millenniums immer ein
Umbruch, der das definitve Ende des vergangenen und den Anfang eines
neuen Zeitalters bedeutete. Gerade diese mystischen Ereignisse sind
es, die den Weisen und Historikern zur Zeit den Angstschwei"s auf
der Stirn stehen lassen. Zur Zeit schreiben wir das Jahr 3.997 und
bis zum Beginn des n"achsten Jahrtausend sind es nur noch drei
Jahre. Was wird uns erwarten?

% Erstes Zeitalter
\begin{longtable}{|r|p{10cm}|}
\hline
Jahr   & Wichtige Ereignisse \\
\hline
0      & Erscheinen der Drachen; Beginn des Jahrhundertkrieges\\
397    & Endg"ultige Versklavung der Menschheit\\
ab 407 & Bau der Unterirdischen Katakomben und Labyrinthe\\
647    & erste Freiheitsk"ampfergruppen bilden sich\\
870    & Angriff auf das Zentrum der Drachen\\
876    & Erscheinen der Urgewalten\\
877    & Niederschlagung aller Widerst"ande und endg"ultiger Fall der Menschheit in die Sklaverei\\
992    & Ein Magier in der Sklaverei unter Antares Truppen erkennt als Erster Mensch den Einfluss der Urgewalten auf die magischen St"urme und erforscht diese im Geheimen\\
\hline
\caption[Zeitleiste: Erstes Zeitalter]{Wichtigste Ereignisse im ersten Zeitalter}
\label{tabelle_zeitalter1}
\end{longtable}
% Zweites Zeitalter
\begin{longtable}{|r|p{10cm}|}
\hline
Jahr   & Wichtige Ereignisse \\
\hline
0      & Erste Forschungsergebnisse bei der Nutzung der neuen magischen Quellen, erster Hoffnugsschimmer zur Nutzung dieser f"ur die Freiheitserringung\\
1      & Erkennung erster Nebenwirkungen bei der Nutzung der neuen magischen Quellen\\
12     & Entwicklung des ersten Zaubers\\
63     & Einer gro"se Gruppe von Freiheitsk"ampfern gelingt auf magische Weise der Weg in die Freiheit, Widerstandsaufbau beginnt, in den n"achsten Jahren finden immer wieder Befreiungen von Gefangenen aus den Festungen statt\\
80     & Bau erster befestigten Siedlungen in still gelegten Minenwerken\\
750    & Erstes Auftauchen der Priesterschaft der G"otter, wirksame Wunder "uberzeugen die Menscheit von der Existenz der G"otter\\
956    & Die Urgewalten finden sich alle in der Zitadelle Orgoths in den Westlanden ein\\
957    & Sturz der Festung Luciens an der Ostk"uste\\
999    & Versammlung der Truppen in den Ebenen; finale Schlacht am Trauerfelsen\\
\hline
\caption[Zeitleiste: Zweites Zeitalter]{Wichtigste Ereignisse im zweiten Zeitalter}
\label{tabelle_zeitalter2}
\end{longtable}
% Drittes Zeitalter
\begin{longtable}{|r|p{10cm}|}
\hline
Jahr   & Wichtige Ereignisse \\
\hline
0      & Vernichtung / Verbannung der Drachen; Zauberspruch des Millenniums\\
16     & Gr"undung des 18er-Rats\\
18     & Errichtung der Festung Patria Pacis im Zentrum des gro"sen Kontinents\\
25     & Gr"undung der ersten neuen Siedlungen in den Zitadellen der Urgewalten. Sie tragen die Namen Pax Antares\index{P!Pax Antares} (Reich der Mitte), Pax Par\index{P!Pax Par} (S"udk"uste), Pax Lucien\index{P!Pax Lucien} (Ostk"uste), Pax Epher\index{P!Pax Epher} (Der Hohe Norden) und Pax Orgoth\index{P!Pax Orgoth} (Westlande). Sie gelten als die Hauptst"adte der 5 gro"sen Reiche\\
28     & Gr"undung einer neuen Magiergilde, zur Wahrung und Erforschung der neuen Magiequelle, die als \textit{\glqq Wahre Magie\grqq} bekannt wurde\\
29     & Errichtung des Turms der Magier im \textit{Land der Toten} als Sitz des 18er-Rats und Forschungs- und Schulungszentrum f"ur die \textit{Wahre Magie}\\
30     & Wahl des ersten Kaisers "uber die 5 Reiche und Wahl der 5 K"onige zur Verwaltung dieser\\
75     & R"ucktritt des Kaisers mit 128 Lebensjahren in den Ruhestand, Aufl"osung dieses Titels und damit Gr"undung von 5 unabh"angigen K"onigreichen ohne gemeinsame Verwaltung\\
997    & aktuelles Zeitgeschehen\\
\hline
\caption[Zeitleiste: Drittes Zeitalter]{Wichtigste Ereignisse im dritten Zeitalter}
\label{tabelle_zeitalter3}
\end{longtable}


\begin{quotation}
\par \textit{\glqq Nun, mein Sch"uler, hast du eine Menge "uber die Historie unserer V"olker gelernt. So steht es geschrieben und so ist es geschehen.\grqq}
\par Arkan setzte sich zur"uck in seinen Stuhl und legte die H"ande gefaltet auf den Tisch. Herausfordernd blickte er seinen Sch"uler an. Arn war aufgew"uhlt. Noch nie zuvor hatte er die Geschichte in solcher Vollkommenheit geh"ort. Trotzdem hatte er das Gef"uhl, als wenn es etwas fehlen wrde.
\par \textit{\glqq Meister\grqq}, z"ogernd sprach er weiter, \textit{\glqq Was ist mit den anderen Legenden, die in den B"uchern stehen und von Generation zu Generation weitergegeben werden. Was ist davon wahr und was erfunden?\grqq}
\par \textit{\glqq Nun, mein junger ungeduldiger Sch"uler\grqq}, Arn meinte ein leichtes zufriedenes L"acheln "uber des Meisters Mund huschen zu sehen, \textit{\glqq Wie du wei"st sind unsere L"ander und Geschichten voll von Legenden. Jede einzelne zu rezitieren w"urde den Rest deines Studiums in Anspruch nehmen und ich habe noch einiges anderes f"ur dich auf dem Lehrplan stehen. Es gibt durchaus noch Wichtigeres. Wie weit bist du mit der Abschrift aus dem Buche \textbf{Georn}?\grqq}
\par Entt"auschung breitete sich auf Arns Gesicht aus. \textit{\glqq Ihr habt nat"urlich recht Meister. Die Abschrift wird nocht etwas Zeit in Anspruch nehmen. Ich werde mich allerdings sofort daran setzen.\grqq}
\par \textit{\glqq Den Flei"s lobe ich mir. Deswegen schlage ich vor, dass wir uns morgen Abend zur selben Stunde wieder hier treffen, und ich will sehen, ob ich einige interessante Legenden finde.\grqq}
\par Arn konnte sich ein leises Schmunzeln nicht verkraften. Sein Meister ist einfach unberechenbar. Aber mit dieser Vorfreude im Kopf geht die Arbeit an den Abschriften gleich doppelt so leicht.

\end{quotation}
\section{Die Mondkinder\index{M!Mondkinder}} 
\label{mondkinder}

\section{Die Harlekine\index{H!Harlekine}}
\label{harlekine}

% Elfen sie versuchen die schwarze Seele zu kontrollieren

\section{Der Turm der Magier\index{T!Turm der Magier}}
\label{turmdermagier}

\parpic[l]{\epsfig{file=pics/capitals/d.eps, scale=0.5}}er \textit{Turm der Magier} wurde Anfang dieses Jahrtausends als
Forschungsst"atte f"ur die \textit{Wahre Magie} errichtet. Er ist
gleichzeitig auch der Sitz des 18er-Rats.
\par Der Rat der 18 setzte sich urspr"unglich aus den
Aufstandsf"uhrern von damals zusammen. Da f"ur die normale
Menschheit die Bedeutung der \textit{Wahren Magie} und die Wahrung
derer, sowie der Schutz vor den Drachen und die Aufrechterhaltung
des Bannzaubers in den Hintergrund geraten ist, sind dessen
Mitglieder nur noch Mitglieder der Magierzirkel. Tats"achlich ist es
so, dass viele nicht einmal mehr von der Existenz dieses Rates,
geschweige denn um seine Bedeutung wissen.
\par Im Gegensatz zu den T"urmen und Schulungseinrichtungen der Zirkel
ist dieser Turm nicht einer bestimmten Ausrichtung gewidmet, denn
der Zweck gilt nicht der "ublichen Magie. Hinter den T"uren dreht
sich in Forschung und Schulung alles um die \textit{Wahre Magie}.
Dieser Umstand ist nicht zuletzt einer der Gr"unde, warum der Turm
im \textit{Land der Toten}\index[regionen]{L!Land der Toten}, also im westlichen Reich errichtet
wurde. Hier sind die Str"omungen der Magie am st"arksten, da hier
auch vor knapp 1000 Jahren die entscheidende Schlacht gegen die
Drachen statt fand.
\par Ein zweiter Grund ist in dem Bannzauber zu finden, der zum Wohle 
aller gesprochen wurde. Dieser wird aus der W"uste gespeist, dem Ort 
des Sprechens, und f"ur dessen Aufrechterhaltung ist es notwendig, 
dass sich jederzeit m"achtige Zauberk"unstler in dessen N"ahe befinden.

\section{Das Portal\index{P!Portal, das}}
\label{dasportal} [Ein Bericht von \mbox{Graf} von
\mbox{Hohenbr"uck}\index[personen]{H!von Hohenbr�ck}, Kartograph im Auftrag des k"oniglichen Amtes
f"ur Landerfassung]\\

\parpic[l]{\epsfig{file=pics/capitals/d.eps, scale=0.5}}ie Reisenden
des "ostlichen Reiches berichten von der Entedckung eines
gigantischen Bauwerkes in den Ebenen. Seine Turmspitzen ragen weit
in den Himmel hinein und seine Erscheinung ist faszinierend und
erschreckend zu gleich.

\par Doch warum ist dieses Bauwerk noch nicht in den Karten verzeichnet?

\par Im Auftrag unseres Amtes aus dem Jahre 956 machte ich mich mit
einer Expedition auf den Weg, den Standort des Turmes zu vermessen
und ihn in die Karten zu "ubertragen. Ein Unternehmen, was sich als
weitaus schwieriger herausstellen sollte, als ich es zuerst
angenommen habe.

Einige Reisende der Strecke berichteten diesen Turm niemals gesehen
zu haben, und dass obwohl sie die identische Strecke gefahren sind
wie andere Reisende. Andere berichteten das Bauwerk nur auf einer
Strecke gesehen zu haben, eine letzte Gruppe sah es mal auf der
einen, mal auf der andere Strecke.

Schon im Laufe der Voruntersuchungen verweifelte ich an diesen
Informationen. Wurde ich losgeschickt ein Phantom zu verzeichnen? Da
ich die Hoffnung aufgegeben hatte verl"assliche Informationen "uber
den Standort zu bekommen, stellte ich eine Expedition auf, um den
Standort pers"onlich ausfindig zu machen. Die ben"otigten
finanziellen Mittel wurden mir zur Verf"ugung gestellt und der
Aufbruch in das fr"uhe n"achste Jahr datiert. F"ur die Dauer der
Expedition werde ich ein Forschungsbuch anlegen und unsere Funde
genauestens dokumentieren.

\begin{quotation}
\begin{flushright}\textit{997-7-7}\end{flushright}
\textit{Mein K"onig und Gebieter,}\\

\textit{mit Freuden bin ich Ihrer Anforderung nachgekommen und habe die Unterlagen und Manuskripte 
nach der von Ihnen gew"unschten Expedition durchsucht. Doch zu meinem gro"sen Leid kann ich nur wenig 
berichten.}\\
\textit{Aus den Unterlagen geht hervor, dass Graf von Hohenbr"uck seinerzeit als vermisst gemeldet 
wurde, R"uckmeldungen gab es keine. Auch Nachforschungen bei den Nachfahren des ehrw"urdigen Grafen 
verliefen ergebnislos.}\\
\textit{Schon wollte ich die Suche abbrechen und euch die Ergebnisse mitteilen, als mich der Zufall 
zu einigen noch nicht katalogisierten Dokumenten f"uhrte, die ich unseren Archiven ausmachen konnte. 
Meinem Erstaunen kann gar nicht genug Ausdruck verliehen werden und schon gar nicht kann ich in Worte 
fassen was ich fand. Zu urteilen obliegt jedoch nur Euch, meine k"onigliche Hoheit, und deswegen "ubersende 
ich euch beiliegend das Manuskript zur Durchsicht.}\\

\textit{Hochachtungsvoll,}\\
\textit{Pablo Hiluia\index[personen]{P!Pabloa Hiluia}}
\end{quotation}

\par Mit dem Schreiben des Bibliothekars seiner Hoheit wurde ihm eine gebundene Mappe mit einigen 
vergilbten Bl�ttern �berreicht. Das Deckblatt tr�gt die Insignien des Grafen von Hohenbr�ck und der 
Themenbeschreibung folgend scheint es sich hierbei um das Forschungsbuch zu handeln. Die Bl�tter sind 
arg in Mitleidenschaft gezogen worden, die Schrift ist jedoch noch gut zu lesen. Die Seiten machen 
einen sortierten Eindruck, jeder Eintrag wurde auf einer eigenen Seite dokumentiert.


\begin{longtable}{|p{15cm}|}
\hline
5. Tag im ersten Monat des 957. Jahres im Zeitaltes des Umbruchs\\
\\
Nachdem ich mich nun entschlossen habe, dem Ph�nomen des Turmes pers�nlich auf den Grund zu gehen, 
habe ich eine Expedition zusammengestellt. Morgen wird der erste Tag unserer Reise sein und wir sind 
voller Hoffnung, das Ziel unserer Reise in einigen Wochen erreicht zu haben.\\
\hline
\end{longtable}

\begin{longtable}{|p{15cm}|}
\hline
6. Tag im ersten Monat des 957. Jahres im Zeitaltes des Umbruchs\\
\\
Der Aufbruch ging planm��ig voran, in der Morgend�mmerung haben wir Pax Lucien verlassen und sind mit 
der Karawane in Richtung Westen auf den bekannten Wegen aufgebrochen. Die Stimmung ist gut.\\
\hline
\end{longtable}

\begin{longtable}{|p{15cm}|}
\hline
14. Tag im ersten Monat des 957. Jahres im Zeitaltes des Umbruchs\\
\\
Heute sahen wir zum ersten Mal den Turm am Horizont, wir haben also das erste Etappenziel unserer 
Reise erreicht. Aus der Ferne betrachtet wirkt der Turm gewaltig, selbst mit unseren Me�instrumenten 
k�nnen wir nicht ausmachen, welche H�he er hat. Wir vermuten ihn jedoch noch in weiter Ferne, da 
keinerlei Struktur auf seiner Oberfl�che erkennbar ist.\\
\hline
\end{longtable}

\begin{longtable}{|p{15cm}|}
\hline
15. Tag im ersten Monat des 957. Jahres im Zeitaltes des Umbruchs\\
\\
Entsetzen machte sich heute nach dem Erwachen breit, unser Ziel war am Horizont verschwunden. 
Einige der abergl�uberischen Karawanisten konnten nur mit M�he davon �berzeugt werden, dass es 
wohl an einer Wetterfront liegt, die den weit entfernt liegenden Turm verdeckt.\\
Ich f�r meinen Teil habe� Zweifel an meiner Notl�ge und besinne mich der Erz�hlungen �ber diesen Turm.\\
\hline
\end{longtable}

\begin{longtable}{|p{15cm}|}
\hline
21. Tag im ersten Monat des 957. Jahres im Zeitaltes des Umbruchs\\
\\
Wie bereits dem Eintrag vom Vortag zu entnehmen ist, haben wir dem Turm seit gestern Abend wieder in 
Sicht. Er liegt immer noch auf der gleichen Route, doch scheinen wir ihm in den letzten Tagen nicht 
n�her gekommen zu sein. Der Mi�mut in der Gruppe macht sich weiter breit.\\
\hline
\end{longtable}

\begin{longtable}{|p{15cm}|}
\hline
23. Tag im ersten Monat des 957. Jahres im Zeitaltes des Umbruchs\\
\\
�berraschen und Entsetzen macht sich heute nach dem Erwachen breit, unser Lager befindet sich am 
Fu�e des Turmes. Einige unserer Leute haben sich im fr�hen Morgen alleine auf den R�ckweg gemacht, 
nur der Kern der Expedition bleibt zur�ck. Am Fu�e des Turmes befindet sich ein gro�es Tor, wir 
werden im Laufe des Tages einige Untersuchungen anstellen und schauen, ob wir das Tor �ffnen k�nnen. 
Durch das pl�tzliche Auftauchen dieses Bauwerks, ist es mir jedoch nicht m�glich, seine genaue Lage zu 
verzeichnen, vielleicht auf dem R�ckweg.\\
\hline
\end{longtable}

\par Hier endet des Manuskript. Angeh�ngt befinden sich einige Dokumente, die den Ursprung der Unterlagen 
dokumentieren und dessen Einlagerung in die Bibliothek im Jahre 954. �ber den Verbleib des Verfassers 
oder seiner Expedition ist leider nichts bekannt. Auch wird nicht erw�hnt, wie die Dokumente in den 
Besitz der Bibliothek kamen.

\section{Das Tal der Verlorenen und Verdammten\index[regionen]{T!Tal der Verlorenen und Verdammten}}
\label{verloreneundverdammte}
\parpic[l]{\epsfig{file=pics/capitals/h.eps, scale=0.5}}inter dem
\textit{Steinernen Schwert}\index[regionen]{S!steinerne Schwert, das} gelegen, findet der Reisende das
\textit{Tal der Verlorenen und Verdammten}. Ein recht seltsamer Name
f"ur eine unbekannte Gegend, aber wir wollen in den folgenden
Abs"atzen seine Bedeutung erl"autern. Diese ist n"amlich nicht
unbedingt eindeutig.
\begin{figure}[hbtp]
\begin{center}
\epsfig{file=pics/wueste2.eps, scale=0.5}
\caption{Der Eingang zum Tal der Verlorenen \& Verdammten}
\end{center}
\end{figure}
\par Die Ureinwohner dieser Grenzregion pflegen die sterblichen �berreste
ihrer Toten zu verbrennen und in Urnen zu f�llen. Diese werden auf
Wagen geladen und in regelm"a"sigen Abst"anden durch den Gebirgspass
zum Taltor gebracht. Dort wird der Wagen, der von einem Maultier
gezogen wird, seiner selbst "uberlassen. Die Tiere trotten voraus
ins Unbekannte.
\par Uneingeweihte fragen sich nun: Was macht das f"ur einen Sinn?
\par Ganz einfach. Der Glaube der Einwohner besagt, dass es an einem
Ort nicht gen"ugend Platz f"ur die Seelen aller Toten gibt. Aus
diesem Grund werden nur H"ohergestellte auf den lokalen Friedh"ofen
beigesetzt. Normale B"urger werden den langen Weg in den S"uden
gebracht und dort in den W"aldern verscharrt. Durch die
weitl"aufigen und hohen B"aume finden hier mehr Tote Platz.
Diejenigen, die kein reines Leben f"uhrten und sich zu Lebzeiten
etwas zu Schulden haben kommen lassen, werden auf den Weg in den
Norden geschickt. Immerhin verdient es Anerkennung, was f�r ein
Aufwand betrieben wird, um die Ausgesto�enen zu Grabe zu tragen.

\par Der normale Reisende ermittelt die Bedeutung des Tals aus
einem anderen Umstand. Expeditionen, die in diese Gegen gef"uhrt
wurden, sind niemals zur"uckgekehrt.

\section{Helden und Schurken}
\label{heldenundschurken}
\par Die folgenden Kapitel stellen ein paar der bekannten und
legend�ren Personen und Gruppierungen der letzten Jahrhunderte vor.
Einiges basiert auf Tatsachen, anderes m�ge Legende sein.

\subsection{Der Orden der schwarzen Rose\index{O!Orden der schwarzen Rose}}
\parpic[l]{\epsfig{file=pics/capitals/d.eps, scale=0.5}}er Orden der schwarzen Rose ist ein Zusammenschluss von
Magiewirkern aus allen m�glichen Fachrichtungen. Sein Ziel ist das
Studium und das damit verbundene Erlangen von Wissen.
\par Diese Tatsachen alleine w�rde die Gemeinschaft nicht
spektakul�rer erscheinen lassen, als jeden x-beliebigen Zirkel von
Magier, h�tte sie sich nicht den Fachrichtungen Dimension und
Beschw�rung verschrieben.
\par Die Mitglieder der schwarzen Rose operieren im Geheimen.
Niemand wei�, wo sie sitzen, niemand wei� wer sie sind. Man sagt
jedoch, hinter vorgehaltener Hand, dass es hochrangige Mitglieder der
Zirkel seinen, die den Orden gegr�ndet haben und ihn f�hren.
\par Neue Mitglieder werden gezielt ausgew�hlt. Nach einer Reihe
geheimer Pr�fungen, von denen die Aspiranten nichts mitbekommen
werden, wird ihnen die Mitglidschaft angeboten. Was mit denen
passiert, die ablehnen ist ebenfalls nur Spekulation.

\subsection{Die Sturmreiter}

\subsection{Die Freibeuter des S�dens}

\subsection{Das Phantom der Ostk�ste}

\subsection{Die Ritter der Nordmark\index{R!Ritter der Nordmark}}
\parpic[l]{\epsfig{file=pics/capitals/d.eps, scale=0.5}}ie Ritter der Nordmark haben bei der Verteigung des �stlichen
Reiches gegen die Orkhorden 867 ihrem F�rsten gro�en Ruhm und Ehre
gegen�ber dem K�nig gebracht. Zahlreiche marodierende und
brandschatzende Horden von Orks fielen von Norden in das Reich ein,
ihre Zahl war so gewaltig, dass kein Heer alleine ausreichte, um sie
zu stoppen. Den Rittern der Nordmark um F�rst Reinier zur Nordmark\index[personen]{P!Reinier zur Nordmark},
zu dieser Zeit noch ein kleiner Orden, gelang es die Orks mit
einigen Kriegslisten aus dem Reich zur�ck in den Norden zu
vertreiben.
\par Renier wurde von dem K�nig f�r die Tapferkeit mit
einer Reichserweiterung belohnt, man bot ihm sogar einen gr��eren
Titel. Reinier lehnte dankend ab, sein Lohn sei der Triumph, er habe
nur im Dienste des K�nigs f�r den K�nig gehandelt, wie es sich f�r
einen Ritter geh�rt. Mehr als beeindruckt von diesen Worten erlie�
der K�nig einen Beschluss, der die Ritter der Nordmark zu den
Rittern des Reiches ernannte. Er stellte Reinier die n�tigen Mittel
zur Verf�gung den Orden auszubauen und noch weitere Krieger
aufzunehmen.
\par Der Orden wurde seit dem st�ndig erweitert und stellt seit
Mitte dieses Jahrhunderts das Heer des Ostreiches. Die Krieger des
Ordens sind f�r ihre Volksn�he und ihre Ritterlichkeit bekannt.

\section{Nachtklinge\index{N!Nachtklinge}}
\label{nachtklinge} [Aus dem Tagebuch des \mbox{Alessandro}, Aufzeichnung ohne Datierung, zweites Zeitalter]\\

\parpic[l]{\epsfig{file=pics/capitals/n.eps, scale=0.5}}achtklinge\index{N!Nachtklinge}, das Schwert des Ritters
Gunter zu Hohenaspen\index[personen]{G!Gunter zu Hohenaspen}. Legenden ranken sich um diese Waffe aus den Drachenkriegen. Gefertigt von den Kriegsschmieden\index{K!Kriegsschmied}
der Zwerge, gesegnet mit ihren einzigartigen Runen geh�rt Nachtklinge zu den Waffen, die in der Schlacht an vorderster Front gegen die
Drachen gef�hrt wurden. Man berichtet, dass der Tr�ger dieses Schmuckst�ckes sich furchtlos in den Zweikampf mit
diesen Bestien st�rzt und Nachtklinges scharfe Schneide ihre Schuppen durchbricht.

\par Wie jede Waffe aus der Schmiede der Zwerge ist Nachtklinge das Ergebnis bester Schmiedekunst und als solches
leicht und behende zu f�hren. Das Material welches bei ihrer Erschaffung benutzt wurde ist nicht zu bestimmen, man
vermutet jedoch, dass es jenes geheimnisvolle Meteoreisen\index{M!Meteoreisen} ist, welches von den G�ttern in unsere Welt geschickt wurde.
Nachtklinges matt schwarze Oberfl�che l�sst keinerlei Reflektionen zu. Ihr Name ist in Zwergenrunen\index{Z!Zwergenrune} in die Klinge
gepr�gt.

\subsection*{Nachtklinge im Spiel}
\par Nachtklinge verleht seinem Tr�ger unb�ndigen Mut im Kampf gegen Drachen, so dass er keine Schw�che zeigt und
in ihrer Gegenwart nicht von Drachenangst befallen wird. Zudem bietet sie ihm eine gewisse Resistenz gegen Feuer,
so dass jeglicher Feuerschaden halbiert wird.

\begin{tabular}{l|c|c|c|c|c|c}
           & Ini  & Bonus     & Schaden & AbB    & BF    & Last\\
\hline
einh�ndig  &  1   & Str 10/15 & 3W+1    & 12/7/2 & 42/62 & 1\\
zweih�ndig & -1   & Str 10/14 & 3W+3    & 10/5/0 & 42/62 & 1\\
\end{tabular}\\

\par Wie bei jedem magischen Artefakt werden die F�higkeiten erst aktiviert, wenn der Tr�ger Kenntnis von den
F�higkeiten hat.
