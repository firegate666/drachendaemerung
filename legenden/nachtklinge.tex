\section{Nachtklinge\index{N!Nachtklinge}}
\label{nachtklinge} [Aus dem Tagebuch des \mbox{Alessandro}, Aufzeichnung ohne Datierung, zweites Zeitalter]\\

\parpic[l]{\epsfig{file=pics/capitals/n.eps, scale=0.5}}achtklinge\index{N!Nachtklinge}, das Schwert des Ritters
Gunter zu Hohenaspen\index[personen]{G!Gunter zu Hohenaspen}. Legenden ranken sich um diese Waffe aus den Drachenkriegen. Gefertigt von den Kriegsschmieden\index{K!Kriegsschmied}
der Zwerge, gesegnet mit ihren einzigartigen Runen geh�rt Nachtklinge zu den Waffen, die in der Schlacht an vorderster Front gegen die
Drachen gef�hrt wurden. Man berichtet, dass der Tr�ger dieses Schmuckst�ckes sich furchtlos in den Zweikampf mit
diesen Bestien st�rzt und Nachtklinges scharfe Schneide ihre Schuppen durchbricht.

\par Wie jede Waffe aus der Schmiede der Zwerge ist Nachtklinge das Ergebnis bester Schmiedekunst und als solches
leicht und behende zu f�hren. Das Material welches bei ihrer Erschaffung benutzt wurde ist nicht zu bestimmen, man
vermutet jedoch, dass es jenes geheimnisvolle Meteoreisen\index{M!Meteoreisen} ist, welches von den G�ttern in unsere Welt geschickt wurde.
Nachtklinges matt schwarze Oberfl�che l�sst keinerlei Reflektionen zu. Ihr Name ist in Zwergenrunen\index{Z!Zwergenrune} in die Klinge
gepr�gt.

\subsection*{Nachtklinge im Spiel}
\par Nachtklinge verleht seinem Tr�ger unb�ndigen Mut im Kampf gegen Drachen, so dass er keine Schw�che zeigt und
in ihrer Gegenwart nicht von Drachenangst befallen wird. Zudem bietet sie ihm eine gewisse Resistenz gegen Feuer,
so dass jeglicher Feuerschaden halbiert wird.

\begin{tabular}{l|c|c|c|c|c|c}
           & Ini  & Bonus     & Schaden & AbB    & BF    & Last\\
\hline
einh�ndig  &  1   & Str 10/15 & 3W+1    & 12/7/2 & 42/62 & 1\\
zweih�ndig & -1   & Str 10/14 & 3W+3    & 10/5/0 & 42/62 & 1\\
\end{tabular}\\

\par Wie bei jedem magischen Artefakt werden die F�higkeiten erst aktiviert, wenn der Tr�ger Kenntnis von den
F�higkeiten hat.