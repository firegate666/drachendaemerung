\section{Das Tal der Verlorenen und Verdammten\index[regionen]{T!Tal der Verlorenen und Verdammten}}
\label{verloreneundverdammte}
\parpic[l]{\epsfig{file=pics/capitals/h.eps, scale=0.5}}inter dem
\textit{Steinernen Schwert}\index[regionen]{S!steinerne Schwert, das} gelegen, findet der Reisende das
\textit{Tal der Verlorenen und Verdammten}. Ein recht seltsamer Name
f"ur eine unbekannte Gegend, aber wir wollen in den folgenden
Abs"atzen seine Bedeutung erl"autern. Diese ist n"amlich nicht
unbedingt eindeutig.
\begin{figure}[hbtp]
\begin{center}
\epsfig{file=pics/wueste2.eps, scale=0.5}
\caption{Der Eingang zum Tal der Verlorenen \& Verdammten}
\end{center}
\end{figure}
\par Die Ureinwohner dieser Grenzregion pflegen die sterblichen �berreste
ihrer Toten zu verbrennen und in Urnen zu f�llen. Diese werden auf
Wagen geladen und in regelm"a"sigen Abst"anden durch den Gebirgspass
zum Taltor gebracht. Dort wird der Wagen, der von einem Maultier
gezogen wird, seiner selbst "uberlassen. Die Tiere trotten voraus
ins Unbekannte.
\par Uneingeweihte fragen sich nun: Was macht das f"ur einen Sinn?
\par Ganz einfach. Der Glaube der Einwohner besagt, dass es an einem
Ort nicht gen"ugend Platz f"ur die Seelen aller Toten gibt. Aus
diesem Grund werden nur H"ohergestellte auf den lokalen Friedh"ofen
beigesetzt. Normale B"urger werden den langen Weg in den S"uden
gebracht und dort in den W"aldern verscharrt. Durch die
weitl"aufigen und hohen B"aume finden hier mehr Tote Platz.
Diejenigen, die kein reines Leben f"uhrten und sich zu Lebzeiten
etwas zu Schulden haben kommen lassen, werden auf den Weg in den
Norden geschickt. Immerhin verdient es Anerkennung, was f�r ein
Aufwand betrieben wird, um die Ausgesto�enen zu Grabe zu tragen.

\par Der normale Reisende ermittelt die Bedeutung des Tals aus
einem anderen Umstand. Expeditionen, die in diese Gegen gef"uhrt
wurden, sind niemals zur"uckgekehrt.
