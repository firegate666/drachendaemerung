\section{Das Portal\index{P!Portal, das}}
\label{dasportal} [Ein Bericht von \mbox{Graf} von
\mbox{Hohenbr"uck}\index[personen]{H!von Hohenbr�ck}, Kartograph im Auftrag des k"oniglichen Amtes
f"ur Landerfassung]\\

\parpic[l]{\epsfig{file=pics/capitals/d.eps, scale=0.5}}ie Reisenden
des "ostlichen Reiches berichten von der Entedckung eines
gigantischen Bauwerkes in den Ebenen. Seine Turmspitzen ragen weit
in den Himmel hinein und seine Erscheinung ist faszinierend und
erschreckend zu gleich.

\par Doch warum ist dieses Bauwerk noch nicht in den Karten verzeichnet?

\par Im Auftrag unseres Amtes aus dem Jahre 956 machte ich mich mit
einer Expedition auf den Weg, den Standort des Turmes zu vermessen
und ihn in die Karten zu "ubertragen. Ein Unternehmen, was sich als
weitaus schwieriger herausstellen sollte, als ich es zuerst
angenommen habe.

Einige Reisende der Strecke berichteten diesen Turm niemals gesehen
zu haben, und dass obwohl sie die identische Strecke gefahren sind
wie andere Reisende. Andere berichteten das Bauwerk nur auf einer
Strecke gesehen zu haben, eine letzte Gruppe sah es mal auf der
einen, mal auf der andere Strecke.

Schon im Laufe der Voruntersuchungen verweifelte ich an diesen
Informationen. Wurde ich losgeschickt ein Phantom zu verzeichnen? Da
ich die Hoffnung aufgegeben hatte verl"assliche Informationen "uber
den Standort zu bekommen, stellte ich eine Expedition auf, um den
Standort pers"onlich ausfindig zu machen. Die ben"otigten
finanziellen Mittel wurden mir zur Verf"ugung gestellt und der
Aufbruch in das fr"uhe n"achste Jahr datiert. F"ur die Dauer der
Expedition werde ich ein Forschungsbuch anlegen und unsere Funde
genauestens dokumentieren.

\begin{quotation}
\begin{flushright}\textit{997-7-7}\end{flushright}
\textit{Mein K"onig und Gebieter,}\\

\textit{mit Freuden bin ich Ihrer Anforderung nachgekommen und habe die Unterlagen und Manuskripte 
nach der von Ihnen gew"unschten Expedition durchsucht. Doch zu meinem gro"sen Leid kann ich nur wenig 
berichten.}\\
\textit{Aus den Unterlagen geht hervor, dass Graf von Hohenbr"uck seinerzeit als vermisst gemeldet 
wurde, R"uckmeldungen gab es keine. Auch Nachforschungen bei den Nachfahren des ehrw"urdigen Grafen 
verliefen ergebnislos.}\\
\textit{Schon wollte ich die Suche abbrechen und euch die Ergebnisse mitteilen, als mich der Zufall 
zu einigen noch nicht katalogisierten Dokumenten f"uhrte, die ich unseren Archiven ausmachen konnte. 
Meinem Erstaunen kann gar nicht genug Ausdruck verliehen werden und schon gar nicht kann ich in Worte 
fassen was ich fand. Zu urteilen obliegt jedoch nur Euch, meine k"onigliche Hoheit, und deswegen "ubersende 
ich euch beiliegend das Manuskript zur Durchsicht.}\\

\textit{Hochachtungsvoll,}\\
\textit{Pablo Hiluia\index[personen]{P!Pabloa Hiluia}}
\end{quotation}

\par Mit dem Schreiben des Bibliothekars seiner Hoheit wurde ihm eine gebundene Mappe mit einigen 
vergilbten Bl�ttern �berreicht. Das Deckblatt tr�gt die Insignien des Grafen von Hohenbr�ck und der 
Themenbeschreibung folgend scheint es sich hierbei um das Forschungsbuch zu handeln. Die Bl�tter sind 
arg in Mitleidenschaft gezogen worden, die Schrift ist jedoch noch gut zu lesen. Die Seiten machen 
einen sortierten Eindruck, jeder Eintrag wurde auf einer eigenen Seite dokumentiert.


\begin{longtable}{|p{15cm}|}
\hline
5. Tag im ersten Monat des 957. Jahres im Zeitaltes des Umbruchs\\
\\
Nachdem ich mich nun entschlossen habe, dem Ph�nomen des Turmes pers�nlich auf den Grund zu gehen, 
habe ich eine Expedition zusammengestellt. Morgen wird der erste Tag unserer Reise sein und wir sind 
voller Hoffnung, das Ziel unserer Reise in einigen Wochen erreicht zu haben.\\
\hline
\end{longtable}

\begin{longtable}{|p{15cm}|}
\hline
6. Tag im ersten Monat des 957. Jahres im Zeitaltes des Umbruchs\\
\\
Der Aufbruch ging planm��ig voran, in der Morgend�mmerung haben wir Pax Lucien verlassen und sind mit 
der Karawane in Richtung Westen auf den bekannten Wegen aufgebrochen. Die Stimmung ist gut.\\
\hline
\end{longtable}

\begin{longtable}{|p{15cm}|}
\hline
14. Tag im ersten Monat des 957. Jahres im Zeitaltes des Umbruchs\\
\\
Heute sahen wir zum ersten Mal den Turm am Horizont, wir haben also das erste Etappenziel unserer 
Reise erreicht. Aus der Ferne betrachtet wirkt der Turm gewaltig, selbst mit unseren Me�instrumenten 
k�nnen wir nicht ausmachen, welche H�he er hat. Wir vermuten ihn jedoch noch in weiter Ferne, da 
keinerlei Struktur auf seiner Oberfl�che erkennbar ist.\\
\hline
\end{longtable}

\begin{longtable}{|p{15cm}|}
\hline
15. Tag im ersten Monat des 957. Jahres im Zeitaltes des Umbruchs\\
\\
Entsetzen machte sich heute nach dem Erwachen breit, unser Ziel war am Horizont verschwunden. 
Einige der abergl�uberischen Karawanisten konnten nur mit M�he davon �berzeugt werden, dass es 
wohl an einer Wetterfront liegt, die den weit entfernt liegenden Turm verdeckt.\\
Ich f�r meinen Teil habe� Zweifel an meiner Notl�ge und besinne mich der Erz�hlungen �ber diesen Turm.\\
\hline
\end{longtable}

\begin{longtable}{|p{15cm}|}
\hline
21. Tag im ersten Monat des 957. Jahres im Zeitaltes des Umbruchs\\
\\
Wie bereits dem Eintrag vom Vortag zu entnehmen ist, haben wir dem Turm seit gestern Abend wieder in 
Sicht. Er liegt immer noch auf der gleichen Route, doch scheinen wir ihm in den letzten Tagen nicht 
n�her gekommen zu sein. Der Mi�mut in der Gruppe macht sich weiter breit.\\
\hline
\end{longtable}

\begin{longtable}{|p{15cm}|}
\hline
23. Tag im ersten Monat des 957. Jahres im Zeitaltes des Umbruchs\\
\\
�berraschen und Entsetzen macht sich heute nach dem Erwachen breit, unser Lager befindet sich am 
Fu�e des Turmes. Einige unserer Leute haben sich im fr�hen Morgen alleine auf den R�ckweg gemacht, 
nur der Kern der Expedition bleibt zur�ck. Am Fu�e des Turmes befindet sich ein gro�es Tor, wir 
werden im Laufe des Tages einige Untersuchungen anstellen und schauen, ob wir das Tor �ffnen k�nnen. 
Durch das pl�tzliche Auftauchen dieses Bauwerks, ist es mir jedoch nicht m�glich, seine genaue Lage zu 
verzeichnen, vielleicht auf dem R�ckweg.\\
\hline
\end{longtable}

\par Hier endet des Manuskript. Angeh�ngt befinden sich einige Dokumente, die den Ursprung der Unterlagen 
dokumentieren und dessen Einlagerung in die Bibliothek im Jahre 954. �ber den Verbleib des Verfassers 
oder seiner Expedition ist leider nichts bekannt. Auch wird nicht erw�hnt, wie die Dokumente in den 
Besitz der Bibliothek kamen.
