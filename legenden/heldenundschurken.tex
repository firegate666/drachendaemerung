\section{Helden und Schurken}
\label{heldenundschurken}
\par Die folgenden Kapitel stellen ein paar der bekannten und
legend�ren Personen und Gruppierungen der letzten Jahrhunderte vor.
Einiges basiert auf Tatsachen, anderes m�ge Legende sein.

\subsection{Der Orden der schwarzen Rose\index{O!Orden der schwarzen Rose}}
\parpic[l]{\epsfig{file=pics/capitals/d.eps, scale=0.5}}er Orden der schwarzen Rose ist ein Zusammenschluss von
Magiewirkern aus allen m�glichen Fachrichtungen. Sein Ziel ist das
Studium und das damit verbundene Erlangen von Wissen.
\par Diese Tatsachen alleine w�rde die Gemeinschaft nicht
spektakul�rer erscheinen lassen, als jeden x-beliebigen Zirkel von
Magier, h�tte sie sich nicht den Fachrichtungen Dimension und
Beschw�rung verschrieben.
\par Die Mitglieder der schwarzen Rose operieren im Geheimen.
Niemand wei�, wo sie sitzen, niemand wei� wer sie sind. Man sagt
jedoch, hinter vorgehaltener Hand, dass es hochrangige Mitglieder der
Zirkel seinen, die den Orden gegr�ndet haben und ihn f�hren.
\par Neue Mitglieder werden gezielt ausgew�hlt. Nach einer Reihe
geheimer Pr�fungen, von denen die Aspiranten nichts mitbekommen
werden, wird ihnen die Mitglidschaft angeboten. Was mit denen
passiert, die ablehnen ist ebenfalls nur Spekulation.

\subsection{Die Sturmreiter}

\subsection{Die Freibeuter des S�dens}

\subsection{Das Phantom der Ostk�ste}

\subsection{Die Ritter der Nordmark\index{R!Ritter der Nordmark}}
\parpic[l]{\epsfig{file=pics/capitals/d.eps, scale=0.5}}ie Ritter der Nordmark haben bei der Verteigung des �stlichen
Reiches gegen die Orkhorden 867 ihrem F�rsten gro�en Ruhm und Ehre
gegen�ber dem K�nig gebracht. Zahlreiche marodierende und
brandschatzende Horden von Orks fielen von Norden in das Reich ein,
ihre Zahl war so gewaltig, dass kein Heer alleine ausreichte, um sie
zu stoppen. Den Rittern der Nordmark um F�rst Reinier zur Nordmark\index[personen]{P!Reinier zur Nordmark},
zu dieser Zeit noch ein kleiner Orden, gelang es die Orks mit
einigen Kriegslisten aus dem Reich zur�ck in den Norden zu
vertreiben.
\par Renier wurde von dem K�nig f�r die Tapferkeit mit
einer Reichserweiterung belohnt, man bot ihm sogar einen gr��eren
Titel. Reinier lehnte dankend ab, sein Lohn sei der Triumph, er habe
nur im Dienste des K�nigs f�r den K�nig gehandelt, wie es sich f�r
einen Ritter geh�rt. Mehr als beeindruckt von diesen Worten erlie�
der K�nig einen Beschluss, der die Ritter der Nordmark zu den
Rittern des Reiches ernannte. Er stellte Reinier die n�tigen Mittel
zur Verf�gung den Orden auszubauen und noch weitere Krieger
aufzunehmen.
\par Der Orden wurde seit dem st�ndig erweitert und stellt seit
Mitte dieses Jahrhunderts das Heer des Ostreiches. Die Krieger des
Ordens sind f�r ihre Volksn�he und ihre Ritterlichkeit bekannt.
