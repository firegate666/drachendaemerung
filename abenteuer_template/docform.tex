\documentclass[a4paper,openany]{book} % DIN A4 Buch, Kapitelbeginn auch auf geraden Seiten (links)
\usepackage[T1]{fontenc}
\usepackage[latin1]{inputenc}

% f"ur Trennung der W"orter nach neuer dt. Rechtschreibung
\usepackage{ngerman}

% Grafik packages
\usepackage{lscape} % f"ur die Verwendung der Landscape-Umgebung
\usepackage{color} % f"ur farbigen Text
\usepackage[dvips]{graphicx}
\usepackage{picinpar} % f"ur Text um Grafik
\usepackage{picins} % f"ur Text um Grafik
\usepackage{wrapfig}

% Seitengr"o"se
\renewcommand{\evensidemargin}{0in} % Abstand nach au"sen bei ungeraden Seiten
\renewcommand{\oddsidemargin}{0in} % Abstand nach au"sen bei geraden Seiten
\renewcommand{\textwidth}{16cm} % Breite des Textblocks ohne Kopf- und Fu"szeile
\renewcommand{\textheight}{22cm} % Hhe des Textblocks ohne Kopf- und Fu"szeile
\renewcommand{\footskip}{1cm}

% f"ur die Verwendung von mehrspaltigen Texten
\usepackage{multicol}

% f"ur die Indexerstellung/-darstellung
\usepackage{makeidx}
\makeindex

% \usepackage{showidx} % einbinden, um indizierte Worte auf den Seiten anzuzeigen
% Pakete zur Verwaltung der Kopf- und Fu"szeilen
% for linux use this
%\usepackage{fancyheadings}
% for windows use this
\usepackage{fancyhdr}
\pagestyle{fancyplain}

% Einfaches Einbinden von EPS-Dateien
\usepackage{epsfig}

% Zugriff auf Anzahl Seiten und
% Seitennummer der letzten Seite
\usepackage{lastpage} % \pageref{LastPage} -> Seitennummer der letzten Seite
\usepackage{count1to} % \pageref{TotalPages} -> Gesamtzahl Seiten

% Kopfzeilendefinition
% [] f"ur gerade Seiten
% {} f"ur ungerade Seiten
\lhead[\let\uppercase\relax\leftmark]{\let\uppercase\relax\rightmark} % links-oben
\chead{} % mitte-oben
\rhead[\let\uppercase\relax\rightmark]{\let\uppercase\relax\leftmark} % rechts-oben

% Fu"szeilendefinition
% [] f"ur gerade Seiten
% {} f"ur ungerade Seiten
\lfoot[Seite \thepage]{} % links-unten
\cfoot{Drachend"ammerungabenteuer - Hier kommt der Name hin} % mitte-unten
\rfoot[]{Seite \thepage} % rechts-unten

% Definition der Kopfzeilenlinien
\renewcommand{\headrulewidth}{1pt}
\renewcommand{\plainheadrulewidth}{1pt}

% Definition der Fu�eilenlinien
\renewcommand{\footrulewidth}{1pt}
\renewcommand{\plainfootrulewidth}{1pt}

% Gro�giges Verhalten beim Zeilenumbruch
% L�t auch gro� Lcken zwischen Worten zu
%\sloppy

% Erweiterte Tabellenumgebungen
\usepackage{array}
\usepackage{tabularx}
\usepackage{longtable}
