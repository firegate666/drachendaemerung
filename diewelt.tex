\chapter{Die Welt}
\label{diewelt}
\par Im folgenden Kapitel stelle ich die dem Reisenden bekannte Welt vor.
Wobei der Ausdruck Welt in diesem Zusammenhang wohl etwas
"ubertrieben w"are, handelt es sich doch eher um einen Teil eines Kontinentes. Da
die Bev"olkerung in letzter Zeit mehr damit besch"aftigt war, ihr
Leben zu retten, als Welterkundungen durchzuf"uhren, hat sich an der
Karte in den letzten Jahrhunderten auch nicht viel getan.

\begin{figure}[hbtp]
\begin{center}
\epsfig{file=pics/map.eps, scale=0.12}
\end{center}
\caption{Weltkarte klein - Gesamt}
\label{worldmapsmall}
\end{figure}

\par Das bekannte Reich misst in Nord-S"ud Richtung ungef"ahr 1500
Meilen und in Ost-West Richtung ungef�hr 2000 Meilen. Die Weltkarte wird f"ur die
folgenden Beschreibung in die 5 K"onigreiche aufgegliedert. (Eine gr��ere Ausf�hrung 
der Karte ist im Anhang auf Seite \pageref{worldmap} zu finden.)

\section{Die 5 Reiche}

\subsection{Der Hohe Norden}
\parpic[l]{\epsfig{file=pics/capitals/i.eps, scale=0.5}}m Norden durch
das gewaltige Felsmassiv \glqq Das steinerne
Schwert\index[regionen]{S!steinerne Schwert, das}\grqq{} begrenzt, hat diese
Region ansonsten nicht viel an Attraktionen zu bieten. Ein bekannter
Weg w"urde weiter in n"ordlicher Richtung f"uhren. Von Reisenden und
Forschern jedoch eher gemieden, nennt sich das dahinter liegende Tal
\glqq Das Tal der Verlorenen und Verdammten\grqq\index[regionen]{T!Tal der
Verlorenen und Verdammten}, eine nicht gerade einladende
Bezeichnung.
\par Das Zentrum des Nordreiches und damit seine Hauptstadt ist 
Pax Epher\index[regionen]{P!Pax Epher}
\par Im S�den des Landes an der Grenze zum Reich der Mitte befindet sich das
Erzmassiv an dessen Randausl�ufern sich der Eingang der Zwergenfestung Bergfried\index[regionen]{B!Bergfried}
befindet. Bergfried\index[regionen]{B!Bergfried} ist die Stadt des Sturmhammerclans\index{S!Sturmhammerclan}.


\subsection{Das Reich der Mitte}
\parpic[l]{\epsfig{file=pics/capitals/d.eps, scale=0.5}}as bl"uhende Reich
unter der Sonne. So, oder so "ahnlich betitelte einmal ein Barde das
Reich um die sch�ne Stadt Patria Pacis\index[regionen]{P!Patria Pacis} und
die Hauptstadt dieser Region Pax Antares\index[regionen]{P!Pax Antares}. Das
ehemalige Reich des m"achtigsten der Drachen kann sich "uber seinen
Wirtschaftszustand nicht beklagen. Gro"e St"adte und weite Wiesen
tragen das Lebensgef"uhl der Bewohner nach au"sen. Einzig eine
K"ustenanbindung fehlt noch zum Gl"uck. Aber der regierende K"onig Siegfried II.\index[personen]{S!Siegfried II.}
schielt nicht ohne Grund auf das angrenzende K"onigreich im Osten.


\subsection{Die S�dk�ste}
\parpic[l]{\epsfig{file=pics/capitals/d.eps, scale=0.5}}as Land der Sonne
und der Palmen, hier l"asst es sich gut leben. Jedenfalls, wenn man
mit dem vorherrschenden Klima gut auskommt. Das S"udreich, dessen
Hauptstadt Pax Par\index[regionen]{P!Pax Par} ist, lebt vom Handel mit den
angrenzenden Staaten. Zu gro"sem Reichtum kam es dabei leider nicht,
da auf Grund des Klimas wichtige Nahrungsmittel nicht angebaut,
sondern importiert werden m"ussen. Wie gewonnen, so zerronnen.
\par Der Bev"olkerung geht es trotzdem nicht schlecht, denn ein Gro"steil der
 Seefahrer geht einem etwas un"ublichem Beruf nach, Freibeuterei. Die Piraten dieser 
 Region sind bei den Handelsflotten der L"ander gef"urchtet und verhasst. Doch leider 
 ist es bisher noch niemandem gelungen, dieses Treiben einzugrenzen. Der Sultan von 
 Pax Par\index[regionen]{P!Pax Par} w"ascht seine H"ande in Unschuld, \glqq Aus meinem Lande kommen diese Verr"uckten nicht!\grqq.


\subsection{Die Ostk�ste}
\parpic[l]{\epsfig{file=pics/capitals/d.eps, scale=0.5}}irekt an der K"uste
gelegen stellt die Hauptstadt Pax Lucien\index[regionen]{P!Pax Lucien} den
wohl "ostlichsten Punkt des Kontinents dar (Karte siehe S.
\pageref{fig_map_east}). Mit seiner g"unstigen Meeresposition ist
Pax Lucien\index[regionen]{P!Pax Lucien} damit Ausgangspunkt f"ur jeglichen Seehandel mit den
anderen Reichen und f"ur alle, wenn auch selten stattfindenden,
Schiffsexpeditionen ins offene Meer. Pax Lucien dient ebenfalls als
Zwischenstopp f"ur Schiffsreisen von und zum Nordreich.

\begin{figure}[hbtp]
\begin{center}
\epsfig{file=pics/map_east.eps, scale=0.5}
%width=4.8in, height=3.45in}
\end{center}
\caption{Weltkarte - Ostk�ste} \label{fig_map_east}
\end{figure}

\par Die politische
Situation der Ostk"uste ist gef"ahrdet. Dem Land geht es nicht
unbedingt gut und im Verh"altnis zu den anderen Reichen ist es ein
eher kleines Land. Da hilft es ihm auch nicht, da"s sich hier die
gr"o"ste Seemacht des Kontinents befindet, wenn potetielle Feinde
"ubers Land kommen. Gl"ucklicherweise gab es bis jetzt noch keine
feindlichen "Ubergriffe.

\par Die Hauptstadt des Landes beherbergt eines der gr"o"sten Kloster
der Neuzeit, die \glqq Feste des Wissens\grqq\index{F!feste des Wissens}. Die M"onche hier
wachen "uber die zweitgr"o"ste Bibliothek des Kontinents. Die
bewahrten Sch"atze reichen von einfachen Manuskriptsammlungen "uber
Abhandlungen der G"otter bis hin zu einigen Steintafeln, die
tausende von Jahren alt sind.

\par Nicht unweit von Pax Lucien\index[regionen]{P!Pax Lucien} findet sich die Tempelburg \textit{Sturmfeste\index[regionen]{S!Sturmfeste}}.
Hier ist der Sitz der Priesterschaft der G"otter. Von hier aus
findet die Koordination der gesamten 5 Reiche statt.


\subsection{Der weite Westen}
\parpic[l]{\epsfig{file=pics/capitals/s.eps, scale=0.5}}cherzhaft einmal
\glqq Das Land der Toten\grqq{}\index[regionen]{L!Land der Toten} genannt,
nehmen deren Einwohner die Bezeichnung nicht auf die leichte
Schulter.
\parpic[l]{\epsfig{file=pics/wueste1.eps, scale=0.4}}
Tats"achlich pr"agen W"usten und Berglandschaften diesen Teil des
Kontinents. Das Zivilisationsende findet sich weit im Westen am
Beginn einer gewaltigen "Odlandschaft. Zerrissener und zerkl"ufteter
Steinboden bestimmt hier das Bild. \par Hier fand vor knapp 1000
Jahren die entscheidene Schlacht um die Zukunft dieses Landes statt.
So unfruchtbar und unbewohnbar dieser Teil des Westens doch ist, hat
er eine wichtige Bedeutung bekommen. Irgendwo in der Ebene befindet
sich \glqq Der Turm der Magier\grqq\index{T!Turm der Magier}. Hier
leben und forschen die Wahrer der neuen Magiequelle. Hauptstadt des
Westens ist Pax Orgoth\index[regionen]{P!Pax Orgoth}


\section{Eine kleine St�dte�bersicht}

\subsection{Bergfried\index[regionen]{B!Bergfried}}
\par Hier folgt eine kleine einleitende Beschreibung zu der Stadt.

\begin{description}
\item [Geographische Lage] s�dliche Grenze Nordreich
\item [Stadtgr�ndung] ...
\item [Fl�che] ...
\item [H�he] ...
\item [Einwohnerzahl] ...
\item [Besonderes] Zwergenstadt in den Bergen, Hauptstadt des Sturmhammerclans
\end{description}
\subsection{Fischgrund\index[regionen]{F!Fischgrund}}
\par Hier folgt eine kleine einleitende Beschreibung zu der Stadt.

\begin{description}
\item [Geographische Lage] �stliches Reich Ostk�ste
\item [Stadtgr�ndung] ...
\item [Fl�che] ...
\item [H�he] ...
\item [Einwohnerzahl] 47
\item [Besonderes] Fischerdorf
\end{description}
\subsection{Patria Pacis\index[regionen]{P!Patria Pacis}}
\par Hier folgt eine kleine einleitende Beschreibung zu der Stadt.

\begin{description}
\item [Geographische Lage] mitteres Reich
\item [Stadtgr�ndung] ...
\item [Fl�che] ...
\item [H�he] ...
\item [Einwohnerzahl] ...
\item [Besonderes] ...
\end{description}
\subsection{Pax Antares\index[regionen]{P!Pax Antares}}
\par Hier regiest seine Majest�t K�nig Siegfried II.\index[personen]{S!Siegfried II.}
\par Hier folgt eine kleine einleitende Beschreibung zu der Stadt.
\par Zu dieser Stadt ist mir noch nichts anst�ndiges eingefallen wie ich das Element Geist verarbeiten kann.

\begin{description}
\item [Geographische Lage] mittleres Reich
\item [Stadtgr�ndung] ...
\item [Fl�che] ...
\item [H�he] ...
\item [Einwohnerzahl] ...
\item [Besonderes] Hauptstadt
\end{description}
\subsection{Pax Epher\index[regionen]{P!Pax Epher}}
\par Pax Epher, nach ihrem Gr�nder die Stadt des Feuers benannt, nahe am n�rdlichen Ende des Landes gelegen.
Die m�chtigen massiven Steinmauern haben die Stadt schon das eine oder andere Mal vor den Angriffen der Orkhorden
gesch�tzt. Wie auch bei den anderen Hauptst�dten haftet auch dieser Stadt der Makel des Drachen an, der sie erschaffen
hat. Rings um die Stadt herum befindet sich ein ca. 10 Schritt breiter Burggraben, der jedoch nicht mit Wasser gef�llt
ist, sondern dessen Inhalt aus einem unterirdischen Vulkan gespeist wird. Nicht zuletzt kein minder effektiver
Abwehrmechanismus.
\par Der K�nig des Stadt ist ein volksnaher Herrscher. Pax Epher beherbergt keine seperate Festung. Nahe der S�dmauer
befindet sich nur ein kleiner Palast mit einem f�r diese Gegend recht ansehnlichen Ziergarten.

\begin{description}
\item [Geographische Lage] n�rdliches Reich
\item [Stadtgr�ndung] ...
\item [Fl�che] ...
\item [H�he] ...
\item [Einwohnerzahl] ...
\item [Besonderes] Hauptstadt
\end{description}
\subsection{Pax Lucien\index[regionen]{P!Pax Lucien}}
\par Pax Lucien ist die Hauptstadt des �stlichen Reiches direkt an der K�ste gelegen. Geographisch betrachtet
stellt sie den �stlichsten Landpunkt dar.
\par W�hrend die Wasserseite der Stadt von m�chtigen Mauern gegen etwaige Brandungen und Angriffe gesch�tzt wird,
ist die Landseite relativ ungesch�tzt. Die einfachen Mauern w�rden kaum einem geballten Angriff statthalten.
\par Das Zentrum der Stadt ist der Palast des K�nigs und damit auch zugleich die Attraktion schlechthin. Die ehemalige
Festung des Drachen Lucien ist eine schwebende Zitadelle, die durch lange Seilbr�cken mit dem Boden verankert ist.
Um die Festung besser befahren zu k�nnen, wurde Mitte des Jahrtausends ein solider Steinweg hoch zu den Haupttoren
der Stadt gebaut. Um trotzdem die Sicherheit w�hrend eines Angriffes zu gew�hrleisten, wurde dieser jedoch auf
magische Weise so pr�pariert, dass er auf Gehei� der k�niglichen Hofmagier einfach in sich zusammenbrechen w�rde.

\begin{description}
\item [Geographische Lage] �stliches Reich, K�stenstadt
\item [Stadtgr�ndung] ...
\item [Fl�che] ...
\item [H�he] ...
\item [Einwohnerzahl] ...
\item [Besonderes] Hauptstadt
\end{description}
\subsection{Pax Orgoth\index[regionen]{P!Pax Orgoth}}
\par Pax Orgoth, die D�nenstadt und das Unterweltreich.
\par Mitten in der W�ste liegt diese Perle des Westens und so sieht sie auch aus. �u�erlich gleicht sie eher
einer Karawanenstadt denn einem Herrschaftssitz, doch das �u�erliche Bild tr�gt. �nhlich den St�dten der Zwerge
findet Pax Orgoth Ausl�ufer bis weit unter die Erde. Ein ausgekl�geltes Gang- und H�hlensystem stellt die Basis dieser
Stadt dar. Magische Kristalle sorgen f�r dauerhafte Beleuchtung unter Tage, die nur in den Nachstunden ged�mmt wird.
\par Die Festung des K�nigs ist tief in der Erde noch unter der eigentlichen Stadt zu finden.

\begin{description}
\item [Geographische Lage] westliches Reich
\item [Stadtgr�ndung] ...
\item [Fl�che] ...
\item [H�he] ...
\item [Einwohnerzahl] ...
\item [Besonderes] Hauptstadt
\end{description}
\subsection{Pax Par\index[regionen]{P!Pax Par}}
\par Pax Par, die Stadt der Kan�le an der S�dk�ste.
\par Im Hafen von Pax Par liegen das Jahr �ber immer ein gutes Dutzend pr�chtiger Segelschiffe, die das Stadtbild von
der Wasserseite aus pr�gen. Die Stadt selber liegt so dicht am Wasser, dass viele H�user auf Sockeln gebaut werden mussten,
damit man st�ndig Wasser absch�pfen m�sste.
\par Durch die Stadt schl�ngeln sich eine ganze Reihe von Kan�len, so dass es nicht weiter verwunderlich ist, dass das
Fortbewegungsmittel der Wahl Gondeln und kleine F�hren sind.
\par Die Festung des Stadt befindet sich Landeinw�rts n�rdlich in der Stadt inmitten eines kleinen Sees gelegen
und ist nur mit F�hren oder Gondeln zu erreichen. Hinter vorgehaltener Hand berichtet man von geheimen G�ngen unter der
Stadt, die ebenfalls dorthin und zur�ckf�hren.

\begin{description}
\item [Geographische Lage] s�dliches Reich, K�stenstadt
\item [Stadtgr�ndung] ...
\item [Fl�che] ...
\item [H�he] ...
\item [Einwohnerzahl] ...
\item [Besonderes] Hauptstadt
\end{description}
\subsection{Sturmfeste\index[regionen]{S!Sturmfeste}}
\par Hier folgt eine kleine einleitende Beschreibung zu der Stadt.

\begin{description}
\item [Geographische Lage] �stliches Reich n�he Ostk�ste
\item [Stadtgr�ndung] ...
\item [Fl�che] ...
\item [H�he] ...
\item [Einwohnerzahl] ca. 150
\item [Besonderes] Tempelstadt
\end{description}