\chapter{Ma�e, Gewichte, W�hrung - einige Werte}
\section{Ma"seinheiten\index{M!Ma"seinheit}}
\begin{itemize}
\item 1 Meile\index{M!Meile} = 1000 Schritt\index{S!Schritt}
\item 1 Schritt = 3 Ellen\index{E!Elle}
\item 3 Ellen = 30 Finger\index{F!Finger}
\par \textbf{Anmerkung:} Urspr�nglich wurden zur L"angenbestimmung tats"achlich der Unterarm und der
 kleine Finger verwendet. Da es hierbei jedoch �fters zu Unregelm"a"sigkeiten kam, wurden
 sogenannte Normst"abe\index{N!Normstab} entworfen, die zur L"angenbestimmung verwendet werden
\caption{Ma"seinheiten}
\end{itemize}

\section{Gewichte\index{G!Gewicht}}
\begin{itemize}
\item 1 Quader\index{Q!Quader} = 20 Stein\index{S!Stein}
\item 1 Stein = 10 Fund\index{F!Fund}
\item 1 Fund = 20 Unzen\index{U!Unze}
\par \textbf{Anmerkung:} Zur Gewichtsbestimmung werden sogenannte Normgewichte\index{N!Normgewicht}
 verwendet, um Unregelm"a"sigkeiten zu vermeiden. F�r alchimiste Zwecke gibt es spezielle
 Messgef"a"se\index{M!Messgef"a"se}, die Markierungen f�r verschieden F�llmengen haben. Als Ma"seinheit
 werden hierbei Unterteilungen der Unze verwendet. Zur Bestimmung des Gewichtes bei Pulvermengen gibt es
 entsprechend leichte Normgewichte.
\caption{Gewichte}
\end{itemize}

\section{W"ahrung\index{W!W�hrung}}
\begin{itemize}
\item 1 Goldbarren\index{G!Goldbarren} = 1000 Goldst�cke\index{G!Goldstueck}
\item 1 Goldst�ck = 10 Silberlinge\index{S!Silberling}
\item 1 Silberlinge = 10 Bronzem�nzen\index{B!Bronzem�nze}
\item 1 Bronzem�nze = 10 Kupferst�cke\index{K!Kupferstueck}
\par \textbf{Anmerkung:} Als �bliche W"ahrung sind Silberlinge, Bronzem�nzen und Kupferst�cke nahezu �berall zugelassen. In gr�"seren St"adten nehmen Kaufleute auch Goldst�cke an. Nur sehr gro"se Kaufh"auser oder Adlige rechnen und handeln mit Goldbarren.
\caption{W"ahrung}
\end{itemize}

\section{Zum Verst"andnis}
\begin{itemize}
\item 1 Schritt = 1 Meter\index{M!Meter}
\item 1 Stein = 5 Kilo\index{K!Kilo}
\end{itemize}
