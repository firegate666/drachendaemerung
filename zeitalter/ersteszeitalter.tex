\subsection{Erstes Zeitalter\index{Z!Zeitalter, erstes}: Drachenkrieg\index{D!Drachenkrieg}}
\begin{quotation}
\par\textit{\glqq Um etwas Neues zu erschaffen muss Altes zerst"ort werden. In den Existenzebenen ist nur Platz f"ur Eines von Beidem. \grqq}
\end{quotation}
\parpic[l]{\epsfig{file=pics/capitals/n.eps, scale=0.5}}ur wenig ist
bekannt �ber die Welt vor dem ersten Zeitalter\index{Z!Zeitalter,
erstes}, liegt dieses doch mittlerweile knapp 3000 Jahre zur"uck und
die Augenzeugen jener Epoche\index{E!Epoche} sind rar geworden. In
Wahrheit gesprochen sind die einzigen Wesen, die ein solches Alter
erreichen k"onnen die Drachen\index{D!Drache}. Und von denen wei"s
ja nun jedes Kind, dass sie ausahmslos von der Bildfl"ache
verschwunden sind. Was die Chronisten\index{C!Chronist} aus dieser
Zeit zu berichten wissen beruht auf Schriften und einigen
Steinmei"seleien, die in ungef"ahr diese Zeit datiert wurden.
\par In einen sind sich jedoch alle einig: Das erste Zeitalter\index{Z!Zeitalter, erstes} begann mit dem Erscheinen der Drachen\index{D!Drache}.
\par Es muss der Ausbruch des Chaos\index{C!Chaos} gewesen sein.
Begleitet von Springfluten und Vulkanausbr"uchen erschienen sie in gigantischen Schw"armen.
Der Himmel verdunkelte sich und Donnerst"urme erschienen am Horizont.
Mit ihren Schwingen erschufen sie Wirbelst"urme, die das Land zerst"orten.
Und mit Ihnen kamen die Drachenkrieger\index{D!Drachenkrieger}.
Humanoide Lebensformen mit dem Aussehen eines Drachen;
Geschuppte Haut, einige ihrerseits mit Schwingen, andere mit der Gabe des Drachenodems\index{D!Drachenodem}.
\par Keiner kann so genau sagen woher sie gekommen sind oder warum. Doch die Frage nach dem \textit{Warum} schien eher ein verzweifeltes Flehen nach einer Erkl"arung als eine ernst gemeinte Frage zu sein. Die Kreaturen zogen durch das Land und s"ahten Zerst"orung. Wer nicht get"otet wurde, wurde versklavt.
\par Einige Theorien gaben den Schwarzmagiern\index{S!Schwarzmagier} Schuld an der Beschwrung der Drachen. Andere wiederum suchen die Erkl"arung in dem Zorn der G"otter.
\par Nach einem Krieg, der fast vier Jahrhunderte anhielt und dessen Sieger schon bei seinem Ausbruch feststand, schien Ruhe in das Land einzukehren. Dieses lag weniger an der Tatsache, dass der Kampfesmut versiegte sondern eher daran, dass einfach nichts mehr da war, was sie vernichten konnten.
\parpic[r]{\epsfig{file=pics/moonshadow.eps, scale=0.3}}
Die "Uberlebenden dieser Zeit wurden in Lagern als Sklaven gehalten und zum Bau neuer Geb"aude gezwungen. Die Krieger sprachen von den Vorbereitungen zur Ankunft der F"unf\index{F!Fnf, die}. Es wurden gigantische Tempel und unterirdische Anlagen gebaut. Die Oberfl"ache war unbewohnbar geworden.
\par Vor etwas mehr als 2000 Jahren, am Ende des ersten Zeitalters, erschienen sie. Wer bis dahin glaubte, dass die Drachen\index{D!Drache}, die die Kriege f"uhrten zu den gr"o"sten Kreaturen z"ahlte, wurde sp"atestens jetzt eines besseren belehrt. Die Herrscher der Drachen\index{D!Drache}, f"ur die Jahrhunderte eine neue Welt erschaffen wurde, hielten es f"ur an der Zeit sich zu offenbahren. Monstr"ose Kreaturen mit der Macht der Urgewalten\index{U!Urgewalt}. Jede von Ihnen als elementare Kraft. Sie sollten bekannt werden als die Inkarnation dessen, was sp"ater die Magie\index{M!Magie} wurde.
\\
\par \textit{Antares}\index{A!Antares}, die Kraft des Geistes,
\par \textit{Par}\index{P!Par}, die Kraft des Wassers,
\par \textit{Lucien}\index{L!Lucien}, die Kraft der Luft,
\par \textit{Epher}\index{E!Epher}, die Kraft des Feuers und
\par \textit{Orgoth}\index{O!Orgoth}, die Kraft der Erde.
\\
\par Die Welt wurde aufgeteilt und jede der Urgewalten\index{U!Urgewalt} herrschte "uber sein Reich unabh"angig von den Anderen. Zuerst "anderte sich nicht viel an dem Zustand. Die Menschen waren weiter Sklaven\index{I!Index} und jede Stunde ihres erb"armlichen Lebens damit besch"aftigt einen Stein nach dem anderen aus dem Fels zu schlagen, nur um dann einen Stein nach dem anderen "ubereinander zu Mauern zu stapeln.
\par Die meisten Zitadellen und Bergfesten stammen aus jener Zeit.
