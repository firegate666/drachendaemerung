\subsection{Zeitleiste}
\parpic[l]{\epsfig{file=pics/capitals/d.eps, scale=0.5}}ie Zeitrechnung wird mit Beginn eines jeden Zeitalters
neu gez"ahlt. Spricht man vom Jahr 546 ist das Jahr 546 in dem
aktuellen Zeitalter gemeint. Will man "uber ein Jahr in einem
anderen Zeitalter sprechen, dann setzt man die Nummer des Zeitalters
vorweg. Das Jahr 546 im zweiten Zeitalter ist also das Jahr 2.546.
\par Die Angaben in den Tabellen \ref{tabelle_zeitalter1},
\ref{tabelle_zeitalter2} und \ref{tabelle_zeitalter3} (ab Seite
\pageref{tabelle_zeitalter1}) sind historischen Schriften entnommen.
Da gerade in der dunklen Zeit keine M"oglichkeit zur genauen
Zeitbestimmung bestand, k"onnen hier Differenzen zu anderen
Schriften durchaus m"oglich sein.
\par Das geradezu faszinierende an der Vergangenheit dieses
Landes ist die Kontinuit"at mit der die Geschichte geschrieben wird.
In der Vergangenheit stand am Ende eines Millenniums immer ein
Umbruch, der das definitve Ende des vergangenen und den Anfang eines
neuen Zeitalters bedeutete. Gerade diese mystischen Ereignisse sind
es, die den Weisen und Historikern zur Zeit den Angstschwei"s auf
der Stirn stehen lassen. Zur Zeit schreiben wir das Jahr 3.997 und
bis zum Beginn des n"achsten Jahrtausend sind es nur noch drei
Jahre. Was wird uns erwarten?

% Erstes Zeitalter
\begin{longtable}{|r|p{10cm}|}
\hline
Jahr   & Wichtige Ereignisse \\
\hline
0      & Erscheinen der Drachen; Beginn des Jahrhundertkrieges\\
397    & Endg"ultige Versklavung der Menschheit\\
ab 407 & Bau der Unterirdischen Katakomben und Labyrinthe\\
647    & erste Freiheitsk"ampfergruppen bilden sich\\
870    & Angriff auf das Zentrum der Drachen\\
876    & Erscheinen der Urgewalten\\
877    & Niederschlagung aller Widerst"ande und endg"ultiger Fall der Menschheit in die Sklaverei\\
992    & Ein Magier in der Sklaverei unter Antares Truppen erkennt als Erster Mensch den Einfluss der Urgewalten auf die magischen St"urme und erforscht diese im Geheimen\\
\hline
\caption[Zeitleiste: Erstes Zeitalter]{Wichtigste Ereignisse im ersten Zeitalter}
\label{tabelle_zeitalter1}
\end{longtable}
% Zweites Zeitalter
\begin{longtable}{|r|p{10cm}|}
\hline
Jahr   & Wichtige Ereignisse \\
\hline
0      & Erste Forschungsergebnisse bei der Nutzung der neuen magischen Quellen, erster Hoffnugsschimmer zur Nutzung dieser f"ur die Freiheitserringung\\
1      & Erkennung erster Nebenwirkungen bei der Nutzung der neuen magischen Quellen\\
12     & Entwicklung des ersten Zaubers\\
63     & Einer gro"se Gruppe von Freiheitsk"ampfern gelingt auf magische Weise der Weg in die Freiheit, Widerstandsaufbau beginnt, in den n"achsten Jahren finden immer wieder Befreiungen von Gefangenen aus den Festungen statt\\
80     & Bau erster befestigten Siedlungen in still gelegten Minenwerken\\
750    & Erstes Auftauchen der Priesterschaft der G"otter, wirksame Wunder "uberzeugen die Menscheit von der Existenz der G"otter\\
956    & Die Urgewalten finden sich alle in der Zitadelle Orgoths in den Westlanden ein\\
957    & Sturz der Festung Luciens an der Ostk"uste\\
999    & Versammlung der Truppen in den Ebenen; finale Schlacht am Trauerfelsen\\
\hline
\caption[Zeitleiste: Zweites Zeitalter]{Wichtigste Ereignisse im zweiten Zeitalter}
\label{tabelle_zeitalter2}
\end{longtable}
% Drittes Zeitalter
\begin{longtable}{|r|p{10cm}|}
\hline
Jahr   & Wichtige Ereignisse \\
\hline
0      & Vernichtung / Verbannung der Drachen; Zauberspruch des Millenniums\\
16     & Gr"undung des 18er-Rats\\
18     & Errichtung der Festung Patria Pacis im Zentrum des gro"sen Kontinents\\
25     & Gr"undung der ersten neuen Siedlungen in den Zitadellen der Urgewalten. Sie tragen die Namen Pax Antares\index{P!Pax Antares} (Reich der Mitte), Pax Par\index{P!Pax Par} (S"udk"uste), Pax Lucien\index{P!Pax Lucien} (Ostk"uste), Pax Epher\index{P!Pax Epher} (Der Hohe Norden) und Pax Orgoth\index{P!Pax Orgoth} (Westlande). Sie gelten als die Hauptst"adte der 5 gro"sen Reiche\\
28     & Gr"undung einer neuen Magiergilde, zur Wahrung und Erforschung der neuen Magiequelle, die als \textit{\glqq Wahre Magie\grqq} bekannt wurde\\
29     & Errichtung des Turms der Magier im \textit{Land der Toten} als Sitz des 18er-Rats und Forschungs- und Schulungszentrum f"ur die \textit{Wahre Magie}\\
30     & Wahl des ersten Kaisers "uber die 5 Reiche und Wahl der 5 K"onige zur Verwaltung dieser\\
75     & R"ucktritt des Kaisers mit 128 Lebensjahren in den Ruhestand, Aufl"osung dieses Titels und damit Gr"undung von 5 unabh"angigen K"onigreichen ohne gemeinsame Verwaltung\\
997    & aktuelles Zeitgeschehen\\
\hline
\caption[Zeitleiste: Drittes Zeitalter]{Wichtigste Ereignisse im dritten Zeitalter}
\label{tabelle_zeitalter3}
\end{longtable}
