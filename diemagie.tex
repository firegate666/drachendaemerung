\chapter{Die Magie}
\label{diemagie}
\section{Einf�hrung}
\parpic[l]{\epsfig{file=pics/capitals/d.eps, scale=0.5}}ie Magie spielt in Drachend"ammerung die Rolle einer geheimnissvollen,
unbekannten und gleichwohl gef"urchteten Erscheinung. Nur wenig ist
dem gemeinen B"urger "uber die arkanen K"unste\index{A!arkane Kunst}
bekannt.
\par Der Gelehrte unterscheidet zwei Formen der Magie. Auf der
einen Seite steht das Sprechen von
Zauberspr"uchen\index{Z!Zauberspruch} (ab S. \pageref{spruchmagie}),
bei denen der Anwender sich voll und ganz auf seinen Geist und seine
F"ahigkeiten verlassen muss. Mag diese Art der Magie f"ur den
Betrachter schon m"achtig erscheinen, so berichten die
Praktizierenden von einer weitaus m"achtigeren Form der Magie. Um
wahre Magie\index{M!Magie, wahre} (ab S. \pageref{wahremagie})
wirken zu k"onnen, muss sich der Zauberer direkt der Kraft der f"unf
Elemente\index{E!Element} bedienen. Diese Art der Magie ist ungleich
effektiver wie auch gef"ahrlicher. Sie hat schon so manchen Magier
den Verstand wenn nicht sogar den K"orper gekostet.
\par Neben diesen beiden Formen der Magie existieren noch eine Reihe von Abwandlungen\index{M!Magie, Abwandlungen der}. Eine weitaus schw"achere Art der Zauberei\index{Z!Zauberei}, die auch von nicht-Magiern beherrscht werden kann, ist die Illusionsmagie\index{I!Illusionsmagie} (ab S. \pageref{illusionsmagie}). Die dunkle Kunst\index{K!Kunst, dunkle} der Schwarzmagier\index{S!Schwarzmagier} ist die Beschw"ohrung\index{B!Beschw"orung} (siehe S. \pageref{beschwoerungen}) und die Hexen\index{H!Hexe} wenden Fluchmagie\index{F!Fluchmagie} (siehe S. \pageref{fluchmagie}) an.
\par Jeder Magier w"urde es jetzt wahrscheinlich bestreiten, aber die Alchimie\index{A!Alchimie} (siehe S. \pageref{alchimie}) ist auch eine Form der Zauberei\index{Z!Zauberei}. Zur Herstellung wirkungsvoller Tinkturen\index{T!Tinktur} bedarf es etwas mehr als der Kunst des Lesens und des Abf"ullens.
\par Nicht zu vergessen, da sie eine nicht unbedeutende Form der Magie ist, ist die Runenmagie der Zwerge. M"achtige Kriegsschmiede und Thaumaturgen fertigen mit ihrer Hilfe magische Artefakte. Diese Form der Magie ist nur den Zwergen bekannt und wird innerhalb der Familien weitergegeben. Obgleich die Runenmagie\index{R!Runenmagie} der Zwerge wesentlich m"achtiger als unsere Alchimie/Thaumaturgie ist, wird sie im allgemeinen doch dazu gez"ahlt.

\section{Spruchmagie\index{S!Spruchmagie}} \label{spruchmagie}
\parpic[l]{\epsfig{file=pics/capitals/d.eps, scale=0.5}}ie Spruchmagie
ist die aller"ublichste aller Formen der Magie. Sie wird seit
Urzeiten von allen Magiebegabten angewandt und in den Zirkeln
gelehrt. Sie ist sozusagen die Schulmagie\index{S!Schulmagie}.

\par Seit der Entstehung der Magie, wann auch immer das war, gibt es
auch verschieden Ausrichtungen. Es gibt den Zirkel der Schwarzmagier\index{S!Schwarzmagier}
und den Zirkel der Wei"en, die freien Magier und die Ge"achteten,
letztere sind im Grunde eine extreme, nicht gedultete Ausrichtung
der Schwarzmagier und werden offiziell selbst von denen nicht
geduldet. W"ahrend die Zirkel ihre Hauptaufgabe in der Weitergabe
des Wissens und dem Finden neuer und w"urdiger Sch"uler sehen, jeder
Zirkel hat dabei nat"urlich eine eigene Vorstellung was w"urdig ist,
besch"aftigen sich die freien Magier mit der Forschung und arbeiten
dabei unter Umst"anden auch mal enger mit den Zirkeln zusammen.
Freie Magier sind meist Einzelg"anger, bilden jedoch hin auch wieder
auch einzelne Sch"uler aus, wenn sie auf ihren Wanderungen ein Kind
entdecken, dass sich als sehr potent erweist.

\par Die Spruchmagie teilt sich in 18 Bereiche der Magie auf.
\begin{multicols}{3}
\begin{itemize}
\item Beschw�ren
\item Bewegung
\item Binden
\item Dimension
\item Elemente
\item Empathie
\item Gegenmagie
\item Hypnose
\item Illusion
\item Kontrolle
\item K�rperbewu�tsein
\item Materietransformation
\item Natur
\item Regeneration
\item Schock
\item Telepathie
\item Verst�ndnis
\item Wahrnehmung
\end{itemize}
\end{multicols}
\par Eine genaue Beschreibung der einzelnen Magiebereiche ist dem
ERPS Regelwerk\cite{erps1} zu entnehmen.
\par Dem normalen Magiewirkende werden die Bereich Beschw�rung\index{B!Beschw�rung} und
Dimension\index{D!Dimension} in der Regel nicht zug�nglich sein. Das Verst�ndis dieser
Gebiete ist einfach zu komplex, als das diese neben den anderen
Bereichen erlernt werden k�nnen.

\section[Wahre Magie]{\glqq Wahre\grqq{} Magie\index{M!Magie, wahre}}
\label{wahremagie}

\section{Illusionsmagie\index{I!Illusionsmagie}}
\label{illusionsmagie}
\parpic[l]{\epsfig{file=pics/capitals/j.eps, scale=0.5}}eder kennt die Tricks,
die mit Illusionen m"oglich sind. W"ahrend man versucht das Opfer vom eigentlichen
Geschehen abzulenken, holt man unter hohem Geschick die M"unze aus dem "Armel und
\glqq zaubert\grqq{} sie hinter dem Ohr des Opfers wieder hervor. Kinderleicht.

\par Wenn der Illusionist jedoch von einer Illusion spricht, dann meint er damit
etwas ganz anderes. Durch die magische Manipulation seiner Umgebung, ist es dem
Illusionisten m"oglich, wirkliche Trugbilder zu erschaffen, die dem oder den
Opfern sehr reell erscheinen werden. Manchmal auch zu reell.

\section{Beschw"orungen\index{B!Beschw�rung}}
\label{beschwoerungen}
\parpic[l]{\epsfig{file=pics/capitals/n.eps, scale=0.5}}eben der
Schwarzmagie stellt die Beschw"orung eine der gef"ahrlichsten und
unberechenbarsten Formen der Magie dar. Nicht zuletzt wurde den
Beschw"orern die Schuld am Erscheinen der Drachen gegeben. Nur zu
verst"andlich, dass diese seit dem nicht mehr ihren Beruf in die
"Offentlichkeit tragen.
\par Doch neben den schwarzen Schafen gibt es auch die jenigen,
die sich mit dem Beschw"oren niederer Wesen besch"aftigen. Kleine
D"amonen, die als Beobachter oder Dienstboten eingesetzt werden,
Beschw"orung der Geister der Toten, um mit ihnen in Kontakt zu
treten, Kontakaufnahme mit Elemtarwesen\index{E!Elementarwesen} oder aber die Kommunikation
mit den Geistern der Pflanzen, dies sind die Aufgabengebiete der
wei"sen Beschw"orer.
\par Auch die Erschaffung von Golems\index{G!Golem} und anderer widernat�rlicher
Wesen wird als Form der Beschw�rung verstanden.
\par Als letztes sei noch erw"ahnt, dass die Nekromantie\index{N!Nekromantie} ebenfalls
eine Form der Beschw�rung ist.

\subsection{Elementarwesen beschw�ren}
\subsection{D�monen beschw�ren}
\subsection{Magische Wesen erschaffen}
\subsection{Geisterbeschw�rung}
\subsection{Totenbeschw�rung}

\section{Fluchmagie\index{F!Fluchmagie}} \label{fluchmagie}
\parpic[l]{\epsfig{file=pics/capitals/d.eps, scale=0.5}}ie Fluchmagie ist wohl die Form der Magie, 
die in der Bev"olkerung am bekanntesten ist. "Ublicherweise f"allt auch jede andere Form von Magie 
meist unter diese Bezeichnung. Unwissenheit ist hier die Ursache. Wie auch immer.
\par Die Anwender von Fluchmagie sind Hexen und deren m"annliches Gegenst"uck die Hexer.
\par Fluchmagie ist eine "au"serst niedertr"achtige Form der Magie und wird von den Hexen meist zu 
Rachezwecken verwendet, dabei muss in den Wirkungsgraden zwischen \glqq f"ur kurze Zeit\grqq{} und 
permanent unterschieden werden. Ein vom Fluch gepeinigter wird meist unter den folgen extrem zu 
leiden haben. Ob es das Pech ist, was ihn verfolgt, oder die Gicht, die ihn von dem Moment an plagt, 
die Folgen sind schwer wieder zu neutralisieren.
\subsubsection{Fluchmagie im Spiel}
\par Wie jede andere Form der Magie, wird Fluchmagie auch "uber die Psifertigkeiten abgewickelt. Die 
Ergebnisse eines erfolgreich gesprochenen Fluches, sind jedoch von anderer Wirkung. Sie zielen meist 
auf die Peinigung des Opfers ab. Hier die Liste der Fl"uche, die nat"urlich wie andere Psi-Kr"afte auch, 
im Laufe des Spiels um Eigenkreationen erweitert werden kann.
\begin{description}
\item [Hexenschuss (Bewegung)] Das Opfer erleidet einen Hexenschuss. Von nun an erh"alt es 
Abz"uge auf alle k"orperlichen und Waffenf"ahigkeiten, sowie auf die Beweglichkeit.\\
\textbf{Mindestwurf:} $ 17+RW^2+Abzug^2+Dauer $\\
\textbf{Zauberdauer:} sofort wirksam\\
\textbf{Wirkungsdauer:} variabel\\
\textbf{Widerstandswurf:} erlaubt\\
\begin{tabular}{r|l}
Dauer & Wirkungsdauer\\
\hline
1 & 1 Spielrunde\\
2 & 1 Kampf\\
3 & 1 Stunde\\
4 & 1 Tag\\
5 & 1 Woche\\
10 & 1 Monat\\
20 & 1 Jahr\\
50 & 10 Jahre\\
75 & 1 Vierteljahrhundert\\
100 & permanent\\
\end{tabular}
\item [Angst (Schock / Empathie)] (siehe ERPS Regelbuch S. 120)
\item [L"ahmen (Bewegung)] (siehe ERPS Regelbuch S. 116)
\item [Liebeszauber (Empathie / Hypnose)] (siehe ERPS Regelbuch S. 120)
\item [Blindheit verursachen (Schock)] (siehe ERPS Regelbuch S. 125)
\item [Fieber (Schock)] (siehe ERPS Regelbuch S. 125)
\item [H"asslichkeit (Hypnose)] Von Beginn des Fluches an, denkt das Opfer es sei h"asslich.
\item [H"asslichkeit (K"orperbewusstsein)] Von Beginn des Fluches an wird das Opfer mit H"asslichkeit gestraft.
\item [Sinne des Vertrauten (Empathie)] Auch wenn es nicht wirklich ein Fluch ist, so wird es doch hier erw"ahnt, da es ein typischer Hexenzauber ist. Er erm"oglicht der Hexe den Zugriff auf die Sinne ihres Vertrauten. Sie kann durch ihn sehen, h"oren und/oder f"uhlen.\\
\textbf{Mindestwurf:} $ 12+RW^2+Anzahl der Sinne $\\
\textbf{Zauberdauer:} Minutenzauber\\
\textbf{Wirkungsdauer:} aufrechterhalten\\
\textbf{Widerstandswurf:} entf"allt\\
RW wird in Kilometern gez"ahlt.
\end{description}

\section{Alchimie\index{A!Alchimie} und
Thaumaturgie\index{T!Thaumaturgie}} \label{alchimie}
\subsection{Allgemeine Alchimie}
\parpic[l]{\epsfig{file=pics/capitals/h.eps, scale=0.5}}eiltr"anke, Gifte, Tintkturen...
\par Alle diese Sachen fallen in das Aufgabengebiet eines Alchimisten. Dar"uber hinaus 
verf"ugt der Alchimist jedoch auch "uber leichte magische F"ahigkeiten, die es ihm erm"oglichen, 
seine Tr"anke magisch zu verst"arken. Die Palette der m"oglichen Effekte wird hiermit immens 
vergr"o"sert.

\subsection{Die Runenmagie der Zwerge}
\par Legenden berichten von dieser Art von Magie. Geschichten und Mythen erz"ahlen von den m"achtigen 
Artefakten, die die Runenmeister der Zwerge unter Tage schufen und auch noch heute herstellen.
\par Ein Spielercharakter wird wahrscheinlich niemals in den Genuss kommen, bei der Schaffung eines 
solchen Artefakts anwesend zu sein. Die Zwerge h"uten diese Kunst wie ihren Augapfel und selten 
werden Fremde bei einer diese Zeremonien anwesend sein. Viel wahrscheinlicher wird es sein, 
dass ein Abenteurer in den Besitz eines solchen Artefaktes kommt. In der Zeit der Kriege und 
des Widerstands haben die Zwerge viele Waffen, R"ustungen und andere n"utzliche Gegenst"ande 
geschaffen, die jetzt "uber den Kontinent und weiter zerstreut sind.

