\chapter{Impressum}
\par So, nun sind wir also am Ende angekommen und hier kommt das, was so ein Buch erst richtig wertvoll macht. Aus einem Gedanken wurde eine Idee und aus einer Idee wurde ein Anfang und letztendlich dieses hier vorliegenden Buch. Aber, ...
\begin{quotation}
\par \textit{\glqq Kein Buch schreibt sich von von alleine und keiner schreibt alleine ein Buch.\grqq}
\end{quotation}
\par An dieser Stelle m"ochte ich allen danken, die mir bei dieser Arbeit geholfen haben, allen Co-Autoren,
Zeichnern, Spieletestern und Spielleitern und nat"urlich auch allen Helden und Schurken, die im Laufe des
Schaffungs- und Testprozesses ihr Leben lie�en.
\begin{center}
\section*{Credits}
\begin{description}
\item[Autor:]Marco Behnke
\item[Co-Autoren:] Torben Werner (Die Zwerge, S. \pageref{zwerge}, \pageref{diezwerge})
\item[Grafiken und Illustrationen:] Dirk Kultus, \glqq Asamarith\grqq{}\footnote{http://www.razyboard.com/system/user\_asamarith.html}
\end{description}

\section*{Spieletester}
\begin{description}
\item[Spielleitungen:]
\item[Spieler:] Benjamin August, Oliver Joppek, Thimo Altmann, Tim Warszta, Torben Werner
\item[Charaktere:] Francesca Da Corva (menschliche Hexe), Grendal (trollischer Magier), Linflas (elfischer M"onch), Paragon (elfischer Waldl"aufer), Teralk, Sohn des Barvos (zwergischer Paladin)
\end{description}

\section*{Danksagungen}
\begin{description}
\item[Allgemeine Danksagungen an:] Catharina 'Cat' Link, Ernst-Joachim
Preussler (ERPS) und die deutschen Mailingliste\footnote{erps-public@informatik.uni-frankfurt.de}, Tim Warszta
\end{description}
\end{center}

\newpage
\begin{center}
\epsfig{file=pics/dd.eps, scale=0.5}
\end{center}
\begin{center}
ein Projekt von\\
http://www.drachendaemmerung.de
\end{center}
\begin{center}
\epsfig{file=pics/frostwyrm.eps, scale=0.4}
\end{center}