\section{In den n"achsten Kapitel}
\label{diewelteinleitung}

\parpic[l]{\epsfig{file=pics/drache_icon.eps, scale=0.5}}
In den n"achsten Kapiteln findest Du eine Beschreibung der Welt\index{W!Weltbeschreibung}, in der sich die Charaktere befinden. Diese Beschreibung soll allerdings nur als Anreiz und Idee f"ur eine Spielwelt\index{S!Spielwelt} dienen. Sie ist keinesfalls als dogmatische Festlegung von Tatsachen zu verstehen. Jedem Spielleiter\index{S!Spielleiter} und jeder Spielgruppe\index{S!Spielgruppe} steht es frei sich eine eigene Welt zu erschaffen oder die bereits vorhandene zu modifizieren. Verstehe die folgenden Beschreibungen als einen Vorschlag, wie man es machen k"onnte. Die Beschreibung gliedert sich in die 5 Reiche, zu jedem Reich gibt es eine Karte, auf der geographische Gegebenheiten und St"adte zu finden sind. Im Anhang gibt es eine Gesamtkarte (siehe Anhang S. \pageref{worldmap}).

\par Das Kapitel \textit{Die Welt} (siehe S. \pageref{diewelt}) beschreibt die Umgebung, in der die Charaktere sich zu Spielbeginn aufhalten. Ihr findet dort eine Abhandlung "uber die bekannten Reiche sowie Spekulationen "uber die Grenzregionen\index{G!Grenzregionen} und das Unbekannte. Wichtige Personen und Orte finden dort ihren Platz.

\par Anschlie"send folgt eine detailierte Beschreibung der V"olker des Kontinents (siehe S. \pageref{dievoelker}). Diese dient als Erg"anzung zu den einleitenten Worten bei den Rassenbeschreibungen (siehe S. \pageref{dierassen}). Hier sind tiefer greifende Informationen "uber die einzelnen Sippen der Elfen und die Clans der Zwerge zu finden. Wir lernen einiges "uber die Herschaftsstrukturen in den menschlichen K"onigsh"ausern und gewinnen letztendlich auch noch einen Einblick in die Familienstrukturen der Halblinge und Trolle.

\par In den Kapiteln \textit{Tierleben\index{T!Tierleben}} (siehe S. \pageref{tierleben}), \textit{Bestiarium\index{B!Bestiarium}} (siehe S. \pageref{bestiarium}) und \textit{Kr"auterkunde\index{K!Kr"auterkunde}} (siehe S. \pageref{kr"auterkunde}) findest Du Informationen "uber die bekannte Flora\index{F!Flora} und Fauna\index{F!Fauna}.

\par Das Kapitel \textit{Legenden und Geschichten} (siehe S. \pageref{legendenundgeschichten}) erz"ahlt von der Entstehung der Welt, den Drachen, der Magie und deren Untergang. Hier ist das eine oder andere M"archen und die eine oder andere wahre Begebenheit zu finden. Es liegt im Auge des Betrachters hier Phantasie und Wirklichkeit zu trennen.


\begin{description}
\item[Anmerkung:]In den folgenden Abschnitten wird des "ofteren von der Bev"olkerung dieser Welt die Rede sein. Der Einfachheit halber wird diese als \glqq die Menschheit\grqq{} bezeichnet. In diesem Fall sind damit nicht nur die Menschen gemeint, sondern durchaus alle Bewohner wie Elfe, Trolle, Orks, Gnome, Halblinge, Zwerge etc. Es m"oge sich niemand bei der Formulierung verletzt f"uhlen. Sie dient nicht der Unterscheidung oder Ausgrenzung, sondern nur der Vereinfachung. Jedesmal alle aufzuz"ahlen w"urde schlichtweg zu Unleserlichkeit f"uhren und sich einen neuen Sammelbegriff, der allen gerecht wird, auszudenken zu Unverst"andnis.
\end{description}

\par Genug der vielen Worte, nun geht es los mit dem n"achsten Kapitel, der Weltenbeschreibung.
