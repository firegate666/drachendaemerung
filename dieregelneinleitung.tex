\section{In den n"achsten Kapiteln}

\parpic[l]{\epsfig{file=pics/drache_icon.eps, scale=0.5}}
In dem Kapitel \textit{Charakterklassen}\index{C!Charakterklasse} (siehe S. \pageref{charakterklassen}) 
stelle ich alle mit dieser Spielweltbeschreibung zur Verf"ugung stehenden Charakterklassen\index{C!Charakterklasse} 
vor. Zu jeder Klasse\index{K!Klasse} gibt es einen Hintergrund, sowie eine ausreichende Einf�hrung in das 
Rollenspiel\index{R!Rollenspiel} des Charakters\index{C!Charakter}. Gleichzeitig werde ich noch die typischen 
Attribute\index{A!Attribute} und Fertigkeiten\index{F!Fertigkeiten} jeder Klasse\index{K!Klasse} beschreiben.

\par Neben verschiedenen Klassen d"urfen in einem Rollenspiel nat"urlich die Rassen nicht fehlen. Es gibt 
letztendlich nicht nur Zwerge und Krieger, sondern auch Zwergenkrieger etc. Das Kapitel \textit{Die Rassen} 
(siehe S. \pageref{dierassen}) beschreibt diese.

\par Anschlie"send gehen wir im Kapitel \textit{Neue Fertigkeiten\index{F!Fertigkeiten}} 
(siehe S. \pageref{neuefertigkeiten}) auf die mit diesem Regelbuch neu erscheinenden Fertigkeiten ein. 
Wer diese Fertigkeiten verwenden kann, ist in den Abschnitten \textit{Charakterklassen} und \textit{Rassen} 
nachzulesen(siehe S. \pageref{charakterklassen} und S. \pageref{dierassen}).

\par Das Kapitel \textit{Die Magie\index{M!Magie}} (siehe S. \pageref{diemagie}) ist den arkanen K"unstlern 
unter den Spielern gewidmet. Hier folgen Beschreibungen der Anwendung von Magie\index{M!Magie anwenden} und 
bekannten Zauberspr"uchen\index{Z!Zauberspr"uche}. Ein kleiner Teil des Kapitels besch"aftigt sich mit der 
Wissenschaft der Alchimie\index{A!Alchimie}. Genauere Beschreibung der einzelnen Kr"auter\index{K!Kr"auter} 
und Gew"achse befinden sich in dem Kapitel \textit{Kr"auterkunde} (siehe S. \pageref{kraeuterkunde}).
