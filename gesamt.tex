% Def. des Layouts
\documentclass[a4paper,openany]{book} % DIN A4 Buch, Kapitelbeginn auch auf geraden Seiten (links)
\usepackage[T1]{fontenc}
\usepackage[latin1]{inputenc}

% f"ur Trennung der W"orter nach neuer dt. Rechtschreibung
\usepackage{ngerman}

% Grafik packages
\usepackage{lscape} % f"ur die Verwendung der Landscape-Umgebung
\usepackage{color} % f"ur farbigen Text
\usepackage[dvips]{graphicx}
\usepackage{picinpar} % f"ur Text um Grafik
\usepackage{picins} % f"ur Text um Grafik
\usepackage{wrapfig}

% Seitengr"o"se
\renewcommand{\evensidemargin}{0in} % Abstand nach au"sen bei ungeraden Seiten
\renewcommand{\oddsidemargin}{0in} % Abstand nach au"sen bei geraden Seiten
\renewcommand{\textwidth}{16cm} % Breite des Textblocks ohne Kopf- und Fu"szeile
\renewcommand{\textheight}{22cm} % Hhe des Textblocks ohne Kopf- und Fu"szeile
\renewcommand{\footskip}{1cm}

% f"ur die Verwendung von mehrspaltigen Texten
\usepackage{multicol}

% f"ur die Indexerstellung/-darstellung
\usepackage{makeidx}
\makeindex

% \usepackage{showidx} % einbinden, um indizierte Worte auf den Seiten anzuzeigen
% Pakete zur Verwaltung der Kopf- und Fu"szeilen
% for linux use this
%\usepackage{fancyheadings}
% for windows use this
\usepackage{fancyhdr}
\pagestyle{fancyplain}

% Einfaches Einbinden von EPS-Dateien
\usepackage{epsfig}

% Zugriff auf Anzahl Seiten und
% Seitennummer der letzten Seite
\usepackage{lastpage} % \pageref{LastPage} -> Seitennummer der letzten Seite
\usepackage{count1to} % \pageref{TotalPages} -> Gesamtzahl Seiten

% Kopfzeilendefinition
% [] f"ur gerade Seiten
% {} f"ur ungerade Seiten
\lhead[\let\uppercase\relax\leftmark]{\let\uppercase\relax\rightmark} % links-oben
\chead{} % mitte-oben
\rhead[\let\uppercase\relax\rightmark]{\let\uppercase\relax\leftmark} % rechts-oben

% Fu"szeilendefinition
% [] f"ur gerade Seiten
% {} f"ur ungerade Seiten
\lfoot[Seite \thepage]{} % links-unten
\cfoot{Drachend"ammerungabenteuer - Hier kommt der Name hin} % mitte-unten
\rfoot[]{Seite \thepage} % rechts-unten

% Definition der Kopfzeilenlinien
\renewcommand{\headrulewidth}{1pt}
\renewcommand{\plainheadrulewidth}{1pt}

% Definition der Fu�eilenlinien
\renewcommand{\footrulewidth}{1pt}
\renewcommand{\plainfootrulewidth}{1pt}

% Gro�giges Verhalten beim Zeilenumbruch
% L�t auch gro� Lcken zwischen Worten zu
%\sloppy

% Erweiterte Tabellenumgebungen
\usepackage{array}
\usepackage{tabularx}
\usepackage{longtable}


% Titeldefinition
\title{\epsfig{file=pics/dd.eps, scale=0.6}}
\author{%\epsfig{file=pics/drache.eps, scale=0.3}\\\\
      Autor: Marco Behnke\\
      Seitenumfang: \pageref{TotalPages}\\
      }

% Dokumentgliederung
\begin{document}
        \sffamily
        \frontmatter
                \begin{titlepage}
                	\maketitle
                \end{titlepage}
                \setcounter{section}{-1}
                \addcontentsline{toc}{chapter}{Vorwort}
                \section*{Vorwort}
\par Das hier vorliegende Buch stellt kein Rollenspielsystem im eigentlichen
Sinne dar. Es ist vielmehr als eine Weltbeschreibung zu verstehen,
die zus"atzliche Regelerweiterung f"ur beliebige existierende oder
kommende Systeme zur Verf"ugung stellt. Die folgenden Kapitel
beziehen sich in ihrer Verwendung auf das Rollenspielsystem
ERPS\footnote{ERPS : Ernest Role Playing System}, sind jedoch auch
mit wenig M"uhen auf andere Systeme zu "ubertragen.
\par Das in diesem Buch von mir vorgestellte Magiesystem ist unabh"angig von Regelsystemen und braucht nicht angepasst, sondern kann so verwendet werden.
\par Die Verwendung dieser Spielwelt und seiner Inhalte erfolgt auf rein privater Ebene und darf nicht f"ur kommerzielle Zwecke, also zum Verkauf und Vertrieb, kopiert werden.
\par Als gro"ser Fan der Buchserie \textit{Chronik der Drachenlanze} habe ich mich von der Faszination den Drachen gegen"uber sehr anstecken lassen. Die beiden Buchtitel \textit{Drachend"ammerung}\cite{cdd6}\index{D!Drachendaemmerung} und \textit{Drachenkrieg}\cite{cdd5}\index{D!Drachenkrieg} haben in diesem Regelwerk nicht nur als Namensgeber Pate gestanden.
\par Einen besonderen Dank m�chte ich an dieser Stelle Andr� Trettin\footnote{http://www.bangersoft.de} aussprechen, ohne den ich ERPS niemals kennen und sch�tzen gelernt h�tte.
\subsection*{Wozu wieder eine Spielwelt?}
\par Weil es toll ist? Weil Spieler Abwechslung suchen? Weil ich mich verwirklichen m"ochte? Warum auch immer....
\par Wahrscheinlich ein bisschen von jedem. Als langj"ahriger Rollenspieler hat es mich gereizt meine eigenen Ideen zu verwirklichen und auch mit anderen Spielern zu teilen.
\par Wenn dieses Buch der "Offentlichkeit zur Verf"ugung steht, dann hat meine Welt schon einen langen Spieletestprozess durchlaufen und ich hoffe, dass auf diesem Weg viele Fehler und Unstimmigkeiten bereits behoben sind. Ich hoffe mit dieser Idee Anklang bei anderen Rollenspielern zu finden und in gemeinsamer Arbeit dieses System zu einer finalen Reife zu f�hren.
\par Wer mit mir in Kontakt\index{K!Kontakt} treten m"ochte,
kann das auf dem Wege der elektronischen
Post\footnote{Email}\index{E!Email} tun:
\\
\begin{tabbing}
tab1 \= tab2 \= \kill
\> \> Marco Behnke\\
\> \> marco@firegate.de\\
\\
oder besucht einfach meine Homepage\index{H!Homepage} unter
\\
\\
\> \> http://www.drachendaemmerung.de bzw.\\
\> \> http://forum.firegate.de\\
\end{tabbing}
\par In diesem Sinne frohes Schaffen und Spielen.
\par Marco Behnke


                \tableofcontents
                \newpage\section*{In den n"achsten Kapiteln}

\parpic[l]{\epsfig{file=pics/drache_icon.eps, scale=0.5}}
Die Einleitung stellt die M�glichkeit eines schnellen Einstiegs in das 
Spiel. In den folgenden Kapiteln stelle ich das ERPS-System in ein paar 
kurzen S"atzen vor und erl�utere ein paar �nderungen, die ich daran 
vorgenommen habe. Dieses Buch enth�lt jedoch nicht das ERPS-Regelbuch\cite{erps1}, 
dieses kann auf der deutschen ERPS Homepage\footnote{http://www.erps.de} kostenlos 
heruntergeladen werden, wenn man nach diesem spielen m�chte. 
Es folgt eine Auswahl verschiedener im Spiel zur Verf�gung stehenden 
Charakterklassen\index{C!Charakterklasse} vor. 

\par Im zweiten Kapitel der Einleitung folgt dann eine Einf�hrung in die Welt, 
die die Charaktere erwartet (siehe S. \pageref{spielweltbeschreibung}). Hier 
gibt es eine kleine �bersicht �ber die Charaktermotivation; was zieht ihn ins 
Abenteuer etc.
\par Eine umfangreichere Beschreibung zum bekannten Kontinent, dessen Geschichte 
und Bev�lkerung folgt weiter hinten in dem Kapitel \glqq Die Welt\grqq{} 
(siehe S. \pageref{diewelt}).

        \mainmatter
                \part{Einleitung}
                        \chapter{Das Regelsystem}

\section{ERPS-Regelsystem}
\par Diese Spielwelt wurde mit der Vorgabe entwickelt, nicht an ein bestimmtes System gebunden zu sein. Da 
dies jedoch nicht immer m"oglich ist, habe ich mich an einigen Stellen an das ERPS\footnote{Erne(a)st Role 
Playing Game : http://www.erps.de} Regelsystem gebunden. Warum ich mich f"ur ERPS entschlossen habe? Dazu 
hier ein Zitat des Autor:
\begin{quotation}
\par\textit{ERPS ist ein \glqq modernes\grqq{} System, welches Fertigkeiten "uber Attribute stellt, das keine 
Grade oder Stufen im klassischen Sinne kennt und nicht einmal Charakterklassen. Die freie Entfaltung eines 
Charakters, Verbessern von Fertigkeiten durch praxisnahe Anwendung und ein v"ollig offenes Magiesystem sind 
sicherlich die gr"o"sten Pluspunkte von ERPS. Gerade das Magiesystem bietet ungeahnte M"oglichkeiten, fordert 
allerdings auch ein bi"sschen mehr vom Spielleiter als klassische Magiesysteme.}
\end{quotation}
\par ERPS ist also ein sehr modulares und freies System, welches man leicht in ein eigenes Setting setzen kann. 
Es gl�nzt durch die Abwesenheit von tausenden von spielrelevanten Zusatzb�chern, ein Grundregelwelt reicht f�r 
jahrelanges Auskommen.
\par Das ERPS-Regelsystem wird von mir an einigen Stellen erweitert werden. Diese Erweiterungen sind jedoch leicht 
an andere System anzupassen. Abweichungen zum ERPS-System sind unter anderem an folgenden Stellen zu finden:
\begin{multicols}{2}
\begin{itemize}
\item Erschaffung von Charakterklasse
\item Rassenvor- und Nachteile
\item Klassen- und Nachteile
\item Erweiterung des Magiesystems um eine neue Form von Magie
\item ...
\end{itemize}
\end{multicols}

\section{Charakterklassen - eine kleine �bersicht}
\subsection{Die Archetypen\index{A!Archetyp}}
\par Als erstes folgt eine kurze Beschreibung einiger der zur Verf"ugung stehenden Archetypen\index{A!Archetyp}. 
Diese Auflistung soll gen"ugen, um einem kurzen Eindruck "uber die Funktionalit"at des Systems zu gewinnen. Das 
Kapitel \textit{Charakterklassen}\index{C!Charakterklasse} (siehe S. \pageref{charakterklassen}) stellt alle mit 
diesem Regelbuch zur Verf"ugung stehenden Charakterklassen\index{C!Charakterklasse} vor.
\subsubsection{Krieger\index{K!Krieger}}
\par Der Krieger\index{K!Krieger} ist bewandert im Umgang mit der Waffe wie kein anderer Charakter. Seine Ausbildung 
hat ihn gelehrt selbst unter extremen Bedingungen einen klaren Kopf zu bewahren. Selbst die schwerste R"ustung 
behindert ihn nicht sonderlich in seinen Aktionen.
\par Krieger\index{K!Krieger} haben gelernt mit Zweihandwaffen\index{Z!Zweihandwaffe} zu k"ampfen und wie sie 
effektiv ihre Gegner ausschalten.
Sie sind ausdauernd und z"ah. W"ahrend ihrer Ausbildung haben sie allerdings auch gesellschafliche F"ahigkeiten 
erworben, die ihnen in Friedenszeiten von Nutzen sind.
\subsubsection{Magier\index{M!Magier}}
\par Der Magier\index{M!Magier} hat von Kindesbeinen an gelernt die Natur zu kontrollieren. Er ist eins mit den 
arkanen Str"ohmungen\index{A!arkane Strmung}. W"ahrend seiner langen Ausbildungszeit wurde ihm beigebracht diese 
St"urme zu erkennen und f"ur seine Zwecke zu formen. Magier\index{M!Magier} sind intelligent und sinnesscharf. 
Ihnen entgeht nichts. Sie haben allerdings keine Kenntnisse im Umgang mit Waffen.
\par Ein weiterer Teil ihrer Ausbildung war das Studium alter Schriften und Folianten\index{F!Foliant}. 
Magier\index{M!Magier} sind sprachbegabt und kennen Mundarten und Schriftzeichen, von deren Existenz andere 
Sterbliche noch nie etwas geh"ort haben.
\subsubsection{Dieb\index{D!Dieb}}
\par Heimlichkeit und Geschick sind die Disziplinen des Diebes\index{D!Dieb}. Dabei sprechen wir hier jetzt 
nicht von dem Stra"senr"auber\index{S!Stra"snr"auber} oder dem Gelegenheits-Einbrecher\index{E!Einbrecher}. 
Der Dieb\index{D!Dieb} sieht in seinen Taten ein Ritual. Jede Aktion hat ihren eigenen Reiz und ihre eigene 
Atmosph"are. Er stiehlt meist nicht um der Beute willen, sondern um seine Grenzen zu erfahren und sich jedes 
mal von neuem zu �bertreffen.
\par F"ur Diebe\index{D!Dieb} gibt es keine verschlossenen T"uren oder versteckte Fallen. Ihr Instinkt hat 
ihnen bisher noch aus jeder Situation heraus geholfen.

\section{Charaktererschaffung}
\par Den genauen Ablauf und die verwendeten Mechanismen zur Charaktererschaffung sind dem ERPS-Regelwerk\cite{erps1} zu 
entnehmen. Als Erg"anzung daf"ur sind folgende Punkte zu ber"ucksichtigen.
\begin{enumerate}
\item Nach dem Ausw"urfeln der Attribute, muss die Spielerrasse gew"ahlt werden. Entstandene Vor- und Nachteile 
werden auf dem Charakterbogen vermerkt.
\item Anschlie"send entscheidet man sich f"ur eine Charakterklasse, vermerkt ebenfalls die entstandenen "Anderungen. 
Dann geht es weiter mit dem Steigern der Anfangsfertigkeiten.
\end{enumerate}
\par Alle weiteren Regeln, wie z.B. angeborene F"ahigkeiten usw. behalten weiter ihre G"ultigkeit.

                        \chapter{Spielweltbeschreibung\index{S!Spielweltbeschreibung}}
\label{spielweltbeschreibung}
\par Eine Frage, die sich jeder Spieler zu Beginn stellt, ist doch die Frage: \glqq Wo lebt mein Charakter? 
Was motiviert ihn?\grqq
\par In den folgenden Abschnitten werden wir uns kurz mit der Spielwelt\index{S!Spielwelt}, in der die 
Charaktere leben und der Motivation\index{M!Motivation}, die die Charaktere durch das Abenteuer\index{A!Abenteuer} 
lenkt, besch"aftigen. Diese Vorstellung soll nur eine kurze Einleitung in das Spielgeschehen\index{S!Spielgeschehen} 
darstellen. In sp"ateren Kapiteln (ab S. \pageref{diewelteinleitung}) werden wir noch genauer auf das Umfeld des 
Spielsystems\index{S!Spielsystem} zu sprechen kommen.

\section{Die Welt}
\par Drachend"ammerung spielt in ein einer sogenannten High Fantasy Welt. Wenig ist vergleichbar mit dem uns 
Bekannten und Akzeptierten. Die Welt wird bestimmt von uns fremden Kreaturen und Magie. Magie spielt eine der 
gr"o"sten, wenn nicht sogar die gr"o"ste Rolle.
\par Gerade ein Millennium ist seit der gro"sen Befreiung vergangen und die Welt steht am Beginn eines neuen 
Zeitalters. Der Umbruch steht kurz bevor und die Propheten haben nichts Gutes zu berichten. Die Sterne k"undigen 
die R"uckkehr der Peiniger an, denn der Bannzauber beginnt sich aufzul"osen. L"angst vergessen ist das Wissen um 
seine Erschaffung und Erhaltung.

\section{Der Auftrag}
\par Die Motivation der Abenteurer kann vielerlei sein. Die Helden finden sich in einer Welt
 wieder, die in das Vorstadium der Apokalypse eingetreten ist. Niemand kann auch nur erahnen was hinter dieser
 Schwelle verborgen liegt. Viele Fragen gibt es, die zu kl"aren sind.
\par Wo sind die Schriften und B"ucher jener "Ara, die das Geheimniss des Bannzaubers beherbergen? Die Magiergilden 
und insbesondere der Rat der 18 ist fieberhaft auf der Suche nach dem Wissen um seine Aufrechterhaltung und 
Erneuerung. Zu grausam w"are seine Aufl"osung und die damit verbundene R"uckkehr der Urgewalten.
\par Die Armeen des Landes bereiten sich auf einen gro"sen Krieg vor. Sollten die Magier nicht rechtzeitig eine 
L"osung des Problems beibringen k"onnen, wird die Entscheidung in gewaltigen Schlachten fallen. 
Damit man "uberhaupt eine Chance im Kampf hat, ist es wichtig, dass die magischen und geweihten Waffen der Krieger 
des zweiten Millenniums wiedergefunden werden, falls sie nicht zerst"ort wurden.
\par Zu guter letzt darf man nicht vergessen, dass die Helden zu Beginn des Spiels ein Haufen unerfahrener Chaoten 
sind, die in ihrer Zukunft ihre Legenden sehen. Doch es ist noch kein Meister vom Himmel gefallen und wer es zu 
etwas bringen will, der sollte zumindest einen halbwegs gef"ullten Geldbeutel und Magen haben.

                        \newpage\section{In den n"achsten Kapiteln}

\parpic[l]{\epsfig{file=pics/drache_icon.eps, scale=0.5}}
In dem Kapitel \textit{Charakterklassen}\index{C!Charakterklasse} (siehe S. \pageref{charakterklassen}) 
stelle ich alle mit dieser Spielweltbeschreibung zur Verf"ugung stehenden Charakterklassen\index{C!Charakterklasse} 
vor. Zu jeder Klasse\index{K!Klasse} gibt es einen Hintergrund, sowie eine ausreichende Einf�hrung in das 
Rollenspiel\index{R!Rollenspiel} des Charakters\index{C!Charakter}. Gleichzeitig werde ich noch die typischen 
Attribute\index{A!Attribute} und Fertigkeiten\index{F!Fertigkeiten} jeder Klasse\index{K!Klasse} beschreiben.

\par Neben verschiedenen Klassen d"urfen in einem Rollenspiel nat"urlich die Rassen nicht fehlen. Es gibt 
letztendlich nicht nur Zwerge und Krieger, sondern auch Zwergenkrieger etc. Das Kapitel \textit{Die Rassen} 
(siehe S. \pageref{dierassen}) beschreibt diese.

\par Anschlie"send gehen wir im Kapitel \textit{Neue Fertigkeiten\index{F!Fertigkeiten}} 
(siehe S. \pageref{neuefertigkeiten}) auf die mit diesem Regelbuch neu erscheinenden Fertigkeiten ein. 
Wer diese Fertigkeiten verwenden kann, ist in den Abschnitten \textit{Charakterklassen} und \textit{Rassen} 
nachzulesen(siehe S. \pageref{charakterklassen} und S. \pageref{dierassen}).

\par Das Kapitel \textit{Die Magie\index{M!Magie}} (siehe S. \pageref{diemagie}) ist den arkanen K"unstlern 
unter den Spielern gewidmet. Hier folgen Beschreibungen der Anwendung von Magie\index{M!Magie anwenden} und 
bekannten Zauberspr"uchen\index{Z!Zauberspr"uche}. Ein kleiner Teil des Kapitels besch"aftigt sich mit der 
Wissenschaft der Alchimie\index{A!Alchimie}. Genauere Beschreibung der einzelnen Kr"auter\index{K!Kr"auter} 
und Gew"achse befinden sich in dem Kapitel \textit{Kr"auterkunde} (siehe S. \pageref{kraeuterkunde}).

                \part{Die Regeln}
                        \chapter{Charakterklassen\index{C!Charakterklasse}}
\label{charakterklassen}
\section{Einleitung}
\par In den ersten Kapiteln haben wir bereits die drei grundlegenden Abenteurerklassen\index{A!Abenteurerklassen} 
vorgestellt, die in keinem High Fantasy Rollenspiel fehlen sollten: Krieger, Magier und Dieb.
\par Der folgende Abschnitt stellt nun die mit diesem Regelwerk erscheinenden Charakterklassen in ihrer F"ulle 
vor und gibt damit sowohl dem Spieler als auch dem Spielleiter ein kleines Sammelwerk f"ur ihre Abenteurerrunde 
mit. Das diese Sammlung nicht als vollst"andig zu betrachten ist, brauche ich wohl nicht extra zu betonen. 
Es obliegt der Absprache zwischen Spieler und Spielleiter weitere Klassen zu kreieren oder g�nzlich auf eine
zu verzichten.
\par Das n"achste Kapitel (ab S. \pageref{dierassen}) stellt die zur Verf"ugung stehenden Rassen vor.
\par Jede Klassenbeschreibung\index{K!Klassenbeschreibung} folgt der unten aufgelisteten Struktur.
\begin{description}
\item[Beschreibung:]Umfassende Beschreibung der Charakterklasse
\item[Wesen:]rollenspieltechnische Hinweise f"ur den Spieler; Beschreibung der allgemeinen 
Wesensz"uge\index{W!Wesensz"uge}, St"arken\index{S!St"arken} und Schw"achen\index{S!Schw"achen}
\item[Ausbildungsweg\index{A!Ausbildungsweg}:]Wie erfolgte die Ausbildung\index{A!Ausbildung}? Wann begann 
diese, wie lange dauerte sie?
\item[Vorz"uge\index{V!Vorz"uge} / Nachteile\index{N!Nachteile}:]Hier stehen Informationen "uber Fertigkeiten, 
die die Charakterklasse besser oder schlechter beherrscht
\item[Besonderes:]Hier wird etwas stehen, falls die Charakterklasse Besonderheiten, wie \mbox{z. B.} besondere 
F"ahigkeiten besitzt oder �ber spezielle Eigenarten verf"ugt.
\end{description}

\newpage
\section{K�mpferklassen}
\subsection{Der klassische Krieger\index{K!Krieger}}
\subsubsection{Beschreibung}
\par Der Archetyp Krieger lie"se sich m"uhelos noch weiter klassifizieren, doch w"urde dies 
nur mehr Namen f"ur ein und die selbe Person in anderer Kleidung mit sich bringen.
\parpic[r]{\epsfig{file=pics/krieger.eps, scale=0.6}}
Wer kennt ihn nicht, den stolzen Krieger der Palastgarde, den k"onigstreuen Ritter\index{R!Ritter}, den aufbrausenden 
Hochlandbarbaren\index{B!Barbar} oder aber den S"oldner\index{S!S�ldner} f"ur den Ehre einfach nur eine Frage des Lohnes ist. All diese 
z"ahlen wir zur Klasse des Kriegers.
\par Sie haben alle eines gemeinsam: Eine hervoragende Ausbildung an den Waffen.
\par Krieger ziehen aus vielerlei Gr"unden auf Abenteuer. Der Ritter zieht aus, um auf Gehei"s seines 
K"onigs einen fl"uchtigen Schwerverbrecher zu fassen oder aber auf einer Queste ein seltenes und doch 
wichtiges Artefakt zu bergen. Der Barbar wird sich mit Sicherheit auf einer Reise befinden, um neue 
Gegenden kennenzulernen oder er ist der Sohn des H"auptlings und er muss sich in der Welt beweisen, 
bevor er dessen Nachfolge antreten kann. Und was den S"oldner betrifft...
\subsubsection{Wesen}
\subsubsection{Ausbildungsweg}
\subsubsection{Vorz"uge / Nachteile}
\subsubsection{Besonderes}

\newpage
\subsection{Der Ordenskrieger\index{O!Ordenskrieger} / Paladin\index{P!Paladin}}
\subsubsection{Beschreibung}
\parpic[r]{\epsfig{file=pics/paladin.eps, scale=0.2}}
%Alternatives Bild eines Paladin
%\parpic[r]{\epsfig{file=pics/paladin2.eps, scale=0.5}}
Der Ordenskrieger/Ordensritter oder auch besser bekannt als Paladin steht als Symbol f�r den kriegerischen 
Arm der Kirche. Vom Prinzip her ist ein Paladin nat�rlich auch nicht viel anders als der bereits beschriebene 
Krieger. Doch es gibt feine Unterschiede, die eine weiter Klassifizierung rechtfertigen: Der Paladin k�mpft im 
Namen der G�tter und ist somit von ihnen bevorzugt. Oder auch anders gesagt: Er k�mpft im Namen der Kirche und 
darf sich auch mal mehr erlauben.
\par Wie die Priester auch verf�gt der Paladin �ber die Gabe seinee G�tter in Form von Wundern um Hilfe zu 
bitten. Dabei ist diese Kraft oder besser gesagt die Beg�nstigung nicht so weit fortgeschritten wie die eines 
wirklichen Geistlichen. Schlie�lich liegt die St�rke des Paladins im F�hren der Klinge.
\par Der Paladin ist unterwegs, um auf der einen Seite das Wort seines Gottes oder seiner G�tter zu verbreiten. 
Auf der anderen Seite ist er st�ndig auf der Suche nach Zeichen, die die G�tter auf der Welt hinterlassen haben, 
meist in Form von heiligen Reliquien.
\par Paladine genie�en ein hohes Ansehen in der Bev�lkerung und werden von der Kirche hervoragend ausgestattet.
\subsubsection{Wesen}
\par Paladine sind edle K�mpfer f�r die Gerechtigkeit. Sie treten f�r die Bed�rftigen ein und preisen das 
Wort ihrer G�tter. Das jedenfalls erz�hlen die Paladine von sich. In der Praxis beschr�nkt sich ihre T�tigkeit 
jedoch meist auf letzteres. Von diesem Vorurteil seien jedoch nicht alle betroffen. Unter den Paladinen gibt es 
immer noch jene noblen Recken, von denen die Legenden berichten.
\subsubsection{Ausbildungsweg}
\par Kinder aller Bev�lkerungsschichten und egal welchen Geschlechts k�nnen bei einem Alter von 5 Jahren in die 
Kirche zur Ausbildung gebracht werden. Dort werden sie von den Priestern aufgenommen und im Laufe ihrer Ausbildung 
den verschiedenen Berufungen zugeteilt. Die Grundausbildung im Tempel dauert 7 Jahre. In diesen Jahren lernen die 
Kinder Lesen, Schreiben und Rechnen, werden gesellschaftlich geschult und in Fremdsprachen unterrichtet.
\par Kinder, die dann zum Paladin ausgebildet werden sollen, werden mit dem 12. Lebensjahr zum Knappen ernannt. Von 
nun an beginnt ihre Grundausbildung im Kampf und im Reiten, die erst im Alter von 18 Jahren ihr Ende finden wird. 
Parallel dazu werden sie im Tempel von den Priestern im Wirken von Wundern geschult.
\par Erreicht der junge Sch�ler den Rang eines Paladins, so mu� er sich fortan auf einer 2-j�hrigen Reise beweisen. 
Hierzu erh�lt er gerade mal das N�tigste. Von nun an mu� er sich auf sich allein verlassen.
\par Kehrt er von der Reise zur�ck und ist stark im Glauben geblieben, so wird er in den Tempel als vollwertiger 
Paladin aufgenommen.
\subsubsection{Vorz�ge / Nachteile}
\par Paladine, die auf dem rechtschaffenden Wege wandern, sind in der Lage um Wunder von ihren G�ttern zu bitten, 
verlassen sie diesen Weg, werden sie vor�bergehend von ihren G�ttern verlassen.
\par Paladine k�nnen einfache Wunder wie \textit{Wunden heilen} oder \textit{Magie erkennen} erlernen oder aber auch 
um Wunder bitten, die sie im Kampf st�rken.
\subsubsection{Besonderes}
\par Paladine erhalten eine Reihe von Auflagen, die ihnen von ihrem Glauben und ihrer Kirche nicht nur auferlegt 
sind, sondern die sie in ihrem Wesen verewigt haben.
\begin{itemize}
\item Ein Paladin betet mindestens 3 mal pro Tag zu den G�ttern und dankt ihnen f�r den Rat und den Beistand, den 
sie ihm auf seinen Reisen gew�hren
\item Ein Paladin wird niemals von sich aus einen Kampf beginnen, auch nicht wenn er dazu herausgefordert wird. Er 
wird sich allerdings sehr wohl auf eine Verteidigung vorbereiten; eine Ausnahme stellt der Kreuzzug dar. Er wird auch 
niemals einen unbegr�ndeten Kampf f�hren.
\item Ein Paladin der k�mpft, wird diesen Kampf ehrbar durchf�hren. Er wird keinen Gegner mit einer �bermacht 
angreifen und auch nicht gegen Gegner vorgehen, die ihm offensichtlich unterlegen sind.
\end{itemize}
\par Es sei noch zu ber�cksichtigen, dass die Aufgabe des Paladins nicht haupts�chlich darin liegt, Gl�ubige zu 
finden und zu bekehren, die ist vorrangig die Aufgabe der Kleriker. Der Paladin hat das Wort der G�tter zu sch�tzen 
und dar�ber zu wachen, dass es erhalten und nicht geschm�lert wird. Haupts�chlich wird er jedoch der Aufgabe folgen, 
die ihm der Tempel auferlegt hat.


\newpage
\section{Abenteurer/Gl�cksritter}
\subsection{Der Barde\index{B!Barde}}
\begin{quotation}
\textit{Wo ich bin gef�llt es mir, doch will ich nicht lang
bleiben}\\
\textit{Doch wo ich nie gewesen bin, dort will ich verweilen}
\end{quotation}
\subsubsection{Beschreibung}
\par Egal ob Poet\index{P!Poet}, Minnes"anger\index{M!Minnes"anger} oder Musiker\index{M!Musiker}, sie alle 
sind Barden und fr"ohnen der Kunstrichtung Musik und Dichtung. Hoffnungslos romantisch und mit unbeschreiblicher 
Vorstellungskraft gesegnet, so sehen sich jedenfalls die meisten Barden.
\parpic[r]{\epsfig{file=pics/barde2.eps, scale=0.5}}
Unter ihnen gibt es selbstverst"andlich auch die wandernden Musiker. Sie begleiten Abenteurergruppen, immer 
auf der Suche nach neuen Geschichten.
\par Die Barden sind es, die die Geschichten weitererz"ahlen und neue Legenden schaffen. Vielleicht liegt in 
ihrem Wissen noch ein wichtiger Schl"ussel f"ur die Zukunft?
\subsubsection{Wesen}
\par Barden sind von Grund auf aufgeschlossen. Es liegt in ihrer Natur auf die Menschen zuzugehen und sich mit 
ihnen zu unterhalten. Auf diese Art und Weise nehmen sie Informationen auf. Einen schlecht gelaunten Barden wird 
man ebenso oft treffen wie einen riesenw�chsigen Zwerg, eigentlich nie und es bedarf schon einer besonderen 
Einwirkung von Au"sen, um dies zu bewerkstelligen.
\subsubsection{Ausbildungsweg}
\par Das Talent eines Barden wird sich schon im fr"uhen jugendlichen Alter zu erkennen geben. Musikalisches 
Talent, Redegewandheit und die Kunst zu dichten und zu schreiben zeichen einen jungen Barden aus. Meist wird 
er dann die Hilfe eines erfahrenen Barden aufsuchen und bei ihm in die Lehre gehen. Mindestens genauso viele 
bringen sich alles n�tige selber bei. \par Als n"achstes wird es ihn in die Ferne ziehen, wo er sich einer 
Gruppe von Abenteurern anschlie"sen wird, dies ist der zweite und letzte Teil seiner Ausbildung. So lernt er 
die Welt kennen, bilder seine Talente weiter und wird Geschichten schreiben.
\subsubsection{Vorz"uge / Nachteile}
\subsubsection{Besonderes}

\newpage
\subsection{Der Dieb\index{D!Dieb}}
\subsubsection{Beschreibung}
\par Heimlichkeit und Geschick sind die Disziplinen des Diebes. Dabei sprechen wir hier jetzt nicht von dem
 Stra"senr"auber\index{S!Stra"snr"auber} oder dem Gelegenheits-Einbrecher\index{E!Einbrecher}. Der Dieb sieht 
 in seinen Taten ein Ritual.
\parpic[r]{\epsfig{file=pics/dieb.eps, scale=1}}
Jede Aktion hat ihren eigenen Reiz und ihre eigene Atmosph"are. Er stiehlt meist nicht um der Beute willen, 
sondern um seine Grenzen zu erfahren und sich jedes mal von neuem zu �bertreffen.
\par F"ur Diebe gibt es keine verschlossenen T"uren oder versteckte Fallen. Ihr Instinkt hat ihnen bisher noch
 aus jeder Situation heraus geholfen.
\subsubsection{Wesen}
\subsubsection{Ausbildungsweg}
\subsubsection{Vorz"uge / Nachteile}
\par Fast jeder Dieb ist ein Mitglied \textit{der Schatten}\index{S!Schatten, die}. Die Schatten sind die 
Kontinentweite Gilde der Diebe. Mitglieder dieser Gilde k"onnen jederzeit auf deren Hilfe bauen. Ob es nun 
um Unterst"utzung, Unterschlupf, Transportm"oglichkeiten oder einfach nur Informationen geht. Im Gegenzug 
\glqq spendet\grqq{} jeder Dieb einen Zehnt seines \glqq Verdienstes\grqq{} der Gilde. Um die Loyalit"at 
ihrer Mitglieder zu wahren, werden die Neuank"ommlinge magisch an den Gildenrat gebunden. Dieses macht es 
dem Einzelnen unm"oglich ihnen gegen"uber die Unwahrheit zu sprechen.
\par Gildenlose Diebe haben es schwer. Sie stehen nicht unter territorialer Kontrolle und k"onnen nicht auf 
die Unterst"utzung der Gilde oder deren Mitglieder bauen, unter ihnen gelten sie als ge"achtet. Im Gegenzug 
k"onnen sie ihre Eink"unfte f"ur sich behalten.
\subsubsection{Besonderes}

\newpage
\subsection{Der Gl�cksritter\index{G!Gl�cksritter} / Streuner\index{S!Streuner}}
\subsubsection{Beschreibung}
\par F"ur den Gl"ucksritter gibt es viele Namen und er tritt in mindestens so vielen Variationen in den 
unterschiedlichsten Gegenden auf. Er ist ein Kind der Strasse, ein Abenteurer, ein Wanderer und letztendlich 
auch ein Freibeuter auf der See. Der typische Mantel und Degen Held eben.
\parpic[r]{\epsfig{file=pics/streuner.eps, scale=0.6}}
Wir sehen hier vor uns die typischen Bilder der Menschen, die niemals das Abenteuer missen wollen, die es nicht 
ertragen k"onnen lange an einem Ort zu verweilen oder aber in ihrer Freiheit eingeschr"ankt zu werden.
\subsubsection{Wesen}
\par Ein gro"ses Maul hat er wohl und mit der Wahrheit nimmt er es auch nicht immer zu ernst. Und wenn es mal 
drauf an kommt, dann nimmt er auch schon mal die Beine in die Hand. So jedenfalls wird man einen Streuner meist 
erleben.
\par Im Laufe seiner Abenteurerkarriere wird er jedoch versuchen, und das sehr erfolgreich, von seinen 
Weggef"ahrten zu lernen, was zu lernen ist. Und fr"uher oder sp"ater wird er merken, dass der Weg nach vorne 
auch zum Ziel f"uhren kann.
\subsubsection{Ausbildungsweg}
\par Es gibt keine bessere Ausbildung zum Gl"ucksritter, als die Stra"se selbst. Hier lernt der Gl"ucksritter 
von Kindesbeinen an auf eigene Faust zu "uberleben. Obwohl sein "Uberlebenstraining in den ersten Jahren wohl 
haupts"achlich aus Nahrungsbeschaffung besteht. Mit sehr gro"ser Wahrscheinlichkeit ist er in den ersten Jahren 
auch in Waisenh"ausern untergekommen, wo er mit Gl"uck einige Kenntnisse in Wort und Schrift gelernt hat, lange 
geblieben sein wird er dort jedoch nicht.
\subsubsection{Vorz"uge / Nachteile}
\subsubsection{Besonderes}


\newpage
\section{Der Waldl�ufer}
\subsection{Der klassische Waldl�ufer\index{W!Waldl�ufer}}
\subsubsection{Beschreibung}
\par J�ger und Sammler, F�hrtenleser, ein Kind der Wildniss. Waldl�ufer verstehen sich auf das �berleben 
in der Wildniss und im Wald. Sie verstehen die F�hrten wilder Tiere zu lesen und sie verstehen es, Spuren im 
Wald zu deuten.
\parpic[r]{\epsfig{file=pics/waldlaeufer.eps, scale=0.7}}
Im Grunde seines Herzens ist der Waldl�ufer ein geselliger Typ, auch wenn er nach au�en hin einen Einzelg�nger 
reflektiert. Doch die Meinung, die die gemeine Bev�lkerung von diesem verschrobenen gr�nen M�nnchen hat, ist 
pr�gend f�r sein Auftreten.
\subsubsection{Wesen}
\par Obwohl der Waldl�ufer allgemeinhin als Einzelg�nger und verschrobener Einsiedler gilt, so lernen diejenigen, 
die mit ihm Freundschaft schlie�en, eine ganz andere Seite kennen. Er ist Freunden gegen�ber aufgeschlossen und 
wei� sich in eine Gruppe einzugliedern. Er bietet seine Dienste an, wo er kann, bringt jedoch auch eine gesundes 
Ma� an Misstrauen an den Tag, wenn ihm etwas suspekt erscheint.
\par Der Waldl�ufer ist niemand, der intuitiv handelt. Jeder Schritt ist geplant und weitergedacht, immer auf 
m�gliche Folgen fixiert. Wer jetzt denkt, dass der Waldl�ufer langsam handelt, der liegt falsch, denn der Waldl�ufer 
kalkuliert schnell und zuverl�ssig.
\subsubsection{Ausbildungsweg}
\par Die Ausbildung des Waldl�ufers erfolgt, wer h�tte es gedacht, im Wald. Waldl�ufer gehen nicht auf die 
Suche nach neuen Sch�lern, sie erwarten vielmehr, dass diese den Weg zu ihnen finden. Bevor der Waldl�ufer den 
Sch�ler annimmt, muss dieser eine Aufnahmepr�fung bestehen. Diese besteht meist aus verschiedenen Jagd- und 
Suchaufgaben im Wald. F�llt diese erfolgreich aus, ist der halbe Weg schon gemacht. Der zuk�nftige Sch�ler 
muss den Lehrmeister nur noch �berzeugen, warum er gerade ihn als Lehrling annehmen soll. Im Normalfall hat ein 
Waldl�ufer immer 3 bis 4 Sch�ler in seinem Haus.
\par Nach Ende der Ausbildung, im Regelfall nach 5 Jahren, verabschiedet sich der Sch�ler von seinem Lehrmeister 
und zieht auf eine mindestens 10-j�hrige Wandertour, um fremde Gegenden, Tiere und Pflanzen kennen zu lernen. 
Anschlie�end suchen sich die meisten Waldl�ufer eine ruhige Waldgegend und lassen sich nieder, um ihr Wissen 
an andere weiterzugeben oder stellen sich in den Dienst von ans�ssigen F�rsten. Nur wenige bleiben weiter auf 
dem Pfad des Abenteurers.
\subsubsection{Vorz�ge / Nachteile}
\par Dem Waldl�ufer ist es m�glich, selbst dann noch Spuren zu erkennen, wenn der menschliche Verstand bereits 
versagt. Er erh�lt magisches Spuren lesen (+1) und kann die Fertigkeit Jagen erlernen. Genaue Informationen zu 
neuen Fertigkeiten sind im Abschnitt \textit{Neue Fertigkeiten} (siehe S. \pageref{neuefertigkeiten}) zu finden.
\subsubsection{Besonderes}

\newpage
\subsection{Der Tiermeister\index{T!Tiermeister}}
\subsubsection{Beschreibung}
\par Obwohl er viel mit dem Waldl�ufer gemeinsam hat, so gibt es doch auch ein paar gravierende Unterschiede. 
W�hrend der Waldl�ufer zwar in der Lage ist, aus dem Verhalten der Tiere zu schlie�en, hat der Tiermeister 
jedoch gelernt mit ihnen zu kommunizieren.
\parpic[r]{\epsfig{file=pics/tiermeister.eps, scale=0.4}}
Der Tiermeister scheint �ber �bersinnliche F�higkeiten zu verf�gen, die ihm eine Kommunikation mit Tieren 
oder \glqq intelligenten\grqq{} Pflanzen erlaubt.
\par Tiermeister ziehen nur aus zwei Gr�nden in das Abenteurerleben, entweder er ist auf einem Vergeltungszug f�r 
eine gro�e Missetat, die den Tieren seines Waldes wiederfahren ist oder er ist auf der Suche nach seinem Sch�ler.
\subsubsection{Wesen}
\par Haftet dem Waldl�ufer das Bild des Einzelg�ngers an, so stellt der Tiermeister den typischen Einzelg�nger 
dar. Ihm f�llt es schwer, Kontakte zu anderen Menschen aufzubauen. Seine Welt und seine Gef�hrten sind die Tiere.
 Eine Freundschaft zu einem Tiermeister baut sich nur langsam auf und erfordert ein hohes Ma� an Geduld und 
 Vertrauen. Sollte man den Tiermeister darin entt�uschen, hat man ihn sich zu einem erbitterten Feind gemacht. 
 Eine aufgebautet Freundschaft bleibt ein Leben lang erhalten und der Tiermeister ist ein loyaler Freund.
\par Tiermeister schlie�en sich nichts desto trotz �fters Abenteurergruppen an. N�mlich immer dann, wenn sie 
auf Wanderschaft gehen (siehe oben). Er wird sie in jedem Fall als seine Gef�hrten akzeptieren und wenn er es 
als n�tig und sinnvoll erachtet, seine Dienste und F�higkeiten anbieten. In jedem Fall steht er der Gruppe 
loyal gegen�ber, so lange er das Gef�hl hat, sich auf sie verlassen zu k�nnen. Verl�sst ihn das Gef�hl, 
so wird er sie in der n�chten Nacht still und heimlich verlassen.
\subsubsection{Ausbildungsweg}
\par Gr��er k�nnte der Gegensatz zum Waldl�ufer nicht sein. Niemand kommt zu einem Tiermeister, um bei ihm 
zu lernen. Vielmehr findet der Tiermeister seine Sch�ler und das auf nicht ganz konventionelle Art und Weise. 
Jeder Tiermeister bildet in seinem Leben nur einen Sch�ler aus und h�lt zu diesem sein Leben lang einen engen 
Kontakt.
\par Irgendwann wird der Tiermeister von einem Traum heimgesucht, der ihn mit seinem zuk�nftigen Sch�ler bekannt
 macht. Er bekommt vage Visionen �ber den derzeitigen Aufenthaltsort, sollte diese schon geboren sein, und sein 
 Aussehen. Ab dem n�chsten Tag begibt sich der Tiermeister auf eine lange Wanderschaft, die bis zur Findung seines 
 Sch�lers andauert. Dies k�nnte sich nat�rlich umso schwieriger gestalten, sollte der Zuk�nftige erst in kurzer 
 Zeit geboren werden.
\par Hat er seinen Sch�ler gefunden, dieser wird zum Zeitpunkt des Findens zwischen einem Monat und 5 Jahren alt 
sein, �berzeugt der Tiermeister die Eltern des Kindes von 
dessen Talent. Sollte ihm dies gelingen, zieht er mit seinem neuen Sch�ler zur�ck zu seiner Wohnst�tte und 
beginnt die Ausbildung. Ein Tiermeister der seinen Sch�ler zu sp�t findet - Kinder �ber 5 Jahre haben nicht mehr
die Gabe die Sprache des Waldes zu lernen - oder aber die Eltern nicht von dessen Talent �berzeugen kann, wird
 ebenfalls wieder zur�ckkehren und ein normales Leben 
fortsetzen, er wird allerdings niemals wieder eine Vision von einem Sch�ler bekommen und somit auch nie einen
 Nachfolger ausbilden.
\subsubsection{Vorz�ge / Nachteile}
\subsubsection{Besonderes}
\par Tiermeister haben die besondere F�higkeit, mit Tieren kommunizieren zu k�nnen.
\par Besonders alte Tiermeister ziehen noch einmal los, um ihre letzte Ruhest�tte aufzusuchen. Diese ist niemanden 
bekannt, der jetzt noch unter uns weilt, und dem Sterbenden offenbart sie sich auch erst in den letzten Wochen 
seines Lebens. An diesem geheimnisumwobenen Ort legt er sich zur letzten Ruhe nieder. Seine Seele vereint sich mit 
dem gro�en Geist der Tierwelt und er wird im Moment seines Todes im K�rper eines Tieres wiedergeboren.


\newpage
\section{Magiebegabte}
\subsection{Der Magier\index{M!Magier}}
\subsubsection{Beschreibung}
\par Der Magier\index{M!Magier} hat von Kindesbeinen an gelernt
die Natur zu kontrollieren. Er ist eins mit den arkanen
Str"ohmungen. W"ahrend seiner langen
Ausbildungszeit wurde ihm beigebracht diese St"urme zu erkennen und
f"ur seine Zwecke zu formen. Magier sind intelligent
und sinnesscharf. Ihnen entgeht nichts. Sie haben allerdings keine
Kenntnisse im Umgang mit Waffen.
\par Ein weiterer Teil ihrer Ausbildung war das Studium
alter Schriften und Folianten.
Magier sind sprachbegabt und kennen Mundarten und
Schriftzeichen, von deren Existenz andere Sterbliche noch nie etwas
geh"ort haben.
\subsubsection{Wesen}
\par Der Magier ist ein ewiger Student. Er hat sein Leben der Magie
gewidmet. Sein Studium der Magie hat ihm die Macht seiner Kunst
gelehrt, er wei� sie mit Bedacht einzusetzen und ist sich ihrer
Gefahren bewusst. Die meisten Magier betrachten sich als
Wissenschaftler und Magie als Wissenschaft, sie w�rden sich niemals
dazu hinrei�en lassen, sie verschwenderisch oder gar prahlerisch zu
nutzen.
\subsubsection{Ausbildungsweg}
\par Das Studium der Magie ist ein langer und schwerer
Ausbildungsweg. Nur wenige Sch�ler der Akademien k�nnen sich r�hmen
am Ende auch einen verbrieften Abschluss in H�nden zu halten. Die
meisten werden fr�her oder sp�ter in einer der P�fungen scheitern
und fortan dem Selbststudium verfallen und als reisende Forscher
unterwegs sein oder als wissenschaftliche Mitarbeiter in der
Akademie ihr Studium der Magie fortsetzen, sich somit von Pr�fung zu
Pr�fung schleppen, in der Hoffnung doch noch mal den Abschluss zu
erreichen.
\par Das magische Potential muss bereits in fr�hen Jahren erkannt
und gef�rdert werden, um es voll aussch�pfen zu k�nnen. Im Regelfall
beginnt die Ausbildung des neuen Sch�lers nach bestandenem
Aufnahmetest mit ca. 4 Jahren. In den ersten 10 Jahren lernt der
Sch�ler neben einigen niederen Magie�bungen zum Erhalt des
Potentials haupts�chlich allgemeines Wissen der Wissenschaften
Mathematik und Alchimie und lernt Sprachen, Lesen und Schreiben und
Magietheorie.
\subsubsection{Vorz"uge / Nachteile}
\par Der Magier kann Spr�che aus nahezu allen Psi-Bereiche
mit Ausnahme von Dimension und Beschw"orung erlernen. Zu Spielbeginn
w"ahlt er frei 3 Zauber aus der Liste aus.
\subsubsection{Besonderes}
\par Jeder Magier muss zu Beginn des Spieles w�hlen, ob er einem
Zirkel angeh�rt oder ein freier Magier ist (siehe dazu das Kapitel
\textit{Die Magie} ab S. \pageref{spruchmagie}), und ob er einen Abschluss
gemacht hat oder nicht.

\newpage
\subsection{Der Beschw�rer\index{B!Beschw�rer, der}}
\subsubsection{Beschreibung}
\par Erst einmal denkt jeder bei der Bezeichnung \glqq Beschw�rung\grqq{}
an schwarze Magie. Man denkt an D�monen\index{D!D�mon},
Geister\index{G!Geist}, Unholde\index{U!Unhold} und
Elementarwesen\index{E!Elementarwesen}. Also an alles
�bernat�rliche, was die eigene Phantasie so her gibt.
\parpic[r]{\epsfig{file=pics/beschwoerer.eps, scale=0.6}}
Und was soll ich sagen, sie haben alle Recht. Das ist exakt das
Gebiet, mit dem sich dieser Zweig der Magiewirker besch�ftigt und
damit ist in der Vergangenheit so mancher Beschw"orer nicht zu
Unrecht auf dem Scheiterhaufen einer seelischen Grundreinigung
unterzogen worden.
\par Doch wir wollen nicht gleich alle auf einmal zum Teufel w"unschen.
Die Gilde der Beschw�rer teilt sich ebenso wie die Gilde der Magier
in verschiedene Ausrichtungen auf und die meisten besch"aftigen sich,
zum Wohle aller, mit der wei"sen Magie. Also die Beschw"orung der
Geister der Toten f"ur Befragungszwecke oder die Erschaffung von
"ubernat"urlichen Wesen f"ur Boteng"ange. Trotz allem bleiben
nat"urlich noch die Individuen, die der Schwarzmagie fr"onen und
diese gilt es zu meiden.
\subsubsection{Wesen}
\par Stumm und verschlossen. Dies sind wahrscheinlich die ersten
Eindr"ucke, die man von einem Beschw"orer bekommen wird und
wahrscheinlich auch meist die Einzigen. Selten wird er sich
ungefragt an einer Diskussion beteiligen und seine Meinung "au"sern.
Viel n"aher liegt es ihm, alles was gesagt wird, sich zu merken.
Spricht man ihn an und fragt ihn etwas, so antwortet er in kurzen,
aber pr"azisen S"atzen oder aber in einer so komplizierten
Formulierung, dass es besser w"are, man h"atte gar nicht gefragt,
denn hinterher versteht man weniger als vorher.
\par Ein Beschw"orer der sich einer Gruppe von Abenteurer
angeschlossen hat, wird ihnen gegen"uber ein wenig aufgeschlossener
sein. Er wei"s, dass es n"otig ist im Team zu arbeiten, damit er in
Notsituationen auf ihre Hilfe setzen kann. Geheimnisse wird er
nichts desto trotz eher f"ur sich behalten, die anderen wissen eh
nicht damit umzugehen.
\subsubsection{Ausbildungsweg}
\subsubsection{Vorz"uge / Nachteile}
\par Der Beschw"orer kann nur die Psi-Bereiche Elemente und
Beschw"oren erlernen. Beschw"orer sind im Umgang mit Elementarwesen
und D"amonen geschult und zeigen keine Furcht vor ihrem Erscheinen
und Auftreten. In Gegenwart dieser Wesen erhalten sie einen Bonus
auf Geistige Stabilit"at (+2).
\subsubsection{Besonderes}

\newpage
\subsection{Der Druide\index{D!Druide}}
\subsubsection{Beschreibung}
\par Verkannt, missverstanden und ignoriert. Besser kann man den Druiden wahrscheinlich nicht beschreiben.
\parpic[r]{\epsfig{file=pics/druide.eps, scale=1}}
Von allen anderen Magiebegabten wird der Druide eher mit einem L"acheln bedacht. Nur all zu gut kennen sie 
das Bild des Mistel schneidenden Alten. Dabei sollte man seine Macht nicht untersch"atzen. Die Druidenzirkel 
haben "uber die Jahrhunderte einen Fundus an historischem Wissen angesammelt, bei dessen Anblick, und dies ist 
nur bildlich gesprochen, denn Druiden geben ihr Wisssen nur m"undlich weiter und schreiben es nicht nieder, 
die Gelehrten erblassen w"urden.
\par Vom einfachen Volk wird der Druide meist als m"annliches Gegenst"uck der Hexe genannt. Mit diesem 
Vorurteil k"ampfen die Zirkel schon seit ihrer Entstehung.
\subsubsection{Wesen}
\subsubsection{Ausbildungsweg}
\par Die Druidenzirkel suchen ihre Sch"uler im einfachen Volk, alle anderen sind zu weit weg vom echten Leben.
 Meistens suchen sie Waisenh"auser auf und unterhalten sich dort mit den Kindern. Die Kinder, die sich als
  w"urdig und geeignet f"ur eine Ausbildung erweisen, werden von den Zirkeln adoptiert. Bei der Auswahl der
   Kinder wird darauf geachtet, dass sie nicht "uber 6 Jahren sind, ein gutes Verh"altnis zu Tieren und der 
   Natur zeigen und von einem ungestillten Wissensdurst getrieben werden. Die Ausbildung eines Druiden dauert 
   im Grunde ein Leben lang, w"urde jedenfalls der Druide sagen, fragt man ihn direkt, man hat nie zu Ende 
   gelernt. Grunds"atzlich gilt aber, dass die Grundausbildung ca. 8 jahre dauert. In dieser Zeit lernt der 
   Sch"uler alles notwendige Grundwissen in Naturkunde, Wissenschaft und Zauberei.
\subsubsection{Vorz"uge / Nachteile}
\subsubsection{Besonderes}

\newpage
\subsection{Der Hexer / Die Hexe\index{H!Hexe}\index{H!Hexer}}
\subsubsection{Beschreibung}
\par Hexe! Dieses Wort treibt der Menschheit die Panik in die
Augen. Wer an Hexe denkt an Fl"uche, wer Hexer h"ort sieht den
Schwarzmagier vor Augen.
\parpic[r]{\epsfig{file=pics/hexe.eps, scale=0.5}}
Kein Archetyp ist gef"urchteter und wahrscheinlich mehr gehasst als
dieser. Dabei sehen sich die Hexen verkannt. Ihre Absichten gleichen
sich in keiner Weise mit dem Bild, dass man von ihnen auf dem Banner
tr"agt.
\par Doch woran erkennt man Hexen? Junge Hexen sind grunds"atzlich
bildsch"on und alte Hexen haben eine krumme Nase. Naja, stimmt auch
nicht immer, aber kann will man schon gegen Vorurteile machen.
\subsubsection{Wesen}
\par Das Wesen einer Hexe ist so wandelbar wie das Schicksal dieses
Kontinents. Ihr Gem"ut ist von ihrer Tagesform abh"angig und meist
wird man die Wahrheit nicht erkennen, denn Hexen sind zu dem sehr
gute Schauspieler und verstehen es, ihre Mitmenschen zu
manipulieren. Entweder durch ihren Charme oder aber durch Magie.
\par Man sollte sich jedoch nie den Zorn einer Hexe auf sich ziehen,
denn dieser wird immer auf einen zur�ck kommen und das dann meist in
Form eines Fluches.
\subsubsection{Ausbildungsweg}
\par Hexen suchen sich keine Sch"uler. Das Wissen wird nur innerhalb
der Familien weiter gegeben. Dabei ist es jedoch nicht zwingend
notwendig, dass der Partner ebenfalls ein Hexer oder eine Hexe ist,
damit die Kinder die magische Veranlagung erben, ein Elternteil
reicht aus.
\par So ungef"ahr im Alter von 6 Jahren beginnt die Ausbildung der
jungen Sch"uler. Um das magische Potential der Kinder zu aktivieren,
bedarf es eines besonderen Rituales. Der Hexenerlternteil tritt mit
seinem Kind oder seinen Kindern, so es Zwillinge sind, eine Reise
zur allj"ahrlichen Hexennacht an. Diese findet immer zum
Jahreswechsel statt.
\subsubsection{Vorz"uge / Nachteile}
\par W"ahrend ihrer Ausbildung lernen die Hexen auf Besen zu fliegen
(+1). Eine genaue Beschreibung dieser Fertigkeit ist im Kapitel
\textit{Neue Fertigkeiten} (siehe S. \pageref{neuefertigkeiten}) zu
finden.
\subsubsection{Besonderes}
\par Die Hexe w"ahlt sich w"ahrend ihrer Ausbildung einen Vertrauten.
Meist einen Vogel oder eine Katze. Mit diesem kann sie auf
telepathischem Wege kommunizieren. Er versteht einfache Befehle und
antwortet in einfachen S"atzen.
\par Die Hexe kann Fluchmagie verwenden. Eine Beschreibung dieser
Magierichtung, sowie einiger Fl"uche findet sich im Kapitel
\textit{Fluchmagie} (siehe S. \pageref{fluchmagie})

\newpage
\subsection{Der Illusionist\index{I!Illusionist}}
\subsubsection{Beschreibung}
\parpic[r]{\epsfig{file=pics/illusionist.eps, scale=0.55}}
Die Magier k"onnten sich vor Lachen kaum halten, wenn der Illusionist in ihrer Gegenwart verlauten 
lassen w"urde, er sei ein Magier. Doch die Kraft der Illusionisten darf nicht untersch"atzt werden. 
Schon so manch lachender Magier ist an der pl�tzlichen F"ullung des Mundes mit einem nie enden 
wollenden Wasserstrahl erstickt. Nachweisen konnte man dem Illusionisten jedoch nichts, Fl"ussigkeit 
wurde keine gefunden.
\par Den wandernden Illusionisten trifft man entweder unter den Mitgliedern des fahrenden Volkes oder 
aber in einer Horde schatzsuchender Abenteurer.
\subsubsection{Wesen}
\par Auf seinen Reisen wird der Illusionist sich tarnen, um nicht sein wahres \glqq Ich\grqq{} preiszugeben. 
Denn ist ein Illusionist erst einmal als solcher erkannt, wird es ihm schwer fallen seine Gegen"uber zu verwirren.
\subsubsection{Ausbildungsweg}
\par Es gibt keine direkte Ausbildung f�r Illusionisten, irgendwann im Laufe ihrer Kindheit erkennen sie ihr 
Talent selbst.
\subsubsection{Vorz"uge / Nachteile}
Illusionisten sind resistenter gegen die Werke ihrer Kollegen und erhalten immer einen
Bonus von +2 auf Psiwiderstandsw�rfe gegen den Magiebereich Illusion.
\subsubsection{Besonderes}
\par Obwohl der Illusionist niemals auf seine Illusionen hereinfallen wird, so wird er sie doch immer sehen 
k"onnen. Er k"onnte ja auch schlecht mit ihnen arbeiten, w"are es nicht so. 

\newpage
\subsection{Der Schamane\index{S!Schamane}}
\subsubsection{Beschreibung}
\begin{quotation}
\par\textit{\glqq Bei dem Schamahn kann man sich streyten, ob man ihn nun zu den Magiern oder zu den Geweihten zelt. Er hat eigentlich von beidn etwas.}
\parpic[r]{\epsfig{file=pics/schamane.eps, scale=0.6}}
\textit{Typische Vertreter der Schamahn kommen aus den sogenannten
Naturvoelkern. Dabei handelt es sich um eine nette Umschreibung, die sich
die Gelehrten ausgedacht haben, um die Wildlebenden wie die Barbaren
und Orks zu umschreyben.}
\par\textit{Wenn die Meisten an einen Schamahn denken, dann sehen
sie einen in Fell gehuellten und mit Knochen behaengten
Eingeborenen, der in unverstentlichen Phrasen versucht die Geister
seiner Ahnen zu beschwoeren. Wahr oder unwahr sey hier mal
dahingestellt. Fakt ist jedoch, das der Schamahn ein Wesen mit
magischen Kreften ist, die in grossen Teilen sogar ueber das
Verstentnis der Schulmagie hinauswachsen.}
\par\textit{In seinem Stamm geniesst der Schamahn einen hohen Stand,
meist sogar hoeher als der Heupdling. Er uebt in Friedenszeyten die
Funktion des Heylers und des Beraters aus. In Zeyten des Kampfes
steht er nebn den Kriegern seines Stammes in der ersten Reye und
sterkt sie mit seinen Zaubern, wehrent er Flueche auf seine Feinde
wirft.\grqq} [\mbox{Lofen} \mbox{Dulsbart}, Knappe im Orden der
Sterne in einem Aufsatz ueber Schamanen]
\end{quotation}
\par Jeder kann f"ur sich selbst entscheiden, was er dem oben genannten Zitat als bare M"unze entgegen nimmt.
\subsubsection{Wesen}
\subsubsection{Ausbildungsweg}
\par Jedes Volk, das an die Geister der Natur glaubt, hat in seinem Stamm einen Schamanen. 
Er hat die Funktion eines Beraters und Weisen inne.
\par Im Laufe seines Lebens w"ahlt der Schamane einen Sch"uler aus den Reihen des Stammes, 
meistens einen Jungen. Dieser wird von ihm in die Geheimnisse der Geister der 
Natur\index{G!Geister der Natur}\index{N!Naturgeister} eingeweiht. Er lernt die Geschichte 
des Stammes und die Kunst des Heilens. Um seinen Geist und K"orper ganz der Natur hinzugeben, 
entsagt der Sch"uler allen k"orperlichen Verlangen und wird sich niemals eine Frau oder einen 
Mann f"ur eine Partnerschaft erw"ahlen. \par Nach 10 Jahren der Aubildung wird es f"ur den 
heranwachsenden Schamanen Zeit, in die Wildnis zu gehen und sich den Geistern der Natur 
vorzustellen. Er sucht einen stillen und abgelegenen Platz in der Steppe auf und begibt 
sich direkt nach Sonnenuntergang in einen meditativen Zustand. So nimmt er Kontakt mit den 
Geistern auf. Diese nehmen Notiz von dem Anw"arter und pr"ufen ihn und seinen Geist. Wird 
er f"ur gut befunden, dann gelangt auf eine spirituell h"ohere Stufe und ist nun bereit den 
Weg des Schamanen zu gehen, wird er abgelehnt, zerst"oren sie seinen Geist und er wird wirr 
oder stirbt sofort. In diesem Fall ist er f"ur den Stamm verloren.
\par Kehrt er als Schamane zu seinem Stamm zur"uck, wird er nun in die tiefen Geheimnisse des 
Schamanismus eingeweiht.
\par In ganz seltenen F"allen wird ein abgelehnter Schamane von den D"amonen\index{D!Daemon} 
der Wildniss aufgenommen. Diese geben ihm seinen Geist zur"uck und vergiften ihn mit ihrer Lehre. 
Solch einen Schamamen kennt man unter dem Begriff \textit{Tza'Sin}\index{T!Tza'Sin} 
(vom Teufel bekehrt). Er kehrt ebenfalls nicht zu seinem Stamm zur"uck.
\subsubsection{Vorz"uge / Nachteile}
\subsubsection{Besonderes}

\newpage
\subsection{Der Seher\index{S!Seher} / Wahrsager\index{W!Wahrsager}}
\subsubsection{Beschreibung}
\par Gestraft oder beg"unstigt? "Uber die F"ahigkeiten des Sehers l"asst sich streiten. W"ahrend der Gro"steil 
der Bev"olkerung die Kraft des Sehers eher als eine Gabe sieht, ist so mancher Seher an eben dieser zu Grunde 
oder in den Wahnsinn gegangen.
\parpic[r]{\epsfig{file=pics/seher.eps, scale=0.6}}
Die jenigen, die mit der Gabe des zweiten Gesichtes beschenkt wurden, leiden meist unter unkontrolliert
 auftretenden Visionen "uber Dinge, die gerade an einem anderen Ort geschehen, geschehen werden oder aber 
 auch geschehen sind. Die Gilde der Seher hat es sich zur Aufgabe gemacht, diese Gabe zu erforschen und es 
 dem Sch"uler zu erm"oglichen, sie zu kontrollieren.
\par Damit sind Seher gern gesehene G"aste in den H"ausern der Reichen, so lange sie nur Gutes zu erz"ahlen 
haben, sie sind jedoch ebenso gef"urchtet, decken sie doch Geheimnisse aus der Dunkelheit auf. Das wiederum
 macht sie zum Lieblingsgast des "ortlichen Sherrifs. Schon so manche als unaufzukl"arend abgelegte Tat, 
 wurde doch noch entschl"usselt und der T"ater seiner gerechten Strafe zugef"uhrt. Diese Tatsache f"uhrt 
jedoch dazu, dass der eine oder andere Seher auch schon mal unschuldig Opfer eines Attentates wurde, 
vorbeugend sozusagen.
\subsubsection{Wesen}
\subsubsection{Ausbildungsweg}
\par Wie bereits oben beschrieben wurde, ist die Gilde der Seher auf der Suche nach neuen Sch"ulern, 
um ihre Kr"afte in kontrollierte Bahnen zu lenken. Dabei ist es besonders wichtig, dass die Sch"uler 
schon in fr"uhen Jahren entdeckt werden, damit ihr Geist nicht schon zu sehr unter den Visionen gelitten hat.
\subsubsection{Vorz"uge / Nachteile}
\subsubsection{Besonderes}
\par Der Seher geh�rt zu den wenigen Begabten, die den magiebereich Dimension\index{D!Dimension} erlernen k�nnen.


\newpage
\section{Geweihte und Geistliche}
\subsection{Der Heiler\index{H!Heiler}}
\subsubsection{Beschreibung}
\par Heiler sind vielerorts gerne gesehen. Sie heilen Wunden, kurieren Krankheiten oder versorgen Br"uche und was sonst noch so alles an Blessuren auftreten kann. Anders als die Bediensteten der Tempel und G"otter nutzen die Heiler keine "ubernat"urlichen F"ahigkeiten. Sie greifen haupts"achlich auf ein gro"ses Fachwissen, welches Jahrhunderte innerhalb der Gilde weitergegeben wurde, zur"uck. Das hei"st allerdings nicht, dass ein Heiler es verschm"aht, einem Verwundeten einen alchimistischen Heiltrank einzufl"o"sen oder in besonders schwerwiegenden F"allen die Hilfe von Klerikern in Anspruch zu nehmen. Heiler sind frei von Vorurteilen und nehmen anders die Kleriker gerne die Hilfe anderer in Anspruch.
\subsubsection{Wesen}
\subsubsection{Ausbildungsweg}
\subsubsection{Vorz"uge / Nachteile}
\subsubsection{Besonderes}

\newpage
\subsection{Der Priester\index{P!Priester} / Kleriker\index{K!Kleriker}}
\subsubsection{Beschreibung}
\parpic[r]{\epsfig{file=pics/kleriker.eps, scale=0.55}}
Der Kleriker ist der Diener der G"otter. Schon seit fr"uher Kindheit pflegt er das Studium der heiligen Schriften
 und die Beziehung zwischen Kirche, Staat und der gemeinen Bev"olkerung. Durch ihre Bindung zu den G"ottern und ihre 
 F"ahigkeiten Wunder zu wirken, genie"sen sie ein hohes Ansehen beim Volk.
\subsubsection{Wesen}
\subsubsection{Ausbildungsweg}
\par Die Ausbildung zu einem Kleriker dauert lang und beginnt bereits in den fr�hen Kindesjahren.
\subsubsection{Vorz"uge / Nachteile}
\subsubsection{Besonderes}
\par Wie die Paladine auch, k�nnen Priester ihre G�tter um Wunder bitten. Neben den einfachen Wundern der,
werden den Priestern jedoch auch gro�e und manchmal auch au�ergew�hnliche Wunder gew�hrt. Man hat schon Kleriker
gesehen, die wandelnden Schrittes auf der Oberfl�che eines Flusses die Uferseiten gewechselt haben oder auch 
Gl�ubige, die lebend aus einem zusammengebrochenen Stollen zur�ckgekehrt sind.

\newpage
\subsection{Der M�nch\index{M!M�nch} / Scholar\index{S!Scholar}}
\subsubsection{Beschreibung}
\par M"onche leben zur"uckgezogen in ihren Klosterfestungen und gehen stiller Meditation nach. Nicht selten haben sie auch ein Schweigegel"ubte abgelegt. Sie studieren dort die Schriften vergangener Zeiten, halten Neues fest oder haben sich der Forschung auf wissentschaftlichen Gebieten verschrieben.
\parpic[r]{\epsfig{file=pics/moench.eps, scale=0.6}}
Wenn man einen M"onch auf Reisen trifft, dann hat seine Wanderung immer einen Auftrag des Klosters als Hintergrund. Auffinden von Wissen, Erforschung, Bergung von Artefakten oder "ahnliches. W"unscht es der Suchenden, so wird er f"ur die Dauer seines Auftrages vom Schweigegel"ubte befreit, die Kommunikation f"allt so einfacher.
\par Die verschiedenen Kl"oster haben unterschiedliche Glaubensausrichtungen und eigentlich gibt es zu jedem Glauben auch mindestens eine Gruppe von M"onchen, die sich ihm verschrieben hat.
\subsubsection{Wesen}
\par Im Wesen eher ruhig und im Geiste stark. Durch sein abgeschiedenes Studium und die Meditation ist der M"onch ein eher ruhiger Zeitgenosse. Auch wenn er f"ur die Dauer seiner Reise vom Schweigegel"ubte befreit ist, so wird er dennoch keine unn"otigen Energien in "uberfl"ussige Konversationen verschwenden, sondern statt dessen ruhig in sich gehen. Das soll nicht hei"sen, dass er sich gar nicht unterh"alt. Gerade wenn es wissenschaftliche Diskussionen geht, wird er sich angeregt beteiligen und sein Wissen mit den anderen teilen. Geht es darum, ein R"atsel zu l"osen, wird er ebenfalls versuchen mit seinen Kenntnissen zu helfen.
\par M"onche sind in sich zur"uckgezogen, Angeberei und Prahlerei sind ihnen fern. Sie lassen sich nicht von Herausforderungen locken und werden versuchen Konfliktsituationen friedlich zu l"osen. Kommt es dennoch zu einem Kampf, so werden sie keine unn"otigen Energien in weitere Warnungen verschwenden und den Kampf versuchen so schnell wie m"oglich und so effektiv wie m"oglich zu beenden, dann jedoch auch ohne R"ucksicht auf die Gesundheit des Gegeners. Vom T"oten versuchen sie jedoch Abstand zu halten.
\subsubsection{Ausbildungsweg}
\par W"ahrend seiner Ausbildung lernt der M"onch nicht nur eine Reihe an Wissensfertigkeiten, sondern auch den Umgang im waffenlosen Kampf. Die Handhabung einer Waffe verweigern sie grunds"atzlich, egal ob f"ur den Nah- oder Fernkampf geeignet.
\subsubsection{Vorz"uge / Nachteile}
\par Neben den "ublichen waffenlosen Kampfarten (Boxen und Ringen) beherrschen die M"onche noch eine Reihe weiterer waffenloser Kampftechniken (+1). Eine genaue Beschreibung dazu ist im Kapitel \textit{Neue Fertigkeiten} (siehe S. \pageref{neuefertigkeiten}) zu finden.
\par Auf Grund ihrer Einstellung zu Waffen, ist es den M"onchen nicht m"oglich Waffen irgendeiner Kategorie zu f"uhren oder auch nur eine Fertigkeit aus diesem Bereich zu steigern. Ausnahme dazu sind die waffenlosen Kampftechniken (siehe oben).
\subsubsection{Besonderes}
\par M"onche sind Meister der K"operbeherrschung. Diese spiegelt sich in ihrem Chi Wert wieder. Die Handhabung dieses Wertes und seine Auswirkungen werden im Kapitel \textit{Neue Fertigkeiten} (siehe S. \pageref{neuefertigkeiten}) beschrieben.

                        \chapter{Die Rassen\index{R!Rassen}}
\label{dierassen}
\begin{quotation}
\par\textit{\glqq Es nicht nur die Klasse, die einen Menschen ausmacht.\grqq}
\end{quotation}
\section{Einleitung}
\par So oder so "ahnlich, oder aber auch gar nicht kann man es auch auf das Rollenspiel umsetzen. Der Abenteurer will nicht nur Dieb, Krieger oder einfach nur Elf sein. Die Vielfalt bestimmt mehr. Es gibt auch Elfen, die Krieger sind. Es gibt, ja wirklich, Zwerge, die Magie anwenden und und und ...
\par Der folgende Abschnitt stellt alle dem Spieler zur Verf"ugung stehenden Rassen vor. In dem sp"ateren Kapitel \textit{Das Bestiarium} (ab S. \pageref{bestiarium}) stelle ich noch einige Rassen vor, die nur dem Spielleiter vorbehalten sind.
\par Jede Klassenbeschreibung\index{K!Klassenbeschreibung} folgt der unten aufgelisteten Struktur.
\begin{description}
\item[Beschreibung:]Umfassende Beschreibung der Rasse
\item[Herkunft:]Manche Rassen sind normalerweise nicht "uberall auf dem Welt zu treffen, deswegen befindet sich hier eine Beschreibung "uber die "ortliche Herkunft sowie des Lebensraums.
\item[Erscheinungsbild:]"Au"serliche Erscheinungsmerkmale wie Augen-, Haar- und Hautfarbe, Gr"o"se, Gewicht, K"orperform und bevorzugte Kleidung; andere Merkmale wie Alter und in dem Zusammenhang Vollj"ahrigkeit etc.
\item[Wesen:]rollenspieltechnische Hinweise f"ur den Spieler; Beschreibung der allgemeinen Wesensz"uge\index{W!Wesensz"uge}, St"arken\index{S!St"arken} und Schw"achen\index{S!Schw"achen}
\item[Besonderes:]Hier wird etwas stehen, falls eine Rasse Besonderheiten, wie \mbox{z.B.} besondere F"ahigkeiten besitzt oder "uber spezielle Eigenarten verf"ugt. Hier wird ebenfalls stehen, welche Sprache die Muttersprache ist. Spieltechnische Auswirkungen sind ebenfalls hier zu finden.
\end{description}

\newpage
\section{Der Elf\index{E!Elf}}
\subsubsection{Beschreibung}
\par Hochgewachsen, stolz, allwissend und unfehlbar.
Fragt man einen Elfen sind das die Attributen mit denen er sich und
seine Artgenossen beschreiben wird. Zumindest die ersten beiden
Punkte kann man undiskutiert als Tatsache niederschreiben. Doch ein
Elf wird noch mit einer ganzen Reihe anderer Eigenschaften
verbunden, einige davon wird er z�hneknirschend akzeptieren und
gegen andere wird er sich mit H�nden und F��en zur Wehr setzen.
Die Wahrheit steht wie immer zwischen den Zeilen und manchmal auch
im \textit{Dunklen}.
\subsubsection{Herkunft}
\par Die Elfen des Landes geh�ren einem von zwei V�lkern an, den Hochland- oder den Waldelfen.
Innerhalb dieser unterteilen sie sich in verschiedene Clans, die an
verschiedenen Orten dieser Welt leben. Elfen sind jedoch so gut wie
nie in ihren Clans anzutreffen. Jeder Clan setzt sich aus
verschiedenen Sippen\footnote{Sippe=Familie} zusammen, in denen die
Elfen leben.
\par Jedem Clan steht ein H�uptling vor und in jedem Clan gibt
es mindestens einen Schamanen. Bei dem H�uptling handelt es sich
eher weniger um den allein herrschenden Monarchen, als um einen
weisen Ratgeber. Er ist der Anlaufpunkt f�r alle Wissbegierigen und
ein Ratgeber f�r schwierige Fragen. Meist ist der H�uptling der
Clan�lteste. Die Schamanen des Clans bewahren die Chroniken auf. In
ihnen wird die Geschichte des Clans und des Kontinents weiter
geschrieben.
\subsubsection{Erscheinungsbild}
\par Wo immer ein Elf auftritt sorgt er f�r Aufsehen. Selten
werden diese Gesch�pfe in belebten Gegenden gesehen. Der Elf misst
im Schnitt 2 Meter und ist von schlankem K�rperbau. Seine Kraft
sollte man jedoch nicht untersch�tzen. Elfen sind stolze und
gef�hrliche Krieger. Ihre Haare werden sehr of lang getragen und es
treten die �blichen Haar- und Augenfarben auf. Ihre Haut
ist meist blass.
\par Ihre bevorzugte Kleidung ist Leder oder grober Stoff,
meist in den Farben des Waldes. Alles was sie bei sich tragen ist
einem Zweck gewidmet. Der Elf hasst nichts mehr, als �berfl�ssigen
Balast. Eine Ausnahme dieser Regel stellen Kunstwerke dar.
\par Elfen, die sich innerhalb ihrer Sippe der Jagd gewidmet haben, 
treten oft in Begleitung eines Jagdtieres auf. Dabei w�hlen sie genauso 
oft den Hund oder den Falken wie den Puma oder Bergl�wen.
\subsubsection{Wesen}
\par Kunst ist ihre Leidenschaft und auch ihr Laster. F�r gute
Kunst geben die Elfen alles. Dabei sind sie selbst auch sehr
geschickte K�nstler. Es spielt keine Rolle, ob es sich um Musik,
Bildhauerei oder die Kunst der Malerei handelt.
\par Der Elf f�r sich genommen ist im Herzen ein Einzelg�nger.
Zusammenschl�sse findet man nur innerhalb der Sippe oder in
Zweckgemeinschaften. Der Elf ist von tiefem Stolz �ber seine
Herkunft erf�llt und legt ein gro�es Ehrbewusstsein an den Tag.
Auf das Wort eines Elfen ist Verlass, auch wenn es gro�e Worte
sind.
\par Da die Elfen zu den am l�ngsten auf diesem Kontinent lebenden
V�lkern geh�ren, verf�gen sie �ber gro�es Wissen der
Geschichte, welches von ihnen zugleich wie ein Geheimnis geh�tet
wird. So bewahren sie das Wissen f�r sich. Auch bei anderen
Anl�ssen versucht der Elf nur so viel Informationen wie n�tig
Preis zu geben.
\subsubsection{Besonderes}
\par Obgleich ihnen viele magische F�higkeiten angedichtet werden,
sind Elfen auch nur \glqq normale Menschen\grqq{}. Sie verf�gen �ber
keine au�ergew�hnlichen magischen F�higkeiten, obwohl es durchaus
auch Magier oder Druiden unter ihnen gibt. Elfen verf�gen jedoch
�ber hervorragend ausgebildete Sinne. Insbesondere ihre Intuition
hat ihnen in so mancher Lage die richtige Entscheidung beschert.
Elfen haben eine erh�hte Geschicklichkeit (+1) und d�rfen 5 Punkte
bei den Sinnen verschieben.
\par Elfen erreichen im Alter von 30 Jahren die Vollj�hrigkeit und
damit das Recht, die Sippe zu verlassen. Sie haben eine sehr hohe
Lebenserwartung. Der �lteste bekannte noch lebende Elf z�hlt
wahrscheinlich �ber 600 Lebensjahre, sein genaues Alter ist jedoch
nicht bekannt, schweigt er sich doch dar�ber aus.
\subsubsection{Die dunkle Seite des Elfen}
\par Trotz der Sch�hnheit und Grazie, die ein Elf reflektiert,
verbirgt sich hinter der Fassade doch ein dunkles Geheimnis. Ein
schweres Los wurde dem hochgewachsenen Volk auferlegt. Niemand au�er
den Stammes�ltesten der Elfen wei� �ber die Herkunft dieses
Schicksals Bescheid.
\par Elfen sind ein diszipliniertes Volk. Sie h�ten ihre
Gef�hlsregungen und halten auf diese Weise ihre dunkle Seite in
Schach. Elfen die gro�es Leid oder gro�e Schmerzen erfahren, werden
von ihrer schwarzen Seele\index{S!Schwarze Seele} �bermannt. Jeder
Elf tr�gt diese Seele in sich, die den Sagen nach ein �berbleibsel
der Sklaverei ist. Jeder Elf, der auf diese Art und Weise die
Kontrolle �ber sich verliert, st�rzt sich, ohne R�cksicht auf
sich und seine Umwelt, auf den Leidensbringer und h�lt erst inne,
wenn Vergeltung erreicht wurde. Im Falle der Ausl�schung der ganzen
Sippe kann dies auch ohne weiteres ein dauerhafter Zustand werden.
Je l�nger die schwarze Seele ausgelebt wird, umso schwieriger ist
die R�ckkehr f�r den Elfen.
\par Ein Elf der dieser Raserei verf�llt, ver�ndert sich auch
�u�serlich. Die Gesichtsz�ge werden harte und schmerzerf�llte
Mimiken. Eine spielrelevante Beschreibung und deren Auswirkungen
befindet sich im Kapitel \textit{Die Elfen} (siehe S.
\pageref{dieelfen}).
\par Unter den Elfen gibt es einen Zusammenschlu� von Individuen,
die sich nicht wie die "Ubrigen vor ihrer dunlen Seite verstecken,
sondern sich aktiv versuchen dagegen zu wehren. Sich sind auf der
Suche nach dem Ursprung und der Heilung gegen diese Krankheit und
unter der Bezeichnung \textit{Harlekine}\index{H!Harlekine} (siehe
S. \pageref{harlekine}) bekannt.


\newpage
\section{Der Halbling\index{H!Halbling, der}}
\subsubsection{Beschreibung}
\par \textit{\glqq Von der Hand in der Tasche.\grqq} So oder
"ahnlich wird wohl das Lebensmotto der Halblinge lauten. Flink und
diebisch sind die 2 Worte, die jedem sofort einfallen, wenn man an
Halblinge denkt. Ausgenommen nat�rlich Halblinge. Sie sehen sich
eher als ein Volk voller missverstandener kleiner liebenswerter
Wesen, die einfach nur ernst genommen werden wollen. Und wer sehr
genau hinschaut, wenn sie dies sagen, wird ein leichtes L"acheln in
den Mundwinkeln erkennen.
\subsubsection{Herkunft}
\par Wo die Halblinge herkamen, wei"s eigentlich keiner mehr so
genau. Aber wahrscheinlicher ist, dass sie schon immer da waren,
auch wenn sich niemand daran erinnern kann, jemals Halblinge
w"ahrend der Herrschaft der Finsternis\index{H!Herrschafft der
Finsternis} in Gefangeschaft gesehen zu haben.
\subsubsection{Erscheinungsbild}
\par Halblinge sind zwischen einem halben und einem Meter gro"s,
Ihre Statur reicht von gertenschlank bis gedrungen. Alles in allem
ist ihre Artenvielfalt "ahnlich gro"s, wie die der Menschen.
Halblinge haben meist dunkle Haar- und Augenfarben. Selten sieht man
einen blonden Halbling. Ungew"ohnlich hoch ist der Anteil von
Albinos innerhalb des Volkes.
\par Halblinge erreichen sehr fr"uh das Erwachsenenalter, die meisten
verlassen bereits mit 10 Jahren das elterliche Haus und ziehen in
die Welt. Ihre Lebenserwartung reicht im Schnitt bis hin zu 40
Jahren.
\subsubsection{Wesen}
\par Fr"ohlich, unbedarft und von kindlichem Gem"ut; Dies sind
sind die wesentlichen Z"uge eines Halblings. Wer ihnen jedoch keine
Ernsthaftigkeit nachsagt, hat weit gefehlt. Halblinge wissen den
Ernst einer Lage sehr wohl einzusch"atzen, auch wenn sie versuchen
ihn zu "uberspielen. Sollte ein Halbling es jedoch in einer
Situation f"ur sinnvoll erachten, wird er ihr den n"otigen Respekt
entgegenbringen.
\par Alles in allem leben Halblinge jedoch auf der sonnigen Seite
des Lebens. Gewalt ist ihnen ein Greuel und nur in wirklicher Not
wird von den Waffen Gebrauch gemacht, die sie gerne als
Schmuckst"ucke und Zierde am K"orper tragen.
\subsubsection{Besonderes}
\par Halblinge sind "uberaus geschickte Wesen. Eine Eigenschaft
die man einigen ob ihres K"orperbaus nicht unbedingt ansehen wird.
Die meisten Halblinge gehen durchaus b"urgerlichen Berufen nach und
verdienen sich als Feinmechaniker oder Goldschmied ihren
Lebensunterhalt.


\newpage
\section{Der Mensch\index{M!Mensch, der}}
\subsubsection{Beschreibung}
\par Die Menschen stellen den gr"o"sten Teil der Bev"olkerung
dieses Kontinents. Sie sind "uberall und in den unterschiedlichsten
Formen und Entwicklungsstadien anzutreffen. Angefangen beim
Ureinwohner in den Nordlanden bis hin zu den Kaufleuten im Reich der
Mitte. Welche davon in ihrer Entwicklung weiter sind "uberlasse ich
dem Leser zu beurteilen.
\subsubsection{Herkunft}
\subsubsection{Erscheinungsbild}
\par Ein ausgewachsener Mensch liegt in der Regel zwischen 160cm und
190cm. Regionale Unterschiede k"onnen auch Extrema hervorbringen.
Unter den Seefahrer des S"udens hat man auch schon Individuen
gesehen, die die 2 Meter Marke locker "uberschreiten, w"ahrend die
Bewohner der westlichen W"usten eher kleiner sind.
\par Es sind alle Naturfarben als Augen- und Haarfarbe vertreten.
\par Mit 16 Jahren gilt ein Mensch als erwachsen, hier gibt es jedoch
auch regionale Unterschiede, wo besonders in den Naturv"olkern ein
niedrigeres Alter angesetzt wird. Die durchschnittliche
Lebenserwartung liegt bei 70 Jahren.
\subsubsection{Wesen}
\subsubsection{Besonderes}


\newpage
\section{Der Troll\index{T!Troll, der}}
\subsubsection{Beschreibung}
\par Die Trolle sind magische Wesen von zierlicher Statur.
B"osartige Zungen nennen Vergleiche mit den Kobolden\index{K!Kobold,
der}, doch im Gegensatz zu ihnen ziehen die Trolle es vor, anderen
Wesen nicht zu schaden. Sie sind friedliche Wesen. Von Magie
erschaffen, von Magie getrieben ist der Troll eine mystische
Kreatur, die noch nicht lange auf dem Kontinent bekannt ist.
\subsubsection{Herkunft}
\par Eigentlich wei"s keiner so genau von wo die Trolle
kommen oder wohin sie gehen. Es gibt keine Informationen "uber ihre
Lebensgewohnheiten oder sonst irgend etwas.
\par Auch ist wenig "uber ihr soziales Verhalten in den
B"ucher niedergeschrieben. Was man wei"s ist, dass die Trolle nicht
zwischen Mann und Frau unterscheiden. Da sie durch und durch von
magischer Natur sind, vermehren sie sich auch nicht auf dem
herk�mmlichen Wege.
\par Trolle werden magisch geboren, sozusagen auf unbekanntem Wege
in die Welt beschworen.
\subsubsection{Erscheinungsbild}
\par Die gr��ten Trolle, die man gesehen hat, waren
bis zu einem halben Meter hoch. Da Trolle es jedoch meist vorziehen
unsichtbar zu reisen, ist es einsehbar, dass die Menschheit noch
l�ngst nicht alle Trolle gesehen hat. Im Durchschnitt misst ein
ausgewachsener Troll 30 bis 40 cm. Diejenigen, die schon einmal
einen Troll zu Gesicht bekommen haben, berichten von Ihrer blassen,
leicht bl"aulichen Hautfarbe und dem kleinen Paar H"orner, welches
ihr Haupt ziert. Trolle haben meist sehr dunkle bis schwarze Augen
und keine K"orperbehaarung.
\par "Uber die Lebenserwartung eines Trolles gibt es keine
Angaben, ebenso wenig "uber ein Erwachsenenalter. Trolle sind seit
dem Augenblick ihrer \glqq Geburt\grqq{} voll ausgewachsen.
\subsubsection{Wesen}
\par Trolle sind im Wesen mysteri"ose und seltsame Kreaturen,
sie sind von Magie durchflossen und werden von ihr ebenso bestimmt,
wie sie es verstehen sie zu manipulieren. Diese F�higkeit hat ihnen
einen gro�en Forscherdrang verliehen. Sie geben sich meist nicht mit
oberfl�chlichen Erkl�rungen zufrieden, sondern werden
Ungereimtheiten zielstrebig auf den Grund gehen.
\par Der Umstand, dass Trolle als erwachsene Lebewesen in die Welt
geboren werden, l�sst ihr Leben in einem anderen Bild erscheinen,
als das der normal Sterblichen. Sie sind vom ersten Augenblick ihres
Lebens an erwachsen und komplett entwickelt. Ihre enorme
Lernf�higkeit erlaubt es ihnen sich in wesentlich k�rzerer Zeit
Wissen anzueignen und so Ausbildungen schneller abzuschlie�en. Da
Trolle noch nicht viel von der Welt gesehen haben, kann man ihnen
einen gewissen Hang zur Neugier nicht absprechen. Ihre Aktionen sind
jedoch immer von Bedacht gedeckt, ein Troll w�rde sein junges Leben
nicht unbegr�ndet in Gefahr bringen.
\par Andere Rassen treten den Trollen mit einer Art gesundem Respekt
gegen�ber. Auf der einen Seite birgt ein Troll das Unbekannte, so
selten werden sie doch gesehen, so gerne w�rde man mehr �ber sie
erfahren. Auf der anderen Seite steht das Misstrauen gegen sie, zu
sonderbar ist ihr Auftreten und Verhalten, zu unbekannt ihre
F�higkeiten.
\par Trolle werden niemals einen kriegerischen Beruf aus"uben.
\subsubsection{Besonderes}
\par Aufgrund ihrer Natur sind Trolle nur f�r eine kleine Reihe von
Berufen geeignet.
\par Trolle haben magische F"ahigkeiten\index{T!Trollmagie},
jedoch auf ihre Art und Weise, die sich vollst"andig von denen der
anderen Magier abhebt. Es ist keine Spruchmagie und sie bedienen
sich auch nicht der \glqq Wahren Magie\grqq. Ihre Magie ist anders,
sie kommt von innen und scheint so etwas wie eine F"ahigkeit zu
sein, die sie nicht zu lernen brauchen. Trolle haben au"serdem die
au"sergew"ohnliche F"ahigkeit sich unsichtbar zu machen. Nicht
selten hat ihnen dies ihr kostbares Leben gerettet. Spieltechnisch
bedeutet dies, dass jeder Troll den Zauberspruch Unsichtbarkeit bereits
auf 50\% erlernt hat.
\par Bei der Berechnung der Tragkraft (TRK) wird
ein Maximalwert von St"arke 10 ber"ucksichtigt, jeder Punkt"uber 10
verf"allt bei der Berechnung. Da die Trolle "uber eine Art 6. Sinn
verf"ugen, erhalten sie eine erh"ohte Intuition (+1).

\newpage
\section{Der Zwerg\index{Z!Zwerg}} \label{zwerge}
\subsubsection{Beschreibung}
\par Die Zwerge sind ein starkes Volk von Kriegern. Wenn alle Zwerge etwas gemeinsam haben, dann ist das ihre Liebe zum Kampf und dem guten Zwergenbr"au. Die Zwerge sind gesellige und laute Zeitgenossen, die wissen wie Feste zu feiern sind, und dieses tun sie in Friedenzeiten zu jedem passenden Anla"s.
\par Mal abgesehen von den Elfen, denen die meisten Zwerge einfach zu \glqq lebhaft\grqq{} sind, halten die anderen Rassen die Zwerge f"ur freundlich und ehrbar.
\par Eine genaue Beschreibung der einzelnen Clans ist unter \textit{Die Clans der Zwerge} (siehe S. \pageref{diezwerge}) zu finden.
\subsubsection{Herkunft}
\par Die Sieben Clans der Zwerge sind "uber die Welt der Drachend"ammerung verteilt. Die Zwerge des Sturmhammer\index{S!Sturmhammerclan}-, Flammenaxt\index{F!Flammenaxtclan}-, Kristallklingen\index{K!Kristallklingenclan}- und Eisenfaustclans\index{E!Eisenfaustclan} leben in den gro"sen Gebirgsketten der Welt. Der Schattenwolfclan\index{S!Schattenwolfclan} lebt in einem riesigen Wald des Landes und die Zwerge des Sch"adelsammlerclans\index{S!Schaedelsammlerclan} f"uhren ein eher nomadisches S"oldnerdasein.
\subsubsection{Erscheinungsbild}
\begin{quotation}
\par \textit{Zwerge sind klein, tragen rote spitz zulaufende M"utzen, einen Spaten und grinsen stets...irgendwie habe ich mir die Zwerge dann doch folgendermassen vorgestellt:}
\end{quotation}
\par Auch wenn sich das Erscheinungsbild zwischen den einzelnen Clans mehr oder weniger unterscheidet, so kann der gemeine Zwerg recht einfach beschrieben werden. Der gemeine Zwerg ist zwischen 90 und 140 cm gro"s und wiegt bis zu 160 kg. Das enorme Gewicht liegt zum einen an der ausgepr"agten Muskelstruktur sowie an dem ber"uhmten Zwergenbr"au. Mal abgesehen von blondem Haar sind alle Farben vertreten und die Augen sind generell dunkel. Die Zwerge des Hochlandes sind von eher blasser, die des Tieflandes von dunklerer Hautfarbe.
\par W"ahrend die Hochlandzwerge\index{H!Hochlandzwer} dem Betrachter in ihrer gut geschnitteten Tuch- und Lederkleidung eher gesittet erscheinen, so sind die Tieflandzwerge\index{T!Tieflandzwerg} von eher wilder Erscheinung. Sie tragen T"atowierungen zur Schau, welche ihre Arme und oftmals kahlgeschorenen K"opfe zieren. Dazu tragen sie eher Felle und schm"ucken sich mit vielerlei Troph"aen; barbarisch sagen manche!
\par Zwerge sind mit 40 Jahren vollj�hrig und k�nnen den Clan auf eigene Faust verlassen oder eigenst�ndig Berufe aus�ben. Die durchschnittliche Lebenserwartung liegt bei 250-300 Jahren
\subsubsection{Wesen}
\par Der gemeine Zwerg ist generell aufbrausend und laut und wird eine Ungerechtigkeit ihm gegen"uber lauthals anmahnen. Wie ich bereits erw"ahnte feiern Zwerge gerne und sind einem guten Schluck Br"au nie abgeneigt.
\par Zwerge sind stolz auf das was sie tuen. Ihre Schmiedkunst steht dabei im Vordergrund, denn kein anderes Volk verarbeitet Stahl in entsprechender Qualit"at.
\par Wenn man die Freundschaft eines Zwergen f"ur sich gewonnen hat, so h"alt diese ein Leben lang. Dummerweise gilt dies auch f"ur eine Feindschaft und ich versichere, da"s der entsprechende Zwerg keine Gnade walten lassen wird!
\subsubsection{Besonderes}
\par Die hohe Ausdauer, Z"ahigkeit und St"arke der Zwerge sind schon besondere Attribute (Konstitution + 1). Doch etwas anderes m"ochte ich an dieser Stelle nennen:
\par Die Zwerge verf"ugen auf Grund ihrer Natur "uber die F"ahigkeit aus jeder H"ohle, die sie betreten, wieder heraus zu finden, egal wie weitl"aufig diese ist. Au"serdem verf"ugen Sie "uber erstaunliche Sicht in fast v"olliger Dunkelheit. Genauere Informationen zu diesen F"ahigkeiten im Kapitel \textit{Neue Fertigkeiten} (siehe S. \pageref{neuefertigkeiten}).
\par Neben den besonderen k"orperlichen Attributen sind es die Zwerge, welche die Runenmagie\index{R!Runenmagie} beherrschen. Die Thaumaturgen\index{T!Thaumaturg} der Zwerge besitzen die F"ahigeit Gegenst"ande durch anbringen magischer Symbole (Runen) mit zum Teil erstaunlichen Eigenschaften auszustatten. Damit nicht genug, beherrschen die Thaumaturgen die F"ahigkeit Runen in Form von T"atowierungen auf die Haut einer Person aufzutragen, welche ebenfalls magische Eigenschaften bei dem Tr"ager hervorrufen.
\par Die Art und Wirkungsweise der toten und lebenden Runen ist
unterschiedlich. Eines haben die Runen\index{R!Runen} allerdings
gemeinsam: die Herstellung ist auf Grund der Substanzen zum Teil
sehr teuer.
\par Zwei Beispiele hierzu:
\begin{description}
\item [tote Rune\index{R!Rune, tote}] Rune der Flamme und Rune des Wortes angebracht an einer Waffe\\
Die Waffe erh"alt auf Befehl des Tr"agers (z.B. Flamme) eine brennende Klinge, die zus"atzlichen Schaden verursacht
\item [lebende Rune\index{R!Rune, lebende}] Rune der Kampfeslust auf dem Arm des Zwerges\\
Der Zwerg erh"alt einen Bonus auf den Angriff einer mit diesem Arm gef"uhrten Waffe, die Rune wird in einer Farbe entsprechend der T"atowierung erstrahlen
\end{description}


\newpage
\section{Einschr�nkungen der Charakterklassen}
\par Nicht alle Rassen k"onnen jeder Berufung folgen.
Alle nicht-m�glichen Kombinationen sind der unten stehenden Tabelle
zu entnehmen.
\begin{longtable}{r|c|c|c|c|c}
            & Elf & Halbling & Mensch & Troll & Zwerg \\
\hline
\hline
            &   &   &   &   &   \\
Barde       &   &   &   &   &   \\
Beschw�rer  & x & X &   & X & x \\
Dieb        &   &   &   &   &   \\
Druide      &   &   &   & x &   \\
Heiler      &   &   &   &   &   \\
Hexe        &   &   &   & x &   \\
Illusionist &   &   &   &   &   \\
Krieger     &   & x &   & x &   \\
Kleriker    &   & x &   & x & x \\
Magier      &   & x &   &   &   \\
M"onch      &   &   &   & x &   \\
Paladin     & x & x &   & x &   \\
%Prediger    & x & x &   & x & x \\
Schamane    &   &   &   & x & x \\
Seher       &   &   &   &   &   \\
Streuner    &   &   &   &   &   \\
Tiermeister &   &   &   & x &   \\
Waldl�ufer  &   &   &   & x &   \\
            &   &   &   &   &   \\
\caption{Rassenbeschr�nkungen f�r Charakterklassen}
\label{tabelle_rassenbeschraenkungen}
\end{longtable}

                        \chapter{Neue Fertigkeiten\index{F!Fertigkeiten}}
\label{neuefertigkeiten}

\section{Waffenfertigkeiten}

\section{K�rperliche Fertigkeiten}
\subsection{Waffenlose Kampftechniken\index{W!waffenlose Kampftechniken}}
\par (BEW 10/17) Unter diesem Begriff werden die verschiedenen waffenlosen Kampftechniken der M"onche zusammengefasst.\\
\begin{tabular}{l|c|c|c|c|r}
Waffe & INI & TbB & Schaden\\
\hline
Schlag, Tritt & +2 & 14/22 & 2W TP\\
Wurf & +3 & 14/22 & speziell \\
Hebel & - & - & 1W SP\\
\end{tabular}

\begin{description}
\item [Verwendete Charakterklassen und Rassen:] M"onch

\item [W"urfe] Beschreibung...\\
\textbf{Mindestwurf:} NK-MW 14\\
\textbf{Wirkung:} Gegner geht zu Boden, 2W SP (Metallr"ustung sch"utzt nicht), immer Rumpftreffer\\
\textbf{Spezial:} Ein Wurf kann einleitend f"ur einen Hebel sein. Um einen Hebel anzusetzen, muss der K"ampfer nach einem gelungenen Wurf einen gezielten Angriff gegen ein Bein oder einen Arm durchf"uhren. Der Angriff erfolgt noch in der selben Runde und der Gegener gilt als am Boden liegend.\\
Wird der Hebel erfolgreich angesetzt, kann in den folgenden Kampfrunden 1W SP verursacht werden. Der Gegner kann in seiner Runde versuchen, sich zu befreien. Der Angreifer f"uhrt einen PW(GES) durch, der Verteidiger einen PW(STR). Die Differenz des W"urfelergebnisses zum Attributswert des Angreifers gibt die Modifikation f"ur den am Boden Liegenden an.\\
Kann kein Hebel angesetzt werden, so bekommt das Opfer einen Malus von 2 auf seine n"achste Aktion.

\item [Feger] Ein Feger dient dazu, durch Wegrei"sen des Standbeines den Gegner aus dem Gleichgewicht und somit zu Fall zu bringen.\\
\textbf{Mindestwurf:} NK-MW 15\\
\textbf{Wirkung:} Gegner geht zu Boden

\item [Tritte, Schl"age] Beschreibung...\\
\textbf{Mindestwurf:} NK-MW 14\\
\textbf{Wirkung:} siehe Tabelle

\end{description}

\section{Geistige Fertigkeiten}
\subsection{Chi\index{C!Chi}}
\par (PSI 8/16) Chi erm"oglicht es dem M"onch, "uber seine normalen k"orperlichen F"ahigkeiten hinaus zu wachsen.

\begin{description}

\item [Verwendete Charakterklassen und Rassen:] M"onch

\item [Verbesserte Initiative] Beschreibung....\\
\textbf{Mindestwurf:} 14\\
\textbf{Wirkung:} 1W zus"atzlich bei Initiative\\
\textbf{Kosten:} 2 PP

\item [Verbesserung k"orperlicher Fertikeiten] Beschreibung....\\
\textbf{Mindestwurf:} 14\\
\textbf{Vorbereitungszeit:} 1 Runde, keine Runde bei 110\% MW\\
\textbf{Wirkung:} Bonus in H"ohe von max. Chi-Wert auf k"orperliche Fertigkeit\\
\textbf{Kosten:} Bonus in PP

\item [Schmerz"uberwindung] Beschreibung....\\
\textbf{Mindestwurf:} 14+Modifikator\\
\textbf{Wirkung:} Schadenreduzierung bis max. Chi-Wert\\
\textbf{Kosten:} 1 PP pro TP\\
\par \begin{tabular}{r|l}
Modifikator & Wunde\\
\hline
1 & leichte Wunde 1-10 TP\\
2 & mittlere Wunde 11-20 TP\\
3 & schwere Wunde 21+ TP\\
\end{tabular}

\item [Verbesserte St"arke] Beschreibung....\\
\textbf{Mindestwurf:} 14\\
\textbf{Vorbereitungszeit:} 1 Runde, keine Runde bei 110\% MW\\
\textbf{Wirkung:} Bonus in H"ohe von max. Chi-Wert auf St"arkeattribut\\
\textbf{Kosten:} Bonus in PP

\end{description}

\subsection{Geographie\index{G!Geographie}}
\par (WIS 6/14) Hier kommt die Beschreibung.....
\begin{description}
\item [Verwendete Charakterklassen und Rassen:] alle
\end{description}

\subsection{Hexenbesen fliegen\index{H!Hexenbesen}}
\par (PSI 12/18) Hier kommt die Beschreibung.....
\begin{description}
\item [Verwendete Charakterklassen und Rassen:] Hexe
\end{description}

\subsection{Magische Spuren lesen}
\par Mit Hilfe des magischen Spurenlesens kann der Waldl"aufer auch sehr versteckte oder verwehte Spuren 
rekonstruieren. Es ist ihm sogar m"oglich, Spuren wieder sichtbar zu machen, die schon komplett entfernt 
wurden. Die Fertigkeitsprobe wird mit Spuren lesen und
Psieinsatz abgelegt.\\
\textbf{Mindestwurf:} $ 15+Zeit+Modifikator $\\
\textbf{Zauberdauer:} sofort\\
\textbf{Wirkungsdauer:} aufrecht erhalten\\
\textbf{Widerstandswurf:} erlaubt\\
\par \begin{tabular}{r|lr|l}
Zeit & vergangene Zeit & Modifikator & Beschreibung\\
\hline
1 & PSI in Minuten & +1 & Spuren wurden absichtlich verwischt\\
2 & PSI in Stunden & +2 & Spuren magisch verwischt\\
4 & PSI in Tagen   & -1 & Waldl"aufer wei"s von den Spuren\\
8 & PSI in Wochen  & -2 & Aufnahme einer verlorenen F"ahrte\\
16 & PSI in Monaten\\
32 & PSI in Jahren\\
\end{tabular}
\begin{description}
\item [Verwendete Charakterklassen und Rassen:] Waldl"aufer
\end{description}

\subsection{Stollennavigation\index{S!Stollennavigation}}
\par (WIS 11/16) Hier kommt die Beschreibung.....
\begin{description}
\item [Verwendete Charakterklassen und Rassen:] Zwerg
\end{description}

\section{Psifertigkeiten}

\section{Sonstiges}
\par Fertigkeiten dieser Kategorie haben keinen Fertigkeitswert.

\subsection{Nachtsicht\index{N!Nachtsicht}}
\par Wesen mit Nachtsicht ist es m"oglich in fast v"olliger Dunkelheit die gleiche Sicht zu haben, wie bei Tageslicht. Sie erhalten keine Modifikationen auf W"urfe in Dunkelheit, so lange noch Restlich vorhanden ist. Dabei reicht ein seichtes Leuchten von einem Punkt v"ollig aus. Bei kompletter oder magischer Dunkelheit gelten f"ur sie die normalen Behinderungen.
\begin{description}
\item [Verwendete Charakterklassen und Rassen:] Zwerg
\end{description}
                        \chapter{Die Magie}
\label{diemagie}
\section{Einf�hrung}
\parpic[l]{\epsfig{file=pics/capitals/d.eps, scale=0.5}}ie Magie spielt in Drachend"ammerung die Rolle einer geheimnissvollen,
unbekannten und gleichwohl gef"urchteten Erscheinung. Nur wenig ist
dem gemeinen B"urger "uber die arkanen K"unste\index{A!arkane Kunst}
bekannt.
\par Der Gelehrte unterscheidet zwei Formen der Magie. Auf der
einen Seite steht das Sprechen von
Zauberspr"uchen\index{Z!Zauberspruch} (ab S. \pageref{spruchmagie}),
bei denen der Anwender sich voll und ganz auf seinen Geist und seine
F"ahigkeiten verlassen muss. Mag diese Art der Magie f"ur den
Betrachter schon m"achtig erscheinen, so berichten die
Praktizierenden von einer weitaus m"achtigeren Form der Magie. Um
wahre Magie\index{M!Magie, wahre} (ab S. \pageref{wahremagie})
wirken zu k"onnen, muss sich der Zauberer direkt der Kraft der f"unf
Elemente\index{E!Element} bedienen. Diese Art der Magie ist ungleich
effektiver wie auch gef"ahrlicher. Sie hat schon so manchen Magier
den Verstand wenn nicht sogar den K"orper gekostet.
\par Neben diesen beiden Formen der Magie existieren noch eine Reihe von Abwandlungen\index{M!Magie, Abwandlungen der}. Eine weitaus schw"achere Art der Zauberei\index{Z!Zauberei}, die auch von nicht-Magiern beherrscht werden kann, ist die Illusionsmagie\index{I!Illusionsmagie} (ab S. \pageref{illusionsmagie}). Die dunkle Kunst\index{K!Kunst, dunkle} der Schwarzmagier\index{S!Schwarzmagier} ist die Beschw"ohrung\index{B!Beschw"orung} (siehe S. \pageref{beschwoerungen}) und die Hexen\index{H!Hexe} wenden Fluchmagie\index{F!Fluchmagie} (siehe S. \pageref{fluchmagie}) an.
\par Jeder Magier w"urde es jetzt wahrscheinlich bestreiten, aber die Alchimie\index{A!Alchimie} (siehe S. \pageref{alchimie}) ist auch eine Form der Zauberei\index{Z!Zauberei}. Zur Herstellung wirkungsvoller Tinkturen\index{T!Tinktur} bedarf es etwas mehr als der Kunst des Lesens und des Abf"ullens.
\par Nicht zu vergessen, da sie eine nicht unbedeutende Form der Magie ist, ist die Runenmagie der Zwerge. M"achtige Kriegsschmiede und Thaumaturgen fertigen mit ihrer Hilfe magische Artefakte. Diese Form der Magie ist nur den Zwergen bekannt und wird innerhalb der Familien weitergegeben. Obgleich die Runenmagie\index{R!Runenmagie} der Zwerge wesentlich m"achtiger als unsere Alchimie/Thaumaturgie ist, wird sie im allgemeinen doch dazu gez"ahlt.

\section{Spruchmagie\index{S!Spruchmagie}} \label{spruchmagie}
\parpic[l]{\epsfig{file=pics/capitals/d.eps, scale=0.5}}ie Spruchmagie
ist die aller"ublichste aller Formen der Magie. Sie wird seit
Urzeiten von allen Magiebegabten angewandt und in den Zirkeln
gelehrt. Sie ist sozusagen die Schulmagie\index{S!Schulmagie}.

\par Seit der Entstehung der Magie, wann auch immer das war, gibt es
auch verschieden Ausrichtungen. Es gibt den Zirkel der Schwarzmagier\index{S!Schwarzmagier}
und den Zirkel der Wei"en, die freien Magier und die Ge"achteten,
letztere sind im Grunde eine extreme, nicht gedultete Ausrichtung
der Schwarzmagier und werden offiziell selbst von denen nicht
geduldet. W"ahrend die Zirkel ihre Hauptaufgabe in der Weitergabe
des Wissens und dem Finden neuer und w"urdiger Sch"uler sehen, jeder
Zirkel hat dabei nat"urlich eine eigene Vorstellung was w"urdig ist,
besch"aftigen sich die freien Magier mit der Forschung und arbeiten
dabei unter Umst"anden auch mal enger mit den Zirkeln zusammen.
Freie Magier sind meist Einzelg"anger, bilden jedoch hin auch wieder
auch einzelne Sch"uler aus, wenn sie auf ihren Wanderungen ein Kind
entdecken, dass sich als sehr potent erweist.

\par Die Spruchmagie teilt sich in 18 Bereiche der Magie auf.
\begin{multicols}{3}
\begin{itemize}
\item Beschw�ren
\item Bewegung
\item Binden
\item Dimension
\item Elemente
\item Empathie
\item Gegenmagie
\item Hypnose
\item Illusion
\item Kontrolle
\item K�rperbewu�tsein
\item Materietransformation
\item Natur
\item Regeneration
\item Schock
\item Telepathie
\item Verst�ndnis
\item Wahrnehmung
\end{itemize}
\end{multicols}
\par Eine genaue Beschreibung der einzelnen Magiebereiche ist dem
ERPS Regelwerk\cite{erps1} zu entnehmen.
\par Dem normalen Magiewirkende werden die Bereich Beschw�rung\index{B!Beschw�rung} und
Dimension\index{D!Dimension} in der Regel nicht zug�nglich sein. Das Verst�ndis dieser
Gebiete ist einfach zu komplex, als das diese neben den anderen
Bereichen erlernt werden k�nnen.

\section[Wahre Magie]{\glqq Wahre\grqq{} Magie\index{M!Magie, wahre}}
\label{wahremagie}

\section{Illusionsmagie\index{I!Illusionsmagie}}
\label{illusionsmagie}
\parpic[l]{\epsfig{file=pics/capitals/j.eps, scale=0.5}}eder kennt die Tricks,
die mit Illusionen m"oglich sind. W"ahrend man versucht das Opfer vom eigentlichen
Geschehen abzulenken, holt man unter hohem Geschick die M"unze aus dem "Armel und
\glqq zaubert\grqq{} sie hinter dem Ohr des Opfers wieder hervor. Kinderleicht.

\par Wenn der Illusionist jedoch von einer Illusion spricht, dann meint er damit
etwas ganz anderes. Durch die magische Manipulation seiner Umgebung, ist es dem
Illusionisten m"oglich, wirkliche Trugbilder zu erschaffen, die dem oder den
Opfern sehr reell erscheinen werden. Manchmal auch zu reell.

\section{Beschw"orungen\index{B!Beschw�rung}}
\label{beschwoerungen}
\parpic[l]{\epsfig{file=pics/capitals/n.eps, scale=0.5}}eben der
Schwarzmagie stellt die Beschw"orung eine der gef"ahrlichsten und
unberechenbarsten Formen der Magie dar. Nicht zuletzt wurde den
Beschw"orern die Schuld am Erscheinen der Drachen gegeben. Nur zu
verst"andlich, dass diese seit dem nicht mehr ihren Beruf in die
"Offentlichkeit tragen.
\par Doch neben den schwarzen Schafen gibt es auch die jenigen,
die sich mit dem Beschw"oren niederer Wesen besch"aftigen. Kleine
D"amonen, die als Beobachter oder Dienstboten eingesetzt werden,
Beschw"orung der Geister der Toten, um mit ihnen in Kontakt zu
treten, Kontakaufnahme mit Elemtarwesen\index{E!Elementarwesen} oder aber die Kommunikation
mit den Geistern der Pflanzen, dies sind die Aufgabengebiete der
wei"sen Beschw"orer.
\par Auch die Erschaffung von Golems\index{G!Golem} und anderer widernat�rlicher
Wesen wird als Form der Beschw�rung verstanden.
\par Als letztes sei noch erw"ahnt, dass die Nekromantie\index{N!Nekromantie} ebenfalls
eine Form der Beschw�rung ist.

\subsection{Elementarwesen beschw�ren}
\subsection{D�monen beschw�ren}
\subsection{Magische Wesen erschaffen}
\subsection{Geisterbeschw�rung}
\subsection{Totenbeschw�rung}

\section{Fluchmagie\index{F!Fluchmagie}} \label{fluchmagie}
\parpic[l]{\epsfig{file=pics/capitals/d.eps, scale=0.5}}ie Fluchmagie ist wohl die Form der Magie, 
die in der Bev"olkerung am bekanntesten ist. "Ublicherweise f"allt auch jede andere Form von Magie 
meist unter diese Bezeichnung. Unwissenheit ist hier die Ursache. Wie auch immer.
\par Die Anwender von Fluchmagie sind Hexen und deren m"annliches Gegenst"uck die Hexer.
\par Fluchmagie ist eine "au"serst niedertr"achtige Form der Magie und wird von den Hexen meist zu 
Rachezwecken verwendet, dabei muss in den Wirkungsgraden zwischen \glqq f"ur kurze Zeit\grqq{} und 
permanent unterschieden werden. Ein vom Fluch gepeinigter wird meist unter den folgen extrem zu 
leiden haben. Ob es das Pech ist, was ihn verfolgt, oder die Gicht, die ihn von dem Moment an plagt, 
die Folgen sind schwer wieder zu neutralisieren.
\subsubsection{Fluchmagie im Spiel}
\par Wie jede andere Form der Magie, wird Fluchmagie auch "uber die Psifertigkeiten abgewickelt. Die 
Ergebnisse eines erfolgreich gesprochenen Fluches, sind jedoch von anderer Wirkung. Sie zielen meist 
auf die Peinigung des Opfers ab. Hier die Liste der Fl"uche, die nat"urlich wie andere Psi-Kr"afte auch, 
im Laufe des Spiels um Eigenkreationen erweitert werden kann.
\begin{description}
\item [Hexenschuss (Bewegung)] Das Opfer erleidet einen Hexenschuss. Von nun an erh"alt es 
Abz"uge auf alle k"orperlichen und Waffenf"ahigkeiten, sowie auf die Beweglichkeit.\\
\textbf{Mindestwurf:} $ 17+RW^2+Abzug^2+Dauer $\\
\textbf{Zauberdauer:} sofort wirksam\\
\textbf{Wirkungsdauer:} variabel\\
\textbf{Widerstandswurf:} erlaubt\\
\begin{tabular}{r|l}
Dauer & Wirkungsdauer\\
\hline
1 & 1 Spielrunde\\
2 & 1 Kampf\\
3 & 1 Stunde\\
4 & 1 Tag\\
5 & 1 Woche\\
10 & 1 Monat\\
20 & 1 Jahr\\
50 & 10 Jahre\\
75 & 1 Vierteljahrhundert\\
100 & permanent\\
\end{tabular}
\item [Angst (Schock / Empathie)] (siehe ERPS Regelbuch S. 120)
\item [L"ahmen (Bewegung)] (siehe ERPS Regelbuch S. 116)
\item [Liebeszauber (Empathie / Hypnose)] (siehe ERPS Regelbuch S. 120)
\item [Blindheit verursachen (Schock)] (siehe ERPS Regelbuch S. 125)
\item [Fieber (Schock)] (siehe ERPS Regelbuch S. 125)
\item [H"asslichkeit (Hypnose)] Von Beginn des Fluches an, denkt das Opfer es sei h"asslich.
\item [H"asslichkeit (K"orperbewusstsein)] Von Beginn des Fluches an wird das Opfer mit H"asslichkeit gestraft.
\item [Sinne des Vertrauten (Empathie)] Auch wenn es nicht wirklich ein Fluch ist, so wird es doch hier erw"ahnt, da es ein typischer Hexenzauber ist. Er erm"oglicht der Hexe den Zugriff auf die Sinne ihres Vertrauten. Sie kann durch ihn sehen, h"oren und/oder f"uhlen.\\
\textbf{Mindestwurf:} $ 12+RW^2+Anzahl der Sinne $\\
\textbf{Zauberdauer:} Minutenzauber\\
\textbf{Wirkungsdauer:} aufrechterhalten\\
\textbf{Widerstandswurf:} entf"allt\\
RW wird in Kilometern gez"ahlt.
\end{description}

\section{Alchimie\index{A!Alchimie} und
Thaumaturgie\index{T!Thaumaturgie}} \label{alchimie}
\subsection{Allgemeine Alchimie}
\parpic[l]{\epsfig{file=pics/capitals/h.eps, scale=0.5}}eiltr"anke, Gifte, Tintkturen...
\par Alle diese Sachen fallen in das Aufgabengebiet eines Alchimisten. Dar"uber hinaus 
verf"ugt der Alchimist jedoch auch "uber leichte magische F"ahigkeiten, die es ihm erm"oglichen, 
seine Tr"anke magisch zu verst"arken. Die Palette der m"oglichen Effekte wird hiermit immens 
vergr"o"sert.

\subsection{Die Runenmagie der Zwerge}
\par Legenden berichten von dieser Art von Magie. Geschichten und Mythen erz"ahlen von den m"achtigen 
Artefakten, die die Runenmeister der Zwerge unter Tage schufen und auch noch heute herstellen.
\par Ein Spielercharakter wird wahrscheinlich niemals in den Genuss kommen, bei der Schaffung eines 
solchen Artefakts anwesend zu sein. Die Zwerge h"uten diese Kunst wie ihren Augapfel und selten 
werden Fremde bei einer diese Zeremonien anwesend sein. Viel wahrscheinlicher wird es sein, 
dass ein Abenteurer in den Besitz eines solchen Artefaktes kommt. In der Zeit der Kriege und 
des Widerstands haben die Zwerge viele Waffen, R"ustungen und andere n"utzliche Gegenst"ande 
geschaffen, die jetzt "uber den Kontinent und weiter zerstreut sind.


                        \newpage\section{In den n"achsten Kapitel}
\label{diewelteinleitung}

\parpic[l]{\epsfig{file=pics/drache_icon.eps, scale=0.5}}
In den n"achsten Kapiteln findest Du eine Beschreibung der Welt\index{W!Weltbeschreibung}, in der sich die Charaktere befinden. Diese Beschreibung soll allerdings nur als Anreiz und Idee f"ur eine Spielwelt\index{S!Spielwelt} dienen. Sie ist keinesfalls als dogmatische Festlegung von Tatsachen zu verstehen. Jedem Spielleiter\index{S!Spielleiter} und jeder Spielgruppe\index{S!Spielgruppe} steht es frei sich eine eigene Welt zu erschaffen oder die bereits vorhandene zu modifizieren. Verstehe die folgenden Beschreibungen als einen Vorschlag, wie man es machen k"onnte. Die Beschreibung gliedert sich in die 5 Reiche, zu jedem Reich gibt es eine Karte, auf der geographische Gegebenheiten und St"adte zu finden sind. Im Anhang gibt es eine Gesamtkarte (siehe Anhang S. \pageref{worldmap}).

\par Das Kapitel \textit{Die Welt} (siehe S. \pageref{diewelt}) beschreibt die Umgebung, in der die Charaktere sich zu Spielbeginn aufhalten. Ihr findet dort eine Abhandlung "uber die bekannten Reiche sowie Spekulationen "uber die Grenzregionen\index{G!Grenzregionen} und das Unbekannte. Wichtige Personen und Orte finden dort ihren Platz.

\par Anschlie"send folgt eine detailierte Beschreibung der V"olker des Kontinents (siehe S. \pageref{dievoelker}). Diese dient als Erg"anzung zu den einleitenten Worten bei den Rassenbeschreibungen (siehe S. \pageref{dierassen}). Hier sind tiefer greifende Informationen "uber die einzelnen Sippen der Elfen und die Clans der Zwerge zu finden. Wir lernen einiges "uber die Herschaftsstrukturen in den menschlichen K"onigsh"ausern und gewinnen letztendlich auch noch einen Einblick in die Familienstrukturen der Halblinge und Trolle.

\par In den Kapiteln \textit{Tierleben\index{T!Tierleben}} (siehe S. \pageref{tierleben}), \textit{Bestiarium\index{B!Bestiarium}} (siehe S. \pageref{bestiarium}) und \textit{Kr"auterkunde\index{K!Kr"auterkunde}} (siehe S. \pageref{kr"auterkunde}) findest Du Informationen "uber die bekannte Flora\index{F!Flora} und Fauna\index{F!Fauna}.

\par Das Kapitel \textit{Legenden und Geschichten} (siehe S. \pageref{legendenundgeschichten}) erz"ahlt von der Entstehung der Welt, den Drachen, der Magie und deren Untergang. Hier ist das eine oder andere M"archen und die eine oder andere wahre Begebenheit zu finden. Es liegt im Auge des Betrachters hier Phantasie und Wirklichkeit zu trennen.


\begin{description}
\item[Anmerkung:]In den folgenden Abschnitten wird des "ofteren von der Bev"olkerung dieser Welt die Rede sein. Der Einfachheit halber wird diese als \glqq die Menschheit\grqq{} bezeichnet. In diesem Fall sind damit nicht nur die Menschen gemeint, sondern durchaus alle Bewohner wie Elfe, Trolle, Orks, Gnome, Halblinge, Zwerge etc. Es m"oge sich niemand bei der Formulierung verletzt f"uhlen. Sie dient nicht der Unterscheidung oder Ausgrenzung, sondern nur der Vereinfachung. Jedesmal alle aufzuz"ahlen w"urde schlichtweg zu Unleserlichkeit f"uhren und sich einen neuen Sammelbegriff, der allen gerecht wird, auszudenken zu Unverst"andnis.
\end{description}

\par Genug der vielen Worte, nun geht es los mit dem n"achsten Kapitel, der Weltenbeschreibung.

                \part{Die Welt}
                        \chapter{Die Welt}
\label{diewelt}
\par Im folgenden Kapitel stelle ich die dem Reisenden bekannte Welt vor.
Wobei der Ausdruck Welt in diesem Zusammenhang wohl etwas
"ubertrieben w"are, handelt es sich doch eher um einen Teil eines Kontinentes. Da
die Bev"olkerung in letzter Zeit mehr damit besch"aftigt war, ihr
Leben zu retten, als Welterkundungen durchzuf"uhren, hat sich an der
Karte in den letzten Jahrhunderten auch nicht viel getan.

\begin{figure}[hbtp]
\begin{center}
\epsfig{file=pics/map.eps, scale=0.12}
\end{center}
\caption{Weltkarte klein - Gesamt}
\label{worldmapsmall}
\end{figure}

\par Das bekannte Reich misst in Nord-S"ud Richtung ungef"ahr 1500
Meilen und in Ost-West Richtung ungef�hr 2000 Meilen. Die Weltkarte wird f"ur die
folgenden Beschreibung in die 5 K"onigreiche aufgegliedert. (Eine gr��ere Ausf�hrung 
der Karte ist im Anhang auf Seite \pageref{worldmap} zu finden.)

\section{Die 5 Reiche}

\subsection{Der Hohe Norden}
\parpic[l]{\epsfig{file=pics/capitals/i.eps, scale=0.5}}m Norden durch
das gewaltige Felsmassiv \glqq Das steinerne
Schwert\index[regionen]{S!steinerne Schwert, das}\grqq{} begrenzt, hat diese
Region ansonsten nicht viel an Attraktionen zu bieten. Ein bekannter
Weg w"urde weiter in n"ordlicher Richtung f"uhren. Von Reisenden und
Forschern jedoch eher gemieden, nennt sich das dahinter liegende Tal
\glqq Das Tal der Verlorenen und Verdammten\grqq\index[regionen]{T!Tal der
Verlorenen und Verdammten}, eine nicht gerade einladende
Bezeichnung.
\par Das Zentrum des Nordreiches und damit seine Hauptstadt ist 
Pax Epher\index[regionen]{P!Pax Epher}
\par Im S�den des Landes an der Grenze zum Reich der Mitte befindet sich das
Erzmassiv an dessen Randausl�ufern sich der Eingang der Zwergenfestung Bergfried\index[regionen]{B!Bergfried}
befindet. Bergfried\index[regionen]{B!Bergfried} ist die Stadt des Sturmhammerclans\index{S!Sturmhammerclan}.


\subsection{Das Reich der Mitte}
\parpic[l]{\epsfig{file=pics/capitals/d.eps, scale=0.5}}as bl"uhende Reich
unter der Sonne. So, oder so "ahnlich betitelte einmal ein Barde das
Reich um die sch�ne Stadt Patria Pacis\index[regionen]{P!Patria Pacis} und
die Hauptstadt dieser Region Pax Antares\index[regionen]{P!Pax Antares}. Das
ehemalige Reich des m"achtigsten der Drachen kann sich "uber seinen
Wirtschaftszustand nicht beklagen. Gro"e St"adte und weite Wiesen
tragen das Lebensgef"uhl der Bewohner nach au"sen. Einzig eine
K"ustenanbindung fehlt noch zum Gl"uck. Aber der regierende K"onig Siegfried II.\index[personen]{S!Siegfried II.}
schielt nicht ohne Grund auf das angrenzende K"onigreich im Osten.


\subsection{Die S�dk�ste}
\parpic[l]{\epsfig{file=pics/capitals/d.eps, scale=0.5}}as Land der Sonne
und der Palmen, hier l"asst es sich gut leben. Jedenfalls, wenn man
mit dem vorherrschenden Klima gut auskommt. Das S"udreich, dessen
Hauptstadt Pax Par\index[regionen]{P!Pax Par} ist, lebt vom Handel mit den
angrenzenden Staaten. Zu gro"sem Reichtum kam es dabei leider nicht,
da auf Grund des Klimas wichtige Nahrungsmittel nicht angebaut,
sondern importiert werden m"ussen. Wie gewonnen, so zerronnen.
\par Der Bev"olkerung geht es trotzdem nicht schlecht, denn ein Gro"steil der
 Seefahrer geht einem etwas un"ublichem Beruf nach, Freibeuterei. Die Piraten dieser 
 Region sind bei den Handelsflotten der L"ander gef"urchtet und verhasst. Doch leider 
 ist es bisher noch niemandem gelungen, dieses Treiben einzugrenzen. Der Sultan von 
 Pax Par\index[regionen]{P!Pax Par} w"ascht seine H"ande in Unschuld, \glqq Aus meinem Lande kommen diese Verr"uckten nicht!\grqq.


\subsection{Die Ostk�ste}
\parpic[l]{\epsfig{file=pics/capitals/d.eps, scale=0.5}}irekt an der K"uste
gelegen stellt die Hauptstadt Pax Lucien\index[regionen]{P!Pax Lucien} den
wohl "ostlichsten Punkt des Kontinents dar (Karte siehe S.
\pageref{fig_map_east}). Mit seiner g"unstigen Meeresposition ist
Pax Lucien\index[regionen]{P!Pax Lucien} damit Ausgangspunkt f"ur jeglichen Seehandel mit den
anderen Reichen und f"ur alle, wenn auch selten stattfindenden,
Schiffsexpeditionen ins offene Meer. Pax Lucien dient ebenfalls als
Zwischenstopp f"ur Schiffsreisen von und zum Nordreich.

\begin{figure}[hbtp]
\begin{center}
\epsfig{file=pics/map_east.eps, scale=0.5}
%width=4.8in, height=3.45in}
\end{center}
\caption{Weltkarte - Ostk�ste} \label{fig_map_east}
\end{figure}

\par Die politische
Situation der Ostk"uste ist gef"ahrdet. Dem Land geht es nicht
unbedingt gut und im Verh"altnis zu den anderen Reichen ist es ein
eher kleines Land. Da hilft es ihm auch nicht, da"s sich hier die
gr"o"ste Seemacht des Kontinents befindet, wenn potetielle Feinde
"ubers Land kommen. Gl"ucklicherweise gab es bis jetzt noch keine
feindlichen "Ubergriffe.

\par Die Hauptstadt des Landes beherbergt eines der gr"o"sten Kloster
der Neuzeit, die \glqq Feste des Wissens\grqq\index{F!feste des Wissens}. Die M"onche hier
wachen "uber die zweitgr"o"ste Bibliothek des Kontinents. Die
bewahrten Sch"atze reichen von einfachen Manuskriptsammlungen "uber
Abhandlungen der G"otter bis hin zu einigen Steintafeln, die
tausende von Jahren alt sind.

\par Nicht unweit von Pax Lucien\index[regionen]{P!Pax Lucien} findet sich die Tempelburg \textit{Sturmfeste\index[regionen]{S!Sturmfeste}}.
Hier ist der Sitz der Priesterschaft der G"otter. Von hier aus
findet die Koordination der gesamten 5 Reiche statt.


\subsection{Der weite Westen}
\parpic[l]{\epsfig{file=pics/capitals/s.eps, scale=0.5}}cherzhaft einmal
\glqq Das Land der Toten\grqq{}\index[regionen]{L!Land der Toten} genannt,
nehmen deren Einwohner die Bezeichnung nicht auf die leichte
Schulter.
\parpic[l]{\epsfig{file=pics/wueste1.eps, scale=0.4}}
Tats"achlich pr"agen W"usten und Berglandschaften diesen Teil des
Kontinents. Das Zivilisationsende findet sich weit im Westen am
Beginn einer gewaltigen "Odlandschaft. Zerrissener und zerkl"ufteter
Steinboden bestimmt hier das Bild. \par Hier fand vor knapp 1000
Jahren die entscheidene Schlacht um die Zukunft dieses Landes statt.
So unfruchtbar und unbewohnbar dieser Teil des Westens doch ist, hat
er eine wichtige Bedeutung bekommen. Irgendwo in der Ebene befindet
sich \glqq Der Turm der Magier\grqq\index{T!Turm der Magier}. Hier
leben und forschen die Wahrer der neuen Magiequelle. Hauptstadt des
Westens ist Pax Orgoth\index[regionen]{P!Pax Orgoth}


\section{Eine kleine St�dte�bersicht}

\subsection{Bergfried\index[regionen]{B!Bergfried}}
\par Hier folgt eine kleine einleitende Beschreibung zu der Stadt.

\begin{description}
\item [Geographische Lage] s�dliche Grenze Nordreich
\item [Stadtgr�ndung] ...
\item [Fl�che] ...
\item [H�he] ...
\item [Einwohnerzahl] ...
\item [Besonderes] Zwergenstadt in den Bergen, Hauptstadt des Sturmhammerclans
\end{description}
\subsection{Fischgrund\index[regionen]{F!Fischgrund}}
\par Hier folgt eine kleine einleitende Beschreibung zu der Stadt.

\begin{description}
\item [Geographische Lage] �stliches Reich Ostk�ste
\item [Stadtgr�ndung] ...
\item [Fl�che] ...
\item [H�he] ...
\item [Einwohnerzahl] 47
\item [Besonderes] Fischerdorf
\end{description}
\subsection{Patria Pacis\index[regionen]{P!Patria Pacis}}
\par Hier folgt eine kleine einleitende Beschreibung zu der Stadt.

\begin{description}
\item [Geographische Lage] mitteres Reich
\item [Stadtgr�ndung] ...
\item [Fl�che] ...
\item [H�he] ...
\item [Einwohnerzahl] ...
\item [Besonderes] ...
\end{description}
\subsection{Pax Antares\index[regionen]{P!Pax Antares}}
\par Hier regiest seine Majest�t K�nig Siegfried II.\index[personen]{S!Siegfried II.}
\par Hier folgt eine kleine einleitende Beschreibung zu der Stadt.
\par Zu dieser Stadt ist mir noch nichts anst�ndiges eingefallen wie ich das Element Geist verarbeiten kann.

\begin{description}
\item [Geographische Lage] mittleres Reich
\item [Stadtgr�ndung] ...
\item [Fl�che] ...
\item [H�he] ...
\item [Einwohnerzahl] ...
\item [Besonderes] Hauptstadt
\end{description}
\subsection{Pax Epher\index[regionen]{P!Pax Epher}}
\par Pax Epher, nach ihrem Gr�nder die Stadt des Feuers benannt, nahe am n�rdlichen Ende des Landes gelegen.
Die m�chtigen massiven Steinmauern haben die Stadt schon das eine oder andere Mal vor den Angriffen der Orkhorden
gesch�tzt. Wie auch bei den anderen Hauptst�dten haftet auch dieser Stadt der Makel des Drachen an, der sie erschaffen
hat. Rings um die Stadt herum befindet sich ein ca. 10 Schritt breiter Burggraben, der jedoch nicht mit Wasser gef�llt
ist, sondern dessen Inhalt aus einem unterirdischen Vulkan gespeist wird. Nicht zuletzt kein minder effektiver
Abwehrmechanismus.
\par Der K�nig des Stadt ist ein volksnaher Herrscher. Pax Epher beherbergt keine seperate Festung. Nahe der S�dmauer
befindet sich nur ein kleiner Palast mit einem f�r diese Gegend recht ansehnlichen Ziergarten.

\begin{description}
\item [Geographische Lage] n�rdliches Reich
\item [Stadtgr�ndung] ...
\item [Fl�che] ...
\item [H�he] ...
\item [Einwohnerzahl] ...
\item [Besonderes] Hauptstadt
\end{description}
\subsection{Pax Lucien\index[regionen]{P!Pax Lucien}}
\par Pax Lucien ist die Hauptstadt des �stlichen Reiches direkt an der K�ste gelegen. Geographisch betrachtet
stellt sie den �stlichsten Landpunkt dar.
\par W�hrend die Wasserseite der Stadt von m�chtigen Mauern gegen etwaige Brandungen und Angriffe gesch�tzt wird,
ist die Landseite relativ ungesch�tzt. Die einfachen Mauern w�rden kaum einem geballten Angriff statthalten.
\par Das Zentrum der Stadt ist der Palast des K�nigs und damit auch zugleich die Attraktion schlechthin. Die ehemalige
Festung des Drachen Lucien ist eine schwebende Zitadelle, die durch lange Seilbr�cken mit dem Boden verankert ist.
Um die Festung besser befahren zu k�nnen, wurde Mitte des Jahrtausends ein solider Steinweg hoch zu den Haupttoren
der Stadt gebaut. Um trotzdem die Sicherheit w�hrend eines Angriffes zu gew�hrleisten, wurde dieser jedoch auf
magische Weise so pr�pariert, dass er auf Gehei� der k�niglichen Hofmagier einfach in sich zusammenbrechen w�rde.

\begin{description}
\item [Geographische Lage] �stliches Reich, K�stenstadt
\item [Stadtgr�ndung] ...
\item [Fl�che] ...
\item [H�he] ...
\item [Einwohnerzahl] ...
\item [Besonderes] Hauptstadt
\end{description}
\subsection{Pax Orgoth\index[regionen]{P!Pax Orgoth}}
\par Pax Orgoth, die D�nenstadt und das Unterweltreich.
\par Mitten in der W�ste liegt diese Perle des Westens und so sieht sie auch aus. �u�erlich gleicht sie eher
einer Karawanenstadt denn einem Herrschaftssitz, doch das �u�erliche Bild tr�gt. �nhlich den St�dten der Zwerge
findet Pax Orgoth Ausl�ufer bis weit unter die Erde. Ein ausgekl�geltes Gang- und H�hlensystem stellt die Basis dieser
Stadt dar. Magische Kristalle sorgen f�r dauerhafte Beleuchtung unter Tage, die nur in den Nachstunden ged�mmt wird.
\par Die Festung des K�nigs ist tief in der Erde noch unter der eigentlichen Stadt zu finden.

\begin{description}
\item [Geographische Lage] westliches Reich
\item [Stadtgr�ndung] ...
\item [Fl�che] ...
\item [H�he] ...
\item [Einwohnerzahl] ...
\item [Besonderes] Hauptstadt
\end{description}
\subsection{Pax Par\index[regionen]{P!Pax Par}}
\par Pax Par, die Stadt der Kan�le an der S�dk�ste.
\par Im Hafen von Pax Par liegen das Jahr �ber immer ein gutes Dutzend pr�chtiger Segelschiffe, die das Stadtbild von
der Wasserseite aus pr�gen. Die Stadt selber liegt so dicht am Wasser, dass viele H�user auf Sockeln gebaut werden mussten,
damit man st�ndig Wasser absch�pfen m�sste.
\par Durch die Stadt schl�ngeln sich eine ganze Reihe von Kan�len, so dass es nicht weiter verwunderlich ist, dass das
Fortbewegungsmittel der Wahl Gondeln und kleine F�hren sind.
\par Die Festung des Stadt befindet sich Landeinw�rts n�rdlich in der Stadt inmitten eines kleinen Sees gelegen
und ist nur mit F�hren oder Gondeln zu erreichen. Hinter vorgehaltener Hand berichtet man von geheimen G�ngen unter der
Stadt, die ebenfalls dorthin und zur�ckf�hren.

\begin{description}
\item [Geographische Lage] s�dliches Reich, K�stenstadt
\item [Stadtgr�ndung] ...
\item [Fl�che] ...
\item [H�he] ...
\item [Einwohnerzahl] ...
\item [Besonderes] Hauptstadt
\end{description}
\subsection{Sturmfeste\index[regionen]{S!Sturmfeste}}
\par Hier folgt eine kleine einleitende Beschreibung zu der Stadt.

\begin{description}
\item [Geographische Lage] �stliches Reich n�he Ostk�ste
\item [Stadtgr�ndung] ...
\item [Fl�che] ...
\item [H�he] ...
\item [Einwohnerzahl] ca. 150
\item [Besonderes] Tempelstadt
\end{description}
                        \chapter{Die V�lker des Kontinents}
\label{dievoelker}

\section{Die Zwerge}
\label{diezwerge}
\subsection{Die Clans der Zwerge - ein kurzer Abriss\index{Z!Zwergenclan}}

\parpic[l]{\epsfig{file=pics/capitals/a.eps, scale=0.5}}uch wenn sich
die verschiedenen Clans untereinander nicht immer einig sind und zum
Teil eine gewisse Feindschaft herrscht, so werden die Clans doch von
einem gemeinsamen K�nig regiert. Der K�nig wurde einstmals von dem
Rat der Alten, dem sieben Weise (Ja genau, die sieben Zwerge!)
angeh�ren, ernannt. Seit dieser Zeit entstammt der K�nig aus der
selben Linie und ein Ende dieser Ahnenreihe ist nicht abzusehen. Zur
Zeit ist die Thronfolge durch zwei gesunde Prinzen geregelt. Die
K�nigsfamilie entstammte dem Clan der
Sturmhammer\index{S!Sturmhammerclan}, ist aber allen Clans
gleichermassen verpflichtet. Der Rat der Alten ist das Sprachrohr
der Clans und steht der K�nigsfamilie beratend zur Seite.
\par Jeder Clan wird vor Ort von einem Stellvertreter gef�hrt, der dem Rat 
wichtige Vorkommnisse �bermittelt.
\par Es folgt nun eine kurzer, aber �bersichtlicher Abriss �ber die Clans 
der Zwerge. F�r weiterf�hrende Informationen sei das 2007 erscheinende Buch 
\textit{Drachend�mmerung - Eine Abhandlung �ber die Clans der Zwerge}\cite{dddz} 
empfohlen.

\subsubsection{Sturmhammerclan\index{S!Sturmhammerclan}}
\par Dies ist der gr��te und bekannteste Clan. Nicht zuletzt, da sie sich r�hmen 
das K�nigsgeschlecht hervor gebracht zu haben. Ihre Stadt Bergfried\index[regionen]{B!Bergfried}
befindet sich in einem Felsmassiv n�rdlich von Patria Pacis\index[regionen]{P!Patria Pacis}.
\par Sie besitzen die ertragsreichsten Minen und f�hren regen Handel mit anderen 
V�lkern und den Clans. Ihre Sturmhammergarde ist sehr gut ausger�stet und k�mpft 
ehrenhaft an der Seite der Schwachen. Traditionell k�mpfen die Zwerge dieses Clans 
mit dem zweih�ndig gef�hrten Hammer. Veteranen erhalten oftmals die gef�rchteten 
Sturmhammer, welche auf Grund der Runenmagie schneller als �bliche Zweihandwaffen 
gef�hrtwerden k�nnen.

\begin{tabular}{l|c|c|c|c|c|c}
               & Ini  & Bonus     & Schaden & AbB    & BF    & Last\\
\hline
Kriegshammer   &  0   & Str  6/14 & 2W+3    & 14/7/ 0 & 28/46 & 1\\
Zweihandhammer & -3   & Str  9/14 & 3W+3    &  9/4/-2 & 35/53 & 3\\
Sturmhammer    &  0   & Str  8/13 & 3W+3    & 10/5/ 0 & 42/70 & 3\\
\end{tabular}\\


\par Sturmhammer werden nur selten zu S�ldnern, da diese in den Augen des Clans 
unehrenhafte K�mpfer sind. Dies erkl�rt die feindschaftliche Haltung dem Clan der 
Sch�delsammler\index{S!Schaedelsammlerclan} gegen�ber, aber dazu sp�ter mehr.
\par Das Wappen zeigt einen schweren goldenen Hammer auf dunkelblauem Hintergrund. 
Rechts neben dem Hammer ist eine ebenfalls g�ldene Krone abgebildet.

\subsubsection{Flammenaxtclan\index{F!Flammenaxtclan}}
\par Dieser Clan ist f�r seine hei�bl�tige Art bekannt, die sich vor allem im Kampf 
mit wilden Schreien bemerkbar macht. Die hei�bl�tigsten unter ihnen werden nicht 
selten als Berserker bezeichnet, aber diese Erfahrung solltet ihr selber machen, 
oder auch nicht?! Alle Zwerge dieses Clans haben feuerrotes Haar. Die Schmiede dieses 
Clans bringen die eindruckvollsten R�stungen  auf den Markt.
\par Ihr Hauptsitz ist die Bergfestung Mataterra\index[regionen]{M!Mataterra} s�dwestlich 
von Patria Pacis\index[regionen]{P!Patria Pacis}.
\par Als echter Flammenaxtzwerg ist das Tragen einer gro�en zweih�ndigen Axt im Kampf 
ehrenhaft. Auch in diesem Clan sind es die Veteranen, die eine echte Flammenaxt tragen 
d�rfen. So mancher Gegner wurde von den m�chtigen Waffen zu einem h�ufchen Asche.

\begin{tabular}{l|c|c|c|c|c|c}
            & Ini  & Bonus     & Schaden & AbB    & BF    & Last\\
\hline
Zweihandaxt & -3   & Str 10/15 & 4W-2    & 8/3/-4 & 35/53 & 3\\
Flammenaxt  &  0   & Str  9/14 & 3W+3    & 9/4/-2 & 42/70 & 3\\
\end{tabular}\\

\par Dieser Clan f�hrt nur wenig Handel mit anderen Rassen. Er beschr�nkt sich auf den 
Handel innerhalb der Zwergenclans. Die �bergabe einer Flammenaxtr�stung an einen 
Angeh�rigen einer anderen Rasse gilt als besondere Ehre.
\par Das Wappen zeigt eine schwarze einklingige Axt auf wei�em Hintergrund. Die Axt 
wird von einer Sph�re aus Feuer umgeben.

\subsubsection{Rammspornclan\index{R!Rammspornclan}}
\par Dieser Clan lebt zur�ckgezogen in den Bergen und f�hrt nur wenig Handel. 
Bekannt ist allerdings, dass er einige der wirkungsvollsten Kriegsmaschinen herstellt.
\par Ihre Festung Lukator\index[regionen]{L!Lukator} liegt in der Gebirgsregion zwischen 
zwischen Pax Orgoth\index[regionen]{P!Pax Orgoth} und Pax Epher\index[regionen]{P!Pax Epher}.
\par Das Wappen zeigt ein mit Dornen bewehrtes h�lzernes Rad auf rotem Hintergrund.

\subsubsection{Kristallklingenclan\index{K!Kristallklingenclan}}
\par Dieser Clan hat sich neben der Bearbeitung von Erzen und Metallen der Gestaltung 
von kristallinen Strukturen gewidmet. Hierbei haben die Schmiede in Zusammenarbeit mit 
den Thaumaturgen\index{T!Thaumaturg} kristalline Waffen und R�stungen erschaffen. 
Diese sind besonders wirksam und erheblich teurer als die metallernen Gegenst�cke. 
Allerdings haben die Kristalle diverse Vorteile, \mbox{z.B.} h�here Stabilit�t, 
Regeneration bei Besch�digung und geringes Gewicht.
\par Diese Kristallwaffen\index{K!Kristallwaffe} stehen der Elite des Clans zur Verf�gung. 
Andere Krieger des Clans nutzen Waffen mit kristalliner Schneide und R�stungen deren Teile 
mit einem feinem Netz durchwebt sind.
\par Die beliebten Waffen dieses Clans sind brachiale meist zweih�ndig gef�hrte Breitschwerter.
\par Die Kleidung ist der kalten Gebirgslandschaft angepasst und in graut�nen gehalten. Es 
gibt keine Rothaarigen Clansangh�rigen und die Augenfarbe ist von stets silbergauem Ton.
\par Als Wappen tragen die Zwerge eine gr�nfunkelnde Klinge auf hellgrauem Hintergrund.
\par Der Heimatsitz diser Zwerge ist die Bergfestung Grangorsch\index[regionen]{G!Grangorsch},
bennannt nach seinem Gr�nder, sie liegt nahe des Ostreiches auf H�he von Nordmark\index[regionen]{N!Nordmark}.

\subsubsection{Eisenfaustclan\index{E!Eisenfaustclan}}
\par �ber die Struktur dieses Clans ist nichts weiter bekannt. Diese Zwerge leben f�r sich 
und treiben keinen Handel mit au�enstehenden. Einzig und allein die Tatsache, dass diese 
Zwerge in Kriegszeiten nur dann in Erscheinung treten, wenn ihr Heim in Gefahr ist, ist 
bekannt. Dann aber sind sie in erster Reihe anzutreffen. Die Besonderheit Ihres Kampfstiles 
ist dann gut zu beobachten. Sie tragen keine Waffen im herk�mmlichen Sinne. Im Kampf tragen 
sie schwere Panzerhandschuhe, welche mit Dornen, Stacheln und kleinen Klingen in verschiedenster 
Form best�ckt sind.
\par Das Wappen zeigt eine st�hlerne Faust auf schwarzem Hintergrund.
\par Ihr Heimatsitz wird von den Clansbr�der geheim gehalten, so dass niemand so genau sagen, wo er liegt.

\subsubsection{Schattenwolfclan\index{S!Schattenwolfclan}}
\par Die Schattenw�lfe sind im Tiefland ans�ssig. Sie haben sich an die Umgebung innerhalb eines 
riesigen Waldes gew�hnt. Ihren Namen erhalten sie durch den Umgang mit den Schattenw�lfen\index{S!Schattenwolf} 
des Waldes. Diese liefern ihnen das Material der seltenen Schattenm�ntel, welche den Tr�ger in 
vollkommender Dunkelheit fast unsichtbar machen. Nat�rlich stehen die M�ntel, welche 
thaumaturgisch hergestellt werden nicht jedem Zwerg zur Verf�gung. Damit wir uns an dieser 
Stelle nicht falsch verstehen: die Zwerge jagen bzw. verteidigen sich gegen die W�lfe. 
Ausgewachsen misst die Schulterh�he eines Schattenwolfes 260 cm. Die W�lfe sind sehr kr�ftig 
und stellen ernst zu nehmende Gegner dar.
\par Ihre \glqq Festung\grqq{} liegt in einem verborgenen Tunnelsystem unterhalb des Waldes. 
Sie haben dar�ber die M�glichkeit Truppenbewegungen verdeckt durchzuf�hren und ihr Terretorium 
gut zu verteidigen. Manche halten diese Zwerge f�r Gespenster, was auf ihr pl�tzliches Erscheinen 
und Verschwinden zur�ckzuf�hren ist. Dieses Geheimnis h�ten sie wie ihren Augapfel.
\par Sie sind die geborenen Guerilla-K�mpfer und agieren des �fteren im Dunkeln, um ihre Ziele zu 
erreichen. Handel treiben sie mit jedem der an ihren Waren (z. B. seltene Felle, Federn und Pflanzen) 
interessiert ist.
\par Ihre Kleidung ist in Gr�n- und Braunt�nen gehalten. Ansonsten sind diese Zwerge f�r ihre Rasse 
�u�erst still.
\par Das Wappen besteht aus einem schwarzem Wolfskopf mit aufgerissenem Fang auf beigem Hintergrund.

\subsubsection{Sch�delsammlerclan\index{S!Schaedelsammlerclan}}
\par Dies ist der kleinste der Zwergenclans. Dieser Clan hegt tiefe Feindschaft dem 
Sturmhammerclan\index{S!Sturmhammerclan} gegen�ber. Die Sch�delsammler haben sich ganz aus 
dem Bergbau zur�ckgezogen, manche behaupten sie h�tten ihre Wurzeln verraten. Die verschiedenen 
Sippen dieses Clans ziehen ein nomadisches leben vor und verdienen sich als S�ldnereinheiten ihren 
Unterhalt. Andere f�hren das Leben von R�ubern und Banditen.
\par Eines haben alle Sammler gemeinsam: sie sammeln die Sch�del ihrer Opfer und tragen diese als 
Troph�en zur Schau. Je mehr und je gr��er die Sch�del sind, desto angesehener ist der Zwerg. 
Die barbarische Erscheinung rundet das Bild eines Wilden ab.
\par Sie verf�gen �ber das gr��te Repetoire lebender Runen und gelangen nur selten an die 
Mittel zur Herstellungen toter Runen. Die gr��ten Krieger dieses Clans haben sich das Recht am 
Tragen von lebenden Runen erk�mpft und nur wenige besitzen Runengegenst�nde.
\par Als Wappen bilden diese Zwerge einen beigen geh�rnten Sch�del auf rotem Hintergrund ab.

\subsection{Zwerge und G�tter und Geweihte}
\par Zwerge verehren nicht die G�tter, die den anderen bekannt sind, sondern nur einen, der f�r 
sie als der Sch�pfer und gro�er Vater bekannt ist. Er verk�rpert ihre Aspekte des Krieges, des 
Schmiedens und der Bergbaukunst.
\par Das Wort des Vaters wird von den Paladinen aufrecht erhalten und weiter getragen. Sie sind 
gleichzeit der kriegerische, aber auch der einzige Arm der Kirche. Kleriker sind ihnen fremd. 
�ber den Paladinen steht der �lteste und weiseste aller Paladine. Er hat sich in vielen 
Schlachten und Feldz�gen einen Namen gemacht und ist der spirituelle F�hrer eines Clans.

\subsection{Der zwergische Paladin}
\par Der Paladin ist auch unter den Zwergen ein hoch angesehener Krieger. Es stellt f�r jeden 
Zwerg eine besondere Ehre dar in den Dienst des Allvaters zu treten.
\par Allerdings ist nicht jeder Zwerg f�r diese Aufgabe geschaffen. Ein Anw�rter mu� sich 
einer Reihe von Tests unterziehen, um festzustellen, ob er ein geigneter Kandidat und sein 
Glauben unersch�tterlich ist. So bleiben nach dieser Zeit vielleicht ein Dutzend Anw�rter 
�brig, die die Ausbildung zum Paladin beginnen k�nnen.
\par Da die Pr�fungen nur alle 15 Jahre stattfinden ist es nicht weiter verwunderlich, da� 
die Anzahl der zwergischen Paladine nicht all zu gro� ist.

\par W�hrend der Grundausbildung lernen die Anw�rter sich an die Gebote und ihre Pflichten 
zu halten. Jeder Paladin bestreitet seinen weiteren Weg unter eben diesem Codex:

\begin{enumerate}
\item Sei stark im Glauben, bete!
\item Hilf den Hilfsbedrftigen
\item Streite f�r die Schwachen
\item Ehre das Leben, denn das Leben ist der Allvater
\end{enumerate}

\par Der Paladin wird nicht umsonst als der kriegerische Arm einer Glaubensgemeinschaft 
bezeichnet. Doch unterscheidet sich der Paladin stark von den herk�mmlichen S�ldnern 
und Kriegern. Ein Paladin wird niemals:

\begin{enumerate}
\item Einen Kampf von sich aus beginnen, denn er ist frei von Schuld
\item Einen Gegner auf unfaire Weise bek�mpfen, denn er ist das Licht auf dem Schlachtfeld
\item Hinterr�cks morden, denn er ist die Gerechtigkeit
\item T�ten wenn es vermeidbar ist, denn er ehrt das Leben
\end{enumerate}

\par Nach der Grundausbildung werden die Anw�rter auf eine zweij�hrige Wallfahrt geschickt. 
In dieser Zeit sollen sie sich dem Allvater gegen�ber beweisen und das Gelernte zu einem Teil 
von ihnen werden lassen.
\par Die Anw�rter werden mit nicht mehr als einer einfachen Reisekleidung, einer Clanwaffe, 
einem Wanderstock und dem Talisman des Allvaters auf Reisen geschickt.
\par Der Talisman ist ein Medallion, welches das Wappen des Clans und eine Paladinrune des 
Lichtes ziert. Der Anw�rter hat die M�glichkeit mit den Worten \glqq Allvater weise mir 
den Weg\grqq{} die Magie der Rune zu aktivieren. Danach erstrahlt das Medallion in hellem 
Licht, vergleichbar mit einer Fackel.
\par Der Anw�rter wird w�hrend dieser Zeit seine Fertigkeiten verbessern und sein Wissen 
vergr��ern. Am Ende dieser Zeit kehrt der erfolgreiche Anw�ter zur�ck und ... so viel sei 
gesagt, nicht alle Paladine �berstehen diese letzte Pr�fung. Doch wer frei von Schuld ist...

\subsubsection{Spieltechnisch}
\begin{enumerate}
\item grundlegende Fertigkeiten Kr�uter-, Heilkunde-, Waffenfertigkeiten, Lesen/Schreiben, Rechnen
\item w�hrend der Anw�rterzeit nur einfachste Wunder wirken
\end{enumerate}

\newpage
\section{Die Elfen}
\label{dieelfen}

\subsection{Die V�lker der Elfen}
\begin{quotation}
\par Kas'Ilyoth - St�tte der Freihet
\par Kas'Adunn - St�tte des Friedens
\end{quotation}

\parpic[l]{\epsfig{file=pics/capitals/k.eps, scale=0.5}}as'Ilyoth\index[regionen]{K!Kas'Illyoth}
und Kas'Adunn\index[regionen]{K!Kas'Adunn} sind die Namen der beiden gro�en Elfenreiche, die
Urspr�nge und Lebensr�ume der Hochland- und Waldelfen. Jeder Elf ist
einem der beiden K�nigreiche zugeh�rig. Auch wenn die beiden V�lker
in gro�er Distanz zueinander leben, so ist zwischen ihnen weder
Feindschaft noch Ignoranz. Die Elfen entstammen einem gro�en Volk
und f�hlen sich auch als solches.

\par Vor Urzeiten waren alle Elfen geeint unter der Sonne. Es gab
nur einen Namen f�r alle und diesen trugen sie mit Stolz. Unter der
Herrschaft der Drachen wurden die Elfen versklavt und in weite Teile
des Kontinents vrschleppt. Im Laufe der Befreiungen und des
folgenden Wideraufbaus sammelten sich zwei Gruppen von Elfen
getrennt voneinander, nicht wissend, dass die anderen noch lebten,
und bauten sich eine neue Heimat auf. Kas'Illyoth - St�tte der
Freiheit - errichtet im Nordwesten ward von nun an die Heimat der
Hochlandelfen. Kas'Adunn - St�tte des Friedens - wurde in den
W�ldern des Grenzbereiches zum S�dreich errichtet.

\par Die Elfenk�nigreiche werden von den Menschen in ihrer Existenz
akzeptiert und geachtet. Sie sind autark und f�hren keine
diplomatischen Beziehungen zu den Menschenk�nigen.

\subsection{Hochlandelfen - Steppenelfen}
\par Stolz und �berlegen; Die Elfen des Elfenreiches Kas'Ilyoth sind
die �lteste Dynastie der Elfen. Ihre Stammes�ltesten sind leben
schon ca. ein halbes Jahrtausend. das Wissen das die Elfen h�ten ist
unermesslich. Die Ilyoths, in der allgemeinen Sprache ein Synonym
f�r Hochelfen, leben wie ihre Vorfahren in Abgeschiedenheit.
\par Die Steppen von Kas'Ilyoth werden selten von anderen Rassen betreten, zu
sehr sitzt ihnen die Angst im Nacken. Die Gebiete sind weitl�ufig
und f�r den unerfahrenen Wanderer nicht gerade n�hrreich. M�chte man
den Elfen dort einen Besuch abstatten, ist man gut beraten, dies
anzuk�ndigen und mit dem n�tigen diplomatischen Geschick einen
F�hrer zu erhalten. Trotz ihrer Abgeschiedenheit sind die Elfen
keineswegs als Einsiedler zu verstehen, sie sind Besuchern gegen�ber
gastfreundlich un gew�hren gerne Unterkunft. Trotzdem wird es
Fremden schwer fallen, hier mehr �ber die wohl �lteste Rasse des
Kontinents in Erfahrung zu bringen, zu distanziert sind sie.
\par Kas'Ilyoth birgt kein Hauptstadt oder �berhaupt gr��ere
Ansammlungen. Die Ilyoth leben innerhalb ihrer Clanstruktur mit
verschiedenen Sippen von ungef�hr 150 bis 200 Elfen zusammen. Die
Behausungen der Elfen sind einfach, zweckgebunden. Simple Holzh�user
und Zelte in den Steppen dienen als Unterkunft.
\par Die �lteste Sippe eines jeden Clanes, die im Normalfall die H�uptling stellen,
bilden eine Gemeinschaft. Die Clan haben untereinander keine
Feindschaften und treffen Entscheidungen f�r das Wohl ihrer Rasse in
gemeinsamem EInverst�ndnis.

\subsection{Waldelfen}
\par Ber�hmt ist der Wald von Kas'Adunn, ber�hmt f�r seine weiten
Ausl�ufer, sein dichtes Baumwerk und die Elfen die ihn bewohnen. Die
Waldelfen, Adunn von den Menschen genannt, geh�ren alle einem Clan.
Die meisten von ihnen leben elfenuntypisch in einer gro�en
Gemeinschaft in der Baumstadt von Kas'Adunn, benannt nach dem
Elfenreich. Nur einige wenige Sippen, meist sehr kleine, leben
abgeschieden in den W�ldern von Kas'Adunn. Ganz selten sieht es eine
Elfenfamilie aus den Heimatlichen W�ldern an einen neuen Wohnort.

\subsection{Die Elfen und die G�tter}
\par Elfen haben einen sehr starken und tief verwurzelten Glauben an
die G�tter.

\subsection{Die schwarze Seele\index{S!Schwarze Seele}}
\begin{center}
\textbf{Pohlis fe Hethen Resul ftho ni-resul}\\
\textit{Verdr�ngung ist der Versuch Geschehenes ungeschehen zu
machen}\\
\textbf{Rah fe Loss an dotrth Pohlis}\\
\textit{Vergeltung ist das Resultat einer misslungenen
Verdr�ngung}\\
\textbf{Mthe trth Dracon. Dracon trth an Mthe}\\
\textit{Leben f�r die Qual. Qualen f�r ein Leben}
\end{center}

\par Die Elfen sind gebranntmarkt. Unter ihrer meist ausdruckslosen
Fassade verbergen sie ein schreckliches Geheimnis. Die Sklaverei und
die Folter der Drachen hat sie gepr�gt, nicht von ungef�hrt kommt
es, dass die Begriffe Drache und Qual im elfischen Sprachgebrauch
Synonym benutzt werden.
\par Doch die schwarze Seele ist mehr als nur ein Makel und
Schandfleck der elfischen Natur, gleichzeitig symbolisiert sie
jedoch auch das Band des Elfen zu seiner Familie und seinen
Freunden.
\par Ein Elf der gro�e Schmerzen oder gro�es Leid erf�hrt kann von
dieser Seele �bermannt werden. Sie wird ihn dazu dr�ngen,
Gerechtigkeit zu �ben, um den Schmerz zu lindern. Sie wird erst Ruhe
geben, wenn der oder die Peiniger einer aus seiner Sicht gerechten
Strafe zugef�hrt wurden.

\subsubsection{Die schwarze Seele im Spiel}
\par Erf�hrt ein Elf gro�e Schmerzen oder gro�es Leid muss er einen
modifizierten Test auf die Geistige Stabilit�t ablegen. D.h. er muss
einen offenen Wurf auf 2W10 gegen den GST\footnote{Geistige
Stabilit�t}-Wert machen. Modifikationen werden mit dem W�rfelwurf
verrechnet:

\begin{center}
\begin{tabular}{l|c}
St�rke des Ausl�sers & Modifikator\\
\hline

mittlere Wunde & -4\\
schwere Wunde & -3\\
schwere Verwundung eines Freundes & -2\\
schwere Verwundung eines Familienmitgliedes & 0\\
Tot eines Freundes oder Familienmitglied & +5\\
Tot mehrerer Freunde oder Familienmitglieder & +10\\
Ausl�schung der Familie & +20\\

\end{tabular}
\end{center}

\par Ein Elf der von der schwarzen Seele �bermannt wird, verf�llt in
einen Zustand der der Raserei �hnelt, obgleich kontrollierter. Der
Zustand wird andauern, bis die Seele Vergeltung ge�bt hat.
\par DIe Auswirkungen der Raserei sind vom Erfolg oder Misserfolg
der GST-Probe abh�ngig. Die in der Tabelle angebenen Reaktionen auf
den Ausbruch der schwarzen Seele sind als Kampfmodifikationen zu
verstehen. Au�erhalb eines Kampfes werden die Reaktionen des Elfen
in Absprache mit dem Spielleiter festgelegt.

\begin{longtable}{c|p{10cm}}
Erfolg & Der Elf wendet sich unverz�glich dem Verursacher zu und
wird erst von ihm ablassen, so er kampfunf�hig ist. Anschlie�end
wird er eine aus seinerSicht gerechte Strafe erlassen.\\

\hline

Misserfolg & Der Elf verf�llt in leichte Raserei, in blinder Wut
richtet er seinw Angriffe gegen den Verursacher. Er bekommt einen
Bonus auf die Attacke, ist jedoch auch um einen leichter zu treffen.
Der MW f�r den Gegner wird um einen gesenkt.\\

\hline

kritischer Misserfolg & Der Elf verf�llt in Raserei. Ha�- und
Wuterf�llt st�rzt auf den Verursacher. Er verliert s�mtliche
Paradewerte und schl�gt blindw�tig zu. Er bekommt einen Bonus von
1W10 auf einen Schaden, wird vom Gegner jedoch auch leichter
getroffen (siehe Misserfolg).\\

\hline

2-fach kritischer Misserfolg & Der Elf verliert die Kontrolle �ber
sich. Blind vor Ha� st�rzt er sich auf den Gegner, verursacht
doppelten Schaden, der von keiner R�stung gehalten wird und
ignoriert Verletzungsauswirkungen bis zum Ende des Kampfes.
Auswirkungen werden normal ermittelt, treten jedoch erst sp�ter in
Kraft. Er verzichtet auf jeglich Angriffs- und Paradeboni und wird
vom Gegner immer mit MW6 angegriffen.\\

\caption{Auswirkungen der schwarzen Seele im Kampf}
\end{longtable}


\subsection{Die Sprache der Elfen}
\subsubsection{Die Zahlen}
\par Der Aufbau der Zahlen im elfischen ist an sich recht einfach.
F�r die Ziffern von 0 bis 9 gibt es eigene Namen, wobei diese
jeweils zu drei �hnlich lautenden W�rter zusammengefasst werden
k�nnen. Die Zahlen ab 10 werden aus Kombinationen der zehn Ziffern
gebildet. Wie man leicht erkennen kann, wird das Aussprechen von
sehr gro�en Zahlen sehr viel Zeit in Anspruch nehmen. Elfen haben
ein langes Leben, Elfen haben viel Zeit. Der Umstand, dass die
elische Spache eine recht schnelle Sprache ist, kompensiert diesen
Nachteil jedoch erheblich besser, als es jetzt zu vermuten ist.
\par Die unten stehende Tabelle dient der Veranschaulichung.
\begin{longtable}{c|c|c|c|c|c|c|c|c|c}
0 &  net &1  & an &2  & ani &3 &  ane &4& en\\
\hline
5  & eni &6  & ene &7  & in &8 &ini &9 &ine\\
\hline
10 & an-net& 11&  an-an & $...$ & $...$ &100& an-net-net& $...$ & $...$ \\
\caption{Zahlen in elfischer Sprache}
\end{longtable}


                        \chapter{Kr�uterkunde\index{K!Kr�uterkunde}}
\label{kraeuterkunde}
\parpic[l]{\epsfig{file=pics/capitals/d.eps, scale=0.5}}ie Kunst
der Kr�uterkunde ist eine nicht unwichtige. Das richtige Kraut zur
richtigen Zeit erh�lt das Leben oder verk�rzt es rapide, je nach
Situation und Anwendungswunsch.
\par Viele Kr�utern werden nachweislich erhebliche Heilungskr�fte
nachgesagt, einige dienen den Sinnesfreuden und einige heimt�ckische
Mischungen und Kr�uter werden als Gift verwendet. Manches mal kann
ein Kraut sowohl das eine, als auch das andere, je nach Dosierung.
Es bedarf schon eines Experten auf em Gebiet der Kr�uterkunde, um
auch wirklich den gew�nschten Effekt zu erzielen. KR�ter werden
h�ufig auch dazu verwendet, um Zaubertr�nke oder dergleichen
herzustellen, diese Kunst f�llt jedoch in den Bereich der Alchimie
(siehe S. \pageref{alchimie}) und hier wird nicht weiter darauf
eingegangen. Dieses Kapitel widmet sich der Naturkunst als solche.

\par Im gro�en und ganzen wird in der Kr�uterkunde zwischen 3
Obergruppen unterschieden:
\begin{description}
\item [Wurzeln] alles was unterhalb der Erde w�chst und gedeiht
\item [Beeren] alles was an der Pflanze an Fr�chten w�chst und
gedeiht
\item [Bl�tter] der Rest der Pflanze
\end{description}

\par Auf den folgenden Seiten m�chte ich eine Reihe von Kr�utern und
deren Wirkung kurz darstellen. Die Textpassagen stellen
Zusammenfassungen aus dem \textit{Folianten f�r Wohl- und
Unwohlsein} von \mbox{Krautus} \mbox{Inmixus} dar. Dem geneigten
Leser w�nsche ich nun viel Spa� und viel Erfolg bei ersten eigenen
Verarbeitung der Wurzeln und Beeren.

\section{Wurzeln}

\section{Beeren}

\section{Gr�ser}

                        \chapter{Religion\index{R!Religion}}
\label{religion}
\par Die Menschen der Neuzeit glauben an die neuen G"otter.
Ihre Priesterschaft verbreitet den Glauben und lehrt ihr Wissen. Die
Frage die sich nun stellt: Warum die neuen G�tter? Was ist an diesen
G�tter neu und die noch entscheidenere Frage: Was ist mit den Alten?

\section{Die Alten G�ter}
\parpic[l]{\epsfig{file=pics/capitals/d.eps, scale=0.5}}ie folgenden
Aufzeichnungen stellen eine Sammlung von Geschichten aus vielen
verschiedenen Schriften dar. Es ist wie bei allen diesen
Aufzeichnungen davon auszugehen, dass nur ein Bruchteil von Wahrheit
in ihnen steckt. Nichts desto trotz stellen sie die einzigen
Anhaltspunkte dar, die wir zu der Zeit vor den Drachen haben.

\begin{quotation}
\par Es gab die G�tter, die von jeher �ber die Geschicke der
Menschen wachten. Es waren gutm�tige G�tter, die nichts vor sich in
den Schatten stellten und alles als gleichberechtigt betrachteten.
Vor knapp 3000 Jahren st�rzten die Drachen �ber die Menschheit
einher. Im Laufe der folgenden Jahre verloren die Menschen den
Glauben in die G�tter. Wer sollten denn dieser G�tter sein, die mit
Ansahen, ihre Kinder in ein Leben von Folter und Sklaverei gepresst
wurden?
\par Einer nach dem anderen gab den Glauben auf und mit den Jahren
ging das Wissen von den alten G�ttern unter. Die Priester der Zeit
gaben den Glauben an die G�tter entweder selber auf oder starben im
Glauben an ihre G�tter bei der Erhaltung ihres Wissen, wie etwa
durch heroische Taten, um ihre Existenz zu best�tigen. Jedoch gab es
kein Zeichen der Himmlischen. Letztentlich blieb auch kein Platz
mehr in den K�pfen der Menschen f�r diesen Glauben, die Drachen
nahmen ihnen den Verstand.
\end{quotation}



\section{Die Neuen G�tter}

                        \chapter{Das Tierleben\index{T!Tierleben}}
\label{tierleben}

\parpic[l]{\epsfig{file=pics/capitals/h.eps, scale=0.5}}ier findest Du Informationen zu allen m"oglichen Tieren, die
diese Welt beschreiten. Die Palette ist vielf"altig und reicht von
den einfachen Haus- und Nutztieren bis hin zu den Tieren, die uns
zur Nahrung dienen.
\begin{quotation}
\par\textit{\glqq Der Hund, die Kazz und das gemeine Ferd. Sie alle sind Tiere die uns zu Nuzzen sind. Der Kluhge weiz die Wild und Jagttiere zu unterscheiden um sich wohl zu naehren. \grqq} [\mbox{Lars} \mbox{Hinnerck}, Gro"sgrundbesitzer bei Hittenfelden]
\end{quotation}

\section{Haustiere}

\subsection{Hauskatze}

\subsection{Hund}

\section{Nutztiere}

\subsection{Kuh}

\subsection{Pferd}

\subsection{Hausschwein}

\section{Wilde Tiere}

\subsection{Wildschwein}

\subsection{Wild}

\subsection{Raubkatzen}

                        \chapter{Bestiarium\index{B!Bestiarium}}
\label{bestiarium}
\parpic[l]{\epsfig{file=pics/capitals/d.eps, scale=0.5}}ieses Kapitel
widmet sich allen Kreaturen, die auf unserem Boden wandern und nicht
den Pr�fungen der G�tter am letzten Tage standhalten werden. Doch
bis es soweit ist tut jeder gut daran, diese Untiere zu kennen und
sie zu meiden, denn meist sind sie m�chtiger.
\begin{quotation}
\par\textit{\glqq Unter Bestye oder Untyr im algemeinen
verstehen wir die Kreatuhren, die nicht ihrem Verstant sondern ihrem
Hunger folgen.\grqq} [\mbox{Lexxis} \mbox{Urmbrecht}, Magier der
Hochburg]
\end{quotation}
\section{Wesen der Nacht}

\subsection{Bluttrinker\index{B!Bluttrinker}\index{V!Vampir}}
\par Der Bluttrinker, im Volksmunde auch besser bekannt als Vampir,
ist ein Wesen der Finsternis. "Ahnlich dem D"amon ist sein Ursprung
widernat"urlicher Natur.

\subsection{Grottenolm}

\section{Wesen des Tages}
\subsection{Drachenkrieger\index{D!Drachenkrieger}}
\par beschreibung folgt

\section{�bernat�rliche und D�monen}

\subsection{Phantome\index{P!Phantom}}
\par Phantome sind die Geister verstorbener Lebewesen, die nicht den Weg ins Jenseits gefunden haben.\\
\par \begin{tabular}{p{2cm}p{2cm}p{2cm}p{2cm}}
\hline
\hline
\multicolumn{4}{c}{\textbf{Phantom}}\\
\hline
\hline
BEW & 15 & RS  &  4\\
TP  & 20 & PSI & 10\\
\hline
\multicolumn{4}{l}{magische Ber�hrung 3/0 (2W) AbB 10/6/4,}\\
\multicolumn{4}{l}{in den Gegner einfahren 1/0 (2W) AbB 8/4/0}\\
\multicolumn{4}{l}{dieser Angriff kann nur durch eine gelunge REA Probe}\\
\multicolumn{4}{l}{gegen das Kampfergebnis abgewehrt werden.}\\
\hline
\end{tabular}
\begin{description}
\item[Sonderregel Portal] Phantome erscheinen immer durch ein Nebelportal, durch dass sie sich materialisieren.
Solange das Portal existiert, k�nnen die Phantome nicht vollst�ndig vernichtet werden, sondern kommen in der n�chsten
Runde wieder.
\item[Sonderregel Lebensraub] Jeder verursachte Lebenspunktverlust beim Gegner wird dem Phantom als Lebenspunkt gut
geschrieben (auch �ber den urspr�nglichen Wert hinaus).
\end{description}
                        \chapter{Ma�e, Gewichte, W�hrung - einige Werte}
\section{Ma"seinheiten\index{M!Ma"seinheit}}
\begin{itemize}
\item 1 Meile\index{M!Meile} = 1000 Schritt\index{S!Schritt}
\item 1 Schritt = 3 Ellen\index{E!Elle}
\item 3 Ellen = 30 Finger\index{F!Finger}
\par \textbf{Anmerkung:} Urspr�nglich wurden zur L"angenbestimmung tats"achlich der Unterarm und der
 kleine Finger verwendet. Da es hierbei jedoch �fters zu Unregelm"a"sigkeiten kam, wurden
 sogenannte Normst"abe\index{N!Normstab} entworfen, die zur L"angenbestimmung verwendet werden
\caption{Ma"seinheiten}
\end{itemize}

\section{Gewichte\index{G!Gewicht}}
\begin{itemize}
\item 1 Quader\index{Q!Quader} = 20 Stein\index{S!Stein}
\item 1 Stein = 10 Fund\index{F!Fund}
\item 1 Fund = 20 Unzen\index{U!Unze}
\par \textbf{Anmerkung:} Zur Gewichtsbestimmung werden sogenannte Normgewichte\index{N!Normgewicht}
 verwendet, um Unregelm"a"sigkeiten zu vermeiden. F�r alchimiste Zwecke gibt es spezielle
 Messgef"a"se\index{M!Messgef"a"se}, die Markierungen f�r verschieden F�llmengen haben. Als Ma"seinheit
 werden hierbei Unterteilungen der Unze verwendet. Zur Bestimmung des Gewichtes bei Pulvermengen gibt es
 entsprechend leichte Normgewichte.
\caption{Gewichte}
\end{itemize}

\section{W"ahrung\index{W!W�hrung}}
\begin{itemize}
\item 1 Goldbarren\index{G!Goldbarren} = 1000 Goldst�cke\index{G!Goldstueck}
\item 1 Goldst�ck = 10 Silberlinge\index{S!Silberling}
\item 1 Silberlinge = 10 Bronzem�nzen\index{B!Bronzem�nze}
\item 1 Bronzem�nze = 10 Kupferst�cke\index{K!Kupferstueck}
\par \textbf{Anmerkung:} Als �bliche W"ahrung sind Silberlinge, Bronzem�nzen und Kupferst�cke nahezu �berall zugelassen. In gr�"seren St"adten nehmen Kaufleute auch Goldst�cke an. Nur sehr gro"se Kaufh"auser oder Adlige rechnen und handeln mit Goldbarren.
\caption{W"ahrung}
\end{itemize}

\section{Zum Verst"andnis}
\begin{itemize}
\item 1 Schritt = 1 Meter\index{M!Meter}
\item 1 Stein = 5 Kilo\index{K!Kilo}
\end{itemize}

                        \chapter{Geschichte und Legenden}
\label{legendenundgeschichten}

\begin{quotation}
\par \textit{\glqq Eine Zeit der Drachen\index{D!Drachen} und der Magie\index{M!Magie}.\grqq}\\
Arkan sprach etwas leiser. \textit{\glqq So musst Du Dir vorstellen, war dieses Land vor dem Umbruch.\grqq}
\par Arn wurde aufmerksamer. Wenn sein Meister anfing zu fl"ustern, dann stand eine Geschichte bevor. Eine jener Legenden\index{L!Legende}, die dieses Land gepr"agt und zu dem gemacht haben, was es jetzt ist. \textit{\glqq Was ist dann passiert? Was hat sich ver"andert.\grqq}
\par \textit{\glqq Um zu verstehen warum sich alles ver"anderte, muss man verstehen, wie es vorher war. Vor allem, bevor alles begann ... \grqq}
\end{quotation}

\section{Die Vier Zeitalter\index{Z!Zeitalter, die vier}}

\parpic[l]{\epsfig{file=pics/capitals/d.eps, scale=0.5}}ie Weltgeschichte l"asst sich grob in 3 Zeitabschnitte einteilen und wir stehen bereits am Anfang eines weiteren. Die Chronisten haben mit jedem einschneidenden Erlebniss eine neue Zeitrechnung angefangen, so dass man bis zum heutigen Tage von drei Zeitaltern oder drei Jahrtausenden spricht. Jedes Millenium hat das Land gezeichnet bzw. gepr"agt und die Entwicklung der Bewohner beeinflusst.
\par Die Jahreszahlen, die in den allgemeinen Tabellen verzeichnet sind, sollten f"ur den Leser jedoch weniger von Bedeutung, als die Ereignisse, die dahinter stehen. Da gerade in den Jahrhunderten der Finsterniss und des Krieges etwas weniger Genauigkeit auf das Datum als auf das "Uberleben und die "Uberlieferung gelegt wurde, m"ogen hier gro"sz"ugig einige Abweichungen in Kauf genommen werden.

	\subsection{Erstes Zeitalter\index{Z!Zeitalter, erstes}: Drachenkrieg\index{D!Drachenkrieg}}
\begin{quotation}
\par\textit{\glqq Um etwas Neues zu erschaffen muss Altes zerst"ort werden. In den Existenzebenen ist nur Platz f"ur Eines von Beidem. \grqq}
\end{quotation}
\parpic[l]{\epsfig{file=pics/capitals/n.eps, scale=0.5}}ur wenig ist
bekannt �ber die Welt vor dem ersten Zeitalter\index{Z!Zeitalter,
erstes}, liegt dieses doch mittlerweile knapp 3000 Jahre zur"uck und
die Augenzeugen jener Epoche\index{E!Epoche} sind rar geworden. In
Wahrheit gesprochen sind die einzigen Wesen, die ein solches Alter
erreichen k"onnen die Drachen\index{D!Drache}. Und von denen wei"s
ja nun jedes Kind, dass sie ausahmslos von der Bildfl"ache
verschwunden sind. Was die Chronisten\index{C!Chronist} aus dieser
Zeit zu berichten wissen beruht auf Schriften und einigen
Steinmei"seleien, die in ungef"ahr diese Zeit datiert wurden.
\par In einen sind sich jedoch alle einig: Das erste Zeitalter\index{Z!Zeitalter, erstes} begann mit dem Erscheinen der Drachen\index{D!Drache}.
\par Es muss der Ausbruch des Chaos\index{C!Chaos} gewesen sein.
Begleitet von Springfluten und Vulkanausbr"uchen erschienen sie in gigantischen Schw"armen.
Der Himmel verdunkelte sich und Donnerst"urme erschienen am Horizont.
Mit ihren Schwingen erschufen sie Wirbelst"urme, die das Land zerst"orten.
Und mit Ihnen kamen die Drachenkrieger\index{D!Drachenkrieger}.
Humanoide Lebensformen mit dem Aussehen eines Drachen;
Geschuppte Haut, einige ihrerseits mit Schwingen, andere mit der Gabe des Drachenodems\index{D!Drachenodem}.
\par Keiner kann so genau sagen woher sie gekommen sind oder warum. Doch die Frage nach dem \textit{Warum} schien eher ein verzweifeltes Flehen nach einer Erkl"arung als eine ernst gemeinte Frage zu sein. Die Kreaturen zogen durch das Land und s"ahten Zerst"orung. Wer nicht get"otet wurde, wurde versklavt.
\par Einige Theorien gaben den Schwarzmagiern\index{S!Schwarzmagier} Schuld an der Beschwrung der Drachen. Andere wiederum suchen die Erkl"arung in dem Zorn der G"otter.
\par Nach einem Krieg, der fast vier Jahrhunderte anhielt und dessen Sieger schon bei seinem Ausbruch feststand, schien Ruhe in das Land einzukehren. Dieses lag weniger an der Tatsache, dass der Kampfesmut versiegte sondern eher daran, dass einfach nichts mehr da war, was sie vernichten konnten.
\parpic[r]{\epsfig{file=pics/moonshadow.eps, scale=0.3}}
Die "Uberlebenden dieser Zeit wurden in Lagern als Sklaven gehalten und zum Bau neuer Geb"aude gezwungen. Die Krieger sprachen von den Vorbereitungen zur Ankunft der F"unf\index{F!Fnf, die}. Es wurden gigantische Tempel und unterirdische Anlagen gebaut. Die Oberfl"ache war unbewohnbar geworden.
\par Vor etwas mehr als 2000 Jahren, am Ende des ersten Zeitalters, erschienen sie. Wer bis dahin glaubte, dass die Drachen\index{D!Drache}, die die Kriege f"uhrten zu den gr"o"sten Kreaturen z"ahlte, wurde sp"atestens jetzt eines besseren belehrt. Die Herrscher der Drachen\index{D!Drache}, f"ur die Jahrhunderte eine neue Welt erschaffen wurde, hielten es f"ur an der Zeit sich zu offenbahren. Monstr"ose Kreaturen mit der Macht der Urgewalten\index{U!Urgewalt}. Jede von Ihnen als elementare Kraft. Sie sollten bekannt werden als die Inkarnation dessen, was sp"ater die Magie\index{M!Magie} wurde.
\\
\par \textit{Antares}\index{A!Antares}, die Kraft des Geistes,
\par \textit{Par}\index{P!Par}, die Kraft des Wassers,
\par \textit{Lucien}\index{L!Lucien}, die Kraft der Luft,
\par \textit{Epher}\index{E!Epher}, die Kraft des Feuers und
\par \textit{Orgoth}\index{O!Orgoth}, die Kraft der Erde.
\\
\par Die Welt wurde aufgeteilt und jede der Urgewalten\index{U!Urgewalt} herrschte "uber sein Reich unabh"angig von den Anderen. Zuerst "anderte sich nicht viel an dem Zustand. Die Menschen waren weiter Sklaven\index{I!Index} und jede Stunde ihres erb"armlichen Lebens damit besch"aftigt einen Stein nach dem anderen aus dem Fels zu schlagen, nur um dann einen Stein nach dem anderen "ubereinander zu Mauern zu stapeln.
\par Die meisten Zitadellen und Bergfesten stammen aus jener Zeit.

	\subsection{Zweites Zeitalter\index{Z!Zeitalter, zweites}: Herrschaft der Finsternis\index{H!Herrschafft der Finsternis}}
\par �ber das zweite Zeitalter existieren bisher nur wenige verl�ssliche Berichte, so
dass wir an dieser Stelle erst noch weitere Recherchen anstellen m�ssen.

	\subsection{Drittes Zeitalter\index{Z!Zeitalter, drittes}: Der Umbruch\index{U!Umbruch}}
\par Die Hauptausrichtung des dritten Zeitalters galt dem Wiederaufbau. Die ...

	\subsection{Viertes Zeitalter\index{Z!Zeitalter, viertes}: Drachend"ammerung\index{D!Drachendaemmerung}}
\parpic[l]{\epsfig{file=pics/capitals/d.eps, scale=0.5}}ie Zukunft\index{Z!Zukunft} diese Jahrtausends\index{J!Jahrtausend}
steht noch in den Sternen, hat es doch noch nicht einmal
angebrochen. Dennoch versprechen die Astrologen ein aufreibendes
Zeitalter\index{Z!Zeitalter}.
\par Nicht nur, dass mit diesem Jahrtausend\index{J!Jahrtausend}
der Bannspruch\index{B!Bannspruch} "uber die
Drachen\index{D!Drachen} seine Wirkung verliert, sondern auch die
Konzentration der Magie weckt Bef"urchtungen in den "Altesten. Zu
tief sitzt der Schock der ersten Zeitalter\index{Z!Zeitalter} und
die Unterdr"uckung. Jetzt, da der Bannspruch\index{B!Bannspruch}
bald seine Wirkung verliert, muss dringend ein Weg der Erneuerung
gefunden werden.
\par Doch die Kenntnisse vergangener Tage sind verlorend und die
Zeit zur Entwicklung neuer Zauber knapp. Die weisesten und
m"achtigsten Magier haben sich auf die Suche nach den Artefakten des
zweiten Zeitalters gemacht, um Hinweise auf das Auftauchen und die
Vernichtung der Drachen zu finden. Die Spuren jener Zeit sind stark
verwischt und die Ebenen des Krieges verw"ustet. Viel erschwerender
kommt jedoch hinzu, dass die Magier damals "uber den Kontinent
verteilt waren, als sie den Bannspruch wirkten, ebenso deren
Artefakte.

	\subsection{Zeitleiste}
\parpic[l]{\epsfig{file=pics/capitals/d.eps, scale=0.5}}ie Zeitrechnung wird mit Beginn eines jeden Zeitalters
neu gez"ahlt. Spricht man vom Jahr 546 ist das Jahr 546 in dem
aktuellen Zeitalter gemeint. Will man "uber ein Jahr in einem
anderen Zeitalter sprechen, dann setzt man die Nummer des Zeitalters
vorweg. Das Jahr 546 im zweiten Zeitalter ist also das Jahr 2.546.
\par Die Angaben in den Tabellen \ref{tabelle_zeitalter1},
\ref{tabelle_zeitalter2} und \ref{tabelle_zeitalter3} (ab Seite
\pageref{tabelle_zeitalter1}) sind historischen Schriften entnommen.
Da gerade in der dunklen Zeit keine M"oglichkeit zur genauen
Zeitbestimmung bestand, k"onnen hier Differenzen zu anderen
Schriften durchaus m"oglich sein.
\par Das geradezu faszinierende an der Vergangenheit dieses
Landes ist die Kontinuit"at mit der die Geschichte geschrieben wird.
In der Vergangenheit stand am Ende eines Millenniums immer ein
Umbruch, der das definitve Ende des vergangenen und den Anfang eines
neuen Zeitalters bedeutete. Gerade diese mystischen Ereignisse sind
es, die den Weisen und Historikern zur Zeit den Angstschwei"s auf
der Stirn stehen lassen. Zur Zeit schreiben wir das Jahr 3.997 und
bis zum Beginn des n"achsten Jahrtausend sind es nur noch drei
Jahre. Was wird uns erwarten?

% Erstes Zeitalter
\begin{longtable}{|r|p{10cm}|}
\hline
Jahr   & Wichtige Ereignisse \\
\hline
0      & Erscheinen der Drachen; Beginn des Jahrhundertkrieges\\
397    & Endg"ultige Versklavung der Menschheit\\
ab 407 & Bau der Unterirdischen Katakomben und Labyrinthe\\
647    & erste Freiheitsk"ampfergruppen bilden sich\\
870    & Angriff auf das Zentrum der Drachen\\
876    & Erscheinen der Urgewalten\\
877    & Niederschlagung aller Widerst"ande und endg"ultiger Fall der Menschheit in die Sklaverei\\
992    & Ein Magier in der Sklaverei unter Antares Truppen erkennt als Erster Mensch den Einfluss der Urgewalten auf die magischen St"urme und erforscht diese im Geheimen\\
\hline
\caption[Zeitleiste: Erstes Zeitalter]{Wichtigste Ereignisse im ersten Zeitalter}
\label{tabelle_zeitalter1}
\end{longtable}
% Zweites Zeitalter
\begin{longtable}{|r|p{10cm}|}
\hline
Jahr   & Wichtige Ereignisse \\
\hline
0      & Erste Forschungsergebnisse bei der Nutzung der neuen magischen Quellen, erster Hoffnugsschimmer zur Nutzung dieser f"ur die Freiheitserringung\\
1      & Erkennung erster Nebenwirkungen bei der Nutzung der neuen magischen Quellen\\
12     & Entwicklung des ersten Zaubers\\
63     & Einer gro"se Gruppe von Freiheitsk"ampfern gelingt auf magische Weise der Weg in die Freiheit, Widerstandsaufbau beginnt, in den n"achsten Jahren finden immer wieder Befreiungen von Gefangenen aus den Festungen statt\\
80     & Bau erster befestigten Siedlungen in still gelegten Minenwerken\\
750    & Erstes Auftauchen der Priesterschaft der G"otter, wirksame Wunder "uberzeugen die Menscheit von der Existenz der G"otter\\
956    & Die Urgewalten finden sich alle in der Zitadelle Orgoths in den Westlanden ein\\
957    & Sturz der Festung Luciens an der Ostk"uste\\
999    & Versammlung der Truppen in den Ebenen; finale Schlacht am Trauerfelsen\\
\hline
\caption[Zeitleiste: Zweites Zeitalter]{Wichtigste Ereignisse im zweiten Zeitalter}
\label{tabelle_zeitalter2}
\end{longtable}
% Drittes Zeitalter
\begin{longtable}{|r|p{10cm}|}
\hline
Jahr   & Wichtige Ereignisse \\
\hline
0      & Vernichtung / Verbannung der Drachen; Zauberspruch des Millenniums\\
16     & Gr"undung des 18er-Rats\\
18     & Errichtung der Festung Patria Pacis im Zentrum des gro"sen Kontinents\\
25     & Gr"undung der ersten neuen Siedlungen in den Zitadellen der Urgewalten. Sie tragen die Namen Pax Antares\index{P!Pax Antares} (Reich der Mitte), Pax Par\index{P!Pax Par} (S"udk"uste), Pax Lucien\index{P!Pax Lucien} (Ostk"uste), Pax Epher\index{P!Pax Epher} (Der Hohe Norden) und Pax Orgoth\index{P!Pax Orgoth} (Westlande). Sie gelten als die Hauptst"adte der 5 gro"sen Reiche\\
28     & Gr"undung einer neuen Magiergilde, zur Wahrung und Erforschung der neuen Magiequelle, die als \textit{\glqq Wahre Magie\grqq} bekannt wurde\\
29     & Errichtung des Turms der Magier im \textit{Land der Toten} als Sitz des 18er-Rats und Forschungs- und Schulungszentrum f"ur die \textit{Wahre Magie}\\
30     & Wahl des ersten Kaisers "uber die 5 Reiche und Wahl der 5 K"onige zur Verwaltung dieser\\
75     & R"ucktritt des Kaisers mit 128 Lebensjahren in den Ruhestand, Aufl"osung dieses Titels und damit Gr"undung von 5 unabh"angigen K"onigreichen ohne gemeinsame Verwaltung\\
997    & aktuelles Zeitgeschehen\\
\hline
\caption[Zeitleiste: Drittes Zeitalter]{Wichtigste Ereignisse im dritten Zeitalter}
\label{tabelle_zeitalter3}
\end{longtable}


\begin{quotation}
\par \textit{\glqq Nun, mein Sch"uler, hast du eine Menge "uber die Historie unserer V"olker gelernt. So steht es geschrieben und so ist es geschehen.\grqq}
\par Arkan setzte sich zur"uck in seinen Stuhl und legte die H"ande gefaltet auf den Tisch. Herausfordernd blickte er seinen Sch"uler an. Arn war aufgew"uhlt. Noch nie zuvor hatte er die Geschichte in solcher Vollkommenheit geh"ort. Trotzdem hatte er das Gef"uhl, als wenn es etwas fehlen wrde.
\par \textit{\glqq Meister\grqq}, z"ogernd sprach er weiter, \textit{\glqq Was ist mit den anderen Legenden, die in den B"uchern stehen und von Generation zu Generation weitergegeben werden. Was ist davon wahr und was erfunden?\grqq}
\par \textit{\glqq Nun, mein junger ungeduldiger Sch"uler\grqq}, Arn meinte ein leichtes zufriedenes L"acheln "uber des Meisters Mund huschen zu sehen, \textit{\glqq Wie du wei"st sind unsere L"ander und Geschichten voll von Legenden. Jede einzelne zu rezitieren w"urde den Rest deines Studiums in Anspruch nehmen und ich habe noch einiges anderes f"ur dich auf dem Lehrplan stehen. Es gibt durchaus noch Wichtigeres. Wie weit bist du mit der Abschrift aus dem Buche \textbf{Georn}?\grqq}
\par Entt"auschung breitete sich auf Arns Gesicht aus. \textit{\glqq Ihr habt nat"urlich recht Meister. Die Abschrift wird nocht etwas Zeit in Anspruch nehmen. Ich werde mich allerdings sofort daran setzen.\grqq}
\par \textit{\glqq Den Flei"s lobe ich mir. Deswegen schlage ich vor, dass wir uns morgen Abend zur selben Stunde wieder hier treffen, und ich will sehen, ob ich einige interessante Legenden finde.\grqq}
\par Arn konnte sich ein leises Schmunzeln nicht verkraften. Sein Meister ist einfach unberechenbar. Aber mit dieser Vorfreude im Kopf geht die Arbeit an den Abschriften gleich doppelt so leicht.

\end{quotation}
\section{Die Mondkinder\index{M!Mondkinder}} 
\label{mondkinder}

\section{Die Harlekine\index{H!Harlekine}}
\label{harlekine}

% Elfen sie versuchen die schwarze Seele zu kontrollieren

\section{Der Turm der Magier\index{T!Turm der Magier}}
\label{turmdermagier}

\parpic[l]{\epsfig{file=pics/capitals/d.eps, scale=0.5}}er \textit{Turm der Magier} wurde Anfang dieses Jahrtausends als
Forschungsst"atte f"ur die \textit{Wahre Magie} errichtet. Er ist
gleichzeitig auch der Sitz des 18er-Rats.
\par Der Rat der 18 setzte sich urspr"unglich aus den
Aufstandsf"uhrern von damals zusammen. Da f"ur die normale
Menschheit die Bedeutung der \textit{Wahren Magie} und die Wahrung
derer, sowie der Schutz vor den Drachen und die Aufrechterhaltung
des Bannzaubers in den Hintergrund geraten ist, sind dessen
Mitglieder nur noch Mitglieder der Magierzirkel. Tats"achlich ist es
so, dass viele nicht einmal mehr von der Existenz dieses Rates,
geschweige denn um seine Bedeutung wissen.
\par Im Gegensatz zu den T"urmen und Schulungseinrichtungen der Zirkel
ist dieser Turm nicht einer bestimmten Ausrichtung gewidmet, denn
der Zweck gilt nicht der "ublichen Magie. Hinter den T"uren dreht
sich in Forschung und Schulung alles um die \textit{Wahre Magie}.
Dieser Umstand ist nicht zuletzt einer der Gr"unde, warum der Turm
im \textit{Land der Toten}\index[regionen]{L!Land der Toten}, also im westlichen Reich errichtet
wurde. Hier sind die Str"omungen der Magie am st"arksten, da hier
auch vor knapp 1000 Jahren die entscheidende Schlacht gegen die
Drachen statt fand.
\par Ein zweiter Grund ist in dem Bannzauber zu finden, der zum Wohle 
aller gesprochen wurde. Dieser wird aus der W"uste gespeist, dem Ort 
des Sprechens, und f"ur dessen Aufrechterhaltung ist es notwendig, 
dass sich jederzeit m"achtige Zauberk"unstler in dessen N"ahe befinden.

\section{Das Portal\index{P!Portal, das}}
\label{dasportal} [Ein Bericht von \mbox{Graf} von
\mbox{Hohenbr"uck}\index[personen]{H!von Hohenbr�ck}, Kartograph im Auftrag des k"oniglichen Amtes
f"ur Landerfassung]\\

\parpic[l]{\epsfig{file=pics/capitals/d.eps, scale=0.5}}ie Reisenden
des "ostlichen Reiches berichten von der Entedckung eines
gigantischen Bauwerkes in den Ebenen. Seine Turmspitzen ragen weit
in den Himmel hinein und seine Erscheinung ist faszinierend und
erschreckend zu gleich.

\par Doch warum ist dieses Bauwerk noch nicht in den Karten verzeichnet?

\par Im Auftrag unseres Amtes aus dem Jahre 956 machte ich mich mit
einer Expedition auf den Weg, den Standort des Turmes zu vermessen
und ihn in die Karten zu "ubertragen. Ein Unternehmen, was sich als
weitaus schwieriger herausstellen sollte, als ich es zuerst
angenommen habe.

Einige Reisende der Strecke berichteten diesen Turm niemals gesehen
zu haben, und dass obwohl sie die identische Strecke gefahren sind
wie andere Reisende. Andere berichteten das Bauwerk nur auf einer
Strecke gesehen zu haben, eine letzte Gruppe sah es mal auf der
einen, mal auf der andere Strecke.

Schon im Laufe der Voruntersuchungen verweifelte ich an diesen
Informationen. Wurde ich losgeschickt ein Phantom zu verzeichnen? Da
ich die Hoffnung aufgegeben hatte verl"assliche Informationen "uber
den Standort zu bekommen, stellte ich eine Expedition auf, um den
Standort pers"onlich ausfindig zu machen. Die ben"otigten
finanziellen Mittel wurden mir zur Verf"ugung gestellt und der
Aufbruch in das fr"uhe n"achste Jahr datiert. F"ur die Dauer der
Expedition werde ich ein Forschungsbuch anlegen und unsere Funde
genauestens dokumentieren.

\begin{quotation}
\begin{flushright}\textit{997-7-7}\end{flushright}
\textit{Mein K"onig und Gebieter,}\\

\textit{mit Freuden bin ich Ihrer Anforderung nachgekommen und habe die Unterlagen und Manuskripte 
nach der von Ihnen gew"unschten Expedition durchsucht. Doch zu meinem gro"sen Leid kann ich nur wenig 
berichten.}\\
\textit{Aus den Unterlagen geht hervor, dass Graf von Hohenbr"uck seinerzeit als vermisst gemeldet 
wurde, R"uckmeldungen gab es keine. Auch Nachforschungen bei den Nachfahren des ehrw"urdigen Grafen 
verliefen ergebnislos.}\\
\textit{Schon wollte ich die Suche abbrechen und euch die Ergebnisse mitteilen, als mich der Zufall 
zu einigen noch nicht katalogisierten Dokumenten f"uhrte, die ich unseren Archiven ausmachen konnte. 
Meinem Erstaunen kann gar nicht genug Ausdruck verliehen werden und schon gar nicht kann ich in Worte 
fassen was ich fand. Zu urteilen obliegt jedoch nur Euch, meine k"onigliche Hoheit, und deswegen "ubersende 
ich euch beiliegend das Manuskript zur Durchsicht.}\\

\textit{Hochachtungsvoll,}\\
\textit{Pablo Hiluia\index[personen]{P!Pabloa Hiluia}}
\end{quotation}

\par Mit dem Schreiben des Bibliothekars seiner Hoheit wurde ihm eine gebundene Mappe mit einigen 
vergilbten Bl�ttern �berreicht. Das Deckblatt tr�gt die Insignien des Grafen von Hohenbr�ck und der 
Themenbeschreibung folgend scheint es sich hierbei um das Forschungsbuch zu handeln. Die Bl�tter sind 
arg in Mitleidenschaft gezogen worden, die Schrift ist jedoch noch gut zu lesen. Die Seiten machen 
einen sortierten Eindruck, jeder Eintrag wurde auf einer eigenen Seite dokumentiert.


\begin{longtable}{|p{15cm}|}
\hline
5. Tag im ersten Monat des 957. Jahres im Zeitaltes des Umbruchs\\
\\
Nachdem ich mich nun entschlossen habe, dem Ph�nomen des Turmes pers�nlich auf den Grund zu gehen, 
habe ich eine Expedition zusammengestellt. Morgen wird der erste Tag unserer Reise sein und wir sind 
voller Hoffnung, das Ziel unserer Reise in einigen Wochen erreicht zu haben.\\
\hline
\end{longtable}

\begin{longtable}{|p{15cm}|}
\hline
6. Tag im ersten Monat des 957. Jahres im Zeitaltes des Umbruchs\\
\\
Der Aufbruch ging planm��ig voran, in der Morgend�mmerung haben wir Pax Lucien verlassen und sind mit 
der Karawane in Richtung Westen auf den bekannten Wegen aufgebrochen. Die Stimmung ist gut.\\
\hline
\end{longtable}

\begin{longtable}{|p{15cm}|}
\hline
14. Tag im ersten Monat des 957. Jahres im Zeitaltes des Umbruchs\\
\\
Heute sahen wir zum ersten Mal den Turm am Horizont, wir haben also das erste Etappenziel unserer 
Reise erreicht. Aus der Ferne betrachtet wirkt der Turm gewaltig, selbst mit unseren Me�instrumenten 
k�nnen wir nicht ausmachen, welche H�he er hat. Wir vermuten ihn jedoch noch in weiter Ferne, da 
keinerlei Struktur auf seiner Oberfl�che erkennbar ist.\\
\hline
\end{longtable}

\begin{longtable}{|p{15cm}|}
\hline
15. Tag im ersten Monat des 957. Jahres im Zeitaltes des Umbruchs\\
\\
Entsetzen machte sich heute nach dem Erwachen breit, unser Ziel war am Horizont verschwunden. 
Einige der abergl�uberischen Karawanisten konnten nur mit M�he davon �berzeugt werden, dass es 
wohl an einer Wetterfront liegt, die den weit entfernt liegenden Turm verdeckt.\\
Ich f�r meinen Teil habe� Zweifel an meiner Notl�ge und besinne mich der Erz�hlungen �ber diesen Turm.\\
\hline
\end{longtable}

\begin{longtable}{|p{15cm}|}
\hline
21. Tag im ersten Monat des 957. Jahres im Zeitaltes des Umbruchs\\
\\
Wie bereits dem Eintrag vom Vortag zu entnehmen ist, haben wir dem Turm seit gestern Abend wieder in 
Sicht. Er liegt immer noch auf der gleichen Route, doch scheinen wir ihm in den letzten Tagen nicht 
n�her gekommen zu sein. Der Mi�mut in der Gruppe macht sich weiter breit.\\
\hline
\end{longtable}

\begin{longtable}{|p{15cm}|}
\hline
23. Tag im ersten Monat des 957. Jahres im Zeitaltes des Umbruchs\\
\\
�berraschen und Entsetzen macht sich heute nach dem Erwachen breit, unser Lager befindet sich am 
Fu�e des Turmes. Einige unserer Leute haben sich im fr�hen Morgen alleine auf den R�ckweg gemacht, 
nur der Kern der Expedition bleibt zur�ck. Am Fu�e des Turmes befindet sich ein gro�es Tor, wir 
werden im Laufe des Tages einige Untersuchungen anstellen und schauen, ob wir das Tor �ffnen k�nnen. 
Durch das pl�tzliche Auftauchen dieses Bauwerks, ist es mir jedoch nicht m�glich, seine genaue Lage zu 
verzeichnen, vielleicht auf dem R�ckweg.\\
\hline
\end{longtable}

\par Hier endet des Manuskript. Angeh�ngt befinden sich einige Dokumente, die den Ursprung der Unterlagen 
dokumentieren und dessen Einlagerung in die Bibliothek im Jahre 954. �ber den Verbleib des Verfassers 
oder seiner Expedition ist leider nichts bekannt. Auch wird nicht erw�hnt, wie die Dokumente in den 
Besitz der Bibliothek kamen.

\section{Das Tal der Verlorenen und Verdammten\index[regionen]{T!Tal der Verlorenen und Verdammten}}
\label{verloreneundverdammte}
\parpic[l]{\epsfig{file=pics/capitals/h.eps, scale=0.5}}inter dem
\textit{Steinernen Schwert}\index[regionen]{S!steinerne Schwert, das} gelegen, findet der Reisende das
\textit{Tal der Verlorenen und Verdammten}. Ein recht seltsamer Name
f"ur eine unbekannte Gegend, aber wir wollen in den folgenden
Abs"atzen seine Bedeutung erl"autern. Diese ist n"amlich nicht
unbedingt eindeutig.
\begin{figure}[hbtp]
\begin{center}
\epsfig{file=pics/wueste2.eps, scale=0.5}
\caption{Der Eingang zum Tal der Verlorenen \& Verdammten}
\end{center}
\end{figure}
\par Die Ureinwohner dieser Grenzregion pflegen die sterblichen �berreste
ihrer Toten zu verbrennen und in Urnen zu f�llen. Diese werden auf
Wagen geladen und in regelm"a"sigen Abst"anden durch den Gebirgspass
zum Taltor gebracht. Dort wird der Wagen, der von einem Maultier
gezogen wird, seiner selbst "uberlassen. Die Tiere trotten voraus
ins Unbekannte.
\par Uneingeweihte fragen sich nun: Was macht das f"ur einen Sinn?
\par Ganz einfach. Der Glaube der Einwohner besagt, dass es an einem
Ort nicht gen"ugend Platz f"ur die Seelen aller Toten gibt. Aus
diesem Grund werden nur H"ohergestellte auf den lokalen Friedh"ofen
beigesetzt. Normale B"urger werden den langen Weg in den S"uden
gebracht und dort in den W"aldern verscharrt. Durch die
weitl"aufigen und hohen B"aume finden hier mehr Tote Platz.
Diejenigen, die kein reines Leben f"uhrten und sich zu Lebzeiten
etwas zu Schulden haben kommen lassen, werden auf den Weg in den
Norden geschickt. Immerhin verdient es Anerkennung, was f�r ein
Aufwand betrieben wird, um die Ausgesto�enen zu Grabe zu tragen.

\par Der normale Reisende ermittelt die Bedeutung des Tals aus
einem anderen Umstand. Expeditionen, die in diese Gegen gef"uhrt
wurden, sind niemals zur"uckgekehrt.

\section{Helden und Schurken}
\label{heldenundschurken}
\par Die folgenden Kapitel stellen ein paar der bekannten und
legend�ren Personen und Gruppierungen der letzten Jahrhunderte vor.
Einiges basiert auf Tatsachen, anderes m�ge Legende sein.

\subsection{Der Orden der schwarzen Rose\index{O!Orden der schwarzen Rose}}
\parpic[l]{\epsfig{file=pics/capitals/d.eps, scale=0.5}}er Orden der schwarzen Rose ist ein Zusammenschluss von
Magiewirkern aus allen m�glichen Fachrichtungen. Sein Ziel ist das
Studium und das damit verbundene Erlangen von Wissen.
\par Diese Tatsachen alleine w�rde die Gemeinschaft nicht
spektakul�rer erscheinen lassen, als jeden x-beliebigen Zirkel von
Magier, h�tte sie sich nicht den Fachrichtungen Dimension und
Beschw�rung verschrieben.
\par Die Mitglieder der schwarzen Rose operieren im Geheimen.
Niemand wei�, wo sie sitzen, niemand wei� wer sie sind. Man sagt
jedoch, hinter vorgehaltener Hand, dass es hochrangige Mitglieder der
Zirkel seinen, die den Orden gegr�ndet haben und ihn f�hren.
\par Neue Mitglieder werden gezielt ausgew�hlt. Nach einer Reihe
geheimer Pr�fungen, von denen die Aspiranten nichts mitbekommen
werden, wird ihnen die Mitglidschaft angeboten. Was mit denen
passiert, die ablehnen ist ebenfalls nur Spekulation.

\subsection{Die Sturmreiter}

\subsection{Die Freibeuter des S�dens}

\subsection{Das Phantom der Ostk�ste}

\subsection{Die Ritter der Nordmark\index{R!Ritter der Nordmark}}
\parpic[l]{\epsfig{file=pics/capitals/d.eps, scale=0.5}}ie Ritter der Nordmark haben bei der Verteigung des �stlichen
Reiches gegen die Orkhorden 867 ihrem F�rsten gro�en Ruhm und Ehre
gegen�ber dem K�nig gebracht. Zahlreiche marodierende und
brandschatzende Horden von Orks fielen von Norden in das Reich ein,
ihre Zahl war so gewaltig, dass kein Heer alleine ausreichte, um sie
zu stoppen. Den Rittern der Nordmark um F�rst Reinier zur Nordmark\index[personen]{P!Reinier zur Nordmark},
zu dieser Zeit noch ein kleiner Orden, gelang es die Orks mit
einigen Kriegslisten aus dem Reich zur�ck in den Norden zu
vertreiben.
\par Renier wurde von dem K�nig f�r die Tapferkeit mit
einer Reichserweiterung belohnt, man bot ihm sogar einen gr��eren
Titel. Reinier lehnte dankend ab, sein Lohn sei der Triumph, er habe
nur im Dienste des K�nigs f�r den K�nig gehandelt, wie es sich f�r
einen Ritter geh�rt. Mehr als beeindruckt von diesen Worten erlie�
der K�nig einen Beschluss, der die Ritter der Nordmark zu den
Rittern des Reiches ernannte. Er stellte Reinier die n�tigen Mittel
zur Verf�gung den Orden auszubauen und noch weitere Krieger
aufzunehmen.
\par Der Orden wurde seit dem st�ndig erweitert und stellt seit
Mitte dieses Jahrhunderts das Heer des Ostreiches. Die Krieger des
Ordens sind f�r ihre Volksn�he und ihre Ritterlichkeit bekannt.

\section{Nachtklinge\index{N!Nachtklinge}}
\label{nachtklinge} [Aus dem Tagebuch des \mbox{Alessandro}, Aufzeichnung ohne Datierung, zweites Zeitalter]\\

\parpic[l]{\epsfig{file=pics/capitals/n.eps, scale=0.5}}achtklinge\index{N!Nachtklinge}, das Schwert des Ritters
Gunter zu Hohenaspen\index[personen]{G!Gunter zu Hohenaspen}. Legenden ranken sich um diese Waffe aus den Drachenkriegen. Gefertigt von den Kriegsschmieden\index{K!Kriegsschmied}
der Zwerge, gesegnet mit ihren einzigartigen Runen geh�rt Nachtklinge zu den Waffen, die in der Schlacht an vorderster Front gegen die
Drachen gef�hrt wurden. Man berichtet, dass der Tr�ger dieses Schmuckst�ckes sich furchtlos in den Zweikampf mit
diesen Bestien st�rzt und Nachtklinges scharfe Schneide ihre Schuppen durchbricht.

\par Wie jede Waffe aus der Schmiede der Zwerge ist Nachtklinge das Ergebnis bester Schmiedekunst und als solches
leicht und behende zu f�hren. Das Material welches bei ihrer Erschaffung benutzt wurde ist nicht zu bestimmen, man
vermutet jedoch, dass es jenes geheimnisvolle Meteoreisen\index{M!Meteoreisen} ist, welches von den G�ttern in unsere Welt geschickt wurde.
Nachtklinges matt schwarze Oberfl�che l�sst keinerlei Reflektionen zu. Ihr Name ist in Zwergenrunen\index{Z!Zwergenrune} in die Klinge
gepr�gt.

\subsection*{Nachtklinge im Spiel}
\par Nachtklinge verleht seinem Tr�ger unb�ndigen Mut im Kampf gegen Drachen, so dass er keine Schw�che zeigt und
in ihrer Gegenwart nicht von Drachenangst befallen wird. Zudem bietet sie ihm eine gewisse Resistenz gegen Feuer,
so dass jeglicher Feuerschaden halbiert wird.

\begin{tabular}{l|c|c|c|c|c|c}
           & Ini  & Bonus     & Schaden & AbB    & BF    & Last\\
\hline
einh�ndig  &  1   & Str 10/15 & 3W+1    & 12/7/2 & 42/62 & 1\\
zweih�ndig & -1   & Str 10/14 & 3W+3    & 10/5/0 & 42/62 & 1\\
\end{tabular}\\

\par Wie bei jedem magischen Artefakt werden die F�higkeiten erst aktiviert, wenn der Tr�ger Kenntnis von den
F�higkeiten hat.

        \backmatter
                \appendix
                \pagenumbering{Roman}
                \part{Anhang}
                        \chapter{Weltkarte}
\begin{figure}[hbtp]
\epsfig{file=pics/map.eps, scale=0.22}
%height=4.5in, width=6.7in}
\caption{Weltkarte - Gesamt}
\label{worldmap}
\end{figure}

                        \addcontentsline{toc}{chapter}{Literaturverzeichnis}
\begin{thebibliography}{99}
\bibitem{cdd5}
Margaret Weis, Tracy Hickman: \textit{Drachenkrieg,}

\bibitem{cdd6}
Margaret Weis, Tracy Hickman: \textit{Drachend"ammerung,}

\bibitem{dsap1}
Schmidt Spiele: \textit{DSA-Professional: Schwertmeister Set 1} (1987)

\bibitem{erps1}
Edition Ulisses: \textit{ERPS-Grundregelwerk} (1995); http://www.erps.de

\bibitem{dddz}
NN: \textit{Drachend"ammerung - Eine Abhandlung "uber die Zwerge} (2004)

\end{thebibliography}

                        \addcontentsline{toc}{chapter}{Abbildungsverzeichnis}
                                \listoffigures
                        \addcontentsline{toc}{chapter}{Tabellenverzeichnis}
                                \listoftables
                        \addcontentsline{toc}{chapter}{Index}
                                \printindex
                                \printindex[regionen]
                                \printindex[personen]
                        \chapter{Impressum}
\par So, nun sind wir also am Ende angekommen und hier kommt das, was so ein Buch erst richtig wertvoll macht. Aus einem Gedanken wurde eine Idee und aus einer Idee wurde ein Anfang und letztendlich dieses hier vorliegenden Buch. Aber, ...
\begin{quotation}
\par \textit{\glqq Kein Buch schreibt sich von von alleine und keiner schreibt alleine ein Buch.\grqq}
\end{quotation}
\par An dieser Stelle m"ochte ich allen danken, die mir bei dieser Arbeit geholfen haben, allen Co-Autoren,
Zeichnern, Spieletestern und Spielleitern und nat"urlich auch allen Helden und Schurken, die im Laufe des
Schaffungs- und Testprozesses ihr Leben lie�en.
\begin{center}
\section*{Credits}
\begin{description}
\item[Autor:]Marco Behnke
\item[Co-Autoren:] Torben Werner (Die Zwerge, S. \pageref{zwerge}, \pageref{diezwerge})
\item[Grafiken und Illustrationen:] Dirk Kultus, \glqq Asamarith\grqq{}\footnote{http://www.razyboard.com/system/user\_asamarith.html}
\end{description}

\section*{Spieletester}
\begin{description}
\item[Spielleitungen:]
\item[Spieler:] Benjamin August, Oliver Joppek, Thimo Altmann, Tim Warszta, Torben Werner
\item[Charaktere:] Francesca Da Corva (menschliche Hexe), Grendal (trollischer Magier), Linflas (elfischer M"onch), Paragon (elfischer Waldl"aufer), Teralk, Sohn des Barvos (zwergischer Paladin)
\end{description}

\section*{Danksagungen}
\begin{description}
\item[Allgemeine Danksagungen an:] Catharina 'Cat' Link, Ernst-Joachim
Preussler (ERPS) und die deutschen Mailingliste\footnote{erps-public@informatik.uni-frankfurt.de}, Tim Warszta
\end{description}
\end{center}

\newpage
\begin{center}
\epsfig{file=pics/dd.eps, scale=0.5}
\end{center}
\begin{center}
ein Projekt von\\
http://www.drachendaemmerung.de
\end{center}
\begin{center}
\epsfig{file=pics/frostwyrm.eps, scale=0.4}
\end{center}
\end{document}
