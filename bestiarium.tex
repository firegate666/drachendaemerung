\chapter{Bestiarium\index{B!Bestiarium}}
\label{bestiarium}
\parpic[l]{\epsfig{file=pics/capitals/d.eps, scale=0.5}}ieses Kapitel
widmet sich allen Kreaturen, die auf unserem Boden wandern und nicht
den Pr�fungen der G�tter am letzten Tage standhalten werden. Doch
bis es soweit ist tut jeder gut daran, diese Untiere zu kennen und
sie zu meiden, denn meist sind sie m�chtiger.
\begin{quotation}
\par\textit{\glqq Unter Bestye oder Untyr im algemeinen
verstehen wir die Kreatuhren, die nicht ihrem Verstant sondern ihrem
Hunger folgen.\grqq} [\mbox{Lexxis} \mbox{Urmbrecht}, Magier der
Hochburg]
\end{quotation}
\section{Wesen der Nacht}

\subsection{Bluttrinker\index{B!Bluttrinker}\index{V!Vampir}}
\par Der Bluttrinker, im Volksmunde auch besser bekannt als Vampir,
ist ein Wesen der Finsternis. "Ahnlich dem D"amon ist sein Ursprung
widernat"urlicher Natur.

\subsection{Grottenolm}

\section{Wesen des Tages}
\subsection{Drachenkrieger\index{D!Drachenkrieger}}
\par beschreibung folgt

\section{�bernat�rliche und D�monen}

\subsection{Phantome\index{P!Phantom}}
\par Phantome sind die Geister verstorbener Lebewesen, die nicht den Weg ins Jenseits gefunden haben.\\
\par \begin{tabular}{p{2cm}p{2cm}p{2cm}p{2cm}}
\hline
\hline
\multicolumn{4}{c}{\textbf{Phantom}}\\
\hline
\hline
BEW & 15 & RS  &  4\\
TP  & 20 & PSI & 10\\
\hline
\multicolumn{4}{l}{magische Ber�hrung 3/0 (2W) AbB 10/6/4,}\\
\multicolumn{4}{l}{in den Gegner einfahren 1/0 (2W) AbB 8/4/0}\\
\multicolumn{4}{l}{dieser Angriff kann nur durch eine gelunge REA Probe}\\
\multicolumn{4}{l}{gegen das Kampfergebnis abgewehrt werden.}\\
\hline
\end{tabular}
\begin{description}
\item[Sonderregel Portal] Phantome erscheinen immer durch ein Nebelportal, durch dass sie sich materialisieren.
Solange das Portal existiert, k�nnen die Phantome nicht vollst�ndig vernichtet werden, sondern kommen in der n�chsten
Runde wieder.
\item[Sonderregel Lebensraub] Jeder verursachte Lebenspunktverlust beim Gegner wird dem Phantom als Lebenspunkt gut
geschrieben (auch �ber den urspr�nglichen Wert hinaus).
\end{description}