\section{Der Mensch\index{M!Mensch, der}}
\subsubsection{Beschreibung}
\par Die Menschen stellen den gr"o"sten Teil der Bev"olkerung
dieses Kontinents. Sie sind "uberall und in den unterschiedlichsten
Formen und Entwicklungsstadien anzutreffen. Angefangen beim
Ureinwohner in den Nordlanden bis hin zu den Kaufleuten im Reich der
Mitte. Welche davon in ihrer Entwicklung weiter sind "uberlasse ich
dem Leser zu beurteilen.
\subsubsection{Herkunft}
\subsubsection{Erscheinungsbild}
\par Ein ausgewachsener Mensch liegt in der Regel zwischen 160cm und
190cm. Regionale Unterschiede k"onnen auch Extrema hervorbringen.
Unter den Seefahrer des S"udens hat man auch schon Individuen
gesehen, die die 2 Meter Marke locker "uberschreiten, w"ahrend die
Bewohner der westlichen W"usten eher kleiner sind.
\par Es sind alle Naturfarben als Augen- und Haarfarbe vertreten.
\par Mit 16 Jahren gilt ein Mensch als erwachsen, hier gibt es jedoch
auch regionale Unterschiede, wo besonders in den Naturv"olkern ein
niedrigeres Alter angesetzt wird. Die durchschnittliche
Lebenserwartung liegt bei 70 Jahren.
\subsubsection{Wesen}
\subsubsection{Besonderes}
