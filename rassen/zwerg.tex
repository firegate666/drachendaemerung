\section{Der Zwerg\index{Z!Zwerg}} \label{zwerge}
\subsubsection{Beschreibung}
\par Die Zwerge sind ein starkes Volk von Kriegern. Wenn alle Zwerge etwas gemeinsam haben, dann ist das ihre Liebe zum Kampf und dem guten Zwergenbr"au. Die Zwerge sind gesellige und laute Zeitgenossen, die wissen wie Feste zu feiern sind, und dieses tun sie in Friedenzeiten zu jedem passenden Anla"s.
\par Mal abgesehen von den Elfen, denen die meisten Zwerge einfach zu \glqq lebhaft\grqq{} sind, halten die anderen Rassen die Zwerge f"ur freundlich und ehrbar.
\par Eine genaue Beschreibung der einzelnen Clans ist unter \textit{Die Clans der Zwerge} (siehe S. \pageref{diezwerge}) zu finden.
\subsubsection{Herkunft}
\par Die Sieben Clans der Zwerge sind "uber die Welt der Drachend"ammerung verteilt. Die Zwerge des Sturmhammer\index{S!Sturmhammerclan}-, Flammenaxt\index{F!Flammenaxtclan}-, Kristallklingen\index{K!Kristallklingenclan}- und Eisenfaustclans\index{E!Eisenfaustclan} leben in den gro"sen Gebirgsketten der Welt. Der Schattenwolfclan\index{S!Schattenwolfclan} lebt in einem riesigen Wald des Landes und die Zwerge des Sch"adelsammlerclans\index{S!Schaedelsammlerclan} f"uhren ein eher nomadisches S"oldnerdasein.
\subsubsection{Erscheinungsbild}
\begin{quotation}
\par \textit{Zwerge sind klein, tragen rote spitz zulaufende M"utzen, einen Spaten und grinsen stets...irgendwie habe ich mir die Zwerge dann doch folgendermassen vorgestellt:}
\end{quotation}
\par Auch wenn sich das Erscheinungsbild zwischen den einzelnen Clans mehr oder weniger unterscheidet, so kann der gemeine Zwerg recht einfach beschrieben werden. Der gemeine Zwerg ist zwischen 90 und 140 cm gro"s und wiegt bis zu 160 kg. Das enorme Gewicht liegt zum einen an der ausgepr"agten Muskelstruktur sowie an dem ber"uhmten Zwergenbr"au. Mal abgesehen von blondem Haar sind alle Farben vertreten und die Augen sind generell dunkel. Die Zwerge des Hochlandes sind von eher blasser, die des Tieflandes von dunklerer Hautfarbe.
\par W"ahrend die Hochlandzwerge\index{H!Hochlandzwer} dem Betrachter in ihrer gut geschnitteten Tuch- und Lederkleidung eher gesittet erscheinen, so sind die Tieflandzwerge\index{T!Tieflandzwerg} von eher wilder Erscheinung. Sie tragen T"atowierungen zur Schau, welche ihre Arme und oftmals kahlgeschorenen K"opfe zieren. Dazu tragen sie eher Felle und schm"ucken sich mit vielerlei Troph"aen; barbarisch sagen manche!
\par Zwerge sind mit 40 Jahren vollj�hrig und k�nnen den Clan auf eigene Faust verlassen oder eigenst�ndig Berufe aus�ben. Die durchschnittliche Lebenserwartung liegt bei 250-300 Jahren
\subsubsection{Wesen}
\par Der gemeine Zwerg ist generell aufbrausend und laut und wird eine Ungerechtigkeit ihm gegen"uber lauthals anmahnen. Wie ich bereits erw"ahnte feiern Zwerge gerne und sind einem guten Schluck Br"au nie abgeneigt.
\par Zwerge sind stolz auf das was sie tuen. Ihre Schmiedkunst steht dabei im Vordergrund, denn kein anderes Volk verarbeitet Stahl in entsprechender Qualit"at.
\par Wenn man die Freundschaft eines Zwergen f"ur sich gewonnen hat, so h"alt diese ein Leben lang. Dummerweise gilt dies auch f"ur eine Feindschaft und ich versichere, da"s der entsprechende Zwerg keine Gnade walten lassen wird!
\subsubsection{Besonderes}
\par Die hohe Ausdauer, Z"ahigkeit und St"arke der Zwerge sind schon besondere Attribute (Konstitution + 1). Doch etwas anderes m"ochte ich an dieser Stelle nennen:
\par Die Zwerge verf"ugen auf Grund ihrer Natur "uber die F"ahigkeit aus jeder H"ohle, die sie betreten, wieder heraus zu finden, egal wie weitl"aufig diese ist. Au"serdem verf"ugen Sie "uber erstaunliche Sicht in fast v"olliger Dunkelheit. Genauere Informationen zu diesen F"ahigkeiten im Kapitel \textit{Neue Fertigkeiten} (siehe S. \pageref{neuefertigkeiten}).
\par Neben den besonderen k"orperlichen Attributen sind es die Zwerge, welche die Runenmagie\index{R!Runenmagie} beherrschen. Die Thaumaturgen\index{T!Thaumaturg} der Zwerge besitzen die F"ahigeit Gegenst"ande durch anbringen magischer Symbole (Runen) mit zum Teil erstaunlichen Eigenschaften auszustatten. Damit nicht genug, beherrschen die Thaumaturgen die F"ahigkeit Runen in Form von T"atowierungen auf die Haut einer Person aufzutragen, welche ebenfalls magische Eigenschaften bei dem Tr"ager hervorrufen.
\par Die Art und Wirkungsweise der toten und lebenden Runen ist
unterschiedlich. Eines haben die Runen\index{R!Runen} allerdings
gemeinsam: die Herstellung ist auf Grund der Substanzen zum Teil
sehr teuer.
\par Zwei Beispiele hierzu:
\begin{description}
\item [tote Rune\index{R!Rune, tote}] Rune der Flamme und Rune des Wortes angebracht an einer Waffe\\
Die Waffe erh"alt auf Befehl des Tr"agers (z.B. Flamme) eine brennende Klinge, die zus"atzlichen Schaden verursacht
\item [lebende Rune\index{R!Rune, lebende}] Rune der Kampfeslust auf dem Arm des Zwerges\\
Der Zwerg erh"alt einen Bonus auf den Angriff einer mit diesem Arm gef"uhrten Waffe, die Rune wird in einer Farbe entsprechend der T"atowierung erstrahlen
\end{description}
