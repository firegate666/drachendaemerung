\section{Der Troll\index{T!Troll, der}}
\subsubsection{Beschreibung}
\par Die Trolle sind magische Wesen von zierlicher Statur.
B"osartige Zungen nennen Vergleiche mit den Kobolden\index{K!Kobold,
der}, doch im Gegensatz zu ihnen ziehen die Trolle es vor, anderen
Wesen nicht zu schaden. Sie sind friedliche Wesen. Von Magie
erschaffen, von Magie getrieben ist der Troll eine mystische
Kreatur, die noch nicht lange auf dem Kontinent bekannt ist.
\subsubsection{Herkunft}
\par Eigentlich wei"s keiner so genau von wo die Trolle
kommen oder wohin sie gehen. Es gibt keine Informationen "uber ihre
Lebensgewohnheiten oder sonst irgend etwas.
\par Auch ist wenig "uber ihr soziales Verhalten in den
B"ucher niedergeschrieben. Was man wei"s ist, dass die Trolle nicht
zwischen Mann und Frau unterscheiden. Da sie durch und durch von
magischer Natur sind, vermehren sie sich auch nicht auf dem
herk�mmlichen Wege.
\par Trolle werden magisch geboren, sozusagen auf unbekanntem Wege
in die Welt beschworen.
\subsubsection{Erscheinungsbild}
\par Die gr��ten Trolle, die man gesehen hat, waren
bis zu einem halben Meter hoch. Da Trolle es jedoch meist vorziehen
unsichtbar zu reisen, ist es einsehbar, dass die Menschheit noch
l�ngst nicht alle Trolle gesehen hat. Im Durchschnitt misst ein
ausgewachsener Troll 30 bis 40 cm. Diejenigen, die schon einmal
einen Troll zu Gesicht bekommen haben, berichten von Ihrer blassen,
leicht bl"aulichen Hautfarbe und dem kleinen Paar H"orner, welches
ihr Haupt ziert. Trolle haben meist sehr dunkle bis schwarze Augen
und keine K"orperbehaarung.
\par "Uber die Lebenserwartung eines Trolles gibt es keine
Angaben, ebenso wenig "uber ein Erwachsenenalter. Trolle sind seit
dem Augenblick ihrer \glqq Geburt\grqq{} voll ausgewachsen.
\subsubsection{Wesen}
\par Trolle sind im Wesen mysteri"ose und seltsame Kreaturen,
sie sind von Magie durchflossen und werden von ihr ebenso bestimmt,
wie sie es verstehen sie zu manipulieren. Diese F�higkeit hat ihnen
einen gro�en Forscherdrang verliehen. Sie geben sich meist nicht mit
oberfl�chlichen Erkl�rungen zufrieden, sondern werden
Ungereimtheiten zielstrebig auf den Grund gehen.
\par Der Umstand, dass Trolle als erwachsene Lebewesen in die Welt
geboren werden, l�sst ihr Leben in einem anderen Bild erscheinen,
als das der normal Sterblichen. Sie sind vom ersten Augenblick ihres
Lebens an erwachsen und komplett entwickelt. Ihre enorme
Lernf�higkeit erlaubt es ihnen sich in wesentlich k�rzerer Zeit
Wissen anzueignen und so Ausbildungen schneller abzuschlie�en. Da
Trolle noch nicht viel von der Welt gesehen haben, kann man ihnen
einen gewissen Hang zur Neugier nicht absprechen. Ihre Aktionen sind
jedoch immer von Bedacht gedeckt, ein Troll w�rde sein junges Leben
nicht unbegr�ndet in Gefahr bringen.
\par Andere Rassen treten den Trollen mit einer Art gesundem Respekt
gegen�ber. Auf der einen Seite birgt ein Troll das Unbekannte, so
selten werden sie doch gesehen, so gerne w�rde man mehr �ber sie
erfahren. Auf der anderen Seite steht das Misstrauen gegen sie, zu
sonderbar ist ihr Auftreten und Verhalten, zu unbekannt ihre
F�higkeiten.
\par Trolle werden niemals einen kriegerischen Beruf aus"uben.
\subsubsection{Besonderes}
\par Aufgrund ihrer Natur sind Trolle nur f�r eine kleine Reihe von
Berufen geeignet.
\par Trolle haben magische F"ahigkeiten\index{T!Trollmagie},
jedoch auf ihre Art und Weise, die sich vollst"andig von denen der
anderen Magier abhebt. Es ist keine Spruchmagie und sie bedienen
sich auch nicht der \glqq Wahren Magie\grqq. Ihre Magie ist anders,
sie kommt von innen und scheint so etwas wie eine F"ahigkeit zu
sein, die sie nicht zu lernen brauchen. Trolle haben au"serdem die
au"sergew"ohnliche F"ahigkeit sich unsichtbar zu machen. Nicht
selten hat ihnen dies ihr kostbares Leben gerettet. Spieltechnisch
bedeutet dies, dass jeder Troll den Zauberspruch Unsichtbarkeit bereits
auf 50\% erlernt hat.
\par Bei der Berechnung der Tragkraft (TRK) wird
ein Maximalwert von St"arke 10 ber"ucksichtigt, jeder Punkt"uber 10
verf"allt bei der Berechnung. Da die Trolle "uber eine Art 6. Sinn
verf"ugen, erhalten sie eine erh"ohte Intuition (+1).