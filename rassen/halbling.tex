\section{Der Halbling\index{H!Halbling, der}}
\subsubsection{Beschreibung}
\par \textit{\glqq Von der Hand in der Tasche.\grqq} So oder
"ahnlich wird wohl das Lebensmotto der Halblinge lauten. Flink und
diebisch sind die 2 Worte, die jedem sofort einfallen, wenn man an
Halblinge denkt. Ausgenommen nat�rlich Halblinge. Sie sehen sich
eher als ein Volk voller missverstandener kleiner liebenswerter
Wesen, die einfach nur ernst genommen werden wollen. Und wer sehr
genau hinschaut, wenn sie dies sagen, wird ein leichtes L"acheln in
den Mundwinkeln erkennen.
\subsubsection{Herkunft}
\par Wo die Halblinge herkamen, wei"s eigentlich keiner mehr so
genau. Aber wahrscheinlicher ist, dass sie schon immer da waren,
auch wenn sich niemand daran erinnern kann, jemals Halblinge
w"ahrend der Herrschaft der Finsternis\index{H!Herrschafft der
Finsternis} in Gefangeschaft gesehen zu haben.
\subsubsection{Erscheinungsbild}
\par Halblinge sind zwischen einem halben und einem Meter gro"s,
Ihre Statur reicht von gertenschlank bis gedrungen. Alles in allem
ist ihre Artenvielfalt "ahnlich gro"s, wie die der Menschen.
Halblinge haben meist dunkle Haar- und Augenfarben. Selten sieht man
einen blonden Halbling. Ungew"ohnlich hoch ist der Anteil von
Albinos innerhalb des Volkes.
\par Halblinge erreichen sehr fr"uh das Erwachsenenalter, die meisten
verlassen bereits mit 10 Jahren das elterliche Haus und ziehen in
die Welt. Ihre Lebenserwartung reicht im Schnitt bis hin zu 40
Jahren.
\subsubsection{Wesen}
\par Fr"ohlich, unbedarft und von kindlichem Gem"ut; Dies sind
sind die wesentlichen Z"uge eines Halblings. Wer ihnen jedoch keine
Ernsthaftigkeit nachsagt, hat weit gefehlt. Halblinge wissen den
Ernst einer Lage sehr wohl einzusch"atzen, auch wenn sie versuchen
ihn zu "uberspielen. Sollte ein Halbling es jedoch in einer
Situation f"ur sinnvoll erachten, wird er ihr den n"otigen Respekt
entgegenbringen.
\par Alles in allem leben Halblinge jedoch auf der sonnigen Seite
des Lebens. Gewalt ist ihnen ein Greuel und nur in wirklicher Not
wird von den Waffen Gebrauch gemacht, die sie gerne als
Schmuckst"ucke und Zierde am K"orper tragen.
\subsubsection{Besonderes}
\par Halblinge sind "uberaus geschickte Wesen. Eine Eigenschaft
die man einigen ob ihres K"orperbaus nicht unbedingt ansehen wird.
Die meisten Halblinge gehen durchaus b"urgerlichen Berufen nach und
verdienen sich als Feinmechaniker oder Goldschmied ihren
Lebensunterhalt.
