\chapter{Das Regelsystem}

\section{ERPS-Regelsystem}
\par Diese Spielwelt wurde mit der Vorgabe entwickelt, nicht an ein bestimmtes System gebunden zu sein. Da 
dies jedoch nicht immer m"oglich ist, habe ich mich an einigen Stellen an das ERPS\footnote{Erne(a)st Role 
Playing Game : http://www.erps.de} Regelsystem gebunden. Warum ich mich f"ur ERPS entschlossen habe? Dazu 
hier ein Zitat des Autor:
\begin{quotation}
\par\textit{ERPS ist ein \glqq modernes\grqq{} System, welches Fertigkeiten "uber Attribute stellt, das keine 
Grade oder Stufen im klassischen Sinne kennt und nicht einmal Charakterklassen. Die freie Entfaltung eines 
Charakters, Verbessern von Fertigkeiten durch praxisnahe Anwendung und ein v"ollig offenes Magiesystem sind 
sicherlich die gr"o"sten Pluspunkte von ERPS. Gerade das Magiesystem bietet ungeahnte M"oglichkeiten, fordert 
allerdings auch ein bi"sschen mehr vom Spielleiter als klassische Magiesysteme.}
\end{quotation}
\par ERPS ist also ein sehr modulares und freies System, welches man leicht in ein eigenes Setting setzen kann. 
Es gl�nzt durch die Abwesenheit von tausenden von spielrelevanten Zusatzb�chern, ein Grundregelwelt reicht f�r 
jahrelanges Auskommen.
\par Das ERPS-Regelsystem wird von mir an einigen Stellen erweitert werden. Diese Erweiterungen sind jedoch leicht 
an andere System anzupassen. Abweichungen zum ERPS-System sind unter anderem an folgenden Stellen zu finden:
\begin{multicols}{2}
\begin{itemize}
\item Erschaffung von Charakterklasse
\item Rassenvor- und Nachteile
\item Klassen- und Nachteile
\item Erweiterung des Magiesystems um eine neue Form von Magie
\item ...
\end{itemize}
\end{multicols}

\section{Charakterklassen - eine kleine �bersicht}
\subsection{Die Archetypen\index{A!Archetyp}}
\par Als erstes folgt eine kurze Beschreibung einiger der zur Verf"ugung stehenden Archetypen\index{A!Archetyp}. 
Diese Auflistung soll gen"ugen, um einem kurzen Eindruck "uber die Funktionalit"at des Systems zu gewinnen. Das 
Kapitel \textit{Charakterklassen}\index{C!Charakterklasse} (siehe S. \pageref{charakterklassen}) stellt alle mit 
diesem Regelbuch zur Verf"ugung stehenden Charakterklassen\index{C!Charakterklasse} vor.
\subsubsection{Krieger\index{K!Krieger}}
\par Der Krieger\index{K!Krieger} ist bewandert im Umgang mit der Waffe wie kein anderer Charakter. Seine Ausbildung 
hat ihn gelehrt selbst unter extremen Bedingungen einen klaren Kopf zu bewahren. Selbst die schwerste R"ustung 
behindert ihn nicht sonderlich in seinen Aktionen.
\par Krieger\index{K!Krieger} haben gelernt mit Zweihandwaffen\index{Z!Zweihandwaffe} zu k"ampfen und wie sie 
effektiv ihre Gegner ausschalten.
Sie sind ausdauernd und z"ah. W"ahrend ihrer Ausbildung haben sie allerdings auch gesellschafliche F"ahigkeiten 
erworben, die ihnen in Friedenszeiten von Nutzen sind.
\subsubsection{Magier\index{M!Magier}}
\par Der Magier\index{M!Magier} hat von Kindesbeinen an gelernt die Natur zu kontrollieren. Er ist eins mit den 
arkanen Str"ohmungen\index{A!arkane Strmung}. W"ahrend seiner langen Ausbildungszeit wurde ihm beigebracht diese 
St"urme zu erkennen und f"ur seine Zwecke zu formen. Magier\index{M!Magier} sind intelligent und sinnesscharf. 
Ihnen entgeht nichts. Sie haben allerdings keine Kenntnisse im Umgang mit Waffen.
\par Ein weiterer Teil ihrer Ausbildung war das Studium alter Schriften und Folianten\index{F!Foliant}. 
Magier\index{M!Magier} sind sprachbegabt und kennen Mundarten und Schriftzeichen, von deren Existenz andere 
Sterbliche noch nie etwas geh"ort haben.
\subsubsection{Dieb\index{D!Dieb}}
\par Heimlichkeit und Geschick sind die Disziplinen des Diebes\index{D!Dieb}. Dabei sprechen wir hier jetzt 
nicht von dem Stra"senr"auber\index{S!Stra"snr"auber} oder dem Gelegenheits-Einbrecher\index{E!Einbrecher}. 
Der Dieb\index{D!Dieb} sieht in seinen Taten ein Ritual. Jede Aktion hat ihren eigenen Reiz und ihre eigene 
Atmosph"are. Er stiehlt meist nicht um der Beute willen, sondern um seine Grenzen zu erfahren und sich jedes 
mal von neuem zu �bertreffen.
\par F"ur Diebe\index{D!Dieb} gibt es keine verschlossenen T"uren oder versteckte Fallen. Ihr Instinkt hat 
ihnen bisher noch aus jeder Situation heraus geholfen.

\section{Charaktererschaffung}
\par Den genauen Ablauf und die verwendeten Mechanismen zur Charaktererschaffung sind dem ERPS-Regelwerk\cite{erps1} zu 
entnehmen. Als Erg"anzung daf"ur sind folgende Punkte zu ber"ucksichtigen.
\begin{enumerate}
\item Nach dem Ausw"urfeln der Attribute, muss die Spielerrasse gew"ahlt werden. Entstandene Vor- und Nachteile 
werden auf dem Charakterbogen vermerkt.
\item Anschlie"send entscheidet man sich f"ur eine Charakterklasse, vermerkt ebenfalls die entstandenen "Anderungen. 
Dann geht es weiter mit dem Steigern der Anfangsfertigkeiten.
\end{enumerate}
\par Alle weiteren Regeln, wie z.B. angeborene F"ahigkeiten usw. behalten weiter ihre G"ultigkeit.
