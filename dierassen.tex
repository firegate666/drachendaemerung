\chapter{Die Rassen\index{R!Rassen}}
\label{dierassen}
\begin{quotation}
\par\textit{\glqq Es nicht nur die Klasse, die einen Menschen ausmacht.\grqq}
\end{quotation}
\section{Einleitung}
\par So oder so "ahnlich, oder aber auch gar nicht kann man es auch auf das Rollenspiel umsetzen. Der Abenteurer will nicht nur Dieb, Krieger oder einfach nur Elf sein. Die Vielfalt bestimmt mehr. Es gibt auch Elfen, die Krieger sind. Es gibt, ja wirklich, Zwerge, die Magie anwenden und und und ...
\par Der folgende Abschnitt stellt alle dem Spieler zur Verf"ugung stehenden Rassen vor. In dem sp"ateren Kapitel \textit{Das Bestiarium} (ab S. \pageref{bestiarium}) stelle ich noch einige Rassen vor, die nur dem Spielleiter vorbehalten sind.
\par Jede Klassenbeschreibung\index{K!Klassenbeschreibung} folgt der unten aufgelisteten Struktur.
\begin{description}
\item[Beschreibung:]Umfassende Beschreibung der Rasse
\item[Herkunft:]Manche Rassen sind normalerweise nicht "uberall auf dem Welt zu treffen, deswegen befindet sich hier eine Beschreibung "uber die "ortliche Herkunft sowie des Lebensraums.
\item[Erscheinungsbild:]"Au"serliche Erscheinungsmerkmale wie Augen-, Haar- und Hautfarbe, Gr"o"se, Gewicht, K"orperform und bevorzugte Kleidung; andere Merkmale wie Alter und in dem Zusammenhang Vollj"ahrigkeit etc.
\item[Wesen:]rollenspieltechnische Hinweise f"ur den Spieler; Beschreibung der allgemeinen Wesensz"uge\index{W!Wesensz"uge}, St"arken\index{S!St"arken} und Schw"achen\index{S!Schw"achen}
\item[Besonderes:]Hier wird etwas stehen, falls eine Rasse Besonderheiten, wie \mbox{z.B.} besondere F"ahigkeiten besitzt oder "uber spezielle Eigenarten verf"ugt. Hier wird ebenfalls stehen, welche Sprache die Muttersprache ist. Spieltechnische Auswirkungen sind ebenfalls hier zu finden.
\end{description}

\newpage
\section{Der Elf\index{E!Elf}}
\subsubsection{Beschreibung}
\par Hochgewachsen, stolz, allwissend und unfehlbar.
Fragt man einen Elfen sind das die Attributen mit denen er sich und
seine Artgenossen beschreiben wird. Zumindest die ersten beiden
Punkte kann man undiskutiert als Tatsache niederschreiben. Doch ein
Elf wird noch mit einer ganzen Reihe anderer Eigenschaften
verbunden, einige davon wird er z�hneknirschend akzeptieren und
gegen andere wird er sich mit H�nden und F��en zur Wehr setzen.
Die Wahrheit steht wie immer zwischen den Zeilen und manchmal auch
im \textit{Dunklen}.
\subsubsection{Herkunft}
\par Die Elfen des Landes geh�ren einem von zwei V�lkern an, den Hochland- oder den Waldelfen.
Innerhalb dieser unterteilen sie sich in verschiedene Clans, die an
verschiedenen Orten dieser Welt leben. Elfen sind jedoch so gut wie
nie in ihren Clans anzutreffen. Jeder Clan setzt sich aus
verschiedenen Sippen\footnote{Sippe=Familie} zusammen, in denen die
Elfen leben.
\par Jedem Clan steht ein H�uptling vor und in jedem Clan gibt
es mindestens einen Schamanen. Bei dem H�uptling handelt es sich
eher weniger um den allein herrschenden Monarchen, als um einen
weisen Ratgeber. Er ist der Anlaufpunkt f�r alle Wissbegierigen und
ein Ratgeber f�r schwierige Fragen. Meist ist der H�uptling der
Clan�lteste. Die Schamanen des Clans bewahren die Chroniken auf. In
ihnen wird die Geschichte des Clans und des Kontinents weiter
geschrieben.
\subsubsection{Erscheinungsbild}
\par Wo immer ein Elf auftritt sorgt er f�r Aufsehen. Selten
werden diese Gesch�pfe in belebten Gegenden gesehen. Der Elf misst
im Schnitt 2 Meter und ist von schlankem K�rperbau. Seine Kraft
sollte man jedoch nicht untersch�tzen. Elfen sind stolze und
gef�hrliche Krieger. Ihre Haare werden sehr of lang getragen und es
treten die �blichen Haar- und Augenfarben auf. Ihre Haut
ist meist blass.
\par Ihre bevorzugte Kleidung ist Leder oder grober Stoff,
meist in den Farben des Waldes. Alles was sie bei sich tragen ist
einem Zweck gewidmet. Der Elf hasst nichts mehr, als �berfl�ssigen
Balast. Eine Ausnahme dieser Regel stellen Kunstwerke dar.
\par Elfen, die sich innerhalb ihrer Sippe der Jagd gewidmet haben, 
treten oft in Begleitung eines Jagdtieres auf. Dabei w�hlen sie genauso 
oft den Hund oder den Falken wie den Puma oder Bergl�wen.
\subsubsection{Wesen}
\par Kunst ist ihre Leidenschaft und auch ihr Laster. F�r gute
Kunst geben die Elfen alles. Dabei sind sie selbst auch sehr
geschickte K�nstler. Es spielt keine Rolle, ob es sich um Musik,
Bildhauerei oder die Kunst der Malerei handelt.
\par Der Elf f�r sich genommen ist im Herzen ein Einzelg�nger.
Zusammenschl�sse findet man nur innerhalb der Sippe oder in
Zweckgemeinschaften. Der Elf ist von tiefem Stolz �ber seine
Herkunft erf�llt und legt ein gro�es Ehrbewusstsein an den Tag.
Auf das Wort eines Elfen ist Verlass, auch wenn es gro�e Worte
sind.
\par Da die Elfen zu den am l�ngsten auf diesem Kontinent lebenden
V�lkern geh�ren, verf�gen sie �ber gro�es Wissen der
Geschichte, welches von ihnen zugleich wie ein Geheimnis geh�tet
wird. So bewahren sie das Wissen f�r sich. Auch bei anderen
Anl�ssen versucht der Elf nur so viel Informationen wie n�tig
Preis zu geben.
\subsubsection{Besonderes}
\par Obgleich ihnen viele magische F�higkeiten angedichtet werden,
sind Elfen auch nur \glqq normale Menschen\grqq{}. Sie verf�gen �ber
keine au�ergew�hnlichen magischen F�higkeiten, obwohl es durchaus
auch Magier oder Druiden unter ihnen gibt. Elfen verf�gen jedoch
�ber hervorragend ausgebildete Sinne. Insbesondere ihre Intuition
hat ihnen in so mancher Lage die richtige Entscheidung beschert.
Elfen haben eine erh�hte Geschicklichkeit (+1) und d�rfen 5 Punkte
bei den Sinnen verschieben.
\par Elfen erreichen im Alter von 30 Jahren die Vollj�hrigkeit und
damit das Recht, die Sippe zu verlassen. Sie haben eine sehr hohe
Lebenserwartung. Der �lteste bekannte noch lebende Elf z�hlt
wahrscheinlich �ber 600 Lebensjahre, sein genaues Alter ist jedoch
nicht bekannt, schweigt er sich doch dar�ber aus.
\subsubsection{Die dunkle Seite des Elfen}
\par Trotz der Sch�hnheit und Grazie, die ein Elf reflektiert,
verbirgt sich hinter der Fassade doch ein dunkles Geheimnis. Ein
schweres Los wurde dem hochgewachsenen Volk auferlegt. Niemand au�er
den Stammes�ltesten der Elfen wei� �ber die Herkunft dieses
Schicksals Bescheid.
\par Elfen sind ein diszipliniertes Volk. Sie h�ten ihre
Gef�hlsregungen und halten auf diese Weise ihre dunkle Seite in
Schach. Elfen die gro�es Leid oder gro�e Schmerzen erfahren, werden
von ihrer schwarzen Seele\index{S!Schwarze Seele} �bermannt. Jeder
Elf tr�gt diese Seele in sich, die den Sagen nach ein �berbleibsel
der Sklaverei ist. Jeder Elf, der auf diese Art und Weise die
Kontrolle �ber sich verliert, st�rzt sich, ohne R�cksicht auf
sich und seine Umwelt, auf den Leidensbringer und h�lt erst inne,
wenn Vergeltung erreicht wurde. Im Falle der Ausl�schung der ganzen
Sippe kann dies auch ohne weiteres ein dauerhafter Zustand werden.
Je l�nger die schwarze Seele ausgelebt wird, umso schwieriger ist
die R�ckkehr f�r den Elfen.
\par Ein Elf der dieser Raserei verf�llt, ver�ndert sich auch
�u�serlich. Die Gesichtsz�ge werden harte und schmerzerf�llte
Mimiken. Eine spielrelevante Beschreibung und deren Auswirkungen
befindet sich im Kapitel \textit{Die Elfen} (siehe S.
\pageref{dieelfen}).
\par Unter den Elfen gibt es einen Zusammenschlu� von Individuen,
die sich nicht wie die "Ubrigen vor ihrer dunlen Seite verstecken,
sondern sich aktiv versuchen dagegen zu wehren. Sich sind auf der
Suche nach dem Ursprung und der Heilung gegen diese Krankheit und
unter der Bezeichnung \textit{Harlekine}\index{H!Harlekine} (siehe
S. \pageref{harlekine}) bekannt.


\newpage
\section{Der Halbling\index{H!Halbling, der}}
\subsubsection{Beschreibung}
\par \textit{\glqq Von der Hand in der Tasche.\grqq} So oder
"ahnlich wird wohl das Lebensmotto der Halblinge lauten. Flink und
diebisch sind die 2 Worte, die jedem sofort einfallen, wenn man an
Halblinge denkt. Ausgenommen nat�rlich Halblinge. Sie sehen sich
eher als ein Volk voller missverstandener kleiner liebenswerter
Wesen, die einfach nur ernst genommen werden wollen. Und wer sehr
genau hinschaut, wenn sie dies sagen, wird ein leichtes L"acheln in
den Mundwinkeln erkennen.
\subsubsection{Herkunft}
\par Wo die Halblinge herkamen, wei"s eigentlich keiner mehr so
genau. Aber wahrscheinlicher ist, dass sie schon immer da waren,
auch wenn sich niemand daran erinnern kann, jemals Halblinge
w"ahrend der Herrschaft der Finsternis\index{H!Herrschafft der
Finsternis} in Gefangeschaft gesehen zu haben.
\subsubsection{Erscheinungsbild}
\par Halblinge sind zwischen einem halben und einem Meter gro"s,
Ihre Statur reicht von gertenschlank bis gedrungen. Alles in allem
ist ihre Artenvielfalt "ahnlich gro"s, wie die der Menschen.
Halblinge haben meist dunkle Haar- und Augenfarben. Selten sieht man
einen blonden Halbling. Ungew"ohnlich hoch ist der Anteil von
Albinos innerhalb des Volkes.
\par Halblinge erreichen sehr fr"uh das Erwachsenenalter, die meisten
verlassen bereits mit 10 Jahren das elterliche Haus und ziehen in
die Welt. Ihre Lebenserwartung reicht im Schnitt bis hin zu 40
Jahren.
\subsubsection{Wesen}
\par Fr"ohlich, unbedarft und von kindlichem Gem"ut; Dies sind
sind die wesentlichen Z"uge eines Halblings. Wer ihnen jedoch keine
Ernsthaftigkeit nachsagt, hat weit gefehlt. Halblinge wissen den
Ernst einer Lage sehr wohl einzusch"atzen, auch wenn sie versuchen
ihn zu "uberspielen. Sollte ein Halbling es jedoch in einer
Situation f"ur sinnvoll erachten, wird er ihr den n"otigen Respekt
entgegenbringen.
\par Alles in allem leben Halblinge jedoch auf der sonnigen Seite
des Lebens. Gewalt ist ihnen ein Greuel und nur in wirklicher Not
wird von den Waffen Gebrauch gemacht, die sie gerne als
Schmuckst"ucke und Zierde am K"orper tragen.
\subsubsection{Besonderes}
\par Halblinge sind "uberaus geschickte Wesen. Eine Eigenschaft
die man einigen ob ihres K"orperbaus nicht unbedingt ansehen wird.
Die meisten Halblinge gehen durchaus b"urgerlichen Berufen nach und
verdienen sich als Feinmechaniker oder Goldschmied ihren
Lebensunterhalt.


\newpage
\section{Der Mensch\index{M!Mensch, der}}
\subsubsection{Beschreibung}
\par Die Menschen stellen den gr"o"sten Teil der Bev"olkerung
dieses Kontinents. Sie sind "uberall und in den unterschiedlichsten
Formen und Entwicklungsstadien anzutreffen. Angefangen beim
Ureinwohner in den Nordlanden bis hin zu den Kaufleuten im Reich der
Mitte. Welche davon in ihrer Entwicklung weiter sind "uberlasse ich
dem Leser zu beurteilen.
\subsubsection{Herkunft}
\subsubsection{Erscheinungsbild}
\par Ein ausgewachsener Mensch liegt in der Regel zwischen 160cm und
190cm. Regionale Unterschiede k"onnen auch Extrema hervorbringen.
Unter den Seefahrer des S"udens hat man auch schon Individuen
gesehen, die die 2 Meter Marke locker "uberschreiten, w"ahrend die
Bewohner der westlichen W"usten eher kleiner sind.
\par Es sind alle Naturfarben als Augen- und Haarfarbe vertreten.
\par Mit 16 Jahren gilt ein Mensch als erwachsen, hier gibt es jedoch
auch regionale Unterschiede, wo besonders in den Naturv"olkern ein
niedrigeres Alter angesetzt wird. Die durchschnittliche
Lebenserwartung liegt bei 70 Jahren.
\subsubsection{Wesen}
\subsubsection{Besonderes}


\newpage
\section{Der Troll\index{T!Troll, der}}
\subsubsection{Beschreibung}
\par Die Trolle sind magische Wesen von zierlicher Statur.
B"osartige Zungen nennen Vergleiche mit den Kobolden\index{K!Kobold,
der}, doch im Gegensatz zu ihnen ziehen die Trolle es vor, anderen
Wesen nicht zu schaden. Sie sind friedliche Wesen. Von Magie
erschaffen, von Magie getrieben ist der Troll eine mystische
Kreatur, die noch nicht lange auf dem Kontinent bekannt ist.
\subsubsection{Herkunft}
\par Eigentlich wei"s keiner so genau von wo die Trolle
kommen oder wohin sie gehen. Es gibt keine Informationen "uber ihre
Lebensgewohnheiten oder sonst irgend etwas.
\par Auch ist wenig "uber ihr soziales Verhalten in den
B"ucher niedergeschrieben. Was man wei"s ist, dass die Trolle nicht
zwischen Mann und Frau unterscheiden. Da sie durch und durch von
magischer Natur sind, vermehren sie sich auch nicht auf dem
herk�mmlichen Wege.
\par Trolle werden magisch geboren, sozusagen auf unbekanntem Wege
in die Welt beschworen.
\subsubsection{Erscheinungsbild}
\par Die gr��ten Trolle, die man gesehen hat, waren
bis zu einem halben Meter hoch. Da Trolle es jedoch meist vorziehen
unsichtbar zu reisen, ist es einsehbar, dass die Menschheit noch
l�ngst nicht alle Trolle gesehen hat. Im Durchschnitt misst ein
ausgewachsener Troll 30 bis 40 cm. Diejenigen, die schon einmal
einen Troll zu Gesicht bekommen haben, berichten von Ihrer blassen,
leicht bl"aulichen Hautfarbe und dem kleinen Paar H"orner, welches
ihr Haupt ziert. Trolle haben meist sehr dunkle bis schwarze Augen
und keine K"orperbehaarung.
\par "Uber die Lebenserwartung eines Trolles gibt es keine
Angaben, ebenso wenig "uber ein Erwachsenenalter. Trolle sind seit
dem Augenblick ihrer \glqq Geburt\grqq{} voll ausgewachsen.
\subsubsection{Wesen}
\par Trolle sind im Wesen mysteri"ose und seltsame Kreaturen,
sie sind von Magie durchflossen und werden von ihr ebenso bestimmt,
wie sie es verstehen sie zu manipulieren. Diese F�higkeit hat ihnen
einen gro�en Forscherdrang verliehen. Sie geben sich meist nicht mit
oberfl�chlichen Erkl�rungen zufrieden, sondern werden
Ungereimtheiten zielstrebig auf den Grund gehen.
\par Der Umstand, dass Trolle als erwachsene Lebewesen in die Welt
geboren werden, l�sst ihr Leben in einem anderen Bild erscheinen,
als das der normal Sterblichen. Sie sind vom ersten Augenblick ihres
Lebens an erwachsen und komplett entwickelt. Ihre enorme
Lernf�higkeit erlaubt es ihnen sich in wesentlich k�rzerer Zeit
Wissen anzueignen und so Ausbildungen schneller abzuschlie�en. Da
Trolle noch nicht viel von der Welt gesehen haben, kann man ihnen
einen gewissen Hang zur Neugier nicht absprechen. Ihre Aktionen sind
jedoch immer von Bedacht gedeckt, ein Troll w�rde sein junges Leben
nicht unbegr�ndet in Gefahr bringen.
\par Andere Rassen treten den Trollen mit einer Art gesundem Respekt
gegen�ber. Auf der einen Seite birgt ein Troll das Unbekannte, so
selten werden sie doch gesehen, so gerne w�rde man mehr �ber sie
erfahren. Auf der anderen Seite steht das Misstrauen gegen sie, zu
sonderbar ist ihr Auftreten und Verhalten, zu unbekannt ihre
F�higkeiten.
\par Trolle werden niemals einen kriegerischen Beruf aus"uben.
\subsubsection{Besonderes}
\par Aufgrund ihrer Natur sind Trolle nur f�r eine kleine Reihe von
Berufen geeignet.
\par Trolle haben magische F"ahigkeiten\index{T!Trollmagie},
jedoch auf ihre Art und Weise, die sich vollst"andig von denen der
anderen Magier abhebt. Es ist keine Spruchmagie und sie bedienen
sich auch nicht der \glqq Wahren Magie\grqq. Ihre Magie ist anders,
sie kommt von innen und scheint so etwas wie eine F"ahigkeit zu
sein, die sie nicht zu lernen brauchen. Trolle haben au"serdem die
au"sergew"ohnliche F"ahigkeit sich unsichtbar zu machen. Nicht
selten hat ihnen dies ihr kostbares Leben gerettet. Spieltechnisch
bedeutet dies, dass jeder Troll den Zauberspruch Unsichtbarkeit bereits
auf 50\% erlernt hat.
\par Bei der Berechnung der Tragkraft (TRK) wird
ein Maximalwert von St"arke 10 ber"ucksichtigt, jeder Punkt"uber 10
verf"allt bei der Berechnung. Da die Trolle "uber eine Art 6. Sinn
verf"ugen, erhalten sie eine erh"ohte Intuition (+1).

\newpage
\section{Der Zwerg\index{Z!Zwerg}} \label{zwerge}
\subsubsection{Beschreibung}
\par Die Zwerge sind ein starkes Volk von Kriegern. Wenn alle Zwerge etwas gemeinsam haben, dann ist das ihre Liebe zum Kampf und dem guten Zwergenbr"au. Die Zwerge sind gesellige und laute Zeitgenossen, die wissen wie Feste zu feiern sind, und dieses tun sie in Friedenzeiten zu jedem passenden Anla"s.
\par Mal abgesehen von den Elfen, denen die meisten Zwerge einfach zu \glqq lebhaft\grqq{} sind, halten die anderen Rassen die Zwerge f"ur freundlich und ehrbar.
\par Eine genaue Beschreibung der einzelnen Clans ist unter \textit{Die Clans der Zwerge} (siehe S. \pageref{diezwerge}) zu finden.
\subsubsection{Herkunft}
\par Die Sieben Clans der Zwerge sind "uber die Welt der Drachend"ammerung verteilt. Die Zwerge des Sturmhammer\index{S!Sturmhammerclan}-, Flammenaxt\index{F!Flammenaxtclan}-, Kristallklingen\index{K!Kristallklingenclan}- und Eisenfaustclans\index{E!Eisenfaustclan} leben in den gro"sen Gebirgsketten der Welt. Der Schattenwolfclan\index{S!Schattenwolfclan} lebt in einem riesigen Wald des Landes und die Zwerge des Sch"adelsammlerclans\index{S!Schaedelsammlerclan} f"uhren ein eher nomadisches S"oldnerdasein.
\subsubsection{Erscheinungsbild}
\begin{quotation}
\par \textit{Zwerge sind klein, tragen rote spitz zulaufende M"utzen, einen Spaten und grinsen stets...irgendwie habe ich mir die Zwerge dann doch folgendermassen vorgestellt:}
\end{quotation}
\par Auch wenn sich das Erscheinungsbild zwischen den einzelnen Clans mehr oder weniger unterscheidet, so kann der gemeine Zwerg recht einfach beschrieben werden. Der gemeine Zwerg ist zwischen 90 und 140 cm gro"s und wiegt bis zu 160 kg. Das enorme Gewicht liegt zum einen an der ausgepr"agten Muskelstruktur sowie an dem ber"uhmten Zwergenbr"au. Mal abgesehen von blondem Haar sind alle Farben vertreten und die Augen sind generell dunkel. Die Zwerge des Hochlandes sind von eher blasser, die des Tieflandes von dunklerer Hautfarbe.
\par W"ahrend die Hochlandzwerge\index{H!Hochlandzwer} dem Betrachter in ihrer gut geschnitteten Tuch- und Lederkleidung eher gesittet erscheinen, so sind die Tieflandzwerge\index{T!Tieflandzwerg} von eher wilder Erscheinung. Sie tragen T"atowierungen zur Schau, welche ihre Arme und oftmals kahlgeschorenen K"opfe zieren. Dazu tragen sie eher Felle und schm"ucken sich mit vielerlei Troph"aen; barbarisch sagen manche!
\par Zwerge sind mit 40 Jahren vollj�hrig und k�nnen den Clan auf eigene Faust verlassen oder eigenst�ndig Berufe aus�ben. Die durchschnittliche Lebenserwartung liegt bei 250-300 Jahren
\subsubsection{Wesen}
\par Der gemeine Zwerg ist generell aufbrausend und laut und wird eine Ungerechtigkeit ihm gegen"uber lauthals anmahnen. Wie ich bereits erw"ahnte feiern Zwerge gerne und sind einem guten Schluck Br"au nie abgeneigt.
\par Zwerge sind stolz auf das was sie tuen. Ihre Schmiedkunst steht dabei im Vordergrund, denn kein anderes Volk verarbeitet Stahl in entsprechender Qualit"at.
\par Wenn man die Freundschaft eines Zwergen f"ur sich gewonnen hat, so h"alt diese ein Leben lang. Dummerweise gilt dies auch f"ur eine Feindschaft und ich versichere, da"s der entsprechende Zwerg keine Gnade walten lassen wird!
\subsubsection{Besonderes}
\par Die hohe Ausdauer, Z"ahigkeit und St"arke der Zwerge sind schon besondere Attribute (Konstitution + 1). Doch etwas anderes m"ochte ich an dieser Stelle nennen:
\par Die Zwerge verf"ugen auf Grund ihrer Natur "uber die F"ahigkeit aus jeder H"ohle, die sie betreten, wieder heraus zu finden, egal wie weitl"aufig diese ist. Au"serdem verf"ugen Sie "uber erstaunliche Sicht in fast v"olliger Dunkelheit. Genauere Informationen zu diesen F"ahigkeiten im Kapitel \textit{Neue Fertigkeiten} (siehe S. \pageref{neuefertigkeiten}).
\par Neben den besonderen k"orperlichen Attributen sind es die Zwerge, welche die Runenmagie\index{R!Runenmagie} beherrschen. Die Thaumaturgen\index{T!Thaumaturg} der Zwerge besitzen die F"ahigeit Gegenst"ande durch anbringen magischer Symbole (Runen) mit zum Teil erstaunlichen Eigenschaften auszustatten. Damit nicht genug, beherrschen die Thaumaturgen die F"ahigkeit Runen in Form von T"atowierungen auf die Haut einer Person aufzutragen, welche ebenfalls magische Eigenschaften bei dem Tr"ager hervorrufen.
\par Die Art und Wirkungsweise der toten und lebenden Runen ist
unterschiedlich. Eines haben die Runen\index{R!Runen} allerdings
gemeinsam: die Herstellung ist auf Grund der Substanzen zum Teil
sehr teuer.
\par Zwei Beispiele hierzu:
\begin{description}
\item [tote Rune\index{R!Rune, tote}] Rune der Flamme und Rune des Wortes angebracht an einer Waffe\\
Die Waffe erh"alt auf Befehl des Tr"agers (z.B. Flamme) eine brennende Klinge, die zus"atzlichen Schaden verursacht
\item [lebende Rune\index{R!Rune, lebende}] Rune der Kampfeslust auf dem Arm des Zwerges\\
Der Zwerg erh"alt einen Bonus auf den Angriff einer mit diesem Arm gef"uhrten Waffe, die Rune wird in einer Farbe entsprechend der T"atowierung erstrahlen
\end{description}


\newpage
\section{Einschr�nkungen der Charakterklassen}
\par Nicht alle Rassen k"onnen jeder Berufung folgen.
Alle nicht-m�glichen Kombinationen sind der unten stehenden Tabelle
zu entnehmen.
\begin{longtable}{r|c|c|c|c|c}
            & Elf & Halbling & Mensch & Troll & Zwerg \\
\hline
\hline
            &   &   &   &   &   \\
Barde       &   &   &   &   &   \\
Beschw�rer  & x & X &   & X & x \\
Dieb        &   &   &   &   &   \\
Druide      &   &   &   & x &   \\
Heiler      &   &   &   &   &   \\
Hexe        &   &   &   & x &   \\
Illusionist &   &   &   &   &   \\
Krieger     &   & x &   & x &   \\
Kleriker    &   & x &   & x & x \\
Magier      &   & x &   &   &   \\
M"onch      &   &   &   & x &   \\
Paladin     & x & x &   & x &   \\
%Prediger    & x & x &   & x & x \\
Schamane    &   &   &   & x & x \\
Seher       &   &   &   &   &   \\
Streuner    &   &   &   &   &   \\
Tiermeister &   &   &   & x &   \\
Waldl�ufer  &   &   &   & x &   \\
            &   &   &   &   &   \\
\caption{Rassenbeschr�nkungen f�r Charakterklassen}
\label{tabelle_rassenbeschraenkungen}
\end{longtable}
