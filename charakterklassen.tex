\chapter{Charakterklassen\index{C!Charakterklasse}}
\label{charakterklassen}
\section{Einleitung}
\par In den ersten Kapiteln haben wir bereits die drei grundlegenden Abenteurerklassen\index{A!Abenteurerklassen} 
vorgestellt, die in keinem High Fantasy Rollenspiel fehlen sollten: Krieger, Magier und Dieb.
\par Der folgende Abschnitt stellt nun die mit diesem Regelwerk erscheinenden Charakterklassen in ihrer F"ulle 
vor und gibt damit sowohl dem Spieler als auch dem Spielleiter ein kleines Sammelwerk f"ur ihre Abenteurerrunde 
mit. Das diese Sammlung nicht als vollst"andig zu betrachten ist, brauche ich wohl nicht extra zu betonen. 
Es obliegt der Absprache zwischen Spieler und Spielleiter weitere Klassen zu kreieren oder g�nzlich auf eine
zu verzichten.
\par Das n"achste Kapitel (ab S. \pageref{dierassen}) stellt die zur Verf"ugung stehenden Rassen vor.
\par Jede Klassenbeschreibung\index{K!Klassenbeschreibung} folgt der unten aufgelisteten Struktur.
\begin{description}
\item[Beschreibung:]Umfassende Beschreibung der Charakterklasse
\item[Wesen:]rollenspieltechnische Hinweise f"ur den Spieler; Beschreibung der allgemeinen 
Wesensz"uge\index{W!Wesensz"uge}, St"arken\index{S!St"arken} und Schw"achen\index{S!Schw"achen}
\item[Ausbildungsweg\index{A!Ausbildungsweg}:]Wie erfolgte die Ausbildung\index{A!Ausbildung}? Wann begann 
diese, wie lange dauerte sie?
\item[Vorz"uge\index{V!Vorz"uge} / Nachteile\index{N!Nachteile}:]Hier stehen Informationen "uber Fertigkeiten, 
die die Charakterklasse besser oder schlechter beherrscht
\item[Besonderes:]Hier wird etwas stehen, falls die Charakterklasse Besonderheiten, wie \mbox{z. B.} besondere 
F"ahigkeiten besitzt oder �ber spezielle Eigenarten verf"ugt.
\end{description}

\newpage
\section{K�mpferklassen}
\subsection{Der klassische Krieger\index{K!Krieger}}
\subsubsection{Beschreibung}
\par Der Archetyp Krieger lie"se sich m"uhelos noch weiter klassifizieren, doch w"urde dies 
nur mehr Namen f"ur ein und die selbe Person in anderer Kleidung mit sich bringen.
\parpic[r]{\epsfig{file=pics/krieger.eps, scale=0.6}}
Wer kennt ihn nicht, den stolzen Krieger der Palastgarde, den k"onigstreuen Ritter\index{R!Ritter}, den aufbrausenden 
Hochlandbarbaren\index{B!Barbar} oder aber den S"oldner\index{S!S�ldner} f"ur den Ehre einfach nur eine Frage des Lohnes ist. All diese 
z"ahlen wir zur Klasse des Kriegers.
\par Sie haben alle eines gemeinsam: Eine hervoragende Ausbildung an den Waffen.
\par Krieger ziehen aus vielerlei Gr"unden auf Abenteuer. Der Ritter zieht aus, um auf Gehei"s seines 
K"onigs einen fl"uchtigen Schwerverbrecher zu fassen oder aber auf einer Queste ein seltenes und doch 
wichtiges Artefakt zu bergen. Der Barbar wird sich mit Sicherheit auf einer Reise befinden, um neue 
Gegenden kennenzulernen oder er ist der Sohn des H"auptlings und er muss sich in der Welt beweisen, 
bevor er dessen Nachfolge antreten kann. Und was den S"oldner betrifft...
\subsubsection{Wesen}
\subsubsection{Ausbildungsweg}
\subsubsection{Vorz"uge / Nachteile}
\subsubsection{Besonderes}

\newpage
\subsection{Der Ordenskrieger\index{O!Ordenskrieger} / Paladin\index{P!Paladin}}
\subsubsection{Beschreibung}
\parpic[r]{\epsfig{file=pics/paladin.eps, scale=0.2}}
%Alternatives Bild eines Paladin
%\parpic[r]{\epsfig{file=pics/paladin2.eps, scale=0.5}}
Der Ordenskrieger/Ordensritter oder auch besser bekannt als Paladin steht als Symbol f�r den kriegerischen 
Arm der Kirche. Vom Prinzip her ist ein Paladin nat�rlich auch nicht viel anders als der bereits beschriebene 
Krieger. Doch es gibt feine Unterschiede, die eine weiter Klassifizierung rechtfertigen: Der Paladin k�mpft im 
Namen der G�tter und ist somit von ihnen bevorzugt. Oder auch anders gesagt: Er k�mpft im Namen der Kirche und 
darf sich auch mal mehr erlauben.
\par Wie die Priester auch verf�gt der Paladin �ber die Gabe seinee G�tter in Form von Wundern um Hilfe zu 
bitten. Dabei ist diese Kraft oder besser gesagt die Beg�nstigung nicht so weit fortgeschritten wie die eines 
wirklichen Geistlichen. Schlie�lich liegt die St�rke des Paladins im F�hren der Klinge.
\par Der Paladin ist unterwegs, um auf der einen Seite das Wort seines Gottes oder seiner G�tter zu verbreiten. 
Auf der anderen Seite ist er st�ndig auf der Suche nach Zeichen, die die G�tter auf der Welt hinterlassen haben, 
meist in Form von heiligen Reliquien.
\par Paladine genie�en ein hohes Ansehen in der Bev�lkerung und werden von der Kirche hervoragend ausgestattet.
\subsubsection{Wesen}
\par Paladine sind edle K�mpfer f�r die Gerechtigkeit. Sie treten f�r die Bed�rftigen ein und preisen das 
Wort ihrer G�tter. Das jedenfalls erz�hlen die Paladine von sich. In der Praxis beschr�nkt sich ihre T�tigkeit 
jedoch meist auf letzteres. Von diesem Vorurteil seien jedoch nicht alle betroffen. Unter den Paladinen gibt es 
immer noch jene noblen Recken, von denen die Legenden berichten.
\subsubsection{Ausbildungsweg}
\par Kinder aller Bev�lkerungsschichten und egal welchen Geschlechts k�nnen bei einem Alter von 5 Jahren in die 
Kirche zur Ausbildung gebracht werden. Dort werden sie von den Priestern aufgenommen und im Laufe ihrer Ausbildung 
den verschiedenen Berufungen zugeteilt. Die Grundausbildung im Tempel dauert 7 Jahre. In diesen Jahren lernen die 
Kinder Lesen, Schreiben und Rechnen, werden gesellschaftlich geschult und in Fremdsprachen unterrichtet.
\par Kinder, die dann zum Paladin ausgebildet werden sollen, werden mit dem 12. Lebensjahr zum Knappen ernannt. Von 
nun an beginnt ihre Grundausbildung im Kampf und im Reiten, die erst im Alter von 18 Jahren ihr Ende finden wird. 
Parallel dazu werden sie im Tempel von den Priestern im Wirken von Wundern geschult.
\par Erreicht der junge Sch�ler den Rang eines Paladins, so mu� er sich fortan auf einer 2-j�hrigen Reise beweisen. 
Hierzu erh�lt er gerade mal das N�tigste. Von nun an mu� er sich auf sich allein verlassen.
\par Kehrt er von der Reise zur�ck und ist stark im Glauben geblieben, so wird er in den Tempel als vollwertiger 
Paladin aufgenommen.
\subsubsection{Vorz�ge / Nachteile}
\par Paladine, die auf dem rechtschaffenden Wege wandern, sind in der Lage um Wunder von ihren G�ttern zu bitten, 
verlassen sie diesen Weg, werden sie vor�bergehend von ihren G�ttern verlassen.
\par Paladine k�nnen einfache Wunder wie \textit{Wunden heilen} oder \textit{Magie erkennen} erlernen oder aber auch 
um Wunder bitten, die sie im Kampf st�rken.
\subsubsection{Besonderes}
\par Paladine erhalten eine Reihe von Auflagen, die ihnen von ihrem Glauben und ihrer Kirche nicht nur auferlegt 
sind, sondern die sie in ihrem Wesen verewigt haben.
\begin{itemize}
\item Ein Paladin betet mindestens 3 mal pro Tag zu den G�ttern und dankt ihnen f�r den Rat und den Beistand, den 
sie ihm auf seinen Reisen gew�hren
\item Ein Paladin wird niemals von sich aus einen Kampf beginnen, auch nicht wenn er dazu herausgefordert wird. Er 
wird sich allerdings sehr wohl auf eine Verteidigung vorbereiten; eine Ausnahme stellt der Kreuzzug dar. Er wird auch 
niemals einen unbegr�ndeten Kampf f�hren.
\item Ein Paladin der k�mpft, wird diesen Kampf ehrbar durchf�hren. Er wird keinen Gegner mit einer �bermacht 
angreifen und auch nicht gegen Gegner vorgehen, die ihm offensichtlich unterlegen sind.
\end{itemize}
\par Es sei noch zu ber�cksichtigen, dass die Aufgabe des Paladins nicht haupts�chlich darin liegt, Gl�ubige zu 
finden und zu bekehren, die ist vorrangig die Aufgabe der Kleriker. Der Paladin hat das Wort der G�tter zu sch�tzen 
und dar�ber zu wachen, dass es erhalten und nicht geschm�lert wird. Haupts�chlich wird er jedoch der Aufgabe folgen, 
die ihm der Tempel auferlegt hat.


\newpage
\section{Abenteurer/Gl�cksritter}
\subsection{Der Barde\index{B!Barde}}
\begin{quotation}
\textit{Wo ich bin gef�llt es mir, doch will ich nicht lang
bleiben}\\
\textit{Doch wo ich nie gewesen bin, dort will ich verweilen}
\end{quotation}
\subsubsection{Beschreibung}
\par Egal ob Poet\index{P!Poet}, Minnes"anger\index{M!Minnes"anger} oder Musiker\index{M!Musiker}, sie alle 
sind Barden und fr"ohnen der Kunstrichtung Musik und Dichtung. Hoffnungslos romantisch und mit unbeschreiblicher 
Vorstellungskraft gesegnet, so sehen sich jedenfalls die meisten Barden.
\parpic[r]{\epsfig{file=pics/barde2.eps, scale=0.5}}
Unter ihnen gibt es selbstverst"andlich auch die wandernden Musiker. Sie begleiten Abenteurergruppen, immer 
auf der Suche nach neuen Geschichten.
\par Die Barden sind es, die die Geschichten weitererz"ahlen und neue Legenden schaffen. Vielleicht liegt in 
ihrem Wissen noch ein wichtiger Schl"ussel f"ur die Zukunft?
\subsubsection{Wesen}
\par Barden sind von Grund auf aufgeschlossen. Es liegt in ihrer Natur auf die Menschen zuzugehen und sich mit 
ihnen zu unterhalten. Auf diese Art und Weise nehmen sie Informationen auf. Einen schlecht gelaunten Barden wird 
man ebenso oft treffen wie einen riesenw�chsigen Zwerg, eigentlich nie und es bedarf schon einer besonderen 
Einwirkung von Au"sen, um dies zu bewerkstelligen.
\subsubsection{Ausbildungsweg}
\par Das Talent eines Barden wird sich schon im fr"uhen jugendlichen Alter zu erkennen geben. Musikalisches 
Talent, Redegewandheit und die Kunst zu dichten und zu schreiben zeichen einen jungen Barden aus. Meist wird 
er dann die Hilfe eines erfahrenen Barden aufsuchen und bei ihm in die Lehre gehen. Mindestens genauso viele 
bringen sich alles n�tige selber bei. \par Als n"achstes wird es ihn in die Ferne ziehen, wo er sich einer 
Gruppe von Abenteurern anschlie"sen wird, dies ist der zweite und letzte Teil seiner Ausbildung. So lernt er 
die Welt kennen, bilder seine Talente weiter und wird Geschichten schreiben.
\subsubsection{Vorz"uge / Nachteile}
\subsubsection{Besonderes}

\newpage
\subsection{Der Dieb\index{D!Dieb}}
\subsubsection{Beschreibung}
\par Heimlichkeit und Geschick sind die Disziplinen des Diebes. Dabei sprechen wir hier jetzt nicht von dem
 Stra"senr"auber\index{S!Stra"snr"auber} oder dem Gelegenheits-Einbrecher\index{E!Einbrecher}. Der Dieb sieht 
 in seinen Taten ein Ritual.
\parpic[r]{\epsfig{file=pics/dieb.eps, scale=1}}
Jede Aktion hat ihren eigenen Reiz und ihre eigene Atmosph"are. Er stiehlt meist nicht um der Beute willen, 
sondern um seine Grenzen zu erfahren und sich jedes mal von neuem zu �bertreffen.
\par F"ur Diebe gibt es keine verschlossenen T"uren oder versteckte Fallen. Ihr Instinkt hat ihnen bisher noch
 aus jeder Situation heraus geholfen.
\subsubsection{Wesen}
\subsubsection{Ausbildungsweg}
\subsubsection{Vorz"uge / Nachteile}
\par Fast jeder Dieb ist ein Mitglied \textit{der Schatten}\index{S!Schatten, die}. Die Schatten sind die 
Kontinentweite Gilde der Diebe. Mitglieder dieser Gilde k"onnen jederzeit auf deren Hilfe bauen. Ob es nun 
um Unterst"utzung, Unterschlupf, Transportm"oglichkeiten oder einfach nur Informationen geht. Im Gegenzug 
\glqq spendet\grqq{} jeder Dieb einen Zehnt seines \glqq Verdienstes\grqq{} der Gilde. Um die Loyalit"at 
ihrer Mitglieder zu wahren, werden die Neuank"ommlinge magisch an den Gildenrat gebunden. Dieses macht es 
dem Einzelnen unm"oglich ihnen gegen"uber die Unwahrheit zu sprechen.
\par Gildenlose Diebe haben es schwer. Sie stehen nicht unter territorialer Kontrolle und k"onnen nicht auf 
die Unterst"utzung der Gilde oder deren Mitglieder bauen, unter ihnen gelten sie als ge"achtet. Im Gegenzug 
k"onnen sie ihre Eink"unfte f"ur sich behalten.
\subsubsection{Besonderes}

\newpage
\subsection{Der Gl�cksritter\index{G!Gl�cksritter} / Streuner\index{S!Streuner}}
\subsubsection{Beschreibung}
\par F"ur den Gl"ucksritter gibt es viele Namen und er tritt in mindestens so vielen Variationen in den 
unterschiedlichsten Gegenden auf. Er ist ein Kind der Strasse, ein Abenteurer, ein Wanderer und letztendlich 
auch ein Freibeuter auf der See. Der typische Mantel und Degen Held eben.
\parpic[r]{\epsfig{file=pics/streuner.eps, scale=0.6}}
Wir sehen hier vor uns die typischen Bilder der Menschen, die niemals das Abenteuer missen wollen, die es nicht 
ertragen k"onnen lange an einem Ort zu verweilen oder aber in ihrer Freiheit eingeschr"ankt zu werden.
\subsubsection{Wesen}
\par Ein gro"ses Maul hat er wohl und mit der Wahrheit nimmt er es auch nicht immer zu ernst. Und wenn es mal 
drauf an kommt, dann nimmt er auch schon mal die Beine in die Hand. So jedenfalls wird man einen Streuner meist 
erleben.
\par Im Laufe seiner Abenteurerkarriere wird er jedoch versuchen, und das sehr erfolgreich, von seinen 
Weggef"ahrten zu lernen, was zu lernen ist. Und fr"uher oder sp"ater wird er merken, dass der Weg nach vorne 
auch zum Ziel f"uhren kann.
\subsubsection{Ausbildungsweg}
\par Es gibt keine bessere Ausbildung zum Gl"ucksritter, als die Stra"se selbst. Hier lernt der Gl"ucksritter 
von Kindesbeinen an auf eigene Faust zu "uberleben. Obwohl sein "Uberlebenstraining in den ersten Jahren wohl 
haupts"achlich aus Nahrungsbeschaffung besteht. Mit sehr gro"ser Wahrscheinlichkeit ist er in den ersten Jahren 
auch in Waisenh"ausern untergekommen, wo er mit Gl"uck einige Kenntnisse in Wort und Schrift gelernt hat, lange 
geblieben sein wird er dort jedoch nicht.
\subsubsection{Vorz"uge / Nachteile}
\subsubsection{Besonderes}


\newpage
\section{Der Waldl�ufer}
\subsection{Der klassische Waldl�ufer\index{W!Waldl�ufer}}
\subsubsection{Beschreibung}
\par J�ger und Sammler, F�hrtenleser, ein Kind der Wildniss. Waldl�ufer verstehen sich auf das �berleben 
in der Wildniss und im Wald. Sie verstehen die F�hrten wilder Tiere zu lesen und sie verstehen es, Spuren im 
Wald zu deuten.
\parpic[r]{\epsfig{file=pics/waldlaeufer.eps, scale=0.7}}
Im Grunde seines Herzens ist der Waldl�ufer ein geselliger Typ, auch wenn er nach au�en hin einen Einzelg�nger 
reflektiert. Doch die Meinung, die die gemeine Bev�lkerung von diesem verschrobenen gr�nen M�nnchen hat, ist 
pr�gend f�r sein Auftreten.
\subsubsection{Wesen}
\par Obwohl der Waldl�ufer allgemeinhin als Einzelg�nger und verschrobener Einsiedler gilt, so lernen diejenigen, 
die mit ihm Freundschaft schlie�en, eine ganz andere Seite kennen. Er ist Freunden gegen�ber aufgeschlossen und 
wei� sich in eine Gruppe einzugliedern. Er bietet seine Dienste an, wo er kann, bringt jedoch auch eine gesundes 
Ma� an Misstrauen an den Tag, wenn ihm etwas suspekt erscheint.
\par Der Waldl�ufer ist niemand, der intuitiv handelt. Jeder Schritt ist geplant und weitergedacht, immer auf 
m�gliche Folgen fixiert. Wer jetzt denkt, dass der Waldl�ufer langsam handelt, der liegt falsch, denn der Waldl�ufer 
kalkuliert schnell und zuverl�ssig.
\subsubsection{Ausbildungsweg}
\par Die Ausbildung des Waldl�ufers erfolgt, wer h�tte es gedacht, im Wald. Waldl�ufer gehen nicht auf die 
Suche nach neuen Sch�lern, sie erwarten vielmehr, dass diese den Weg zu ihnen finden. Bevor der Waldl�ufer den 
Sch�ler annimmt, muss dieser eine Aufnahmepr�fung bestehen. Diese besteht meist aus verschiedenen Jagd- und 
Suchaufgaben im Wald. F�llt diese erfolgreich aus, ist der halbe Weg schon gemacht. Der zuk�nftige Sch�ler 
muss den Lehrmeister nur noch �berzeugen, warum er gerade ihn als Lehrling annehmen soll. Im Normalfall hat ein 
Waldl�ufer immer 3 bis 4 Sch�ler in seinem Haus.
\par Nach Ende der Ausbildung, im Regelfall nach 5 Jahren, verabschiedet sich der Sch�ler von seinem Lehrmeister 
und zieht auf eine mindestens 10-j�hrige Wandertour, um fremde Gegenden, Tiere und Pflanzen kennen zu lernen. 
Anschlie�end suchen sich die meisten Waldl�ufer eine ruhige Waldgegend und lassen sich nieder, um ihr Wissen 
an andere weiterzugeben oder stellen sich in den Dienst von ans�ssigen F�rsten. Nur wenige bleiben weiter auf 
dem Pfad des Abenteurers.
\subsubsection{Vorz�ge / Nachteile}
\par Dem Waldl�ufer ist es m�glich, selbst dann noch Spuren zu erkennen, wenn der menschliche Verstand bereits 
versagt. Er erh�lt magisches Spuren lesen (+1) und kann die Fertigkeit Jagen erlernen. Genaue Informationen zu 
neuen Fertigkeiten sind im Abschnitt \textit{Neue Fertigkeiten} (siehe S. \pageref{neuefertigkeiten}) zu finden.
\subsubsection{Besonderes}

\newpage
\subsection{Der Tiermeister\index{T!Tiermeister}}
\subsubsection{Beschreibung}
\par Obwohl er viel mit dem Waldl�ufer gemeinsam hat, so gibt es doch auch ein paar gravierende Unterschiede. 
W�hrend der Waldl�ufer zwar in der Lage ist, aus dem Verhalten der Tiere zu schlie�en, hat der Tiermeister 
jedoch gelernt mit ihnen zu kommunizieren.
\parpic[r]{\epsfig{file=pics/tiermeister.eps, scale=0.4}}
Der Tiermeister scheint �ber �bersinnliche F�higkeiten zu verf�gen, die ihm eine Kommunikation mit Tieren 
oder \glqq intelligenten\grqq{} Pflanzen erlaubt.
\par Tiermeister ziehen nur aus zwei Gr�nden in das Abenteurerleben, entweder er ist auf einem Vergeltungszug f�r 
eine gro�e Missetat, die den Tieren seines Waldes wiederfahren ist oder er ist auf der Suche nach seinem Sch�ler.
\subsubsection{Wesen}
\par Haftet dem Waldl�ufer das Bild des Einzelg�ngers an, so stellt der Tiermeister den typischen Einzelg�nger 
dar. Ihm f�llt es schwer, Kontakte zu anderen Menschen aufzubauen. Seine Welt und seine Gef�hrten sind die Tiere.
 Eine Freundschaft zu einem Tiermeister baut sich nur langsam auf und erfordert ein hohes Ma� an Geduld und 
 Vertrauen. Sollte man den Tiermeister darin entt�uschen, hat man ihn sich zu einem erbitterten Feind gemacht. 
 Eine aufgebautet Freundschaft bleibt ein Leben lang erhalten und der Tiermeister ist ein loyaler Freund.
\par Tiermeister schlie�en sich nichts desto trotz �fters Abenteurergruppen an. N�mlich immer dann, wenn sie 
auf Wanderschaft gehen (siehe oben). Er wird sie in jedem Fall als seine Gef�hrten akzeptieren und wenn er es 
als n�tig und sinnvoll erachtet, seine Dienste und F�higkeiten anbieten. In jedem Fall steht er der Gruppe 
loyal gegen�ber, so lange er das Gef�hl hat, sich auf sie verlassen zu k�nnen. Verl�sst ihn das Gef�hl, 
so wird er sie in der n�chten Nacht still und heimlich verlassen.
\subsubsection{Ausbildungsweg}
\par Gr��er k�nnte der Gegensatz zum Waldl�ufer nicht sein. Niemand kommt zu einem Tiermeister, um bei ihm 
zu lernen. Vielmehr findet der Tiermeister seine Sch�ler und das auf nicht ganz konventionelle Art und Weise. 
Jeder Tiermeister bildet in seinem Leben nur einen Sch�ler aus und h�lt zu diesem sein Leben lang einen engen 
Kontakt.
\par Irgendwann wird der Tiermeister von einem Traum heimgesucht, der ihn mit seinem zuk�nftigen Sch�ler bekannt
 macht. Er bekommt vage Visionen �ber den derzeitigen Aufenthaltsort, sollte diese schon geboren sein, und sein 
 Aussehen. Ab dem n�chsten Tag begibt sich der Tiermeister auf eine lange Wanderschaft, die bis zur Findung seines 
 Sch�lers andauert. Dies k�nnte sich nat�rlich umso schwieriger gestalten, sollte der Zuk�nftige erst in kurzer 
 Zeit geboren werden.
\par Hat er seinen Sch�ler gefunden, dieser wird zum Zeitpunkt des Findens zwischen einem Monat und 5 Jahren alt 
sein, �berzeugt der Tiermeister die Eltern des Kindes von 
dessen Talent. Sollte ihm dies gelingen, zieht er mit seinem neuen Sch�ler zur�ck zu seiner Wohnst�tte und 
beginnt die Ausbildung. Ein Tiermeister der seinen Sch�ler zu sp�t findet - Kinder �ber 5 Jahre haben nicht mehr
die Gabe die Sprache des Waldes zu lernen - oder aber die Eltern nicht von dessen Talent �berzeugen kann, wird
 ebenfalls wieder zur�ckkehren und ein normales Leben 
fortsetzen, er wird allerdings niemals wieder eine Vision von einem Sch�ler bekommen und somit auch nie einen
 Nachfolger ausbilden.
\subsubsection{Vorz�ge / Nachteile}
\subsubsection{Besonderes}
\par Tiermeister haben die besondere F�higkeit, mit Tieren kommunizieren zu k�nnen.
\par Besonders alte Tiermeister ziehen noch einmal los, um ihre letzte Ruhest�tte aufzusuchen. Diese ist niemanden 
bekannt, der jetzt noch unter uns weilt, und dem Sterbenden offenbart sie sich auch erst in den letzten Wochen 
seines Lebens. An diesem geheimnisumwobenen Ort legt er sich zur letzten Ruhe nieder. Seine Seele vereint sich mit 
dem gro�en Geist der Tierwelt und er wird im Moment seines Todes im K�rper eines Tieres wiedergeboren.


\newpage
\section{Magiebegabte}
\subsection{Der Magier\index{M!Magier}}
\subsubsection{Beschreibung}
\par Der Magier\index{M!Magier} hat von Kindesbeinen an gelernt
die Natur zu kontrollieren. Er ist eins mit den arkanen
Str"ohmungen. W"ahrend seiner langen
Ausbildungszeit wurde ihm beigebracht diese St"urme zu erkennen und
f"ur seine Zwecke zu formen. Magier sind intelligent
und sinnesscharf. Ihnen entgeht nichts. Sie haben allerdings keine
Kenntnisse im Umgang mit Waffen.
\par Ein weiterer Teil ihrer Ausbildung war das Studium
alter Schriften und Folianten.
Magier sind sprachbegabt und kennen Mundarten und
Schriftzeichen, von deren Existenz andere Sterbliche noch nie etwas
geh"ort haben.
\subsubsection{Wesen}
\par Der Magier ist ein ewiger Student. Er hat sein Leben der Magie
gewidmet. Sein Studium der Magie hat ihm die Macht seiner Kunst
gelehrt, er wei� sie mit Bedacht einzusetzen und ist sich ihrer
Gefahren bewusst. Die meisten Magier betrachten sich als
Wissenschaftler und Magie als Wissenschaft, sie w�rden sich niemals
dazu hinrei�en lassen, sie verschwenderisch oder gar prahlerisch zu
nutzen.
\subsubsection{Ausbildungsweg}
\par Das Studium der Magie ist ein langer und schwerer
Ausbildungsweg. Nur wenige Sch�ler der Akademien k�nnen sich r�hmen
am Ende auch einen verbrieften Abschluss in H�nden zu halten. Die
meisten werden fr�her oder sp�ter in einer der P�fungen scheitern
und fortan dem Selbststudium verfallen und als reisende Forscher
unterwegs sein oder als wissenschaftliche Mitarbeiter in der
Akademie ihr Studium der Magie fortsetzen, sich somit von Pr�fung zu
Pr�fung schleppen, in der Hoffnung doch noch mal den Abschluss zu
erreichen.
\par Das magische Potential muss bereits in fr�hen Jahren erkannt
und gef�rdert werden, um es voll aussch�pfen zu k�nnen. Im Regelfall
beginnt die Ausbildung des neuen Sch�lers nach bestandenem
Aufnahmetest mit ca. 4 Jahren. In den ersten 10 Jahren lernt der
Sch�ler neben einigen niederen Magie�bungen zum Erhalt des
Potentials haupts�chlich allgemeines Wissen der Wissenschaften
Mathematik und Alchimie und lernt Sprachen, Lesen und Schreiben und
Magietheorie.
\subsubsection{Vorz"uge / Nachteile}
\par Der Magier kann Spr�che aus nahezu allen Psi-Bereiche
mit Ausnahme von Dimension und Beschw"orung erlernen. Zu Spielbeginn
w"ahlt er frei 3 Zauber aus der Liste aus.
\subsubsection{Besonderes}
\par Jeder Magier muss zu Beginn des Spieles w�hlen, ob er einem
Zirkel angeh�rt oder ein freier Magier ist (siehe dazu das Kapitel
\textit{Die Magie} ab S. \pageref{spruchmagie}), und ob er einen Abschluss
gemacht hat oder nicht.

\newpage
\subsection{Der Beschw�rer\index{B!Beschw�rer, der}}
\subsubsection{Beschreibung}
\par Erst einmal denkt jeder bei der Bezeichnung \glqq Beschw�rung\grqq{}
an schwarze Magie. Man denkt an D�monen\index{D!D�mon},
Geister\index{G!Geist}, Unholde\index{U!Unhold} und
Elementarwesen\index{E!Elementarwesen}. Also an alles
�bernat�rliche, was die eigene Phantasie so her gibt.
\parpic[r]{\epsfig{file=pics/beschwoerer.eps, scale=0.6}}
Und was soll ich sagen, sie haben alle Recht. Das ist exakt das
Gebiet, mit dem sich dieser Zweig der Magiewirker besch�ftigt und
damit ist in der Vergangenheit so mancher Beschw"orer nicht zu
Unrecht auf dem Scheiterhaufen einer seelischen Grundreinigung
unterzogen worden.
\par Doch wir wollen nicht gleich alle auf einmal zum Teufel w"unschen.
Die Gilde der Beschw�rer teilt sich ebenso wie die Gilde der Magier
in verschiedene Ausrichtungen auf und die meisten besch"aftigen sich,
zum Wohle aller, mit der wei"sen Magie. Also die Beschw"orung der
Geister der Toten f"ur Befragungszwecke oder die Erschaffung von
"ubernat"urlichen Wesen f"ur Boteng"ange. Trotz allem bleiben
nat"urlich noch die Individuen, die der Schwarzmagie fr"onen und
diese gilt es zu meiden.
\subsubsection{Wesen}
\par Stumm und verschlossen. Dies sind wahrscheinlich die ersten
Eindr"ucke, die man von einem Beschw"orer bekommen wird und
wahrscheinlich auch meist die Einzigen. Selten wird er sich
ungefragt an einer Diskussion beteiligen und seine Meinung "au"sern.
Viel n"aher liegt es ihm, alles was gesagt wird, sich zu merken.
Spricht man ihn an und fragt ihn etwas, so antwortet er in kurzen,
aber pr"azisen S"atzen oder aber in einer so komplizierten
Formulierung, dass es besser w"are, man h"atte gar nicht gefragt,
denn hinterher versteht man weniger als vorher.
\par Ein Beschw"orer der sich einer Gruppe von Abenteurer
angeschlossen hat, wird ihnen gegen"uber ein wenig aufgeschlossener
sein. Er wei"s, dass es n"otig ist im Team zu arbeiten, damit er in
Notsituationen auf ihre Hilfe setzen kann. Geheimnisse wird er
nichts desto trotz eher f"ur sich behalten, die anderen wissen eh
nicht damit umzugehen.
\subsubsection{Ausbildungsweg}
\subsubsection{Vorz"uge / Nachteile}
\par Der Beschw"orer kann nur die Psi-Bereiche Elemente und
Beschw"oren erlernen. Beschw"orer sind im Umgang mit Elementarwesen
und D"amonen geschult und zeigen keine Furcht vor ihrem Erscheinen
und Auftreten. In Gegenwart dieser Wesen erhalten sie einen Bonus
auf Geistige Stabilit"at (+2).
\subsubsection{Besonderes}

\newpage
\subsection{Der Druide\index{D!Druide}}
\subsubsection{Beschreibung}
\par Verkannt, missverstanden und ignoriert. Besser kann man den Druiden wahrscheinlich nicht beschreiben.
\parpic[r]{\epsfig{file=pics/druide.eps, scale=1}}
Von allen anderen Magiebegabten wird der Druide eher mit einem L"acheln bedacht. Nur all zu gut kennen sie 
das Bild des Mistel schneidenden Alten. Dabei sollte man seine Macht nicht untersch"atzen. Die Druidenzirkel 
haben "uber die Jahrhunderte einen Fundus an historischem Wissen angesammelt, bei dessen Anblick, und dies ist 
nur bildlich gesprochen, denn Druiden geben ihr Wisssen nur m"undlich weiter und schreiben es nicht nieder, 
die Gelehrten erblassen w"urden.
\par Vom einfachen Volk wird der Druide meist als m"annliches Gegenst"uck der Hexe genannt. Mit diesem 
Vorurteil k"ampfen die Zirkel schon seit ihrer Entstehung.
\subsubsection{Wesen}
\subsubsection{Ausbildungsweg}
\par Die Druidenzirkel suchen ihre Sch"uler im einfachen Volk, alle anderen sind zu weit weg vom echten Leben.
 Meistens suchen sie Waisenh"auser auf und unterhalten sich dort mit den Kindern. Die Kinder, die sich als
  w"urdig und geeignet f"ur eine Ausbildung erweisen, werden von den Zirkeln adoptiert. Bei der Auswahl der
   Kinder wird darauf geachtet, dass sie nicht "uber 6 Jahren sind, ein gutes Verh"altnis zu Tieren und der 
   Natur zeigen und von einem ungestillten Wissensdurst getrieben werden. Die Ausbildung eines Druiden dauert 
   im Grunde ein Leben lang, w"urde jedenfalls der Druide sagen, fragt man ihn direkt, man hat nie zu Ende 
   gelernt. Grunds"atzlich gilt aber, dass die Grundausbildung ca. 8 jahre dauert. In dieser Zeit lernt der 
   Sch"uler alles notwendige Grundwissen in Naturkunde, Wissenschaft und Zauberei.
\subsubsection{Vorz"uge / Nachteile}
\subsubsection{Besonderes}

\newpage
\subsection{Der Hexer / Die Hexe\index{H!Hexe}\index{H!Hexer}}
\subsubsection{Beschreibung}
\par Hexe! Dieses Wort treibt der Menschheit die Panik in die
Augen. Wer an Hexe denkt an Fl"uche, wer Hexer h"ort sieht den
Schwarzmagier vor Augen.
\parpic[r]{\epsfig{file=pics/hexe.eps, scale=0.5}}
Kein Archetyp ist gef"urchteter und wahrscheinlich mehr gehasst als
dieser. Dabei sehen sich die Hexen verkannt. Ihre Absichten gleichen
sich in keiner Weise mit dem Bild, dass man von ihnen auf dem Banner
tr"agt.
\par Doch woran erkennt man Hexen? Junge Hexen sind grunds"atzlich
bildsch"on und alte Hexen haben eine krumme Nase. Naja, stimmt auch
nicht immer, aber kann will man schon gegen Vorurteile machen.
\subsubsection{Wesen}
\par Das Wesen einer Hexe ist so wandelbar wie das Schicksal dieses
Kontinents. Ihr Gem"ut ist von ihrer Tagesform abh"angig und meist
wird man die Wahrheit nicht erkennen, denn Hexen sind zu dem sehr
gute Schauspieler und verstehen es, ihre Mitmenschen zu
manipulieren. Entweder durch ihren Charme oder aber durch Magie.
\par Man sollte sich jedoch nie den Zorn einer Hexe auf sich ziehen,
denn dieser wird immer auf einen zur�ck kommen und das dann meist in
Form eines Fluches.
\subsubsection{Ausbildungsweg}
\par Hexen suchen sich keine Sch"uler. Das Wissen wird nur innerhalb
der Familien weiter gegeben. Dabei ist es jedoch nicht zwingend
notwendig, dass der Partner ebenfalls ein Hexer oder eine Hexe ist,
damit die Kinder die magische Veranlagung erben, ein Elternteil
reicht aus.
\par So ungef"ahr im Alter von 6 Jahren beginnt die Ausbildung der
jungen Sch"uler. Um das magische Potential der Kinder zu aktivieren,
bedarf es eines besonderen Rituales. Der Hexenerlternteil tritt mit
seinem Kind oder seinen Kindern, so es Zwillinge sind, eine Reise
zur allj"ahrlichen Hexennacht an. Diese findet immer zum
Jahreswechsel statt.
\subsubsection{Vorz"uge / Nachteile}
\par W"ahrend ihrer Ausbildung lernen die Hexen auf Besen zu fliegen
(+1). Eine genaue Beschreibung dieser Fertigkeit ist im Kapitel
\textit{Neue Fertigkeiten} (siehe S. \pageref{neuefertigkeiten}) zu
finden.
\subsubsection{Besonderes}
\par Die Hexe w"ahlt sich w"ahrend ihrer Ausbildung einen Vertrauten.
Meist einen Vogel oder eine Katze. Mit diesem kann sie auf
telepathischem Wege kommunizieren. Er versteht einfache Befehle und
antwortet in einfachen S"atzen.
\par Die Hexe kann Fluchmagie verwenden. Eine Beschreibung dieser
Magierichtung, sowie einiger Fl"uche findet sich im Kapitel
\textit{Fluchmagie} (siehe S. \pageref{fluchmagie})

\newpage
\subsection{Der Illusionist\index{I!Illusionist}}
\subsubsection{Beschreibung}
\parpic[r]{\epsfig{file=pics/illusionist.eps, scale=0.55}}
Die Magier k"onnten sich vor Lachen kaum halten, wenn der Illusionist in ihrer Gegenwart verlauten 
lassen w"urde, er sei ein Magier. Doch die Kraft der Illusionisten darf nicht untersch"atzt werden. 
Schon so manch lachender Magier ist an der pl�tzlichen F"ullung des Mundes mit einem nie enden 
wollenden Wasserstrahl erstickt. Nachweisen konnte man dem Illusionisten jedoch nichts, Fl"ussigkeit 
wurde keine gefunden.
\par Den wandernden Illusionisten trifft man entweder unter den Mitgliedern des fahrenden Volkes oder 
aber in einer Horde schatzsuchender Abenteurer.
\subsubsection{Wesen}
\par Auf seinen Reisen wird der Illusionist sich tarnen, um nicht sein wahres \glqq Ich\grqq{} preiszugeben. 
Denn ist ein Illusionist erst einmal als solcher erkannt, wird es ihm schwer fallen seine Gegen"uber zu verwirren.
\subsubsection{Ausbildungsweg}
\par Es gibt keine direkte Ausbildung f�r Illusionisten, irgendwann im Laufe ihrer Kindheit erkennen sie ihr 
Talent selbst.
\subsubsection{Vorz"uge / Nachteile}
Illusionisten sind resistenter gegen die Werke ihrer Kollegen und erhalten immer einen
Bonus von +2 auf Psiwiderstandsw�rfe gegen den Magiebereich Illusion.
\subsubsection{Besonderes}
\par Obwohl der Illusionist niemals auf seine Illusionen hereinfallen wird, so wird er sie doch immer sehen 
k"onnen. Er k"onnte ja auch schlecht mit ihnen arbeiten, w"are es nicht so. 

\newpage
\subsection{Der Schamane\index{S!Schamane}}
\subsubsection{Beschreibung}
\begin{quotation}
\par\textit{\glqq Bei dem Schamahn kann man sich streyten, ob man ihn nun zu den Magiern oder zu den Geweihten zelt. Er hat eigentlich von beidn etwas.}
\parpic[r]{\epsfig{file=pics/schamane.eps, scale=0.6}}
\textit{Typische Vertreter der Schamahn kommen aus den sogenannten
Naturvoelkern. Dabei handelt es sich um eine nette Umschreibung, die sich
die Gelehrten ausgedacht haben, um die Wildlebenden wie die Barbaren
und Orks zu umschreyben.}
\par\textit{Wenn die Meisten an einen Schamahn denken, dann sehen
sie einen in Fell gehuellten und mit Knochen behaengten
Eingeborenen, der in unverstentlichen Phrasen versucht die Geister
seiner Ahnen zu beschwoeren. Wahr oder unwahr sey hier mal
dahingestellt. Fakt ist jedoch, das der Schamahn ein Wesen mit
magischen Kreften ist, die in grossen Teilen sogar ueber das
Verstentnis der Schulmagie hinauswachsen.}
\par\textit{In seinem Stamm geniesst der Schamahn einen hohen Stand,
meist sogar hoeher als der Heupdling. Er uebt in Friedenszeyten die
Funktion des Heylers und des Beraters aus. In Zeyten des Kampfes
steht er nebn den Kriegern seines Stammes in der ersten Reye und
sterkt sie mit seinen Zaubern, wehrent er Flueche auf seine Feinde
wirft.\grqq} [\mbox{Lofen} \mbox{Dulsbart}, Knappe im Orden der
Sterne in einem Aufsatz ueber Schamanen]
\end{quotation}
\par Jeder kann f"ur sich selbst entscheiden, was er dem oben genannten Zitat als bare M"unze entgegen nimmt.
\subsubsection{Wesen}
\subsubsection{Ausbildungsweg}
\par Jedes Volk, das an die Geister der Natur glaubt, hat in seinem Stamm einen Schamanen. 
Er hat die Funktion eines Beraters und Weisen inne.
\par Im Laufe seines Lebens w"ahlt der Schamane einen Sch"uler aus den Reihen des Stammes, 
meistens einen Jungen. Dieser wird von ihm in die Geheimnisse der Geister der 
Natur\index{G!Geister der Natur}\index{N!Naturgeister} eingeweiht. Er lernt die Geschichte 
des Stammes und die Kunst des Heilens. Um seinen Geist und K"orper ganz der Natur hinzugeben, 
entsagt der Sch"uler allen k"orperlichen Verlangen und wird sich niemals eine Frau oder einen 
Mann f"ur eine Partnerschaft erw"ahlen. \par Nach 10 Jahren der Aubildung wird es f"ur den 
heranwachsenden Schamanen Zeit, in die Wildnis zu gehen und sich den Geistern der Natur 
vorzustellen. Er sucht einen stillen und abgelegenen Platz in der Steppe auf und begibt 
sich direkt nach Sonnenuntergang in einen meditativen Zustand. So nimmt er Kontakt mit den 
Geistern auf. Diese nehmen Notiz von dem Anw"arter und pr"ufen ihn und seinen Geist. Wird 
er f"ur gut befunden, dann gelangt auf eine spirituell h"ohere Stufe und ist nun bereit den 
Weg des Schamanen zu gehen, wird er abgelehnt, zerst"oren sie seinen Geist und er wird wirr 
oder stirbt sofort. In diesem Fall ist er f"ur den Stamm verloren.
\par Kehrt er als Schamane zu seinem Stamm zur"uck, wird er nun in die tiefen Geheimnisse des 
Schamanismus eingeweiht.
\par In ganz seltenen F"allen wird ein abgelehnter Schamane von den D"amonen\index{D!Daemon} 
der Wildniss aufgenommen. Diese geben ihm seinen Geist zur"uck und vergiften ihn mit ihrer Lehre. 
Solch einen Schamamen kennt man unter dem Begriff \textit{Tza'Sin}\index{T!Tza'Sin} 
(vom Teufel bekehrt). Er kehrt ebenfalls nicht zu seinem Stamm zur"uck.
\subsubsection{Vorz"uge / Nachteile}
\subsubsection{Besonderes}

\newpage
\subsection{Der Seher\index{S!Seher} / Wahrsager\index{W!Wahrsager}}
\subsubsection{Beschreibung}
\par Gestraft oder beg"unstigt? "Uber die F"ahigkeiten des Sehers l"asst sich streiten. W"ahrend der Gro"steil 
der Bev"olkerung die Kraft des Sehers eher als eine Gabe sieht, ist so mancher Seher an eben dieser zu Grunde 
oder in den Wahnsinn gegangen.
\parpic[r]{\epsfig{file=pics/seher.eps, scale=0.6}}
Die jenigen, die mit der Gabe des zweiten Gesichtes beschenkt wurden, leiden meist unter unkontrolliert
 auftretenden Visionen "uber Dinge, die gerade an einem anderen Ort geschehen, geschehen werden oder aber 
 auch geschehen sind. Die Gilde der Seher hat es sich zur Aufgabe gemacht, diese Gabe zu erforschen und es 
 dem Sch"uler zu erm"oglichen, sie zu kontrollieren.
\par Damit sind Seher gern gesehene G"aste in den H"ausern der Reichen, so lange sie nur Gutes zu erz"ahlen 
haben, sie sind jedoch ebenso gef"urchtet, decken sie doch Geheimnisse aus der Dunkelheit auf. Das wiederum
 macht sie zum Lieblingsgast des "ortlichen Sherrifs. Schon so manche als unaufzukl"arend abgelegte Tat, 
 wurde doch noch entschl"usselt und der T"ater seiner gerechten Strafe zugef"uhrt. Diese Tatsache f"uhrt 
jedoch dazu, dass der eine oder andere Seher auch schon mal unschuldig Opfer eines Attentates wurde, 
vorbeugend sozusagen.
\subsubsection{Wesen}
\subsubsection{Ausbildungsweg}
\par Wie bereits oben beschrieben wurde, ist die Gilde der Seher auf der Suche nach neuen Sch"ulern, 
um ihre Kr"afte in kontrollierte Bahnen zu lenken. Dabei ist es besonders wichtig, dass die Sch"uler 
schon in fr"uhen Jahren entdeckt werden, damit ihr Geist nicht schon zu sehr unter den Visionen gelitten hat.
\subsubsection{Vorz"uge / Nachteile}
\subsubsection{Besonderes}
\par Der Seher geh�rt zu den wenigen Begabten, die den magiebereich Dimension\index{D!Dimension} erlernen k�nnen.


\newpage
\section{Geweihte und Geistliche}
\subsection{Der Heiler\index{H!Heiler}}
\subsubsection{Beschreibung}
\par Heiler sind vielerorts gerne gesehen. Sie heilen Wunden, kurieren Krankheiten oder versorgen Br"uche und was sonst noch so alles an Blessuren auftreten kann. Anders als die Bediensteten der Tempel und G"otter nutzen die Heiler keine "ubernat"urlichen F"ahigkeiten. Sie greifen haupts"achlich auf ein gro"ses Fachwissen, welches Jahrhunderte innerhalb der Gilde weitergegeben wurde, zur"uck. Das hei"st allerdings nicht, dass ein Heiler es verschm"aht, einem Verwundeten einen alchimistischen Heiltrank einzufl"o"sen oder in besonders schwerwiegenden F"allen die Hilfe von Klerikern in Anspruch zu nehmen. Heiler sind frei von Vorurteilen und nehmen anders die Kleriker gerne die Hilfe anderer in Anspruch.
\subsubsection{Wesen}
\subsubsection{Ausbildungsweg}
\subsubsection{Vorz"uge / Nachteile}
\subsubsection{Besonderes}

\newpage
\subsection{Der Priester\index{P!Priester} / Kleriker\index{K!Kleriker}}
\subsubsection{Beschreibung}
\parpic[r]{\epsfig{file=pics/kleriker.eps, scale=0.55}}
Der Kleriker ist der Diener der G"otter. Schon seit fr"uher Kindheit pflegt er das Studium der heiligen Schriften
 und die Beziehung zwischen Kirche, Staat und der gemeinen Bev"olkerung. Durch ihre Bindung zu den G"ottern und ihre 
 F"ahigkeiten Wunder zu wirken, genie"sen sie ein hohes Ansehen beim Volk.
\subsubsection{Wesen}
\subsubsection{Ausbildungsweg}
\par Die Ausbildung zu einem Kleriker dauert lang und beginnt bereits in den fr�hen Kindesjahren.
\subsubsection{Vorz"uge / Nachteile}
\subsubsection{Besonderes}
\par Wie die Paladine auch, k�nnen Priester ihre G�tter um Wunder bitten. Neben den einfachen Wundern der,
werden den Priestern jedoch auch gro�e und manchmal auch au�ergew�hnliche Wunder gew�hrt. Man hat schon Kleriker
gesehen, die wandelnden Schrittes auf der Oberfl�che eines Flusses die Uferseiten gewechselt haben oder auch 
Gl�ubige, die lebend aus einem zusammengebrochenen Stollen zur�ckgekehrt sind.

\newpage
\subsection{Der M�nch\index{M!M�nch} / Scholar\index{S!Scholar}}
\subsubsection{Beschreibung}
\par M"onche leben zur"uckgezogen in ihren Klosterfestungen und gehen stiller Meditation nach. Nicht selten haben sie auch ein Schweigegel"ubte abgelegt. Sie studieren dort die Schriften vergangener Zeiten, halten Neues fest oder haben sich der Forschung auf wissentschaftlichen Gebieten verschrieben.
\parpic[r]{\epsfig{file=pics/moench.eps, scale=0.6}}
Wenn man einen M"onch auf Reisen trifft, dann hat seine Wanderung immer einen Auftrag des Klosters als Hintergrund. Auffinden von Wissen, Erforschung, Bergung von Artefakten oder "ahnliches. W"unscht es der Suchenden, so wird er f"ur die Dauer seines Auftrages vom Schweigegel"ubte befreit, die Kommunikation f"allt so einfacher.
\par Die verschiedenen Kl"oster haben unterschiedliche Glaubensausrichtungen und eigentlich gibt es zu jedem Glauben auch mindestens eine Gruppe von M"onchen, die sich ihm verschrieben hat.
\subsubsection{Wesen}
\par Im Wesen eher ruhig und im Geiste stark. Durch sein abgeschiedenes Studium und die Meditation ist der M"onch ein eher ruhiger Zeitgenosse. Auch wenn er f"ur die Dauer seiner Reise vom Schweigegel"ubte befreit ist, so wird er dennoch keine unn"otigen Energien in "uberfl"ussige Konversationen verschwenden, sondern statt dessen ruhig in sich gehen. Das soll nicht hei"sen, dass er sich gar nicht unterh"alt. Gerade wenn es wissenschaftliche Diskussionen geht, wird er sich angeregt beteiligen und sein Wissen mit den anderen teilen. Geht es darum, ein R"atsel zu l"osen, wird er ebenfalls versuchen mit seinen Kenntnissen zu helfen.
\par M"onche sind in sich zur"uckgezogen, Angeberei und Prahlerei sind ihnen fern. Sie lassen sich nicht von Herausforderungen locken und werden versuchen Konfliktsituationen friedlich zu l"osen. Kommt es dennoch zu einem Kampf, so werden sie keine unn"otigen Energien in weitere Warnungen verschwenden und den Kampf versuchen so schnell wie m"oglich und so effektiv wie m"oglich zu beenden, dann jedoch auch ohne R"ucksicht auf die Gesundheit des Gegeners. Vom T"oten versuchen sie jedoch Abstand zu halten.
\subsubsection{Ausbildungsweg}
\par W"ahrend seiner Ausbildung lernt der M"onch nicht nur eine Reihe an Wissensfertigkeiten, sondern auch den Umgang im waffenlosen Kampf. Die Handhabung einer Waffe verweigern sie grunds"atzlich, egal ob f"ur den Nah- oder Fernkampf geeignet.
\subsubsection{Vorz"uge / Nachteile}
\par Neben den "ublichen waffenlosen Kampfarten (Boxen und Ringen) beherrschen die M"onche noch eine Reihe weiterer waffenloser Kampftechniken (+1). Eine genaue Beschreibung dazu ist im Kapitel \textit{Neue Fertigkeiten} (siehe S. \pageref{neuefertigkeiten}) zu finden.
\par Auf Grund ihrer Einstellung zu Waffen, ist es den M"onchen nicht m"oglich Waffen irgendeiner Kategorie zu f"uhren oder auch nur eine Fertigkeit aus diesem Bereich zu steigern. Ausnahme dazu sind die waffenlosen Kampftechniken (siehe oben).
\subsubsection{Besonderes}
\par M"onche sind Meister der K"operbeherrschung. Diese spiegelt sich in ihrem Chi Wert wieder. Die Handhabung dieses Wertes und seine Auswirkungen werden im Kapitel \textit{Neue Fertigkeiten} (siehe S. \pageref{neuefertigkeiten}) beschrieben.
