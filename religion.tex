\chapter{Religion\index{R!Religion}}
\label{religion}
\par Die Menschen der Neuzeit glauben an die neuen G"otter.
Ihre Priesterschaft verbreitet den Glauben und lehrt ihr Wissen. Die
Frage die sich nun stellt: Warum die neuen G�tter? Was ist an diesen
G�tter neu und die noch entscheidenere Frage: Was ist mit den Alten?

\section{Die Alten G�ter}
\parpic[l]{\epsfig{file=pics/capitals/d.eps, scale=0.5}}ie folgenden
Aufzeichnungen stellen eine Sammlung von Geschichten aus vielen
verschiedenen Schriften dar. Es ist wie bei allen diesen
Aufzeichnungen davon auszugehen, dass nur ein Bruchteil von Wahrheit
in ihnen steckt. Nichts desto trotz stellen sie die einzigen
Anhaltspunkte dar, die wir zu der Zeit vor den Drachen haben.

\begin{quotation}
\par Es gab die G�tter, die von jeher �ber die Geschicke der
Menschen wachten. Es waren gutm�tige G�tter, die nichts vor sich in
den Schatten stellten und alles als gleichberechtigt betrachteten.
Vor knapp 3000 Jahren st�rzten die Drachen �ber die Menschheit
einher. Im Laufe der folgenden Jahre verloren die Menschen den
Glauben in die G�tter. Wer sollten denn dieser G�tter sein, die mit
Ansahen, ihre Kinder in ein Leben von Folter und Sklaverei gepresst
wurden?
\par Einer nach dem anderen gab den Glauben auf und mit den Jahren
ging das Wissen von den alten G�ttern unter. Die Priester der Zeit
gaben den Glauben an die G�tter entweder selber auf oder starben im
Glauben an ihre G�tter bei der Erhaltung ihres Wissen, wie etwa
durch heroische Taten, um ihre Existenz zu best�tigen. Jedoch gab es
kein Zeichen der Himmlischen. Letztentlich blieb auch kein Platz
mehr in den K�pfen der Menschen f�r diesen Glauben, die Drachen
nahmen ihnen den Verstand.
\end{quotation}



\section{Die Neuen G�tter}
