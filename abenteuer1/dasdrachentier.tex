\chapter{Das Drachentier}
\par Die Gangbreite der H"ohle am Anfang entspricht ca. 3 Schritt, die H"ohe ca. 4-5 Schritt.
\par Die W"ande sind feucht und auf dem Boden tummeln sich allerhand \glqq Bewohner\grqq{} (K"afer, W"urmer, ...). Von den W"anden geht ein leichtes phosphorisierendes Leuchten aus. An der Decke ist die eine oder andere Fledermaus. Etwas weiter im Inneren der H"ohle sind die Wagenspuren wieder zu erkennen.
\par Euer Weg f"uhrt Euch zun"achst durch nat"urlich entstandene H"ohleng"ange. Schon nach ein paar Metern im Berginneren werden die W"ande staubtrocken und die Luft stickig. Das Leuchten von den W"anden bleibt.

\begin{description}
\item [Die H"ohle] Ein genauer Plan der H"ohle ist dem Anhang zu entnehmen. Das Leuchten an den W"anden wird von einem Schimmelpilz produziert, dessen Ber"uhrung leicht schmerzhafte Verbrennungen auf der Haut verursacht.

\item [Raum 1] Eine H"ohle von ca. 8 Schritt Breite und 4 Schritt L"ange. Die Deckenh"ohe misst hier gut 6 Schritt.\\
Obwohl der Schimmel den Raum gut ausleuchtet bleiben viele Ecken in den Schatten und hier und dort sehr ihr etwas an den W"anden entlang huschen. Aus der Ferne vernehmt ihr ein Knacken, als ob Knochen brechen. Der Hall h"alt sich f"ur ein paar Sekunden im Raum. Irgendetwas schwirrt recht schnell an Euch vorbei. Vielleicht ja nur ein Luftzug? Auf dem Boden findet ihr vertrocknete Tierexkremente.\\
Die H"ohle kann man Richtung Norden oder S"ud-Osten verlasse. Die Wagenspuren folgen dem zweiten Weg.

\item [Raum 2] Eine kleine H"ohle von ca. 3x3 Schritt, das Leuchten von den W"anden ist hier fast g"anzlich verschwunden.\\ Aus der hintersten Ecke starrt Euch ein gelbes Augenpaar an. Dort befindet sich ein recht abgemagerter Wolf, der auf den knochigen "Uberresten eines Tieres herumkaut. Es stinkt hier schon ein wenig nach Verwesung.\\
Der Wolf knurrt euch an, zeigt aber ansonsten keinerlei Agressionen.

\item [Raum 3] Recht gro"se H"ohle (10x7), kein Leuchten von den W"anden und die Sichtweite betr"agt ca. 1 Schritt.\\
Ein stechender Geruch liegt euch in der Nase und es riecht verbrannt, der Boden ist sandig weich. Ausg"ange nach Norden und S"uden. Bei Licht ist die H"ohe der H"ohle zu erkennen (4 Schritt), der Boden ist bef"ullt mit feinem Sand (ca. 1/10 Schritt hoch). Im Norden ist die Stahlkutsche in einer Extra-H"ohle zu erkennen.

\item [Raum 4] Hier steht die eiserne Kutsche. Dieser Teil ist k"unstlich angelegt, Scharniere an der vorderen Seiten deuten darauf hin, dass hier mal ein Gitter oder ein Tor angebracht war (oder irgendwann mal sein wird?). Die Kutsche ist aus stabilem Eisen gefertigt, welches fl"achendeckend verru"st ist. Die Kutsch mi"st 1,5 x 2 Schritt und ist ca. 2 Schritt hoch, an den beiden Seiten befinden sich jeweils 2 st"ahlerne R"ader. Die Au"senh"ulle ist unverziert und Eing"ange sind keine zu erkennen. An der vermutlichen Vorderseite befinden sich 12 kleine Sch"achte, die wohl in das Innere f"uhren, aber versiegelt sind.

\item [Raum 5] Die nat"urliche H"ohle endet hier. Der folgende Raum ist "ahnlich wie die Garage gemauert. Das Werk hier ist jedoch deutlich neuer und besser gefertigt. Der Raum ist ca 4x4 Schritt gro"s und beinhaltet einige Gem"alde an der Ostwand. Diese sind jedoch so stark verbrannt, dass fast nichts mehr zu erkennen ist. Ein gemauerter Gang f"uhrt nach Westen. Die W"ande sind hier mit verbrannten T"uchern abgeh"angt, leere F"ackelhalter befinden sich dort ebenfalls. Der Gang endet abrupt.

\item [Raum 6] Ihr kommt in einen erleuchteten Raum, in dessen Boden ein seltsames arkanes Muster eingelassen ist. An einer Seite findet ihr einen Kasten mit Kerzen, tw. heruntergebrannt. In der Mitte des Raumes ist angetrocknetes Blut zu finden.\\
Bei genauer Untersuchung des Fu"sbodens/Symbols k"onnt ihr Wachsreste an den Schnitt- und Ber"uhrungspunkten finden. Im Raum h"angt ein leicht schwefliger Geruch.\\
Erst bei genauerer "Uberlegung f"allt euch auf, dass dieser Raum keine nat"urliche Lichtquelle besitzt. Das geheimnissvolle, warme Leuchten, dass den Raum erf"ullt, geht vom Boden aus.
\end{description}

\begin{longtable}{|p{15cm}|}
\hline
\textbf{Meisterinformationen:} Hier sind einige Informationen zu den R"aumen der H"ohle zu finden, die den Charakterem jedoch nur nach entsprechnden Untersuchungen zur Verf"ugung gestellt werden sollten.
\begin{description}
\item [Die Gem"alde] Die Gem"alde an den W"anden sind vebrannt und fast g"anzlich vernichtet. Sie stellen jedoch ein Kopie der Karten dar, die im Beschw"orungsraum zu finden sind. Einzig zu erkennen ist auf allen Bildern noch die untere Zeile (Numerierung).
\end{description}
\begin{description}
\item [Die Geheimt"ur] Die Geheimt"ur ist kein wirkliches Hindernis und auch leicht zu erkennen und zu "offnen. Sie soll nur ein Hindernis f"ur die wilden Tiere der H"ohle darstellen.
\end{description}
\begin{description}
\item [Der Beschw"orungsraum] Unter der Kiste mit den Kerzen findet ihr eine Fallt"ur im Boden. Dadrunter befindet sich ein Geheimfach, in dem ihr eine Schatulle findet. In der Schatulle befinden sich
	\begin{itemize}
	\item 6 Karten, mit verschiedenen Symbolen. Die Karten sind zur Beschw"orung und Austreibung des D"amonen n"otig
	 (siehe Anhang).
	\item Eine Strohpuppemit einer feinen Eisenkette umwickelt (bei genauerer Betrachtung ist der gesch"adigte Wirt
	zu erkennen).
	\item zwei handgro"se Portraits von 2 sch"onen Frauen (T"ochter des Wirtes vom Schwertfisch).
	\item 25 Silberm"unzen
	\item einen Zettel mit verschiedenen Namen (siehe Anhang).
	\end{itemize}
\end{description}
\par Sp"atestens jetzt d"urften den Helden die Zusammenh"ange klar werden. Die gefundenen Utensilien und das Gespr"ach mit dem Gesch"adigten im Dorf, sowie die H"ohle und die Kutsche sollten den Schuldigen eindeutig identifizieren. Es liegt jetzt an den Helden, dieses alles dem Dorfvorsteher vorzulegen und die Tatsachen zu schildern. Wenn die Helden besonders pfiffig sind, vernichten sie den D"amon noch mit Hilfe der Karten. Das Ganze hat nur leider einen kleinen Haken, um ihn auszutreiebn und endg"ultig zu vernichten.
\par Kehren die Helden in das Dorf zur"uck, ohne den D"amon zu vernichten, wird er sp"atestens in der n"achsten Vollmondnacht das Dorf erneut heimsuchen.\\
\hline
\end{longtable}
\par Wie auch immer das Ganze hier enden wird, leben die Helden noch, werden sie sich auf den R"uckweg machen.