\chapter{Die Beschw�rungskarten}

\section{Die Bedeutung der Karten}
\par Die Beschw"orungskarten dienen dem Wirt dazu, dem D"amon, der die Kutsche antreibt, Leben einzuhauchen und ihn aber auch wieder zu vertreiben. Die Reihenfolge, in der die Karten benutzt werden, ist auf ihnen vermerkt, sie sind durchnummeriert. Die Nummer findet sich im unteren Viertel der Karte und wir durch eine entsprechende Zahl an Strichen dargestellt.
\par Um die Karten zu benutzen, muss der Beschw"orer einfach nur den Anweisungen auf den Karten folgen und die dabei zur Mitte des Zirkels pr"asentieren. Der Zirkel aktiviert sich automatisch und "ubernimmt die Kontrolle "uber den Beschw"orer und bedient sich seiner astralen Kraft. Die Beschw"orung kann vom Wirker nur nach einem vollendeten Schritt freiwillig, aber nur widerwillig beendet werden. Verliert er w"ahrend eines Beschw"orungsschrittes aus irgendeinem Grund das Bewusstsein, so ist die Beschw"orung des D"amon unterbrochen, kann jedoch zu einem sp"ateren Zeitpunkt ab diesem Schritt wieder fortgesetzt werde. Die Unterbrechung hat jedoch eine andere unangenehme Nebenwirkung: Es wurde ein Portal zu einer unbekannten Dimension ge"offnet und nicht wieder verschlossen. Es sei der Phantasie jedes Spielleiters "uberlasse, welche Auswirkung das haben kann und wann sie sich bemerkbar machen. Dieses Portal wird jedoch bis zur Beendung des Rituals in Verbindung mit dem Beschw"orer stehen.
\par Die Beschw"orung erfordert die Karten 1 bis 4, w"ahrend seine R"uckrufung die Karte 5 erfordert, die Karte 6 vernichtet in komplett. Nach Durchf"uhrung seines Auftrages kehrt der D"amon von alleine zur"uck.

\subsection{Die Beschw"orung (Karten 1 - 4)}
\begin{description}
\item [Karte 1] Der Beschw"orer stellt sich am Beschw"orungszirkel auf und entz"undet die Kerzen. Anschlie"send wird der Zirkel ihm preisgeben, welche Beschw"orungsformel er zu sprechen hat.
\item [Karte 2] Eine Puppe, die das Opfer darstellt, muss in der Mitte des Zirkels plaziert werden.
\item [Karte 3] Die Unterwerfungsformel muss gesprochen werden (DAEMON IMPERATORI DAEMON DESTRUCTOR - D"amon ich befehle dir, D"amon du bist der Zerst"orer).
\item [Karte 4] Der D"amon l"ost sich aus dem Zirkel, f"ahrt in die Kutsche ein und sucht das Opfer heim.
\end{description}

\subsection{Die Austreibung (Karten 5 und 6)}
\begin{description}
\item [Karte 5] Die magische Formel muss gesprochen werden (DAEMON DESTRUCTORI - D"amon ich vernichte dich), der D"amon wird unverz"uglich im Beschw"orungszirkel verschwinden. M"ochte man den D"amon sp"ater wieder beschw"oren, endet das Ritual hier.
\item [Karte 6] Der Zirkel verliert sein Leuchtkraft und entschwindet schlie"slich g"anzlich.
\end{description}

\newpage
\section{Die Karten}
\begin{figure}[hbtp]
\begin{center}
\epsfig{file=kartensatz.eps, scale=0.6}
\end{center}
\caption{Kartensatz - Beschw"orungskarten}
\label{bkarten}
\end{figure}
