\chapter{Das kleine M�dchen}
\par Bei dem "Uberfall an der Kutsche werden die Charaktere die Bekanntschaft dieses M"adchens machen, welches sie fortan, freiwillig oder unfreiwillig begleiten wird.

\section{Das kleine M"adchen}
\par Das kleine M"adchen ist ungef"ahr 10 Jahre alt, hat lange braune Haare und tr"agt ein einfaches rot-gemustertes Kleid und keine Schuhe.
\parpic[r]{\epsfig{file=amulett.eps, scale=0.5}}
Um den Hals tr"agt es an einem Lederband ein Amulett aus Stein, in das eine Rune eingearbeitet ist. Ein Charakter mit entsprechend religi"oser Kenntnis oder Legendenkunde k"onnte versuchen, diese Rune zu identifizieren. Gelingt dieser komplizierte Vorgang, so meint er sich erinnern zu k"onnen, diese Rune in alten Schriften schon mal mit der Bezeichnung \glqq Auserw"ahlter\grqq{} oder \glqq Gesegneter\grqq{} gesehen zu haben. Jedenfalls dieser grobe Zusammenhang. Mehr ist jedoch nicht herauszuholen. Sollten die Charaktere versuchen ihr die Rune abzunehmen, so wird sie sich wieder schreiend zur Wehr setzen, jedoch gleich wieder zur Ruhe kommen, wenn sie von ihr ablassen.
\par Das M"adchen hat sonst nichts weiter bei sich. Eine genauere Untersuchung ihrer Verletzungen f"uhrt zu der Erkenntnis, dass das M"adchen dem "Uberfall wohl gl"ucklicherweise entkommen ist. Sie ist nicht weiter verwundet und das Blut muss wohl den anderen Passagieren kommen.

\section{Das dunkle Geheimnis}
