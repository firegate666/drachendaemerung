\chapter{Eine Reise in die Berge}
\par Die Dorfbewohner werden den Charakteren weniger behilflich bei etwaigen Vorbereitungen sein. Keiner von ihnen hat Vertrauen in ihre Bem"uhungen. Sollten sie jedoch noch weitere Besorgungen f"ur ihre Reise t"atigen wollen, siegt der Grundsatz: \textit{Der Kunde ist K"onig}.
\par Der Weg, der nun vor ihnen liegt, f"uhrt aus dem Dorf heraus gen Westen. Nach etwa einer halben Tagesreise durch die angrenzenden Felder und Wiesen, wird die Umgebung etwas bewaldeter. Gegen Abend werden die Helden die Ausl"aufer der Gebirgsz"uge des \textit{Dunkelfelsens} erreicht haben. Dort wird zwangsl"aufig eine Rast anstehen, denn die sie werden von den Strapazen des hinter ihnen liegenden Tages ersch"opft und den vor ihnen liegenden niedergeschlagen sein. Der Berg begr"u"st sie zwar mit gut begehbaren Serpentinen, jedoch mit sehr steilen.
\begin{description}
\item [Wiesen und Felder] Die direkt ans Dorf grenzende Umgebung beherbergt die Wiesen und Felder der dort wohnenden Bauern. Dank der Jahreszeit sind diese mit bunten Blumen und Gr"asern "ubers"aht. W"ahrend ihres Marsches k"onnte es sein, dass sie einem Bauern begegnen, dieser inspiziert seiner Felder. Er wird die Helden freundlich begr"u"sen, ihnen sonst jedoch weiter nicht behilflich sein. Auch er hat Misstrauen ihnen gegen"uber und wird sich mit der Ausrede, dass er noch viel zu tun habe, weiter auf den Weg machen. Sollten die Helden auf die Idee kommen, hier jagen zu gehen, so ist dies durchaus m"oglich. Mit etwas G"luck k"onnten sie einen Hasen oder ein Rotwild ausfindig machen. Sie sollten jedoch dar"uber nachdenken, dass diese Felder auf denen sie jagen wollen, den Bauern geh"oren. Vielleicht w"are es sinnvoller derartige Aktionen in den Wald zu verlagern.
\item[Der Wald] Von Beginn des Waldes an wird es nun zunehmend dunkler. Zum einem, weil die Charaktere hier kurz vor Mittag ankommen werden und zum anderen wird das Bl"atterdach zum Gebirge hin immer dichter werden. Der Weg bleibt sch"on breit und begehbar, der Boden ist recht fest, so dass hier nur mit M"uhe Spuren zu finden sein werden. Zwischen dem eigentlichen Weg und dem Beginn des dichten Waldes, finden sich angenehme Pl"atze, die sich f"ur eine Rast anbieten.
\item[Am Fu"se des Berges] Gegen Ende des Weges wird der Wald wieder lichter, es ist jedoch schon ziemlich dunkel geworden, der Abend ist bereits angebrochen. Am Wegesrand werden die Helden nun ein Kutsche entdecken, die zur Seite ins Geb"usch gekippt ist, "uber den Weg liegen Leichen verstreut, die Pferde liegen noch im Geschirr, eine paar Aas fressende V"ogel haben sich bereits herabgelassen.
\par Bei der Untersuchung der Leichen kann festgestellt werden, dass diese durch starke Schnittwunden ums Leben gekommen sind. Diese sind "uberall auf dem K"orper und im Gesicht zu erkennen. Die Kutsche ist nicht weiter besch"adigt, die T"ur zu Wegseite ist aufgerissen worden und der Innenraum der Kutsche ist blutverschmiert.
\par Wie auch immer die Begegnung der Helden hier ausgeht, es wird dunkel und es ist an der Zeit, einen Schlafplatz zu suchen. Vielleicht jedoch nicht in der N"ahe dieses Schlachtplatzes.
\end{description}

\begin{longtable}{|p{15cm}|}
\hline
\textbf{Meisterinformationen:}
Der folgende Abschnitt behandelt ein paar der Begegnungen, die die Helden auf ihrer Reise in die Berge machen werden.\\
\begin{description}
\item [Ork"uberfall] Sollten die Helden rasten, so werden sie w"ahrend ihrer Mittagspause vom einem Pack Orks "uberrascht. Diese tauchen pl"otzlich zu beiden Seiten des Weges aus dem Unterholz auf und fragen ganz h"oflich nach einer Spende in Form von Goldm"unzen. Sollte einer der Charaktere "uber ein sehr gutes Geh"or verf"ugen, k"onnte er bereits fr"uher "uber die Ankunft der ungebetenen G"aste informiert sein. Die Orks werden nicht lange auf ein Antwort warten und schnell zum Angriff "ubergehen. Da ihr Auftritt recht "uberraschend ist, haben die Orks auf jeden Fall die ersten Kampfhandlungen und die Charaktere m"ussen die erst Runde damit verbringen, ihre Waffe zu ziehen. Falls sie schon "uber das kommen der Orks informiert wurden, wird der Kampf normal abgehandelt.
\par Sollten die Orks durch den Kampf mehr als die H"alfte ihrer Kameraden verlieren, werde sie wieder in den Wald fl"uchten, ein schneller Held k"onnte sie jedoch verfolgen. "Uberlebende Orks k"onnen verh"ort werden. Nach einem glaubw"urdigen Versprechen, dass ihr Leben verschont wird, erz"ahlen sie von einer eisernen Kutsche, die sie in der vergangenen Nacht haben vorbeifahren sehen. Mehr n"utzliche Informationen haben sie jedoch nicht. Sollten die Helden die Orks ausbeuten, so finden sie ein paar Kupferm"unzen und noch weniger Silberm"unzen.
\end{description}
\begin{description}
\item [Die Kutsche und der "Uberfall] Zu den Opfern z"ahlen 2 m"annliche K"orper in Kutscherkleidung, ein Mann und eine Frau in feineren Kleidern und ein Diener. Bei der genaueren Untersuchung der Leichen k"onnen fachkundige Helden festellen, dass die Opfer von scharfen Klauen zerrissen worden sind. Bissspuren sind keine zu erkennen. Die K"orper sehen von der Lage so aus, als w"aren sie auf den Boden geschleudert worden, einige weisen Knochenbr"uche auf. Im inneren der Kutsch ist einiges an Reisegep"ack zu finden. 3 Koffer mit Kleidung f"ur die toten Reisenden. Sollten die Leichen untersucht werden, so findet man in der Tasche des m"annlichen Reisenden 5 Silberm"unzen.
\par W"ahrend der Untersuchung wird ein Schluchzen aus dem nahen Geb"usch hinter der Kutsche zu h"oren sein. Dort finden die Charaktere ein junges M"adchen mit langen braunen Haaren und einem einfachen Kleidchen. N"ahere Informationen zu diesem M"adchen sind im Anhang zu finden. Holen die Helden es behutsam aus dem Busch, sehen sie, dass es ebenfalls blutverschmiert ist wie alle Leichen hier. Es befindet sich in einem Schockzustand und ist nicht in der Lage zu sprechen. Sollten die Helden entscheiden, es mitzunehmen, wird es willig folgen. Lassen sie es stehen, dann wird es von alleine folgen. Kommen die Helden auf die Idee, es zurück ins Dorf zu bringen, wird es sich schreiend und weinend dagegen wehren.
\end{description}\\
\hline
\end{longtable}
\par Der eigentliche Aufstieg gestaltet sich v"ollig unproblematisch. Nach ca. 2 Stunden Wanderung wird der Boden wieder weicher und erdiger. Ab hier sind die Wagenspuren wieder zu sehen. Nach einer weiteren Stunde stehen die Helden vor einem H"ohleneingang, vor dem die Spuren enden.