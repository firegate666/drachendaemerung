\chapter{Ankunft im Dorf}
\par Eine lauer Sommertag und eine lange Reise liegen hinter Euch seit \textit{Pax Lucien}. Dort hattet ihr Euch einem Handelszug angeschlossen, um es auf dem Weg nach Norden etwa sicherer zu haben. Wobei Norden dabei eher eine zuf"allige Richtung ist, denn so richtig habt ihr noch kein Ziel vor Augen, aber m"ussen Abenteurer denn ein "ortliches Ziel haben? Eure F"u"se sind schwer vom Gehen und Eure R"ucken schmerzen vom langen Ritt. Seit heute morgen um 7 seit ihr unterwegs und mittlerweile ist es bereits 6 Uhr am Abend und dieser Sklaventreiber von Karawanenf"uhrer hat nur eine Pause zum Mittag gemacht und das doch glatt f"ur eine halbe Stunde. Eure m"uden Augen fallen auf das Tavernenschild am Marktplatz. \textit{Schwertfisch} steht dort in verschn"orkelten Buchstaben aus Muscheln und Fischgr"aten geschrieben. Darum herum sind Verzierungen aus Seefr"uchten und anderen Meerestieren. Das Geb"aude besteht aus 2 Stockwerken. Alles in allem macht die Taverne einen einladenden Eindruck, die G"aste die dort hinein gehen, machen auch einen akzeptablen Eindruck. Viel "uberzeugender ist jedoch der Hunger und der Durst, der Euch plagt. Also hinein dort.

\par Die Luft in der Taverne ist geschw"angert von Rauch, Bier- und Fischgeruch. Eure Blicke schweifen kurz durch den Raum.

\begin{description}
\item [Die Taverne] Das Innere der Taverne ist "ahnlich wie das Eingangsschild reich geschm"uckt mit allem, was das Meer so zu bieten hat. Unter der Decke sind Fischernetze aufgespannt und an der Wand "uber dem Tresen h"angen zwei Schwertfische in Duellmanier mit gekreuzten Schwertern; wahrscheinlich die Namensgeber.
\par Der Raum selbst ist schon recht gef"ullt, was jedoch kein Wunder um diese Uhrzeit ist. Zwei weibliche Bedienungen sind flei"sig zwischen Tresen und den Tischen unterwegs, um die G"aste mit Speisen und Getr"anken zu versorgen. Hinter dem Tresen steht ein "alterer Mann, der die G"laser f"ullt. Vor dem Tresen sind einige Barhocker aufgestellt. Im Raum verteilt stehen verschieden gro"se Tische, an denen die G"aste sitzen.
\par Neben der Theke f"uhrt eine Schwingt"ur in den hinteren Bereich. Dort befindet sich sehr wahrscheinlich der Koch, der die Speisen zubereitet. Direkt neben der Eingangst"ur f"uhrt eine Treppe nach oben.
\par Auf dem oberen Flur befinden sich mehrere T"uren, sehr wahrscheinlich die G"astezimmer.
\item [Der Wirt] Der Wirt ist ein "alterer menschlicher Mann. Er hat schulterlanges, zerzaustes ergrautes Haar. Hier und dort sind lichte Stellen auf seinem Haupt zu sehen. Das Gesicht ist glattrasiert. Sein Name ist \textit{Dorn Blaubauch}. Der Name stammt von einer sehr unangenehmen Begegnung mit einem Schwertfisch, bei der er sich eine "ubliche Stichverletzung in der Magengegend zugezogen hat. Das auslaufende Blut, welches sich in seinem K"orper sammelte, lies seinen Bauch wie einen riesigen blauen Fleck erscheinen. Mittlerweile ist die F"arbung abgeklungen, die 15 cm lange Narbe ist ihm jedoch geblieben. Fr"uher ist er wie alle Fischer zur See gefahren. Die Schwertfische hinter der Theke hat er selbst gefangen.
\item [Die Bedienungen] Die beiden Frauen, die hier den G"asten servieren hei"sen \textit{Anika} und \textit{Magdalen}. Sie sind \textit{Dorns} T"ochter und hier im Dorf mit Fischern verheiratet. F"ur ihr Alter sind sie erstaunlich sch"on geblieben, w"usste man es nicht besser, dann w"urde man sie auf Anfang 20 sch"atzen. Sie sind stets freundlich zu den G"asten, wahren aber die n"otige Distanz.
\end{description}

\par Ihr findet im Gedr"ange noch einen freien Tisch f"ur Euch und kaum dass ihr Euch gesetzt habt, findet sich eine Bedienung an Eurem Tisch ein, um die Bestellung aufzunehmen.

\begin{longtable}{|p{15cm}|}
\hline
\textbf{Meisterinformationen:}
Die Charaktere m"ussen im Laufe des Abend Bekanntschaft mit \textit{Steen Rotbart} machen, der ihnen eine Reihe von Geschichten erz"ahlen wird (siehe Anhang).\\
Es w"are auch gut, aber nicht unbedingt notwendig, wenn die Charaktere in dieser Taverne ihre Zimmer beziehen, eine der Bedienungen wird sie im Laufe des Abends darauf ansprechen, falls sie das Thema nicht von selbst anschneiden.\\
Die Preise dieser Lokalit"at entsprechendem dem Mittel des Landes, nicht zu teuer und nicht zu billig.\\
\hline
\end{longtable}

\par Am Ende dieses anstrengenden Tages, fallt ihr m"ude in Eure Betten. Hoffentlich wird die Reise morgen etwas entspannender. Eine etwas l"angere Pause oder vielleicht zwei w"aren ja nicht schlecht. Lange denkt ihr nicht dar"uber nach, denn der Schlaf holt Euch schon ein.
