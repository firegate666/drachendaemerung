\chapter{Der �berfall}
\par Mitten in der Nacht werdet ihr pl"otzlich durch lautes Geschrei geweckt. Es scheint von drau"sen zu kommen. Viele Stimmen schreien und rufen durcheinander. Der Nachthimmel scheint erleuchtet zu sein, jedenfalls ist drau"sen ein helles Flackern zu erkennen.

\begin{description}
\item [Der Marktplatz] Aus den Fenster ist zu erkennen, das drau"sen ein gewaltiger Aufruhr herrscht. Menschen laufen schreiend durcheinander, zwei Geb"aude auf der anderen Seite des Marktplatzes stehen in Flammen. Einige M"anner tragen Wasser vom Flu"s herbei, um das Feuer unter Kontrolle zu bekommen.\\
Wenn die Helden das Gasthaus verlassen und den Marktplatz betreten, erkennen sie einen Mann in Nachtkleidung, der in der Mitte des Platzes steht und etwas von einem feuerspeienden Ungeheuer erz"ahlt. Er wird jedoch gr"o"stenteils ignoriert.
\item [Die \glqq Opfer\grqq{}] Die H"auser, die in Flammen stehen sind ein Gasthaus und ein Kr"auterladen. Sie sind innerhalb k"urzester Zeit heruntergebrannt. Sp"ater wird man feststellen, dass die wohnhafte Hexe den Flammentod gefunden hat.
\item[Die Gastst"atte] Der Wirt h"alt sich noch im oberen Geschoss des Gasthauses auf, seine Frau und seine beiden Kinder sind bereits ins Untergeschoss geflohen, der Treppenaufgang und der Ausgang werden jedoch von Flammen abgeriegelt.
\end{description}

\par Von den Menschen werdet ihr zum Helfen aufgefordert. Es muss gel"oscht und versorgt werden. Euch werden Eimer und Decken gereicht.
\par Mit der Zeit habt ihr die Flammen unter Kontrolle, ein "Ubergreifen auf die umstehenden H"auser konnte verhindert werden. Die Bewohner fangen langsam an Notiz von dem Mann in der Mitte des Platzes zu nehmen. Sie scharen sich um ihn, und h"oren, was er zu sagen hat. Es ist der Wirt des Gasthauses, welches diese Nacht in Flammen aufgegangen ist.

\begin{description}
\item [Der Wirt, ein "Uberlebender] Er scheint in totaler Panik zu sein und wiederholt sich "ofters bei seinen Erz"ahlungen. Er habe in der Dunkelheit eine Bestie gesehen, die mit Feuerodem die H"auser in Brand gesteckt hat. Er habe auch das Schreien und Kreischen des Tieres geh"ort. Einige der Umstehenden stimmen ihm zu.
\item [Die Dorfbewohner] Die Dorfbewohner sind von diesen Neuigkeiten durchweg geschockt. Alle reden durcheinander. Im Laufe der Unterhaltung werden sich einige an \textit{Steen Rotbarts} Geschichte vom Drachentier erinnern, es wird nicht lange dauern und er wird f"ur dessen Auftauchen verantwortlich gemacht. Die Bewohner schaukeln sich in ihrer Stimmung hoch und einige setzen sich vom Pulk ab und holen \textit{Steen} mit Gewalt herbei. Ihm wird die Beschw"orung dieses Wesens vorgeworfen und eine Bestrafung wird gefordert, sie soll ihn das Leben kosten.
\end{description}

\begin{longtable}{|p{15cm}|}
\hline
\textbf{Meisterinformationen:}
Hier w"are es nun an der Zeit f"ur die Helden, einzuschreiten. Als tapfere und Gerechte Abenteurer k"onnen sie sich f"ur ihn einsetzen und einen fairen Prozess fordern in dessen Verlauf sie sich als Unschuldfinder anbieten. In jedem Fall sollte eine gewaltsame Befreiung vermieden werden.\\
Kommen die Charaktere nicht von alleine auf die Idee, so erkennt \textit{Steen} sie in der Menge und bittet sie um Hilfe. Hilft das immer noch nix, dann wird er anfangen sie der Mitt"aterschaft zu beschuldigen und sp"atestens jetzt ist es auch in ihrem Interesse, dieses Mi"sverst"andnis aufzukl"aren.\\
\hline
\end{longtable}
\par Durch euer Angebot nach Hilfe, kommt wieder etwas Ruhe in den aufgebrachten Mob. Die Dorfbewohner lassen sich auf eure Forderung nach einer Verhandlung mit Beweisf"uhrung ein. Sie geben euch Charakteren die Gelegenheit euch umzuschauen und nach Beweisen f"ur die Unschuld zu suchen. In jedem Fall habt ihr ab jetzt 72 Stunden Zeit, um alles aufzukl"aren.

\begin{description}
\item [Die abgebrannten H"auser] Wie bereits oben erw"ahnt handelt es sich hierbei um das Gasthaus \textit{Dampfende K"uche} und einen Kr"auterladen. Beide Geb"aude sind komplett heruntergebrannt und der einzige "Uberlebende ist der Wirt. Dieser steht jedoch unter Schock und kann zu diesem Zeitpunkt keine anderen, als die oben genannten Aussagen treffen.
\item [Die Hexe] Die Hexe war allen Dorfbewohner bekannt. Sie war eine gutm"utige alte Frau, die ihre Kenntnisse zum Wohle aller zur Verf"ugung gestellt hatte. Sie kannte Rezepte f"ur allerlei Heilmittel und verstand sich auch gut auf den Umgang mit Magie, was in der Vergangenheit besonders wichtig war, als sie einen betr"ugerischen Illusionisten auffliegen lie"s.
\item [Der Wirt] Der Wirt wird mit seiner Familie erst einmal bei seinem Vater unterkommen, dieser ist der Dorfschmied. Wenn er aufgesucht wird, sitzt er in den hinterem R"aumen der Schmiede und sein Blick wandert ins Leere, er ist eigentlich zu keiner sinnvollen "Au"serung f"ahig. Ein einf"uhlsamer Charakter k"onnte von ihm jedoch noch weitere Informationen bekommen. Vor ca. 5 Wochen wurde seine Gastst"atte von einem Haufen durchreisender Seeleute w"ahrend eines Saufgelages in ihre Bestandteile zerlegt. Der Wiederer"offnung stand kurz bevor. Die andere Gastst"atte wurde erst vor 4 Wochen er"offnet
\item [Stens Haus] Kommen die Charaktere bei ihren Untersuchungen auch auf Stens Haus, so ist dies nicht weiter schwer zu finden. Relativ am Dorfrand k"onnen sie eine kleine H"utte ausmachen, komplett aus Holz gebaut mit einem kleinen Garten. Das Ganze wird von einem Zaun umgeben und mit einer recht sch"on verzierten Gartenpforte abgeschlossen. Diese ist jedoch aus den Angeln gerissen und ein regelrechter Trampelpfad f"uhrt durch den Garten in das innere des Hauses. Die Haust"ur wurde ebenfalls niedergerissen, jedoch notd"urftig wieder aufgestellt. Im Inneren der H"utte sind eine Menge Bilder und einfacher Kunstwerke zu finden, die wohl verschiedenen Legenden und Geschichten zuzuordnen sind, nichts von wirklichem Wert. Ansonsten ist die Ausstattung eher arm und das ganze Haus besteht aus einem Zimmer.
\end{description}

\begin{longtable}{|p{15cm}|}
\hline
\textbf{Meisterinformationen:}
Auf den Strassen um die H"auser sind mit etwas M"uhe Wagenspuren zu erkennen, es wurde versucht diese zu verwischen. Ein Charakter mit Kenntnissen im Spuren lesen oder aber ein Magier k"onnen diese entdecken und herausfinden, dass sie von au"serhalb des Dorfes gekommen sind, vor den H"ausern herumf"uhren und auch wieder aus dem Dorf in Richtung Gebirge herausf"uhren\\
Mit ein wenig Geschick und ein paar Untersuchungen kann herausgefunden werden, dass die Flammen, die die H"auser verzehrt haben von unten nach oben gebrannt haben, ungew"ohnlich also f"ur ein riesiges Tier, welches von oben herab Feuer speit. Evtl. finden die Charaktere ja auch noch "Uberreste eines Brand"olsatzes. Dessen Geruch kann hier n"amlich vernommen werden.\\
\hline
\end{longtable}

\par Mit den neu gewonnenen Informationen k"onnen ihr die Bewohner "uberzeugen, euch in die Berge ziehen zu lassen. Dort wollt ihr die Spur aufnehmen.
