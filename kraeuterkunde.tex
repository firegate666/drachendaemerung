\chapter{Kr�uterkunde\index{K!Kr�uterkunde}}
\label{kraeuterkunde}
\parpic[l]{\epsfig{file=pics/capitals/d.eps, scale=0.5}}ie Kunst
der Kr�uterkunde ist eine nicht unwichtige. Das richtige Kraut zur
richtigen Zeit erh�lt das Leben oder verk�rzt es rapide, je nach
Situation und Anwendungswunsch.
\par Viele Kr�utern werden nachweislich erhebliche Heilungskr�fte
nachgesagt, einige dienen den Sinnesfreuden und einige heimt�ckische
Mischungen und Kr�uter werden als Gift verwendet. Manches mal kann
ein Kraut sowohl das eine, als auch das andere, je nach Dosierung.
Es bedarf schon eines Experten auf em Gebiet der Kr�uterkunde, um
auch wirklich den gew�nschten Effekt zu erzielen. KR�ter werden
h�ufig auch dazu verwendet, um Zaubertr�nke oder dergleichen
herzustellen, diese Kunst f�llt jedoch in den Bereich der Alchimie
(siehe S. \pageref{alchimie}) und hier wird nicht weiter darauf
eingegangen. Dieses Kapitel widmet sich der Naturkunst als solche.

\par Im gro�en und ganzen wird in der Kr�uterkunde zwischen 3
Obergruppen unterschieden:
\begin{description}
\item [Wurzeln] alles was unterhalb der Erde w�chst und gedeiht
\item [Beeren] alles was an der Pflanze an Fr�chten w�chst und
gedeiht
\item [Bl�tter] der Rest der Pflanze
\end{description}

\par Auf den folgenden Seiten m�chte ich eine Reihe von Kr�utern und
deren Wirkung kurz darstellen. Die Textpassagen stellen
Zusammenfassungen aus dem \textit{Folianten f�r Wohl- und
Unwohlsein} von \mbox{Krautus} \mbox{Inmixus} dar. Dem geneigten
Leser w�nsche ich nun viel Spa� und viel Erfolg bei ersten eigenen
Verarbeitung der Wurzeln und Beeren.

\section{Wurzeln}

\section{Beeren}

\section{Gr�ser}
