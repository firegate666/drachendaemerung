\chapter{Neue Fertigkeiten\index{F!Fertigkeiten}}
\label{neuefertigkeiten}

\section{Waffenfertigkeiten}

\section{K�rperliche Fertigkeiten}
\subsection{Waffenlose Kampftechniken\index{W!waffenlose Kampftechniken}}
\par (BEW 10/17) Unter diesem Begriff werden die verschiedenen waffenlosen Kampftechniken der M"onche zusammengefasst.\\
\begin{tabular}{l|c|c|c|c|r}
Waffe & INI & TbB & Schaden\\
\hline
Schlag, Tritt & +2 & 14/22 & 2W TP\\
Wurf & +3 & 14/22 & speziell \\
Hebel & - & - & 1W SP\\
\end{tabular}

\begin{description}
\item [Verwendete Charakterklassen und Rassen:] M"onch

\item [W"urfe] Beschreibung...\\
\textbf{Mindestwurf:} NK-MW 14\\
\textbf{Wirkung:} Gegner geht zu Boden, 2W SP (Metallr"ustung sch"utzt nicht), immer Rumpftreffer\\
\textbf{Spezial:} Ein Wurf kann einleitend f"ur einen Hebel sein. Um einen Hebel anzusetzen, muss der K"ampfer nach einem gelungenen Wurf einen gezielten Angriff gegen ein Bein oder einen Arm durchf"uhren. Der Angriff erfolgt noch in der selben Runde und der Gegener gilt als am Boden liegend.\\
Wird der Hebel erfolgreich angesetzt, kann in den folgenden Kampfrunden 1W SP verursacht werden. Der Gegner kann in seiner Runde versuchen, sich zu befreien. Der Angreifer f"uhrt einen PW(GES) durch, der Verteidiger einen PW(STR). Die Differenz des W"urfelergebnisses zum Attributswert des Angreifers gibt die Modifikation f"ur den am Boden Liegenden an.\\
Kann kein Hebel angesetzt werden, so bekommt das Opfer einen Malus von 2 auf seine n"achste Aktion.

\item [Feger] Ein Feger dient dazu, durch Wegrei"sen des Standbeines den Gegner aus dem Gleichgewicht und somit zu Fall zu bringen.\\
\textbf{Mindestwurf:} NK-MW 15\\
\textbf{Wirkung:} Gegner geht zu Boden

\item [Tritte, Schl"age] Beschreibung...\\
\textbf{Mindestwurf:} NK-MW 14\\
\textbf{Wirkung:} siehe Tabelle

\end{description}

\section{Geistige Fertigkeiten}
\subsection{Chi\index{C!Chi}}
\par (PSI 8/16) Chi erm"oglicht es dem M"onch, "uber seine normalen k"orperlichen F"ahigkeiten hinaus zu wachsen.

\begin{description}

\item [Verwendete Charakterklassen und Rassen:] M"onch

\item [Verbesserte Initiative] Beschreibung....\\
\textbf{Mindestwurf:} 14\\
\textbf{Wirkung:} 1W zus"atzlich bei Initiative\\
\textbf{Kosten:} 2 PP

\item [Verbesserung k"orperlicher Fertikeiten] Beschreibung....\\
\textbf{Mindestwurf:} 14\\
\textbf{Vorbereitungszeit:} 1 Runde, keine Runde bei 110\% MW\\
\textbf{Wirkung:} Bonus in H"ohe von max. Chi-Wert auf k"orperliche Fertigkeit\\
\textbf{Kosten:} Bonus in PP

\item [Schmerz"uberwindung] Beschreibung....\\
\textbf{Mindestwurf:} 14+Modifikator\\
\textbf{Wirkung:} Schadenreduzierung bis max. Chi-Wert\\
\textbf{Kosten:} 1 PP pro TP\\
\par \begin{tabular}{r|l}
Modifikator & Wunde\\
\hline
1 & leichte Wunde 1-10 TP\\
2 & mittlere Wunde 11-20 TP\\
3 & schwere Wunde 21+ TP\\
\end{tabular}

\item [Verbesserte St"arke] Beschreibung....\\
\textbf{Mindestwurf:} 14\\
\textbf{Vorbereitungszeit:} 1 Runde, keine Runde bei 110\% MW\\
\textbf{Wirkung:} Bonus in H"ohe von max. Chi-Wert auf St"arkeattribut\\
\textbf{Kosten:} Bonus in PP

\end{description}

\subsection{Geographie\index{G!Geographie}}
\par (WIS 6/14) Hier kommt die Beschreibung.....
\begin{description}
\item [Verwendete Charakterklassen und Rassen:] alle
\end{description}

\subsection{Hexenbesen fliegen\index{H!Hexenbesen}}
\par (PSI 12/18) Hier kommt die Beschreibung.....
\begin{description}
\item [Verwendete Charakterklassen und Rassen:] Hexe
\end{description}

\subsection{Magische Spuren lesen}
\par Mit Hilfe des magischen Spurenlesens kann der Waldl"aufer auch sehr versteckte oder verwehte Spuren 
rekonstruieren. Es ist ihm sogar m"oglich, Spuren wieder sichtbar zu machen, die schon komplett entfernt 
wurden. Die Fertigkeitsprobe wird mit Spuren lesen und
Psieinsatz abgelegt.\\
\textbf{Mindestwurf:} $ 15+Zeit+Modifikator $\\
\textbf{Zauberdauer:} sofort\\
\textbf{Wirkungsdauer:} aufrecht erhalten\\
\textbf{Widerstandswurf:} erlaubt\\
\par \begin{tabular}{r|lr|l}
Zeit & vergangene Zeit & Modifikator & Beschreibung\\
\hline
1 & PSI in Minuten & +1 & Spuren wurden absichtlich verwischt\\
2 & PSI in Stunden & +2 & Spuren magisch verwischt\\
4 & PSI in Tagen   & -1 & Waldl"aufer wei"s von den Spuren\\
8 & PSI in Wochen  & -2 & Aufnahme einer verlorenen F"ahrte\\
16 & PSI in Monaten\\
32 & PSI in Jahren\\
\end{tabular}
\begin{description}
\item [Verwendete Charakterklassen und Rassen:] Waldl"aufer
\end{description}

\subsection{Stollennavigation\index{S!Stollennavigation}}
\par (WIS 11/16) Hier kommt die Beschreibung.....
\begin{description}
\item [Verwendete Charakterklassen und Rassen:] Zwerg
\end{description}

\section{Psifertigkeiten}

\section{Sonstiges}
\par Fertigkeiten dieser Kategorie haben keinen Fertigkeitswert.

\subsection{Nachtsicht\index{N!Nachtsicht}}
\par Wesen mit Nachtsicht ist es m"oglich in fast v"olliger Dunkelheit die gleiche Sicht zu haben, wie bei Tageslicht. Sie erhalten keine Modifikationen auf W"urfe in Dunkelheit, so lange noch Restlich vorhanden ist. Dabei reicht ein seichtes Leuchten von einem Punkt v"ollig aus. Bei kompletter oder magischer Dunkelheit gelten f"ur sie die normalen Behinderungen.
\begin{description}
\item [Verwendete Charakterklassen und Rassen:] Zwerg
\end{description}