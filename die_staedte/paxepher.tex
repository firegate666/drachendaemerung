\subsection{Pax Epher\index[regionen]{P!Pax Epher}}
\par Pax Epher, nach ihrem Gr�nder die Stadt des Feuers benannt, nahe am n�rdlichen Ende des Landes gelegen.
Die m�chtigen massiven Steinmauern haben die Stadt schon das eine oder andere Mal vor den Angriffen der Orkhorden
gesch�tzt. Wie auch bei den anderen Hauptst�dten haftet auch dieser Stadt der Makel des Drachen an, der sie erschaffen
hat. Rings um die Stadt herum befindet sich ein ca. 10 Schritt breiter Burggraben, der jedoch nicht mit Wasser gef�llt
ist, sondern dessen Inhalt aus einem unterirdischen Vulkan gespeist wird. Nicht zuletzt kein minder effektiver
Abwehrmechanismus.
\par Der K�nig des Stadt ist ein volksnaher Herrscher. Pax Epher beherbergt keine seperate Festung. Nahe der S�dmauer
befindet sich nur ein kleiner Palast mit einem f�r diese Gegend recht ansehnlichen Ziergarten.

\begin{description}
\item [Geographische Lage] n�rdliches Reich
\item [Stadtgr�ndung] ...
\item [Fl�che] ...
\item [H�he] ...
\item [Einwohnerzahl] ...
\item [Besonderes] Hauptstadt
\end{description}