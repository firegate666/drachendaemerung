\subsection{Die Ostk�ste}
\parpic[l]{\epsfig{file=pics/capitals/d.eps, scale=0.5}}irekt an der K"uste
gelegen stellt die Hauptstadt Pax Lucien\index[regionen]{P!Pax Lucien} den
wohl "ostlichsten Punkt des Kontinents dar (Karte siehe S.
\pageref{fig_map_east}). Mit seiner g"unstigen Meeresposition ist
Pax Lucien\index[regionen]{P!Pax Lucien} damit Ausgangspunkt f"ur jeglichen Seehandel mit den
anderen Reichen und f"ur alle, wenn auch selten stattfindenden,
Schiffsexpeditionen ins offene Meer. Pax Lucien dient ebenfalls als
Zwischenstopp f"ur Schiffsreisen von und zum Nordreich.

\begin{figure}[hbtp]
\begin{center}
\epsfig{file=pics/map_east.eps, scale=0.5}
%width=4.8in, height=3.45in}
\end{center}
\caption{Weltkarte - Ostk�ste} \label{fig_map_east}
\end{figure}

\par Die politische
Situation der Ostk"uste ist gef"ahrdet. Dem Land geht es nicht
unbedingt gut und im Verh"altnis zu den anderen Reichen ist es ein
eher kleines Land. Da hilft es ihm auch nicht, da"s sich hier die
gr"o"ste Seemacht des Kontinents befindet, wenn potetielle Feinde
"ubers Land kommen. Gl"ucklicherweise gab es bis jetzt noch keine
feindlichen "Ubergriffe.

\par Die Hauptstadt des Landes beherbergt eines der gr"o"sten Kloster
der Neuzeit, die \glqq Feste des Wissens\grqq\index{F!feste des Wissens}. Die M"onche hier
wachen "uber die zweitgr"o"ste Bibliothek des Kontinents. Die
bewahrten Sch"atze reichen von einfachen Manuskriptsammlungen "uber
Abhandlungen der G"otter bis hin zu einigen Steintafeln, die
tausende von Jahren alt sind.

\par Nicht unweit von Pax Lucien\index[regionen]{P!Pax Lucien} findet sich die Tempelburg \textit{Sturmfeste\index[regionen]{S!Sturmfeste}}.
Hier ist der Sitz der Priesterschaft der G"otter. Von hier aus
findet die Koordination der gesamten 5 Reiche statt.
