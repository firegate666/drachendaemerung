\section{Fluchmagie\index{F!Fluchmagie}} \label{fluchmagie}
\parpic[l]{\epsfig{file=pics/capitals/d.eps, scale=0.5}}ie Fluchmagie ist wohl die Form der Magie, 
die in der Bev"olkerung am bekanntesten ist. "Ublicherweise f"allt auch jede andere Form von Magie 
meist unter diese Bezeichnung. Unwissenheit ist hier die Ursache. Wie auch immer.
\par Die Anwender von Fluchmagie sind Hexen und deren m"annliches Gegenst"uck die Hexer.
\par Fluchmagie ist eine "au"serst niedertr"achtige Form der Magie und wird von den Hexen meist zu 
Rachezwecken verwendet, dabei muss in den Wirkungsgraden zwischen \glqq f"ur kurze Zeit\grqq{} und 
permanent unterschieden werden. Ein vom Fluch gepeinigter wird meist unter den folgen extrem zu 
leiden haben. Ob es das Pech ist, was ihn verfolgt, oder die Gicht, die ihn von dem Moment an plagt, 
die Folgen sind schwer wieder zu neutralisieren.
\subsubsection{Fluchmagie im Spiel}
\par Wie jede andere Form der Magie, wird Fluchmagie auch "uber die Psifertigkeiten abgewickelt. Die 
Ergebnisse eines erfolgreich gesprochenen Fluches, sind jedoch von anderer Wirkung. Sie zielen meist 
auf die Peinigung des Opfers ab. Hier die Liste der Fl"uche, die nat"urlich wie andere Psi-Kr"afte auch, 
im Laufe des Spiels um Eigenkreationen erweitert werden kann.
\begin{description}
\item [Hexenschuss (Bewegung)] Das Opfer erleidet einen Hexenschuss. Von nun an erh"alt es 
Abz"uge auf alle k"orperlichen und Waffenf"ahigkeiten, sowie auf die Beweglichkeit.\\
\textbf{Mindestwurf:} $ 17+RW^2+Abzug^2+Dauer $\\
\textbf{Zauberdauer:} sofort wirksam\\
\textbf{Wirkungsdauer:} variabel\\
\textbf{Widerstandswurf:} erlaubt\\
\begin{tabular}{r|l}
Dauer & Wirkungsdauer\\
\hline
1 & 1 Spielrunde\\
2 & 1 Kampf\\
3 & 1 Stunde\\
4 & 1 Tag\\
5 & 1 Woche\\
10 & 1 Monat\\
20 & 1 Jahr\\
50 & 10 Jahre\\
75 & 1 Vierteljahrhundert\\
100 & permanent\\
\end{tabular}
\item [Angst (Schock / Empathie)] (siehe ERPS Regelbuch S. 120)
\item [L"ahmen (Bewegung)] (siehe ERPS Regelbuch S. 116)
\item [Liebeszauber (Empathie / Hypnose)] (siehe ERPS Regelbuch S. 120)
\item [Blindheit verursachen (Schock)] (siehe ERPS Regelbuch S. 125)
\item [Fieber (Schock)] (siehe ERPS Regelbuch S. 125)
\item [H"asslichkeit (Hypnose)] Von Beginn des Fluches an, denkt das Opfer es sei h"asslich.
\item [H"asslichkeit (K"orperbewusstsein)] Von Beginn des Fluches an wird das Opfer mit H"asslichkeit gestraft.
\item [Sinne des Vertrauten (Empathie)] Auch wenn es nicht wirklich ein Fluch ist, so wird es doch hier erw"ahnt, da es ein typischer Hexenzauber ist. Er erm"oglicht der Hexe den Zugriff auf die Sinne ihres Vertrauten. Sie kann durch ihn sehen, h"oren und/oder f"uhlen.\\
\textbf{Mindestwurf:} $ 12+RW^2+Anzahl der Sinne $\\
\textbf{Zauberdauer:} Minutenzauber\\
\textbf{Wirkungsdauer:} aufrechterhalten\\
\textbf{Widerstandswurf:} entf"allt\\
RW wird in Kilometern gez"ahlt.
\end{description}
