\section{Alchimie\index{A!Alchimie} und
Thaumaturgie\index{T!Thaumaturgie}} \label{alchimie}
\subsection{Allgemeine Alchimie}
\parpic[l]{\epsfig{file=pics/capitals/h.eps, scale=0.5}}eiltr"anke, Gifte, Tintkturen...
\par Alle diese Sachen fallen in das Aufgabengebiet eines Alchimisten. Dar"uber hinaus 
verf"ugt der Alchimist jedoch auch "uber leichte magische F"ahigkeiten, die es ihm erm"oglichen, 
seine Tr"anke magisch zu verst"arken. Die Palette der m"oglichen Effekte wird hiermit immens 
vergr"o"sert.

\subsection{Die Runenmagie der Zwerge}
\par Legenden berichten von dieser Art von Magie. Geschichten und Mythen erz"ahlen von den m"achtigen 
Artefakten, die die Runenmeister der Zwerge unter Tage schufen und auch noch heute herstellen.
\par Ein Spielercharakter wird wahrscheinlich niemals in den Genuss kommen, bei der Schaffung eines 
solchen Artefakts anwesend zu sein. Die Zwerge h"uten diese Kunst wie ihren Augapfel und selten 
werden Fremde bei einer diese Zeremonien anwesend sein. Viel wahrscheinlicher wird es sein, 
dass ein Abenteurer in den Besitz eines solchen Artefaktes kommt. In der Zeit der Kriege und 
des Widerstands haben die Zwerge viele Waffen, R"ustungen und andere n"utzliche Gegenst"ande 
geschaffen, die jetzt "uber den Kontinent und weiter zerstreut sind.
