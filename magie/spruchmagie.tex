\section{Spruchmagie\index{S!Spruchmagie}} \label{spruchmagie}
\parpic[l]{\epsfig{file=pics/capitals/d.eps, scale=0.5}}ie Spruchmagie
ist die aller"ublichste aller Formen der Magie. Sie wird seit
Urzeiten von allen Magiebegabten angewandt und in den Zirkeln
gelehrt. Sie ist sozusagen die Schulmagie\index{S!Schulmagie}.

\par Seit der Entstehung der Magie, wann auch immer das war, gibt es
auch verschieden Ausrichtungen. Es gibt den Zirkel der Schwarzmagier\index{S!Schwarzmagier}
und den Zirkel der Wei"en, die freien Magier und die Ge"achteten,
letztere sind im Grunde eine extreme, nicht gedultete Ausrichtung
der Schwarzmagier und werden offiziell selbst von denen nicht
geduldet. W"ahrend die Zirkel ihre Hauptaufgabe in der Weitergabe
des Wissens und dem Finden neuer und w"urdiger Sch"uler sehen, jeder
Zirkel hat dabei nat"urlich eine eigene Vorstellung was w"urdig ist,
besch"aftigen sich die freien Magier mit der Forschung und arbeiten
dabei unter Umst"anden auch mal enger mit den Zirkeln zusammen.
Freie Magier sind meist Einzelg"anger, bilden jedoch hin auch wieder
auch einzelne Sch"uler aus, wenn sie auf ihren Wanderungen ein Kind
entdecken, dass sich als sehr potent erweist.

\par Die Spruchmagie teilt sich in 18 Bereiche der Magie auf.
\begin{multicols}{3}
\begin{itemize}
\item Beschw�ren
\item Bewegung
\item Binden
\item Dimension
\item Elemente
\item Empathie
\item Gegenmagie
\item Hypnose
\item Illusion
\item Kontrolle
\item K�rperbewu�tsein
\item Materietransformation
\item Natur
\item Regeneration
\item Schock
\item Telepathie
\item Verst�ndnis
\item Wahrnehmung
\end{itemize}
\end{multicols}
\par Eine genaue Beschreibung der einzelnen Magiebereiche ist dem
ERPS Regelwerk\cite{erps1} zu entnehmen.
\par Dem normalen Magiewirkende werden die Bereich Beschw�rung\index{B!Beschw�rung} und
Dimension\index{D!Dimension} in der Regel nicht zug�nglich sein. Das Verst�ndis dieser
Gebiete ist einfach zu komplex, als das diese neben den anderen
Bereichen erlernt werden k�nnen.
