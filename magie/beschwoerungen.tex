\section{Beschw"orungen\index{B!Beschw�rung}}
\label{beschwoerungen}
\parpic[l]{\epsfig{file=pics/capitals/n.eps, scale=0.5}}eben der
Schwarzmagie stellt die Beschw"orung eine der gef"ahrlichsten und
unberechenbarsten Formen der Magie dar. Nicht zuletzt wurde den
Beschw"orern die Schuld am Erscheinen der Drachen gegeben. Nur zu
verst"andlich, dass diese seit dem nicht mehr ihren Beruf in die
"Offentlichkeit tragen.
\par Doch neben den schwarzen Schafen gibt es auch die jenigen,
die sich mit dem Beschw"oren niederer Wesen besch"aftigen. Kleine
D"amonen, die als Beobachter oder Dienstboten eingesetzt werden,
Beschw"orung der Geister der Toten, um mit ihnen in Kontakt zu
treten, Kontakaufnahme mit Elemtarwesen\index{E!Elementarwesen} oder aber die Kommunikation
mit den Geistern der Pflanzen, dies sind die Aufgabengebiete der
wei"sen Beschw"orer.
\par Auch die Erschaffung von Golems\index{G!Golem} und anderer widernat�rlicher
Wesen wird als Form der Beschw�rung verstanden.
\par Als letztes sei noch erw"ahnt, dass die Nekromantie\index{N!Nekromantie} ebenfalls
eine Form der Beschw�rung ist.

\subsection{Elementarwesen beschw�ren}
\subsection{D�monen beschw�ren}
\subsection{Magische Wesen erschaffen}
\subsection{Geisterbeschw�rung}
\subsection{Totenbeschw�rung}
