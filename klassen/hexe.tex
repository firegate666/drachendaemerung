\subsection{Der Hexer / Die Hexe\index{H!Hexe}\index{H!Hexer}}
\subsubsection{Beschreibung}
\par Hexe! Dieses Wort treibt der Menschheit die Panik in die
Augen. Wer an Hexe denkt an Fl"uche, wer Hexer h"ort sieht den
Schwarzmagier vor Augen.
\parpic[r]{\epsfig{file=pics/hexe.eps, scale=0.5}}
Kein Archetyp ist gef"urchteter und wahrscheinlich mehr gehasst als
dieser. Dabei sehen sich die Hexen verkannt. Ihre Absichten gleichen
sich in keiner Weise mit dem Bild, dass man von ihnen auf dem Banner
tr"agt.
\par Doch woran erkennt man Hexen? Junge Hexen sind grunds"atzlich
bildsch"on und alte Hexen haben eine krumme Nase. Naja, stimmt auch
nicht immer, aber kann will man schon gegen Vorurteile machen.
\subsubsection{Wesen}
\par Das Wesen einer Hexe ist so wandelbar wie das Schicksal dieses
Kontinents. Ihr Gem"ut ist von ihrer Tagesform abh"angig und meist
wird man die Wahrheit nicht erkennen, denn Hexen sind zu dem sehr
gute Schauspieler und verstehen es, ihre Mitmenschen zu
manipulieren. Entweder durch ihren Charme oder aber durch Magie.
\par Man sollte sich jedoch nie den Zorn einer Hexe auf sich ziehen,
denn dieser wird immer auf einen zur�ck kommen und das dann meist in
Form eines Fluches.
\subsubsection{Ausbildungsweg}
\par Hexen suchen sich keine Sch"uler. Das Wissen wird nur innerhalb
der Familien weiter gegeben. Dabei ist es jedoch nicht zwingend
notwendig, dass der Partner ebenfalls ein Hexer oder eine Hexe ist,
damit die Kinder die magische Veranlagung erben, ein Elternteil
reicht aus.
\par So ungef"ahr im Alter von 6 Jahren beginnt die Ausbildung der
jungen Sch"uler. Um das magische Potential der Kinder zu aktivieren,
bedarf es eines besonderen Rituales. Der Hexenerlternteil tritt mit
seinem Kind oder seinen Kindern, so es Zwillinge sind, eine Reise
zur allj"ahrlichen Hexennacht an. Diese findet immer zum
Jahreswechsel statt.
\subsubsection{Vorz"uge / Nachteile}
\par W"ahrend ihrer Ausbildung lernen die Hexen auf Besen zu fliegen
(+1). Eine genaue Beschreibung dieser Fertigkeit ist im Kapitel
\textit{Neue Fertigkeiten} (siehe S. \pageref{neuefertigkeiten}) zu
finden.
\subsubsection{Besonderes}
\par Die Hexe w"ahlt sich w"ahrend ihrer Ausbildung einen Vertrauten.
Meist einen Vogel oder eine Katze. Mit diesem kann sie auf
telepathischem Wege kommunizieren. Er versteht einfache Befehle und
antwortet in einfachen S"atzen.
\par Die Hexe kann Fluchmagie verwenden. Eine Beschreibung dieser
Magierichtung, sowie einiger Fl"uche findet sich im Kapitel
\textit{Fluchmagie} (siehe S. \pageref{fluchmagie})
