\subsection{Der Gl�cksritter\index{G!Gl�cksritter} / Streuner\index{S!Streuner}}
\subsubsection{Beschreibung}
\par F"ur den Gl"ucksritter gibt es viele Namen und er tritt in mindestens so vielen Variationen in den 
unterschiedlichsten Gegenden auf. Er ist ein Kind der Strasse, ein Abenteurer, ein Wanderer und letztendlich 
auch ein Freibeuter auf der See. Der typische Mantel und Degen Held eben.
\parpic[r]{\epsfig{file=pics/streuner.eps, scale=0.6}}
Wir sehen hier vor uns die typischen Bilder der Menschen, die niemals das Abenteuer missen wollen, die es nicht 
ertragen k"onnen lange an einem Ort zu verweilen oder aber in ihrer Freiheit eingeschr"ankt zu werden.
\subsubsection{Wesen}
\par Ein gro"ses Maul hat er wohl und mit der Wahrheit nimmt er es auch nicht immer zu ernst. Und wenn es mal 
drauf an kommt, dann nimmt er auch schon mal die Beine in die Hand. So jedenfalls wird man einen Streuner meist 
erleben.
\par Im Laufe seiner Abenteurerkarriere wird er jedoch versuchen, und das sehr erfolgreich, von seinen 
Weggef"ahrten zu lernen, was zu lernen ist. Und fr"uher oder sp"ater wird er merken, dass der Weg nach vorne 
auch zum Ziel f"uhren kann.
\subsubsection{Ausbildungsweg}
\par Es gibt keine bessere Ausbildung zum Gl"ucksritter, als die Stra"se selbst. Hier lernt der Gl"ucksritter 
von Kindesbeinen an auf eigene Faust zu "uberleben. Obwohl sein "Uberlebenstraining in den ersten Jahren wohl 
haupts"achlich aus Nahrungsbeschaffung besteht. Mit sehr gro"ser Wahrscheinlichkeit ist er in den ersten Jahren 
auch in Waisenh"ausern untergekommen, wo er mit Gl"uck einige Kenntnisse in Wort und Schrift gelernt hat, lange 
geblieben sein wird er dort jedoch nicht.
\subsubsection{Vorz"uge / Nachteile}
\subsubsection{Besonderes}
