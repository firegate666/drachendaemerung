\subsection{Der Priester\index{P!Priester} / Kleriker\index{K!Kleriker}}
\subsubsection{Beschreibung}
\parpic[r]{\epsfig{file=pics/kleriker.eps, scale=0.55}}
Der Kleriker ist der Diener der G"otter. Schon seit fr"uher Kindheit pflegt er das Studium der heiligen Schriften
 und die Beziehung zwischen Kirche, Staat und der gemeinen Bev"olkerung. Durch ihre Bindung zu den G"ottern und ihre 
 F"ahigkeiten Wunder zu wirken, genie"sen sie ein hohes Ansehen beim Volk.
\subsubsection{Wesen}
\subsubsection{Ausbildungsweg}
\par Die Ausbildung zu einem Kleriker dauert lang und beginnt bereits in den fr�hen Kindesjahren.
\subsubsection{Vorz"uge / Nachteile}
\subsubsection{Besonderes}
\par Wie die Paladine auch, k�nnen Priester ihre G�tter um Wunder bitten. Neben den einfachen Wundern der,
werden den Priestern jedoch auch gro�e und manchmal auch au�ergew�hnliche Wunder gew�hrt. Man hat schon Kleriker
gesehen, die wandelnden Schrittes auf der Oberfl�che eines Flusses die Uferseiten gewechselt haben oder auch 
Gl�ubige, die lebend aus einem zusammengebrochenen Stollen zur�ckgekehrt sind.
