\subsection{Der Magier\index{M!Magier}}
\subsubsection{Beschreibung}
\par Der Magier\index{M!Magier} hat von Kindesbeinen an gelernt
die Natur zu kontrollieren. Er ist eins mit den arkanen
Str"ohmungen. W"ahrend seiner langen
Ausbildungszeit wurde ihm beigebracht diese St"urme zu erkennen und
f"ur seine Zwecke zu formen. Magier sind intelligent
und sinnesscharf. Ihnen entgeht nichts. Sie haben allerdings keine
Kenntnisse im Umgang mit Waffen.
\par Ein weiterer Teil ihrer Ausbildung war das Studium
alter Schriften und Folianten.
Magier sind sprachbegabt und kennen Mundarten und
Schriftzeichen, von deren Existenz andere Sterbliche noch nie etwas
geh"ort haben.
\subsubsection{Wesen}
\par Der Magier ist ein ewiger Student. Er hat sein Leben der Magie
gewidmet. Sein Studium der Magie hat ihm die Macht seiner Kunst
gelehrt, er wei� sie mit Bedacht einzusetzen und ist sich ihrer
Gefahren bewusst. Die meisten Magier betrachten sich als
Wissenschaftler und Magie als Wissenschaft, sie w�rden sich niemals
dazu hinrei�en lassen, sie verschwenderisch oder gar prahlerisch zu
nutzen.
\subsubsection{Ausbildungsweg}
\par Das Studium der Magie ist ein langer und schwerer
Ausbildungsweg. Nur wenige Sch�ler der Akademien k�nnen sich r�hmen
am Ende auch einen verbrieften Abschluss in H�nden zu halten. Die
meisten werden fr�her oder sp�ter in einer der P�fungen scheitern
und fortan dem Selbststudium verfallen und als reisende Forscher
unterwegs sein oder als wissenschaftliche Mitarbeiter in der
Akademie ihr Studium der Magie fortsetzen, sich somit von Pr�fung zu
Pr�fung schleppen, in der Hoffnung doch noch mal den Abschluss zu
erreichen.
\par Das magische Potential muss bereits in fr�hen Jahren erkannt
und gef�rdert werden, um es voll aussch�pfen zu k�nnen. Im Regelfall
beginnt die Ausbildung des neuen Sch�lers nach bestandenem
Aufnahmetest mit ca. 4 Jahren. In den ersten 10 Jahren lernt der
Sch�ler neben einigen niederen Magie�bungen zum Erhalt des
Potentials haupts�chlich allgemeines Wissen der Wissenschaften
Mathematik und Alchimie und lernt Sprachen, Lesen und Schreiben und
Magietheorie.
\subsubsection{Vorz"uge / Nachteile}
\par Der Magier kann Spr�che aus nahezu allen Psi-Bereiche
mit Ausnahme von Dimension und Beschw"orung erlernen. Zu Spielbeginn
w"ahlt er frei 3 Zauber aus der Liste aus.
\subsubsection{Besonderes}
\par Jeder Magier muss zu Beginn des Spieles w�hlen, ob er einem
Zirkel angeh�rt oder ein freier Magier ist (siehe dazu das Kapitel
\textit{Die Magie} ab S. \pageref{spruchmagie}), und ob er einen Abschluss
gemacht hat oder nicht.
