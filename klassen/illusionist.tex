\subsection{Der Illusionist\index{I!Illusionist}}
\subsubsection{Beschreibung}
\parpic[r]{\epsfig{file=pics/illusionist.eps, scale=0.55}}
Die Magier k"onnten sich vor Lachen kaum halten, wenn der Illusionist in ihrer Gegenwart verlauten 
lassen w"urde, er sei ein Magier. Doch die Kraft der Illusionisten darf nicht untersch"atzt werden. 
Schon so manch lachender Magier ist an der pl�tzlichen F"ullung des Mundes mit einem nie enden 
wollenden Wasserstrahl erstickt. Nachweisen konnte man dem Illusionisten jedoch nichts, Fl"ussigkeit 
wurde keine gefunden.
\par Den wandernden Illusionisten trifft man entweder unter den Mitgliedern des fahrenden Volkes oder 
aber in einer Horde schatzsuchender Abenteurer.
\subsubsection{Wesen}
\par Auf seinen Reisen wird der Illusionist sich tarnen, um nicht sein wahres \glqq Ich\grqq{} preiszugeben. 
Denn ist ein Illusionist erst einmal als solcher erkannt, wird es ihm schwer fallen seine Gegen"uber zu verwirren.
\subsubsection{Ausbildungsweg}
\par Es gibt keine direkte Ausbildung f�r Illusionisten, irgendwann im Laufe ihrer Kindheit erkennen sie ihr 
Talent selbst.
\subsubsection{Vorz"uge / Nachteile}
Illusionisten sind resistenter gegen die Werke ihrer Kollegen und erhalten immer einen
Bonus von +2 auf Psiwiderstandsw�rfe gegen den Magiebereich Illusion.
\subsubsection{Besonderes}
\par Obwohl der Illusionist niemals auf seine Illusionen hereinfallen wird, so wird er sie doch immer sehen 
k"onnen. Er k"onnte ja auch schlecht mit ihnen arbeiten, w"are es nicht so. 
