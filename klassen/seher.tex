\subsection{Der Seher\index{S!Seher} / Wahrsager\index{W!Wahrsager}}
\subsubsection{Beschreibung}
\par Gestraft oder beg"unstigt? "Uber die F"ahigkeiten des Sehers l"asst sich streiten. W"ahrend der Gro"steil 
der Bev"olkerung die Kraft des Sehers eher als eine Gabe sieht, ist so mancher Seher an eben dieser zu Grunde 
oder in den Wahnsinn gegangen.
\parpic[r]{\epsfig{file=pics/seher.eps, scale=0.6}}
Die jenigen, die mit der Gabe des zweiten Gesichtes beschenkt wurden, leiden meist unter unkontrolliert
 auftretenden Visionen "uber Dinge, die gerade an einem anderen Ort geschehen, geschehen werden oder aber 
 auch geschehen sind. Die Gilde der Seher hat es sich zur Aufgabe gemacht, diese Gabe zu erforschen und es 
 dem Sch"uler zu erm"oglichen, sie zu kontrollieren.
\par Damit sind Seher gern gesehene G"aste in den H"ausern der Reichen, so lange sie nur Gutes zu erz"ahlen 
haben, sie sind jedoch ebenso gef"urchtet, decken sie doch Geheimnisse aus der Dunkelheit auf. Das wiederum
 macht sie zum Lieblingsgast des "ortlichen Sherrifs. Schon so manche als unaufzukl"arend abgelegte Tat, 
 wurde doch noch entschl"usselt und der T"ater seiner gerechten Strafe zugef"uhrt. Diese Tatsache f"uhrt 
jedoch dazu, dass der eine oder andere Seher auch schon mal unschuldig Opfer eines Attentates wurde, 
vorbeugend sozusagen.
\subsubsection{Wesen}
\subsubsection{Ausbildungsweg}
\par Wie bereits oben beschrieben wurde, ist die Gilde der Seher auf der Suche nach neuen Sch"ulern, 
um ihre Kr"afte in kontrollierte Bahnen zu lenken. Dabei ist es besonders wichtig, dass die Sch"uler 
schon in fr"uhen Jahren entdeckt werden, damit ihr Geist nicht schon zu sehr unter den Visionen gelitten hat.
\subsubsection{Vorz"uge / Nachteile}
\subsubsection{Besonderes}
\par Der Seher geh�rt zu den wenigen Begabten, die den magiebereich Dimension\index{D!Dimension} erlernen k�nnen.
