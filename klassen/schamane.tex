\subsection{Der Schamane\index{S!Schamane}}
\subsubsection{Beschreibung}
\begin{quotation}
\par\textit{\glqq Bei dem Schamahn kann man sich streyten, ob man ihn nun zu den Magiern oder zu den Geweihten zelt. Er hat eigentlich von beidn etwas.}
\parpic[r]{\epsfig{file=pics/schamane.eps, scale=0.6}}
\textit{Typische Vertreter der Schamahn kommen aus den sogenannten
Naturvoelkern. Dabei handelt es sich um eine nette Umschreibung, die sich
die Gelehrten ausgedacht haben, um die Wildlebenden wie die Barbaren
und Orks zu umschreyben.}
\par\textit{Wenn die Meisten an einen Schamahn denken, dann sehen
sie einen in Fell gehuellten und mit Knochen behaengten
Eingeborenen, der in unverstentlichen Phrasen versucht die Geister
seiner Ahnen zu beschwoeren. Wahr oder unwahr sey hier mal
dahingestellt. Fakt ist jedoch, das der Schamahn ein Wesen mit
magischen Kreften ist, die in grossen Teilen sogar ueber das
Verstentnis der Schulmagie hinauswachsen.}
\par\textit{In seinem Stamm geniesst der Schamahn einen hohen Stand,
meist sogar hoeher als der Heupdling. Er uebt in Friedenszeyten die
Funktion des Heylers und des Beraters aus. In Zeyten des Kampfes
steht er nebn den Kriegern seines Stammes in der ersten Reye und
sterkt sie mit seinen Zaubern, wehrent er Flueche auf seine Feinde
wirft.\grqq} [\mbox{Lofen} \mbox{Dulsbart}, Knappe im Orden der
Sterne in einem Aufsatz ueber Schamanen]
\end{quotation}
\par Jeder kann f"ur sich selbst entscheiden, was er dem oben genannten Zitat als bare M"unze entgegen nimmt.
\subsubsection{Wesen}
\subsubsection{Ausbildungsweg}
\par Jedes Volk, das an die Geister der Natur glaubt, hat in seinem Stamm einen Schamanen. 
Er hat die Funktion eines Beraters und Weisen inne.
\par Im Laufe seines Lebens w"ahlt der Schamane einen Sch"uler aus den Reihen des Stammes, 
meistens einen Jungen. Dieser wird von ihm in die Geheimnisse der Geister der 
Natur\index{G!Geister der Natur}\index{N!Naturgeister} eingeweiht. Er lernt die Geschichte 
des Stammes und die Kunst des Heilens. Um seinen Geist und K"orper ganz der Natur hinzugeben, 
entsagt der Sch"uler allen k"orperlichen Verlangen und wird sich niemals eine Frau oder einen 
Mann f"ur eine Partnerschaft erw"ahlen. \par Nach 10 Jahren der Aubildung wird es f"ur den 
heranwachsenden Schamanen Zeit, in die Wildnis zu gehen und sich den Geistern der Natur 
vorzustellen. Er sucht einen stillen und abgelegenen Platz in der Steppe auf und begibt 
sich direkt nach Sonnenuntergang in einen meditativen Zustand. So nimmt er Kontakt mit den 
Geistern auf. Diese nehmen Notiz von dem Anw"arter und pr"ufen ihn und seinen Geist. Wird 
er f"ur gut befunden, dann gelangt auf eine spirituell h"ohere Stufe und ist nun bereit den 
Weg des Schamanen zu gehen, wird er abgelehnt, zerst"oren sie seinen Geist und er wird wirr 
oder stirbt sofort. In diesem Fall ist er f"ur den Stamm verloren.
\par Kehrt er als Schamane zu seinem Stamm zur"uck, wird er nun in die tiefen Geheimnisse des 
Schamanismus eingeweiht.
\par In ganz seltenen F"allen wird ein abgelehnter Schamane von den D"amonen\index{D!Daemon} 
der Wildniss aufgenommen. Diese geben ihm seinen Geist zur"uck und vergiften ihn mit ihrer Lehre. 
Solch einen Schamamen kennt man unter dem Begriff \textit{Tza'Sin}\index{T!Tza'Sin} 
(vom Teufel bekehrt). Er kehrt ebenfalls nicht zu seinem Stamm zur"uck.
\subsubsection{Vorz"uge / Nachteile}
\subsubsection{Besonderes}
