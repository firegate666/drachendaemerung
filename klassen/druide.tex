\subsection{Der Druide\index{D!Druide}}
\subsubsection{Beschreibung}
\par Verkannt, missverstanden und ignoriert. Besser kann man den Druiden wahrscheinlich nicht beschreiben.
\parpic[r]{\epsfig{file=pics/druide.eps, scale=1}}
Von allen anderen Magiebegabten wird der Druide eher mit einem L"acheln bedacht. Nur all zu gut kennen sie 
das Bild des Mistel schneidenden Alten. Dabei sollte man seine Macht nicht untersch"atzen. Die Druidenzirkel 
haben "uber die Jahrhunderte einen Fundus an historischem Wissen angesammelt, bei dessen Anblick, und dies ist 
nur bildlich gesprochen, denn Druiden geben ihr Wisssen nur m"undlich weiter und schreiben es nicht nieder, 
die Gelehrten erblassen w"urden.
\par Vom einfachen Volk wird der Druide meist als m"annliches Gegenst"uck der Hexe genannt. Mit diesem 
Vorurteil k"ampfen die Zirkel schon seit ihrer Entstehung.
\subsubsection{Wesen}
\subsubsection{Ausbildungsweg}
\par Die Druidenzirkel suchen ihre Sch"uler im einfachen Volk, alle anderen sind zu weit weg vom echten Leben.
 Meistens suchen sie Waisenh"auser auf und unterhalten sich dort mit den Kindern. Die Kinder, die sich als
  w"urdig und geeignet f"ur eine Ausbildung erweisen, werden von den Zirkeln adoptiert. Bei der Auswahl der
   Kinder wird darauf geachtet, dass sie nicht "uber 6 Jahren sind, ein gutes Verh"altnis zu Tieren und der 
   Natur zeigen und von einem ungestillten Wissensdurst getrieben werden. Die Ausbildung eines Druiden dauert 
   im Grunde ein Leben lang, w"urde jedenfalls der Druide sagen, fragt man ihn direkt, man hat nie zu Ende 
   gelernt. Grunds"atzlich gilt aber, dass die Grundausbildung ca. 8 jahre dauert. In dieser Zeit lernt der 
   Sch"uler alles notwendige Grundwissen in Naturkunde, Wissenschaft und Zauberei.
\subsubsection{Vorz"uge / Nachteile}
\subsubsection{Besonderes}
