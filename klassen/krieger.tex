\subsection{Der klassische Krieger\index{K!Krieger}}
\subsubsection{Beschreibung}
\par Der Archetyp Krieger lie"se sich m"uhelos noch weiter klassifizieren, doch w"urde dies 
nur mehr Namen f"ur ein und die selbe Person in anderer Kleidung mit sich bringen.
\parpic[r]{\epsfig{file=pics/krieger.eps, scale=0.6}}
Wer kennt ihn nicht, den stolzen Krieger der Palastgarde, den k"onigstreuen Ritter\index{R!Ritter}, den aufbrausenden 
Hochlandbarbaren\index{B!Barbar} oder aber den S"oldner\index{S!S�ldner} f"ur den Ehre einfach nur eine Frage des Lohnes ist. All diese 
z"ahlen wir zur Klasse des Kriegers.
\par Sie haben alle eines gemeinsam: Eine hervoragende Ausbildung an den Waffen.
\par Krieger ziehen aus vielerlei Gr"unden auf Abenteuer. Der Ritter zieht aus, um auf Gehei"s seines 
K"onigs einen fl"uchtigen Schwerverbrecher zu fassen oder aber auf einer Queste ein seltenes und doch 
wichtiges Artefakt zu bergen. Der Barbar wird sich mit Sicherheit auf einer Reise befinden, um neue 
Gegenden kennenzulernen oder er ist der Sohn des H"auptlings und er muss sich in der Welt beweisen, 
bevor er dessen Nachfolge antreten kann. Und was den S"oldner betrifft...
\subsubsection{Wesen}
\subsubsection{Ausbildungsweg}
\subsubsection{Vorz"uge / Nachteile}
\subsubsection{Besonderes}
