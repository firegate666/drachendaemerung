\subsection{Der Barde\index{B!Barde}}
\begin{quotation}
\textit{Wo ich bin gef�llt es mir, doch will ich nicht lang
bleiben}\\
\textit{Doch wo ich nie gewesen bin, dort will ich verweilen}
\end{quotation}
\subsubsection{Beschreibung}
\par Egal ob Poet\index{P!Poet}, Minnes"anger\index{M!Minnes"anger} oder Musiker\index{M!Musiker}, sie alle 
sind Barden und fr"ohnen der Kunstrichtung Musik und Dichtung. Hoffnungslos romantisch und mit unbeschreiblicher 
Vorstellungskraft gesegnet, so sehen sich jedenfalls die meisten Barden.
\parpic[r]{\epsfig{file=pics/barde2.eps, scale=0.5}}
Unter ihnen gibt es selbstverst"andlich auch die wandernden Musiker. Sie begleiten Abenteurergruppen, immer 
auf der Suche nach neuen Geschichten.
\par Die Barden sind es, die die Geschichten weitererz"ahlen und neue Legenden schaffen. Vielleicht liegt in 
ihrem Wissen noch ein wichtiger Schl"ussel f"ur die Zukunft?
\subsubsection{Wesen}
\par Barden sind von Grund auf aufgeschlossen. Es liegt in ihrer Natur auf die Menschen zuzugehen und sich mit 
ihnen zu unterhalten. Auf diese Art und Weise nehmen sie Informationen auf. Einen schlecht gelaunten Barden wird 
man ebenso oft treffen wie einen riesenw�chsigen Zwerg, eigentlich nie und es bedarf schon einer besonderen 
Einwirkung von Au"sen, um dies zu bewerkstelligen.
\subsubsection{Ausbildungsweg}
\par Das Talent eines Barden wird sich schon im fr"uhen jugendlichen Alter zu erkennen geben. Musikalisches 
Talent, Redegewandheit und die Kunst zu dichten und zu schreiben zeichen einen jungen Barden aus. Meist wird 
er dann die Hilfe eines erfahrenen Barden aufsuchen und bei ihm in die Lehre gehen. Mindestens genauso viele 
bringen sich alles n�tige selber bei. \par Als n"achstes wird es ihn in die Ferne ziehen, wo er sich einer 
Gruppe von Abenteurern anschlie"sen wird, dies ist der zweite und letzte Teil seiner Ausbildung. So lernt er 
die Welt kennen, bilder seine Talente weiter und wird Geschichten schreiben.
\subsubsection{Vorz"uge / Nachteile}
\subsubsection{Besonderes}
