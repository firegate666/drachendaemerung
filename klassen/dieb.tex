\subsection{Der Dieb\index{D!Dieb}}
\subsubsection{Beschreibung}
\par Heimlichkeit und Geschick sind die Disziplinen des Diebes. Dabei sprechen wir hier jetzt nicht von dem
 Stra"senr"auber\index{S!Stra"snr"auber} oder dem Gelegenheits-Einbrecher\index{E!Einbrecher}. Der Dieb sieht 
 in seinen Taten ein Ritual.
\parpic[r]{\epsfig{file=pics/dieb.eps, scale=1}}
Jede Aktion hat ihren eigenen Reiz und ihre eigene Atmosph"are. Er stiehlt meist nicht um der Beute willen, 
sondern um seine Grenzen zu erfahren und sich jedes mal von neuem zu �bertreffen.
\par F"ur Diebe gibt es keine verschlossenen T"uren oder versteckte Fallen. Ihr Instinkt hat ihnen bisher noch
 aus jeder Situation heraus geholfen.
\subsubsection{Wesen}
\subsubsection{Ausbildungsweg}
\subsubsection{Vorz"uge / Nachteile}
\par Fast jeder Dieb ist ein Mitglied \textit{der Schatten}\index{S!Schatten, die}. Die Schatten sind die 
Kontinentweite Gilde der Diebe. Mitglieder dieser Gilde k"onnen jederzeit auf deren Hilfe bauen. Ob es nun 
um Unterst"utzung, Unterschlupf, Transportm"oglichkeiten oder einfach nur Informationen geht. Im Gegenzug 
\glqq spendet\grqq{} jeder Dieb einen Zehnt seines \glqq Verdienstes\grqq{} der Gilde. Um die Loyalit"at 
ihrer Mitglieder zu wahren, werden die Neuank"ommlinge magisch an den Gildenrat gebunden. Dieses macht es 
dem Einzelnen unm"oglich ihnen gegen"uber die Unwahrheit zu sprechen.
\par Gildenlose Diebe haben es schwer. Sie stehen nicht unter territorialer Kontrolle und k"onnen nicht auf 
die Unterst"utzung der Gilde oder deren Mitglieder bauen, unter ihnen gelten sie als ge"achtet. Im Gegenzug 
k"onnen sie ihre Eink"unfte f"ur sich behalten.
\subsubsection{Besonderes}
