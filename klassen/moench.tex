\subsection{Der M�nch\index{M!M�nch} / Scholar\index{S!Scholar}}
\subsubsection{Beschreibung}
\par M"onche leben zur"uckgezogen in ihren Klosterfestungen und gehen stiller Meditation nach. Nicht selten haben sie auch ein Schweigegel"ubte abgelegt. Sie studieren dort die Schriften vergangener Zeiten, halten Neues fest oder haben sich der Forschung auf wissentschaftlichen Gebieten verschrieben.
\parpic[r]{\epsfig{file=pics/moench.eps, scale=0.6}}
Wenn man einen M"onch auf Reisen trifft, dann hat seine Wanderung immer einen Auftrag des Klosters als Hintergrund. Auffinden von Wissen, Erforschung, Bergung von Artefakten oder "ahnliches. W"unscht es der Suchenden, so wird er f"ur die Dauer seines Auftrages vom Schweigegel"ubte befreit, die Kommunikation f"allt so einfacher.
\par Die verschiedenen Kl"oster haben unterschiedliche Glaubensausrichtungen und eigentlich gibt es zu jedem Glauben auch mindestens eine Gruppe von M"onchen, die sich ihm verschrieben hat.
\subsubsection{Wesen}
\par Im Wesen eher ruhig und im Geiste stark. Durch sein abgeschiedenes Studium und die Meditation ist der M"onch ein eher ruhiger Zeitgenosse. Auch wenn er f"ur die Dauer seiner Reise vom Schweigegel"ubte befreit ist, so wird er dennoch keine unn"otigen Energien in "uberfl"ussige Konversationen verschwenden, sondern statt dessen ruhig in sich gehen. Das soll nicht hei"sen, dass er sich gar nicht unterh"alt. Gerade wenn es wissenschaftliche Diskussionen geht, wird er sich angeregt beteiligen und sein Wissen mit den anderen teilen. Geht es darum, ein R"atsel zu l"osen, wird er ebenfalls versuchen mit seinen Kenntnissen zu helfen.
\par M"onche sind in sich zur"uckgezogen, Angeberei und Prahlerei sind ihnen fern. Sie lassen sich nicht von Herausforderungen locken und werden versuchen Konfliktsituationen friedlich zu l"osen. Kommt es dennoch zu einem Kampf, so werden sie keine unn"otigen Energien in weitere Warnungen verschwenden und den Kampf versuchen so schnell wie m"oglich und so effektiv wie m"oglich zu beenden, dann jedoch auch ohne R"ucksicht auf die Gesundheit des Gegeners. Vom T"oten versuchen sie jedoch Abstand zu halten.
\subsubsection{Ausbildungsweg}
\par W"ahrend seiner Ausbildung lernt der M"onch nicht nur eine Reihe an Wissensfertigkeiten, sondern auch den Umgang im waffenlosen Kampf. Die Handhabung einer Waffe verweigern sie grunds"atzlich, egal ob f"ur den Nah- oder Fernkampf geeignet.
\subsubsection{Vorz"uge / Nachteile}
\par Neben den "ublichen waffenlosen Kampfarten (Boxen und Ringen) beherrschen die M"onche noch eine Reihe weiterer waffenloser Kampftechniken (+1). Eine genaue Beschreibung dazu ist im Kapitel \textit{Neue Fertigkeiten} (siehe S. \pageref{neuefertigkeiten}) zu finden.
\par Auf Grund ihrer Einstellung zu Waffen, ist es den M"onchen nicht m"oglich Waffen irgendeiner Kategorie zu f"uhren oder auch nur eine Fertigkeit aus diesem Bereich zu steigern. Ausnahme dazu sind die waffenlosen Kampftechniken (siehe oben).
\subsubsection{Besonderes}
\par M"onche sind Meister der K"operbeherrschung. Diese spiegelt sich in ihrem Chi Wert wieder. Die Handhabung dieses Wertes und seine Auswirkungen werden im Kapitel \textit{Neue Fertigkeiten} (siehe S. \pageref{neuefertigkeiten}) beschrieben.
