\subsection{Der klassische Waldl�ufer\index{W!Waldl�ufer}}
\subsubsection{Beschreibung}
\par J�ger und Sammler, F�hrtenleser, ein Kind der Wildniss. Waldl�ufer verstehen sich auf das �berleben 
in der Wildniss und im Wald. Sie verstehen die F�hrten wilder Tiere zu lesen und sie verstehen es, Spuren im 
Wald zu deuten.
\parpic[r]{\epsfig{file=pics/waldlaeufer.eps, scale=0.7}}
Im Grunde seines Herzens ist der Waldl�ufer ein geselliger Typ, auch wenn er nach au�en hin einen Einzelg�nger 
reflektiert. Doch die Meinung, die die gemeine Bev�lkerung von diesem verschrobenen gr�nen M�nnchen hat, ist 
pr�gend f�r sein Auftreten.
\subsubsection{Wesen}
\par Obwohl der Waldl�ufer allgemeinhin als Einzelg�nger und verschrobener Einsiedler gilt, so lernen diejenigen, 
die mit ihm Freundschaft schlie�en, eine ganz andere Seite kennen. Er ist Freunden gegen�ber aufgeschlossen und 
wei� sich in eine Gruppe einzugliedern. Er bietet seine Dienste an, wo er kann, bringt jedoch auch eine gesundes 
Ma� an Misstrauen an den Tag, wenn ihm etwas suspekt erscheint.
\par Der Waldl�ufer ist niemand, der intuitiv handelt. Jeder Schritt ist geplant und weitergedacht, immer auf 
m�gliche Folgen fixiert. Wer jetzt denkt, dass der Waldl�ufer langsam handelt, der liegt falsch, denn der Waldl�ufer 
kalkuliert schnell und zuverl�ssig.
\subsubsection{Ausbildungsweg}
\par Die Ausbildung des Waldl�ufers erfolgt, wer h�tte es gedacht, im Wald. Waldl�ufer gehen nicht auf die 
Suche nach neuen Sch�lern, sie erwarten vielmehr, dass diese den Weg zu ihnen finden. Bevor der Waldl�ufer den 
Sch�ler annimmt, muss dieser eine Aufnahmepr�fung bestehen. Diese besteht meist aus verschiedenen Jagd- und 
Suchaufgaben im Wald. F�llt diese erfolgreich aus, ist der halbe Weg schon gemacht. Der zuk�nftige Sch�ler 
muss den Lehrmeister nur noch �berzeugen, warum er gerade ihn als Lehrling annehmen soll. Im Normalfall hat ein 
Waldl�ufer immer 3 bis 4 Sch�ler in seinem Haus.
\par Nach Ende der Ausbildung, im Regelfall nach 5 Jahren, verabschiedet sich der Sch�ler von seinem Lehrmeister 
und zieht auf eine mindestens 10-j�hrige Wandertour, um fremde Gegenden, Tiere und Pflanzen kennen zu lernen. 
Anschlie�end suchen sich die meisten Waldl�ufer eine ruhige Waldgegend und lassen sich nieder, um ihr Wissen 
an andere weiterzugeben oder stellen sich in den Dienst von ans�ssigen F�rsten. Nur wenige bleiben weiter auf 
dem Pfad des Abenteurers.
\subsubsection{Vorz�ge / Nachteile}
\par Dem Waldl�ufer ist es m�glich, selbst dann noch Spuren zu erkennen, wenn der menschliche Verstand bereits 
versagt. Er erh�lt magisches Spuren lesen (+1) und kann die Fertigkeit Jagen erlernen. Genaue Informationen zu 
neuen Fertigkeiten sind im Abschnitt \textit{Neue Fertigkeiten} (siehe S. \pageref{neuefertigkeiten}) zu finden.
\subsubsection{Besonderes}
