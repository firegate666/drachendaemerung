\subsection{Der Beschw�rer\index{B!Beschw�rer, der}}
\subsubsection{Beschreibung}
\par Erst einmal denkt jeder bei der Bezeichnung \glqq Beschw�rung\grqq{}
an schwarze Magie. Man denkt an D�monen\index{D!D�mon},
Geister\index{G!Geist}, Unholde\index{U!Unhold} und
Elementarwesen\index{E!Elementarwesen}. Also an alles
�bernat�rliche, was die eigene Phantasie so her gibt.
\parpic[r]{\epsfig{file=pics/beschwoerer.eps, scale=0.6}}
Und was soll ich sagen, sie haben alle Recht. Das ist exakt das
Gebiet, mit dem sich dieser Zweig der Magiewirker besch�ftigt und
damit ist in der Vergangenheit so mancher Beschw"orer nicht zu
Unrecht auf dem Scheiterhaufen einer seelischen Grundreinigung
unterzogen worden.
\par Doch wir wollen nicht gleich alle auf einmal zum Teufel w"unschen.
Die Gilde der Beschw�rer teilt sich ebenso wie die Gilde der Magier
in verschiedene Ausrichtungen auf und die meisten besch"aftigen sich,
zum Wohle aller, mit der wei"sen Magie. Also die Beschw"orung der
Geister der Toten f"ur Befragungszwecke oder die Erschaffung von
"ubernat"urlichen Wesen f"ur Boteng"ange. Trotz allem bleiben
nat"urlich noch die Individuen, die der Schwarzmagie fr"onen und
diese gilt es zu meiden.
\subsubsection{Wesen}
\par Stumm und verschlossen. Dies sind wahrscheinlich die ersten
Eindr"ucke, die man von einem Beschw"orer bekommen wird und
wahrscheinlich auch meist die Einzigen. Selten wird er sich
ungefragt an einer Diskussion beteiligen und seine Meinung "au"sern.
Viel n"aher liegt es ihm, alles was gesagt wird, sich zu merken.
Spricht man ihn an und fragt ihn etwas, so antwortet er in kurzen,
aber pr"azisen S"atzen oder aber in einer so komplizierten
Formulierung, dass es besser w"are, man h"atte gar nicht gefragt,
denn hinterher versteht man weniger als vorher.
\par Ein Beschw"orer der sich einer Gruppe von Abenteurer
angeschlossen hat, wird ihnen gegen"uber ein wenig aufgeschlossener
sein. Er wei"s, dass es n"otig ist im Team zu arbeiten, damit er in
Notsituationen auf ihre Hilfe setzen kann. Geheimnisse wird er
nichts desto trotz eher f"ur sich behalten, die anderen wissen eh
nicht damit umzugehen.
\subsubsection{Ausbildungsweg}
\subsubsection{Vorz"uge / Nachteile}
\par Der Beschw"orer kann nur die Psi-Bereiche Elemente und
Beschw"oren erlernen. Beschw"orer sind im Umgang mit Elementarwesen
und D"amonen geschult und zeigen keine Furcht vor ihrem Erscheinen
und Auftreten. In Gegenwart dieser Wesen erhalten sie einen Bonus
auf Geistige Stabilit"at (+2).
\subsubsection{Besonderes}
