\subsection{Der Heiler\index{H!Heiler}}
\subsubsection{Beschreibung}
\par Heiler sind vielerorts gerne gesehen. Sie heilen Wunden, kurieren Krankheiten oder versorgen Br"uche und was sonst noch so alles an Blessuren auftreten kann. Anders als die Bediensteten der Tempel und G"otter nutzen die Heiler keine "ubernat"urlichen F"ahigkeiten. Sie greifen haupts"achlich auf ein gro"ses Fachwissen, welches Jahrhunderte innerhalb der Gilde weitergegeben wurde, zur"uck. Das hei"st allerdings nicht, dass ein Heiler es verschm"aht, einem Verwundeten einen alchimistischen Heiltrank einzufl"o"sen oder in besonders schwerwiegenden F"allen die Hilfe von Klerikern in Anspruch zu nehmen. Heiler sind frei von Vorurteilen und nehmen anders die Kleriker gerne die Hilfe anderer in Anspruch.
\subsubsection{Wesen}
\subsubsection{Ausbildungsweg}
\subsubsection{Vorz"uge / Nachteile}
\subsubsection{Besonderes}
